\documentclass{article}
\usepackage{hyperref}
\usepackage{float}
\usepackage{csquotes}
\usepackage[style=iso]{datetime2}
\usepackage[usenames,dvipsnames]{color}
\usepackage{booktabs}
  \setlength\heavyrulewidth{0.20ex}
  \setlength\cmidrulewidth{0.10ex}
  \setlength\lightrulewidth{0.10ex}

\usepackage[font=normalsize,labelfont={bf}]{caption}
  \captionsetup[table]{aboveskip=3pt}

\hypersetup{
    colorlinks=true,
    linkcolor=blue,
    filecolor=magenta,      
    urlcolor=magenta,
 }
 
\usepackage{graphicx}
\graphicspath{ {img/} }
\renewcommand{\abstractname}{CLAUSE DE NON-RESPONSABILITE}
\title{UN GUIDE PRATIQUE POUR LE THS FEMINISANT}
\author{\href{https://katea.gay/}{Katie Tightpussy}}
\date{\today}
\setcounter{section}{-1}
\setcounter{figure}{-1}
\urlstyle{same}

\begin{document}

\maketitle
\tableofcontents
\begin{abstract}
  Je ne suis pas médecin. Je ne travaille pas dans le domaine de la médecine. Je ne suis pas une professionnelle du domaine médical ou d'un domaine associé. Je suis une profane qui offre des opinions de profane basées sur ce que je sais et sur mon expérience. Toutes les informations et les affirmations ci-dessous doivent donc être traitées comme des opinions et non pas des vérités scientifiques établies ou des avis médicaux. Ce guide priorise les éléments découverts par la communauté, la qualité de la recherche scientifique sur ce sujet nous faisant défaut. En bref, c'est votre dos.
\end{abstract}

\section*{LANGUAGES}
\addcontentsline{toc}{section}{LANGUAGES}

Ce document est disponible dans les langues suivantes : 

\href{https://en.pghrt.diy}{🇺🇸 English}, \href{https://de.pghrt.diy}{🇩🇪 Deutsch}, \href{https://fr.pghrt.diy}{🇫🇷 Français}

\noindent\textbf{Si jamais tu es intéressé.e pour faire une traduction ou une version alternative de ce guide, n'hésite pas à me contacter !}



\section{AVANT-PROPOS}

Le but de ce document est de cataloguer mes pensées et mes opinions à propos du traitement hormonal de subsitution (THS) féminisant, car je pense que les différents wikis déjà existants manquent d'efficacité. Ce sont des ressources inestimables, mais de mon point de vue, ces wikis ne sont pas destinés aux personnes qui cherchent des conseils clairs et rapidement traduisibles en action, mais plutôt à ceux qui cherchent à connaître des processus biologiques liés de près (mais malheureusement des fois de loin) à la transition. Mon but ici est de fournir une référence exhaustive et concise qui contient des réponses à toutes les questions que je vois souvent sur comment faire son THS de manière efficace et sûre, grâce à des connaissances que j'ai accumulé à travers les années, dans le but de démystifier ce processus qui sauve des vies, autant pour les personnes qui cherchent à savoir si le THS est pour eux que pour pour les transexuel.les vétérans. Je pars du principe que tu as déjà une idée de ce que fait le THS. Au cas où tu l'ignorais: le THS a probablement beaucoup plus d'effets que tu ne le penses. Ce qui est une bonne chose. \textbf{Changer son sexe, c'est fun et cool. Vraiment une expérience que je recommande.} Tu mérites des soins de transition de qualité, et tu es la personne la plus capable pour prendre les décisions qui te conviendront le mieux. J'espère que ce document t'aidera pour prendre tes décisions, et un jalon dans ton apprentissage si tu souhaite continuer à accumuler des connaissances dans le domaine.

Aussi, reste loin des sub-reddits trans. Fais-moi confiance là-dessus, ok? Au minimum évite /r/mtf, celui-ci est particulièrement horrible. Ce ne sont pas des endroits sains ou de sources de sagesse. Ces endroits te versent directement des vers pourris dans le cerveau, pendant des années. C'est le meilleur conseil que je puisse te donner.

Pour les mecs, certaines sections de ce documents sont tout de même très à propos, mais il y a évidememnt de grosses différences de but et de résultats. \href{https://docs.google.com/document/d/1DXFxzN0XTudPZez\_SO61fpqncRLPH\_Be\_QG\_8Pcz9LU/edit?tab=t.0}{Ce guide pour le THS masculinisant} \textcolor{red}{(Attention : lien Google Docs (et guide en anglais))} à l'air vraiment bien, mais je ne l'ai pas complètement examiné, donc reste vigilant. Ils devraient vraiment faire un mec trans comme Katie Tightpussy. Un type qui s'appelerait genre Oliver Longdick, ou Xavier pourquoi pas. 

\textbf{Si jamais tu veux aider ce projet,} \href{https://cash.app/Katitties}{CashApp}, \href{https://ko-fi.com/katitties}{Ko-Fi}, et \href{https://account.venmo.com/u/katitties}{Venmo} marchent tous. Merci beaucoup!

\subsection*{Comment utiliser ce document}

Ce document est organisé comme une série de questions/résponses, de telle sorte que chaque question et chaque section forment une suite logique. Je t'encourage fortement à le lire séquentiellement, ça se lit comme une conversation qui devrait répondre à la totalité de vos questions (incluant celles que tu ne savais pas que tu te posais), même si c'est très long. Prends ton temps et lis le à ton rythme. 

Sinon, tu peux utiliser la table des matières pour naviguer directement vers une section ou une question particulière, ce qui est très utile si ce n'est pas ta première visite. Je recommande de garder ce document dans un coin pour y revenir à chaque fois que tu as des questions à propos du THS. C'est un peu lourd à absorber en un coup, donc pas de soucis si ça te prends quelques relectures pour tout intégrer, ne te presse pas. 

\noindent\textbf{\href{https://raw.githubusercontent.com/Juicysteak117/pghrt/refs/heads/main/pdfs/pghrt_fr.pdf}{Ce document peut aussi être téléchargé en pdf en cliquant ici. Vraiment, télécharge-le et garde-le quelque part !}}

\noindent\href{https://pghrt.diy/pghrtgretchensversion.txt}{Si tu préfères, tu peux lire une version de ce texte en .txt style années 90-2000.} Par contre je ne garantis pas qu'il soit à jour. (Le texte est en anglais)



\section*{DEDICACE}
\addcontentsline{toc}{section}{DEDICACE}

Ce document est dédicacé à toutes nos sœurs qui ne sont plus. Puisse-t-on porter la lumière de leur torche dans un jour nouveau.

\section{INTRODUCTION}

\subsection{Est-ce que la prise d'œstrogènes est risquée ?}

Avec des hormones modernes, et donc bioidentiques, le THS est tout ce qu'il y a de plus sûr. Au final, tout ce que tu fais, c'est changer le carburant principal qui fais tourner ton corps pour obtenir un équilibre un peu différent avec des hormones qui étaient déjà dans ton corps. Même si c'est un peu plus complexe que ça quand on va chercher à optimiser le processus, l'idée de base de changer sa biologie permet pas mal de largeurs. Le corps est assez malléable et vas pouvoir ajuster ton traitement vers quelque chose qui te fait aller mieux.

\subsection{Quel voie d'administration prendre pour mes œstrogènes ?}

Par injection. C'est le moyen le plus facile, prédicible, sûr, efficace et abordable de faire une transition hormonale. On dit que pour certains, le moment de l'injection devient un rituel attendu, et que pour d'autres, ça en devient même assez fun !  

\noindent\underline{\textbf{Rappel important : Peu importe la manière dont tu prends des œstrogènes, c'est toujours mieux que de ne pas en prendre.}}

\subsection{Pourquoi est-ce que tu ne recommandes pas les pilules, les patches ou les gels ?}

La raison principale est que toutes les autres voies d'administrations ont des inconvénients que les injections n'ont pas. C'est pas qu'elles ne fonctionnent pas, mais tu mérites mieux que de devoir supporter des inconvénients parfois déplaisants. Mais je le répète : \textbf{toutes les formes de THS permettent d'avoir des résultats satisfaisants}. Ca ne veut pour autant pas dire que toutes les formes de THS se valent.

\subsection{Est-ce que le dosage d'œstrogènes est équivalent selon la voie d'adminstration choisie ?}

Non. Et c'est suffisamment important pour ne pas être relègué dans la Section \ref{MM} "Mythes et autres". \textbf{On ne peut pas comparer les dosages d'œstrogènes sans prendre en compte la voie d'administration choisie}. 1mg en pilule n'est pas la même chose qu'1mg en injections par exemple. Les différentes voies d'administration ont des propriétés différentes, ce qui affecte la quantité d'œstrogènes absorbée dans le corps (\text{"biodisponibilité"}), à quelle vitesse, et donc la demi-vie corresponante.

\subsection{C'est quoi une demi-vie ?}

Pour faire simple, la \textit{demi-vie} d'une substance est le temps qu'il faut pour que la moitié de cette substance soit éliminée. Dans le contexte du THS, c'est ce qui détermine combien de temps un dosage reste actif dans ton système, et donc la fréquence de prise d'hormones. C'est ce qu'on va appeler ton cycle hormonal, qui forme une courbe. Tes niveaux montent, arrivent à un pic, puis retombent. Les propriétés de cette courbe (comment tes niveaux d'œstrogènes changent en fonction du temps) sont importants.

\subsection{C'est quoi le soucis avec les pilules ?}

Leur problème principal, c'est qu'elles sont associées à un risque plus élevé de caillots sanguins dûs à une coagulation dans le foie. On peut diminuer ces risques en partie en prenant ces pilules de façon sublinguale ou buccale (en faisant dissoudre la pilule sous ta langue ou entre ta gencive et ta joue, respectivement), plutôt qu'oralement (en avalant la pilule normalement), pour éviter de passer directement par le foie pour métaboliser les œstrogènes. Cependant, même avec les méthodes des voies sublinguales ou buccales, il est fréquent de tout de même avaler une partie du contenu de la pilule, donc on peut imaginer qu'il reste un risque résiduel. Attention, le risque reste très peu élevé, (e.g. \textit{le paracetamol} est plus risqué que l'œstrogène par un ordre de grandeur), mais \textbf{ces complications sont d'autant plus impactantes qu'elles se multiplent avec les risques liés à la prise de nicotine sous œstrogènes}. Pour en savoir plus, tu peux consulter la Question \ref{11-2}. 

A part ça, il y a d'autres soucis liés aux pilules qui viennent principalement de deux de ses caractéristiques : 1) elles ont une demi-vie très courte et une mauvaise biodisponibilité, et 2) elles nécessitent souvent de les accompagner avec des anti-androgènes. Le premier point fait que les pilules sont difficiles à utiliser en monothérapie (dont on dicutera plus en détail ensuite), comparé aux injections. Le deuxième fait qu'on se retrouve avec un assortimment d'effets secondaires liés aux antiandrogènes que tu prendrais avec (voir section \ref{AA} "ANTIANDROGENES"). Prises ensembles, ces caractéristiques et autres degrés de variabilités font qu'on a plus souvent avec les pilules des mauvais régimes remplis d'effets secondaires (comme par exemple un manque d'énergie/libido et des lenteurs à obtenir des résultats) par rapport à d'autres voies d'administration. Aussi, il est plus difficile de stocker des pilules et selon la façon dont tu les obtiens, elles sont plus chères que les fioles d'injections. A noter aussi que si tu essaye de faire importer de grandes quantités de pilules depuis des distributeurs étrangers, ça pourrait faire tiquer les douanes et mener à une saisie des médicaments, ce qui t'expose à une perte financière voire des risques de poursuites judicières. \textbf{Si on te demande, tu ne sais pas qui a commandé toutes ces pilules.}

\textbf{Si, pour quelque raison que ce soit, tu utilises des pilules, s'il te plaît prends de 4 à 8mg en voie sublinguale en espaçant tes prises dans la journée.} Tout ce qui est en dessous de 4mg n'est probablement pas un dosage suffisant.

\subsection{Qu'est ce qui ne va pas avec les patches ?}

\begin{itemize}
  \item C'est cher (plus encore que les pilules);
  \item Plus difficile de s'en procurer en DIY (seulement via des marchés gris);
  \item Te demandent généralement un antiandrogène (voir la  Section \ref{AA} “ANTIANDROGENES”);
  \item Peuvent causer une irritation de la peau;
  \item Nécessistent d'être appliqués 24h/24;
  \item Se décollent souvent (NdT: si ça t'arrive, demande des pansements qui couvrent le patch à la pharmacie);
  \item Ne permettent pas une absorbtion uniforme (à cause de la chaleur par exemple);
  \item Difficile de faire des stocks (difficile à acheter en grandes quantités);
  \item Ne permettent souvent pas de dépasser des niveaux équivalents à une ménopause même avec plusieurs patches en même temps.
\end{itemize}

\subsection{Et le gel alors ?}

\begin{itemize}
  \item Difficile à doser avec précision, ce qui mène à des niveaux non-uniformes;
  \item Requièrent une application régulière de truc poisseux à cause d'une demi-vie relativement courte;
  \item Peut-être sale (poisseux);
  \item Tu risques d'exposer les autres par contact
  \item Te demande généralement un antiandrogène (voir la  Section \ref{AA} “ANTIANDROGENES”);
\end{itemize}

Il faut cependant noter que le gel nécessite peu d'éléments pour en produire soi-même, ce qui peut-être avantageux selon les circonstances.

\subsection{Pourquoi pas les implants?}

\begin{itemize}
  \item Généralement beaucoup plus cher que n'importe quelle autre option;
  \item Peu de possibilités d'en obtenir;
  \item Les périodes d'ajustements de doses sont très espacées;
  \item Un implant défectueux peut te faire avoir des niveaux insuffisants;
  \item Un implant cassé ou écrasé peut causer des niveaux trop hauts de façon inattendue;
  \item Impossible à obtenir en DIY.
\end{itemize}

Ce dernier point en particulier veut dire qu'il n'est possible de s'en procurer que dans peu d'endroits, et la procédure coûte cher. C'est peut-être même la première fois que tu en entends parler. Tu vois le soucis avec ça ?

\subsection{Et pour les sprays ?}

Ils sont encore assez peu utilisés et on a peu de retours donc il n'y a pas grand-chose à dire à leur propos, mais ils ont pas mal d'avantages / inconvénients en commun avec le gel. Je note ça là essentiellement pour rappeler qu'ils existent.

\subsection{La différence est vraiment si énorme que ça ?}

\textbf{Oui.} C'est tellement le cas que j'ai écrit tout ce truc juste pour arrêter de me répéter tout le temps et envoyer un lien vers ici à la place. Un régime d'injections bien dosé est la meilleure forme d'œtrogènes qu'on a pour obtenir des niveaux corrects en monothérapie.

\subsection{Est-ce que ce tableau est vrai ?}\label{1-12}

\begin{figure}[H]
  \centering
  \includegraphics[width=1\linewidth]{STUPID_CHART_evil_bad_bad_destroy_evil_bad.png}
  \caption{Ce tableau est nul}
  \label{fig:scebbdeb}
\end{figure}

\textbf{Non.} Même si ce guide n'est pas là pour faire le tape-taupe en répondant directement à chaque exemple de désinformation sur les réseaux sociaux, la prévalence de ce tableau à la fois en ligne et sur les ressources données par les médecins partout, ainsi que le volume impressionnant de mal qu'il a causé montre un besoin d'une exception qui va au-delà d'une référence rapide dans la Question \ref{11-6}. 

\textbf{Ce tableau est complètement faux.} Presque chaque aspect de celui-ci est trompeur d'une manière ou d'une autre, l'exception étant qu'il est totalement vrai que les œstrogènes ne causent pas de changements vocaux. Les temps listés pour les "temps attendus" sont trompeurs, et l'idée "d'effet maximum" est tellement trompeur que ça flirte avec le criminel. Une pléthore de changements ne sont pas listés (comme les effets sur ta psyché) ou sont faux (le tableau se contredit sur la dysfonction érectile). Aucune considération n'est donnée sur la qualité du régime par ailleurs.

\textbf{Les aspects clés de ce tableau qui demandent quelques nuances sont abordés dans ce guide, donc la seule chose que tu dois en garder est que les changements arrivent à des rythmes différents.} Ce tableau cause des attentes mal placées, montre une vue incorrecte de ce que le THS fait, et mènent les personnes trans à leur perte en limitant leur compréhension des hormones. S'il te plaît, oublie-le complètement. Maintenant que c'est dit, retournons à notre sujet.
 

\section{POURQUOI LES INJECTIONS}

\subsection{Qu'est ce qui fait que les injections, c'est si bien ?}

\textbf{La régularité.} La régularité est le truc important pour un THS. La régularité, ça veut dire que ton régime est stable, et la stabilité, c'est bien. Même les pires formes d'injections (on détaillera ça plus tard) te créeront un cycle hormonal plus prédictible que toutes les autres voies d'administration, ce qui a beaucoup d'avantages.

\subsection{Est-ce qu'on a besoin d'antiandrogènes avec des injections ?}

En général, non. Si tu as un cycle d'injection qui est correctement dosé avec des prises suffisamment fréquentes, qui donne des niveaux d'œtrogènes suffisamment hauts tout le temps pour arrêter la production naturelle de testostérone, cela rendra les antiandrogènes inutiles, ce qui est préférable dans la plupart des cas. C'est ce qu'on appelle la \textit{“monotherapie”}.

\subsection{La monthérapie, ça marche comment ?}\label{2-3}

Pour faire simple, ton cerveau se fout un peu de savoir quel type d'hormone sexuelle est dans ton corps, du moment que tu en as en assez grande quantité. Si tu as tout le temps assez d'hormones dans ton corps, il ne voit pas d'intérêt à en produire plus. C'est cette "régularité" qui fait que les injections résussissent là où les autres voies d'administration sont plus en peine. Faire une monothérapie avec des pilules, par exemple, est quasiment impossible dans la plupart des situations. Si tu veux que je précise un peu, si on regarde \href{https://en.wikipedia.org/wiki/Hypothalamic-pituitary-gonadal_axis}{l'axe HPG (NdT : je référence l'artice anglais parce que l'équivalent français est vide)}, la production \textit{d'hormone lutéinisante} (LH) et \textit{d'hormone folliculo-stimulante} (FSH) est supprimée par les niveaux élevés \textit{d'estradiol}, ce qui inhibe la production de testosterone dans les testicules. 

\subsection{Comment ça se fait que les injections soient moins risquées pour la santé ?}

Comme on n'a pas besoin de les utiliser avec des antiandrogènes (voir la Section \ref{AA} “ANTIANDROGENES”), le risque à long terme associé à l'utilisation des antiandrogènes sur la santé est évité. Si tu utilises des œstrogènes bioidentiques et qu'elles ne passe pas par ton foie (voir la question \ref{11-1}), alors tu es au plus proche possible d'une production naturelle d'œstrogènes par ton corps, ce qui enlève la plupart des risques pour ta santé.

\subsection{Mais y'a pas des risques associés au fait de s'injecter une substance dans le corps ?}

En soi oui, mais avec un peu de technique, tout ce que tu risques c'est un bleu (see Section \ref{ts} “TECHINQUES ET MATERIEL”). C'est un peu comme faire du vélo, une fois que tu sais faire, faut vraiment le vouloir pour faire n'importe quoi.

\subsection{D'accord, mais en quoi les injections sont plus simples que les autres voies d'administration ?}

Parce qu'une fois que tu as un régime stable, t'es tranquille. Tu t'en occupe moins souvent (e.g. une injection par semaine contre plusieurs pilules par jour), tu as des dosages précis (contrairement au gel), l'apport d'œstrogènes ne s'arrête pas en plein milieu de cycle à cause d'un patch qui se décolle, et tu n'as pas besoin de te déplacer pour aller voir un médecin souvent (contrairement aux implants).

\subsection{Comment ça se fait que les injections soient aussi abordables ?}

Pour faire simple, c'est parce que tu as besoin de moins de principe actif. Une fiole de 5 mL contenant à peu près un an de THS ne contient que 200mg d'œstrogènes, alors que l'équivalent en pilules par exemple (4mg * 365 jours = 1460 mg) en contient beaucoup plus. C'est une comparaison qui a ses limites, mais ça donne une idée. Une autre comparaison marrante : tu peux faire tenir une fiole qui contient 1 à 2 ans d'œstrogènes dans une boîte de 3 mois de pilules. 

\subsection{Je ne peux pas en acheter / me faire rembourser en pharmacie, je fais comment ? (ou mon médecin ne veut pas m'en prescrire)}

Va voir la section \ref{sv} "OU TROUVER DES INJECTABLES". Ca va changer ta vie, et probablement te radicaliser dans la foulée.

\subsection{Est-ce que je peux passer aux injections même après quelques mois / années sous THS ?}

\textbf{Oui.} Je ne garantis rien, mais beaucoup de gens racontent avoir vu une différence notable après être passé.e.s aux injections même après quelques années sous THS. On parle d'un plus grand développement mammaire, une meilleure santé mentale, moins de soucis liés aux effets secondaires des anti-androgènes ou aux autres voies d'administration des œstrogènes. Vraiment, saute le pas, ça en vaut la peine.

\textbf{A noter que le temps sous THS avant d'être sous injections n'est pas "gâché", ni qu'il y a une fenêtre de temps limitée pendant laquelle les œstrogènes sont plus efficaces pour la féminisation.} C'est la régularité qu'apporte les injections qui les rend si intéressantes, mais la destination reste largement la même dans tous les cas. Voir les Questions \ref{11-5} et \ref{11-6}. 

\subsection{Oui mais j'ai peur des piqûres ?}

On va pas se mentir, les piqûres font peur, surtout au début. Beaucoup de gens n'aiment pas ça, parce qu'instintivement tu ne veux pas faire des trous dans ton corps, mais avec une bonne technique et de bons instruments, ça ne fait presque pas mal. Il y a énormément de gens qui avaient des cas sérieux de bélénophobie (peur des aiguilles) et qui maintenant trouvent l'expérience de l'injection ennuyante. C'est une peur commune et normale, mais c'est surmontable et ça vaut le coup de la surmonter. "Oh ce n'est pas si terrible que ça" est une réaction assez commune. Comme le dit le mantra: fais-le en ayant peur, ça ira. 

\subsection{Est que la sensation des injections est semblables à une prise de sang ou à un vaccin ?}

Non. Une prise de sang utilise des aiguilles beaucoup plus larges et on pique dans un endroit beaucoup plus sensible, tout en te drainant du sang, ce qui est déplaisant en temps normal. Les vaccins contiennent des organismes qui causent des réactions immunitaires douloureuses parce que c'est des vaccins. Les injections d'œstrogènes ajoutent une petite quantité d'hormones dans ton corps qui te fait te sentir bien parce qu'après tu as des hormones dans ton corps. Je suis sûre que tu vois la différence. L'acte de s'injecter soi même peut aussi être plus facile plutôt que ce soit quelqu'un d'autre qui le fasse, selon ta personnalité.

\subsection{Est-ce qu'il y a des outils d'accessibilité pour faciliter les injections ?}

Oui. Il existe des auto-injecteurs qui peuvent être très utiles si tu as des soucis moteurs par exemple. Pour plus d'informations, va voir la question \ref{5-21}, ou continue de lire.

\subsection{Tu comprends pas, \textit{je} suis spécial.e et je ne peux pas faire d'injections parce que j'ai des os en verre et une peau en papier et\textemdash{}?}

Je comprends que tu aies peur, mais si tu ne veux pas faire d'injections sous aucune circonstance et que tu n'as pas de contre-indication comme le fait d'être hémophile, alors ne le fais pas. Tu peux juste dire ça, c'est ok. Quand tu changera d'avis, ce guide sera toujours là. Et si tu ne changes pas d'avis, pas de problème.
 

\section{TYPES ET DOSAGES}\label{td}

\subsection*{Vocabulaire clé}
\addcontentsline{toc}{subsection}{\textemdash{} Vocabulaire clé}

\subsection{Quels sont les différents types d'œstrogène injectable ?}

Les quatre types principaux utilisés pour le THS sont \textit{le valerate d'estradiol} (EV), \textit{le cypionate d'estradiol} (EC), \textit{l'enanthate d'estradiol} (EEn) et \textit{l'undecylate d'estradiol} (EUn). Chacun de ces composés est un "ester" de \textit{l'estradiol} et va être converti en \textit{estradiol} dans ton corps.

A noter que dans certaines régions, les pilules sont vendues sous le nom \textit{valerate d'estradiol}, qui peut porter à confusion. Cette section ne se réfère qu'à sa forme injectable.

\subsection{Quelles sont les différences entre les types d'œtrogènes injectables ?}

La seule différence intéressante entre les esters est que chacun a une demi-vie différente, ce qui change leur courbe hormonale résultante, affectant le dosage et sa fréquence.

\subsection{Un type d'œstrogène injectable est-il meillleur pour la féminisation qu'un autre ?}

\textbf{Non.} Les différences affectent uniquement le dosage et la fréquence de prise, ce qui change l'exprérience de manière qualitative. Ca peut rendre un ester préférable à un autre, mais les 4 types fonctionnent convenablement et partagent tous les avantages des injections.

\subsection{Quel type d'œstrogène choisir si j'ai le choix ?}

Si tu as le choix, \textit{l'enanthate d'estradiol} est préférable pour la plupart des gens au vu des niveaux exceptionnelement stables qu'il permet d'avoir, avec l'inconvénient que dans la plupart des pays ce choix est seulement possible si tu le fais en DIY (voir la section \ref{sv} "OU TROUVER DES INJECTABLES"). Si tu passes par un docteur, tu auras probablement l'option de prendre du \textit{cypionate d'estradiol}, mais probablement seulement peu concentrée, ce qui peut être ennuyant en fonction de ta tolérance pour des gros volumes d'injections. L'injectible le plus prescrit (surtout aux USA), \textit{le valerate d'estradiol}, permet toujours d'avoir des bons résultats, mais il est un peu ennuyant sur certains points qui font qu'il n'est pas préférable d'en prendre si tu as le choix (i.e. quand tu fais du DIY). Continue à lire.

Note : en France, étant donné qu'aucune formule injectable n'a reçu d'autorisation de mise sur le marché, et que les pays qui permettaient auparavant d'importer des injections ne le permettent plus, il n'est aujourd'hui pas possible de légalement obtenir des fioles d'injections. La seule option pour nous est donc le DIY. Si tu veux changer ça, rapproche toi des différentes associations d'activistes trans qui se battent pour faire entrendre raison à l'ANSM.
% TODO: localisation

\subsection{C'est quoi la concentration ?}

Les fioles d'œstrogènes sont faites à partir d'œstrogènes contenues dans une solution organique. La \textit{concentration} d'une fiole est la quantité d'œstrogène contenue dans cette solution. C'est un ratio de la masse par rapport au volume de la fiole. En d'autres termes : pour chaque millilitre d'huile (la mesure de volume), il y a tant de milligrammes d'œstrogènes (la mesure de masse). Tu va souvent voir des concentrations listées selon le volume total de la fiole (e.g. 200mg/5mL) mais il est toujours préférable de simplifier cette fraction (donc 40mg/mL dans ce cas). \textbf{Les concentrations communes sont 5mg/mL, 10mg/mL, 20mg/mL, 40mg/mL, et occasionnellement 50mg/mL.}

\subsection{Qu'est ce qu'on veut dire par "dosage et fréquence ?"}

\textit{Le dosage} et la \textit{fréquence} sont les deux facteurs qui déterminent ton cycle hormonal. \textit{Le dosage} est la quantité d'œstrogènes que tu mets dans ton corps (mesuré en mg), et la \textit{fréquence} est à quelle fréquence tu mets des œstrogènes dans ton corps (mesurée en jours ou en semaines). Tu vas souvent entendre le mot "régime d'œstrogènes", ce qui référence tout ce qui est lié au THS que tu prends et à quelle fréquence.

\subsection{Comment est-ce que je sais quel dosage prendre ?}

Tu obtiens ton dosage en multipliant la concentration de ta fiole par le volume que tu es en train d'injecter \[Concentration (mg/mL) * volume (mL) = dosage (mg)\] \textbf{J'insiste sur le fait que le volume à lui seul ne suffit pas pour avoir un dosage correct.} Si tu veux une analogie, tu peux le voir avec la pâtisserie: tu ne peux pas dire à quelqu'un "fais cuire ça au four pendant 45 minutes", sans lui dire à quel température faire chauffer le four.

\subsection{Est-ce que je pourrais avoir un exemple de calcul de dosage ?}

Le calcul est super simple, promis ! J'ai mis une table de référence dessous qui compare en fonction de la concentration et du volume pour quelques dosages communs. Pas la peine d'être plus précis.e que 2 chiffres après la virgule. De toute façon, tu n'auras jamais une seringue assez précise pour doser 0.153mL par exemple. C'est une marge d'erreur acceptable qui ne fera aucune différence pour ce qu'on cherche à faire. Tu peux aussi utiliser \href{https://hrtcafe.net/Calc/}{ce calculateur} comme référence.

\begin{table}[]
\centering
\caption{Exemples de Dosages pour des couples Concentrations / Volume Fréqents}
\label{tab:concentrations}
\begin{tabular}{@{}llllll@{}}
  \toprule
  \multicolumn{1}{c}{} & \multicolumn{5}{c}{Concentrations (mg/ml)} \\
  \cmidrule(rl){2-6}
            & 5    & 10  & 20 & 40 & 50   \\
            \cmidrule(rl){2-6}
Dosage (mg) & \multicolumn{5}{c}{Volume (mL)}  \\
    \cmidrule(r){1-1} \cmidrule(lr){2-6}
4        & 0.8  & 0.4 & 0.20  & 0.10 & 0.08\\
5        & 1.0    & 0.5 & 0.25 & 0.13 &  0.10\\
6        & 1.2  & 0.6   & 0.30  & 0.15 & 0.12   \\
7        & 1.4  & 0.7 & 0.35  & 0.18 & 0.14\\
8        & 1.6  & 0.8   & 0.40  & 0.20 & 0.16  \\
9        & 1.8  & 0.9 & 0.45  & 0.23 & 0.18\\
10       & 2.0    & 1.0   & 0.50  & 0.25 & 0.20 \\
    \bottomrule
\end{tabular}
\end{table}

\textbf{Comment lire ce tableau :} Commence par prendre la dose que tu veux sur la gauche, et en choisissant avec la colonne correspondant à la concentration de ta fiole, tu obtiens le volume correspondant sur la droite. Tu peux déjà remarquer que les volumes nécessaires pour les fioles concentrées à 5mg/mL ne sont pas terribles. C'est parce que les fioles concentrées à 5mg/mL ne sont pas terribles.

\subsection{Comment est-ce que je convertis les dosages entre différents esters ?}

\textbf{Tu ne le fais pas.} Comme ils se comportent de manière différente, on ne peut pas faire de "conversion" de dosages. Si tu changes d'ester, tu dois (re)commencer à un dosage typique pour ce nouvel ester et ensuite retrouver ton cycle de croisière à partir de là. Tu peux faire des comparaisons entre esters, mais on n'a aucune méthode pour convertir des dosages.

\subsection{Comment est-ce que je peux comparer les différentes courbes en fonction du dosage et de l'ester choisi ?}

Si tu veux fouiller le sujet, je recommande \href{http://estrannai.se}{estrannai.se} que je trouve très bon. Garde en tête que ce n'est pas nécessaire, mais c'est un bon outil pour faire quelques comparaisons. \href{https://estrannai.se/\#i0__cu,7,7,1-cu,5,7,3-cu,5,7,2}{Voici un exemple de comparaison entre quelques dosage hebdomadaires typiques} qu'on va détailler un peu plus dès à présent.

\textbf{A noter que les dosages que je vais lister ci-dessous devraient suffire pour une dose minimum dans la plupart des cas.} Commence avec le dosage le plus bas, et augmente si besoin. Plus ne veut pas nécessairement dire mieux, mais on va voir ça plus en détail plus tard. Ces dosages fonctionneront peu importe comment tu as obtenu ta fiole.

\subsection*{A la rencontre de nos Esters}
\addcontentsline{toc}{subsection}{\textemdash{} A la rencontre de nos Esters}

\subsection{Comment est-ce que je dose mon \textit{valerate d'estradiol}?}

Tu peux soit prendre une plus petite dose deux fois par semaine, ou une plus grosse dose une fois par semaine avec du \textit{valerate d'estradiol}. C'est en fonction de ta tolérance et de ton confort. En gros, il te faut 1mg pour chaque jour de ton cycle pour une injection tous les 3 à 7 jours. \textbf{Si tu fais une injection hebdomadaire, je recommande entre 6 et 8 mg}, mais 4 à 5mg tous les 5 jours est tout aussi bien. \textbf{Fais attention à ne pas dépasser 7 jours entre deux injections.} Un cycle hebdomadaire pousse déjà suffisamment la demi-vie de l'ester. Toute fréquence plus basse est hautement déconseillée pour éviter les mêmes effets secondaires que ceux liés à l'irrégularité dans les prises (voir la Question \ref{7-3}).

Je rappelle que dans certains endroits, les pilules sont vendues sous le nom de \textit{valerate d'estradiol}, ce qui peut porter à confusion, cette section ne réfère qu'à la forme injectable du produit.

\subsection{Qu'est-ce qui caractérise la courbe hormonale de \textit{valerate d'estradiol} ?}

\textit{Le valerate d'estradiol} est l'ester qui a le plus une courbe en forme de pointe. Il monte rapidement à un pic quelques jours après l'injection et retombe durement et rapidement. Cette relative instabilité peut être déplaisante en fonction de ta sensibilité, mais tu peux amoindrir les effets en ajustant le dosage et la fréquence.

 \begin{figure}[H]
     \centering
     \includegraphics[width=1\linewidth]{ev.png}
     \caption{Serum Estradiol (pg / ml) de valerate d'estradiol en fonction du temps (jours) }
     \label{fig:ev}
 \end{figure}

\subsection{Comment est-ce que je dose mon \textit{cypionate d'estradiol}?}

\textit{Le cypionate d'estradiol} peut être pris de façon hebdomadaire sans soucis. \textbf{On prend généralement une dose entre 5 et 7 mg.} Je ne recommande pas une fréquence plus basse qu'une fois tous les 7 jours (e.g. une fois tous les 10 jours) parce que l'œstrogène sera moins efficace que prise de façon hebdomadaire, vu qu'il faut une dose de plus en plus haute pour avoir des niveaux acceptables. Toute fréquence plus basse qu'une fois tous les 7 jours est hautement déconseillée pour éviter les mêmes effets secondaires que ceux liés à l'irrégularité dans les prises (voir la Question \ref{7-3}).

\subsection{Qu'est-ce qui caractérise la courbe hormonale de \textit{le cypionate d'estradiol} ?}

\textit{Le cypionate d'estradiol} pardonne un peu plus que \textit{le valerate d'estradiol}. La courbe ne progesse pas aussi vite et varie beaucoup moins entre sa partie haute et sa partie basse, mais on remarque tout de même une montée et descente visibles sur un cycle hebdomadaire typique.

 \begin{figure}[H]
     \centering
     \includegraphics[width=1\linewidth]{ec.png}
     \caption{Serum Estradiol (pg / ml) de cypionate d'estradiol en fonction du temps (jours) }
     \label{fig:ec}
 \end{figure}

\subsection{Comment est-ce que je dose mon \textit{enanthate d'estradiol}?}

\textit{L'enanthate d'estradiol} peut se prendre sur cycle hebdomadaire, qui peut être étendu à une injection tous les 10 jours si ça t'arrange. Tu peux techniquement l'étendre plus que ça, mais je ne le recommande pas parce que tes niveaux seront de plus en plus instables. \textbf{On prend généralement une dose hebdomadaire entre 4 et 6mg}, ou 5 à 7mg pour une injection tous les 10 jours. Je préfère la solution hebdomaire de toute façon pour des raisons de facilité de planification, les extensions jusqu'à 10 jours n'apportant rien de particulier. 

\subsection{Qu'est-ce qui caractérise la courbe hormonale de \textit{l'enanthate d'estradiol} ?}

\textit{L'enanthate d'estradiol} est le standard pour l'œstrogène injectable. Il a une courbe extrêmement plate (i.e. qui varie peu) sur le cycle hebdomadaire typique. Ca permet d'avoir des niveaux très réguliers sans avoir trop des effets secondaires sembables à l'irrégularité dans les prises (voir la Question \ref{7-3}).

 \begin{figure}[H]
     \centering
     \includegraphics[width=1\linewidth]{een.png}
     \caption{Serum Estradiol (pg / ml) d'enanthate d'estradiol en fonction du temps (jours) }
     \label{fig:een}
 \end{figure}

\subsection{Comment est-ce que je dose mon \textit{l'undecylate d'estradiol}?}

\textit{L'undecylate d'estradiol} est capable de dépasser de loin le cycle hebdomadaire pour aller vers le mensuel ou le trimestriel. Cependant, on n'a pas de standard ou de base pour déterminer une dose recommandée. Les éléments qui affectent la façon dont une injection d'œstrogènes est absorbée (\textit{"la pharmacocinétique"}), qui sont négligeables pour les autres esters sont plus influents pour \textit{L'undecylate d'estradiol}. A cause de ça, cet ester est toujours un terrain très expérimental qui dépasse le sujet de ce guide. Honnêtement, tu peux aller consulter un almanac de sorcière, ça ferait partie des meilleurs conseils que tu puisses trouver.

\subsection{Qu'est-ce qui caractérise la courbe hormonale de \textit{l'undecylate d'estradiol} ?}

On sait pas trop. Il y a trop peu de données pour avoir une vision claire, et il y a beaucoup de variables à prendre en considération. C'est quelque chose que tu peux rechercher et expérimenter si ça d'intéresse, mais c'est un saut dans l'inconnu et tu dois d'abord comprendre les risques induits par le fait d'être un rat de laboratoire, donc je ne recommande pas à moins de savoir ce que tu fais.

 \begin{figure}[H]
     \centering
     \includegraphics[width=1\linewidth]{moon.png}
     \caption{La lune}
     \label{fig:moon}
 \end{figure}

 

\section{PRISES DE SANG ET NIVEAUX}

\subsection*{Obtenir des résulats}
\addcontentsline{toc}{subsection}{\textemdash{} Obtenir des résulats}

\subsection{A quelle fréquence dois-je tester mes niveaux ?}

Tant que tu es en train d'affiner ton dosage, tu dois tester relativement fréquemment. Pour suivre chaque ajustement à ton régime, tu devrais laisser 1 ou 2 mois à tes niveaux pour se stabiliser, et tester dès que as atteint un rythme de croisière.

\subsection{Dois-je tester mes niveaux avant de commencer mon THS ?}

Ce n'est pas nécessaire, parce que tes niveaux de testosterone seront trop hauts et ceux d'œstrogène trop bas, donc ce n'est pas une donnée vraiment intéressante. Cependant, des prises de sang de routines (i.e. bilan lipidique et autres) sont recommandés pour ta santé de toute façon. L'exception étant si tu penses avoir une condition intersexe qui peut affecter ton régime de THS vu que c'est quelque chose qu'on peut des fois détecter dans une prise de sang préliminaire.

\subsection{Est-ce que j'ai besoin de tester mes niveaux si je n'ai pas changé mon dosage depuis longtemps ?}

Il n'y a pas vraiment de raison de le faire : vu que tu n'as rien changé, rien n'a changé. Si tu veux avoir l'esprit tranquille ou si tu as changé des aspects de ta routine ou ta façon d'acheter ton THS, et de temps en temps ton docteur va te le demander, mais tu ne devrais pas voir des différences énormes. Par contre, si tu es en train d'expérimenter avec \textit{l'estradiol undecylate}, tu devrais faire une prise de sang au moins tous les 3 mois quoi qu'il arrive. 

\subsection{Je ne peux pas avoir d'ordonnance pour une prise de sang, que faire ?}


Selon le pays / la région où tu es, essaie de voir si tu peux contacter un laboratoire pour faire un check-up hormonal. Le prix / la légalité sera complètement dépendant de la région dans laquelle tu te trouves. 

Note : en France, tu as le droit de demander une prise de sang avec un check-up hormonal au laboratoire sans ordonnance, qui ne sera par contre pas remboursée. Ca te coûtera une trentaine d'euros.
% TODO: localiser

\subsection{Je ne peux pas me permettre de payer une prise de sang / Je ne peux pas en faire, est-ce que je peux faire un THS quand même ?}

Même s'il est évident qu'il vaut mieux avoir l'information que de ne pas l'avoir, un THS est extrêmement sûr et, pour des doses habituelles, ne posera pas de soucis. Tu devras juste plus te baser sur comment tu te sens et ce que tu observes de ton corps.

\subsection{Qu'est-ce que je dois tester ?}

\textit{L'estradiol} (E2) et \textit{testosterone totale} (T) au minimum parce que c'est les choses principales qui nous intéressent. Les niveaux de \textit{globuline liant les hormones sexuelles} (SHBG), de \textit{dihydrotestosterone} (DHT), \textit{d'estrone} (E1), et de \textit{prolactine}(PRL) peuvent aussi être utiles si tu a des complications parce qu'elles peuvent être utiles pour établir un diagnostic. Les niveaux de \textit{d'hormone lutéinisante} (LH) et \textit{d'hormone folliculo-stimulante} (FSH) peuvent te dire si ton axe HPG est inactif, ce qui est la fondation de la monothérapie (voir la question Question \ref{2-3}). Mais je le répète : \textbf{\textit{L'estradiol} and \textit{Testosterone totale} sont les éléments principaux à regarder.} 

\subsection{A quel moment de mon cycle dois-je faire ma prise de sang ?}

A la fin de ton cycle (\textit{"au bout"}). Tu veux être au plus bas possible parce que c'est l'information la plus utile. On peut même dire que c'est la seule information utile, vu que le fait d'avoir des niveaux minimum suffisants est notre problème principal. Par exemple : si tu fais une injection chaque jeudi après-midi, fais ta prise de sang le jeudi dans la matinée ou en début d'après-midi juste avant ta prochaine injection.

\subsection{Mon docteur m'a dit de faire mes prises de sang au milieu de mon cycle / juste après mon injection, est-ce que je fais ça ?}

\textbf{Non.} La mesure du taux d'œstrogène en milieu de cycle ne donne aucune information utile et ne permet que de savoir quel ester tu utilises. En restant gentille, ce genre de demandes est dûe à l'incompétence née de standards de soins datés et conservateurs. En l'étant moins, c'est une volonté malveillante d'asssurer que tu aies des niveaux d'œstrogène trop bas, ce qui va te donner des mauvais résultats, ou te rendre malade. \textbf{Je recommande de mesurer au dernier jour du cycle quand même.} 

\subsection*{Interpréter les Résultats}
\addcontentsline{toc}{subsection}{\textemdash{} Interpréter les Résultats}

\subsection{Quel est le taux d'œstrogène que je dois viser ?}

C'est probablement la question la plus controversée pour ce qui concerne la transition. Pour faire court, tu veux avoir des niveaux suffisants pour te sentir bien, et suffisants pour supprimer la production de testosterone si c'est ce que tu cherches, mais des niveaux plus hauts que ça sont au mieux du gâchis d'horomones, et au pire contreproductifs. C'est cependant une fourchette très large, et avec autant de variables il y a toujours un élément personnel. En d'autre mots, tu veux avoir suffisamment d'œstrogènes pour que tu te sentes bien, c'est à peu près tout.

\subsection{Est-ce que des niveaux d'œstrogène plus hauts me permettent de mieux me féminiser / de le faire plus rapidement ?}

\textbf{Non.} Des niveaux d'œstrogènes plus hauts que nécessaires sont préférés par certains parce qu'iels se sentent mieux comme ça, mais ça n'a aucun intérêt au niveau de la féminisation. En fait, des niveaux trop hauts peuvent causer des troubles de l'humeur ou d'autres effets secondaires non désirés. \textbf{Minimiser les niveaux de testostérone (jusqu'à un certain point) est bien plus important que de maximiser les niveaux d'œstrogène.}

\subsection{D'accord, mais en pratique je veux voir quels chiffres sur mon résultat de prise de sang ?}

En rappelant que le nombre exact n'est pas si important que ça, et que le chiffre sera toujours un peu plus important que ce que tu as dans ton corps au dernier jour du cycle à cause de la latence, et que ce chiffre sera dans un nuage de possibilité basé sur un certain nombre de facteurs, \textbf{Je recommande 200pg/mL (730pmol/L) au minimum au dernier jour du cycle.} C'est une recommandation plutôt large, qui prévoit une bonne marge vu que la suppression de l'axe HPG arrive bien en-dessous de ce niveau. La plupart des gens préfèrent être autour de ce niveau, et certains préfèrent un peu plus ou moins. Je ne pense pas que ça soit un chiffre sur lequel trop se fixer, parce que c'est très dépendant de toi et, au final, le plus important c'est que tu te sentes bien. \textbf{Cependant, au-delà de 300pg/mL (1100pmol/L) au dernier jour du cycle, tu as un niveau certainement trop haut pour tes besoins.} Il y a des exceptions à cela, mais tu n'en fais probablement pas partie. Mais bon, fais ce qui te fait sentir le mieux. Aussi, va voir la Question \ref{11-1}.

\subsection{Quel est le taux de testostérone que je dois viser ?}

La suppression de la testostérone (T) est un prérequis pour une féminisation adéquate, donc descendre en-dessous de 50 ng/dL (1.7 nmol/L) est généralement suffisant. \textbf{A noter que des taux de testostérone approchant 0 ne sont pas désirés.} Voir la section \ref{T} "TESTOSTERONE" 

\subsection{J'ai naturellement des taux de T très hauts / très bas. Est-ce qu'il faut que je change quelque chose à mon dosage ?}

Probablement pas. les taux de testostérone qu'on voit avant le début d'un THS sont quasiment toujours plus hauts que ce qu'on cherche pour une féminisation et vont être réduits quoi qu'il arrive (voir la Question \ref{2-3}). L'exception est si tu as une condition intersexe de quelque variété que ce soit, ce qui va peut-être demander un ajustement plus fin que les préconisations listées dans ce guide et qui dépasse l'objet de celui-ci. Tu n'auras peut-être pas besoin de changer quoi ce soit, mais tu te sentiras peut-être mieux si tu le fais. Voir la Question \ref{9-2}

\subsection{J'ai eu une chirurgie génitale. Est-ce que mes niveaux d'œstrogènes doivent être différents ?}

Vu que la diminution des niveaux de testostérone n'est plus un problème pour toi, tu peux probablement t'en tirer avec des niveaux d'œstrogènes plus bas que pour les autres, mais \textbf{tu as toujours besoin d'œstrogènes.} Puisque tu ne produis plus d'hormones sexuelles, il est crucial que tu maintienne des niveaux d'hormones suffisants pour rester en bonne santé. N'avoir plus aucune hormone ou presque va induire des symptômes de ménopause, raison pour laquelles les femmes cis plus âgées prennent aussi un THS parfois après la ménopause. Ajuste tes niveaux comme tu le sens.

Pour plus de clarté : \textbf{maintenir un minimum autour de 100pg/mL (350 pmol/L) est essentiel pour éviter des problèmes de densité minérale osseuse.} Si la majorité de ta féminisation est déjà faite, alors dans beaucoup d'aspects ton profil hormonal est comparable à celui d'une femme cis ménopausée, donc on peut en apprendre à partir de leurs expériences (voir la Qestion \ref{11-29}). Dans certains cas de fatigue ou de manque d'énergie, se supplémenter en dosages faibles de testostérone peut aider (voir la Question \ref{9-2}). 

\subsection{Est-ce qu'il existe des raisons pour lesquelles une prise de sang pourrait mener à des résultats imprécis ?}

Selon comment le test est conduit, des suppléments de biotine peuvent donner l'impression que les niveaux d'\textit{estradiol} (E2) (parmi d'autres, mais c'est l'\textit{estradiol} qui nous intéresse) sont anormalement hauts. Il n'est pas toujours possible de savoir avec quel méthode le test est fait, donc il vaut mieux arrêter de prendre tes suppléments de biotine quelque jours avant ta prise de sang. Il est possible aussi qu'il y ait eu une erreur avec l'équipement ou l'échantillon, même si c'est beaucoup moins probable.

\subsection{Est-ce qu'on voit quel ester / voie d'administration j'ai choisi sur mes résultats de prise de sang ?}\label{4-16}

Non. Il n'y a pas de moyen de savoir quel type d'œstrogène tu prends juste à partir d'une prise de sang. Tous les ester se transforment en \textit{estradiol} dans ton sang, ce qui est le but recherché, et c'est la même chose pour les pilules, gels, patches, sprays ou quoi que ce soit d'autre que tu pourrais utiliser. Au final, tout n'est qu'œstrogène. 
 

\section{TECHINQUES ET MATERIEL} \label{ts}

\subsection*{Sites d'injections \& Sécurité}
\addcontentsline{toc}{subsection}{\textemdash{} Sites d'injections \& Sécurité}

\subsection{Comment est-ce que je fais une injection en toute sécurité}

\textit{Le FLIRT (@front_transfem sur instagram) et le STRIP ont créé ces super guides d'auto injection}, que tu peux lire ou imprimer, et que je recommande fortement

\href{https://tr.ee/6xXjjsMtPj}{Guide d'auto-injection en sous-cutané}

\href{https://tr.ee/PEeuyOWzvo}{Guide d'auto-injection en intra-musculaire}

Sinon, Je recommande ces deux vidéos (en anglais):

\begin{enumerate}
  \item \href{https://www.youtube.com/watch?v=cBabaGC2Dok}{\textit{"Comment s'auto injecter en intramusculaire (IM)”}}
  \item \href{https://www.youtube.com/watch?v=YfNlAZLxLyw}{\textit{“Une technique d'injection IM sans douleur (pour le moment pour moi)”}}
\end{enumerate}

Avec ces guides et ces deux vidéos, tu devrais avoir tout ce qu'il faut pour t'auto injecter sans trop avoir mal. Prends le temps d'étudier ça et reviens les voir quand tu en as besoin. \textbf{Un des trucs importants quand tu t'injectes est d'avoir le biseau vers le haut pour réduire la douleur.} Je m'explique : l'aiguille se termine en un point, et tu veux que ça soit ça qui touche ta peau en premier. Tu veux que le parcours de ton aiguille suive une ligne bien droite depuis le trou que tu vas créer. Tu peux imaginer le mouvement de ta main si ça t'aide, mais c'est essentiellement de la mémoire musculaire que tu vas de mieux en mieux maîtriser avec le temps.

\textbf{Rappelles-toi : s'auto-injecter s'apprend ! } Tu vas t'améliorer avec le temps, et ça ne prendra pas longtemps avant que tu saches faire. Ca va le faire.

\subsection{Est-ce que je dois m'injecter exactement comme ça ?}

Non, tu peux personnaliser comme tu veux. Au final, quand il s'agit de se faire des trous, il y a plein de manières d'y arriver. Trouve la manière qui marche le mieux pour toi. Faire un mouvement rapide de piqué marche habituellement le mieux mais, si tu préfère aller plus lentement, ça marche aussi. 

\subsection{Comment est-ce que je fais pour passer outre l'anxiété au moment de l'injection ?}

Je suggère de créer un rituel autour de l'injection. En formant une routine, le processus va devenir une seconde nature pour toi. Si ça marche pour toi de te distraire en mettant de la musique, en ayant une conversation, en regardant une série, ou en faisant autre chose qui marche pour toi et qui laisse la mémoire musculaire prendre le dessus, c'est super ! Sinon, tu peux te faire aider par un.e ami.e ou un.e proche pour faire tes premières injections, ça aide aussi. La première injection est celle qui fait le plus peur. Habituellement, les gens disent "Ah c'est tout ?", parce que c'est jamais aussi terrible que ce qu'on croit.

\subsection{Est-ce que l'endoit où je m'injecte a une importance ?}

Oui et non. Il faut rester dans des zones sûres, mais à part ça, ça dépend surtout de ta mobilité, du volumes que tu injectes, de ton combo seringue/aiguille et de ce que tu préfères. En tout cas, \textbf{assure-toi de changer de site d'injection a chaque fois.} Alterne le côté de ton corps que tu choisis, par exemple, si tu injectes sur ta jambe droite une semaine, utilise ta jambe gauche la suivante. C'est pour éviter d'avoir des risques à long terme.

\subsection{Quels sont les sites d'injection sûrs ?}

Les opinions varient en fonction des autorités médicales, et la composition de ton corps a un impact. Je recommande de t'injecter sur le côté de la jambe comme on le voit sur le guide et les vidéos, c'est le plus simple pour la plupart des gens et permet de faire tout le temps des injections sans douleur une fois que tu as pris l'habitude de la technique, mais certaines personnes préfèrent le faire sur la fesse ou le ventre. \href{https://vertisis.com/articles/how-to-self-administer-a-subcutaneous-injection}{Ce site/vidéo (en anglais)} montre d'autres sites d'injections qui peuvent être acceptables en fonction de l'équipement utilisé. Expérimente et trouve ce qui te plaît le plus.

\subsection{Qu'est-ce que "intramusculaire" (IM) et "sous-cutané" (SubQ/SC) veulent dire ?}

Tu vas souvent entendre ces termes dans le contexte des injections. \textit{intramusculaire} veut dire injecté dans le muscle et \textit{sous-cutané} veut dire injecté dans le tissu gras en-dessous de ta peau.

\subsection{Quelle est la différence entre une injection intra-musculaire (IM) et sous-cutanée (SubQ/SC) ?}

\textbf{En ce qui concerne le THS, il n'a aucune différence significative entre une injection sous-cutanée et intramusculaire.} Les injections sous-cutanées sont absorbées plus lentement que les injections intramusculaires, mais c'est généralement pas une différence suffisamment significative pour impacter ton dosage. A noter aussi que ton injection est rarement totalement déposée dans le muscle ou dans le tissu sous-cutané, ce qui brouille d'autant plus les différences possible entre les deux types d'injections.

En passant, les sources pharmaceutiques pour les fioles d'œstrogènes vont souvent dire que leur produit est déstiné à des injections intramusculaires seulement, parce qu'elles ont reçu uniquement une autorisation pour cet usage. Ca n'a cependant aucune importance, le contenu de la fiole est le même quoi qu'il arrive, et c'est au final tout ce qui compte. Continue à lire.

\subsection{Est-ce que je devrais plutôt faire une injection intramusculaire (IM) ou  sous-cutanée (SubQ/SC) ?}

\textbf{Tu te poses la mauvaise question.} \textbf{Une injection est une injection.} Les injections sous-cutanées sont souvent recommandées parce que les gens croient qu'elles permettent de faire des injections moins douloureuses par le simple fait d'être sous-cutanées, mais ça ne fait en fait aucune différence. \textbf{Les avantages dont les gens parlent à ce sujet ne sont pas inhérent au site de dépôt d'injection; ils sont dépendants d'autres facteurs qui eux, affectent les douleurs liés à l'injection.} La meilleure question serait "Comment puis-je minimiser la douleur liée à une injection ?", j'y viens, je réponds à deux autres questions d'abord. 

\subsection{Est-ce que l'angle d'injection / ma méthode préférée d'injection à une incidence ?}

Non. Je me répète, mais la partie la plus importante pour faire une injection est que tu perce ta peau avec une aiguille pour déposer un fluide dans ton corps. Si le fluide ne ressort pas (ou en tout cas pas trop) et ça ne fait pas mal (ou en tout cas pas trop), alors tu as fait un travail formidable. \textbf{Je ne peux pas le répéter assez, le choix de faire ton injection en intramusculaire / sous-cutané importe peu et n'impacte pas de façon significative l'efficacité de l'œstrogène injectable. } Le cas de l'\textit{undecylate d'estradiol} est le seul où le site de dépôt à l'air de vraiment affecter l'absorption mais, même là, on ne connait pas vraiment les détails. En bref : reste concentré sur les choses qui comptent, et pas celles qui ne changent rien.

\subsection{Est-ce que je dois créer une aspiration ?}

Non. "l'aspiration", c'est quand tu tire sur le piston de la seringue juste après avoir percé ta peau, avant d'injecter du fluide pour s'assurer que tu n'injectes rien dans un vaisseau sanguin. La nécessité d'une telle procédure est controversée, mais en ce qui concerne les procédures standard d'injection d'hormones, les avantages ne couvrent pas les inconvénients. Les sites d'injections standards ont peu de risques de toucher un vaisseau sanguin, risque encore amoindri par la petite taille des aiguilles, donc cette pratique n'est plus recommandée par les organisations médicales.

\subsection{Comment puis-je minimiser la douleur liée à une injection ?}

A part accumuler de l'expérience pour améliorer ta technique, le facteur principal d'inconfort pendant une injection est la combinaison aiguille / seringue que tu utilises. \textbf{Pour minimiser l'inconfort, prends le calibre d'aiguille le plus élevé que ton huile de solvant est capable de tolérer, avec une seringue et une aiguille de taille appropriée.} Tu dois te demander : "Quel calibre et quelle longeur d'aiguille choisir pour mon injection ?" Pour répondre à cela, parlons un peu du fonctionnement des aiguilles.

\subsection*{Connaitre tes aiguilles}
\addcontentsline{toc}{subsection}{\textemdash{} Connaitre tes aiguilles}

\subsection{Un calibre d'aiguille, c'est quoi ?}

\textit{Le calibre d'aiguille} est une mesure de la largeur de ton aiguille. Plus le nombre est haut, plus l'aiguille est fine. Une aiguille 25G est plus fine qu'une aiguille 20G, par exemple. Les aiguilles de calibre élevé sont généralement plus courtes car les aiguilles trop longues ont tendance à se plier, donc leur longeur à un maximum atteint plus rapidement. Sans surprise, les aiguilles les plus fines font le moins mal. A noter que le calibre de ton (ou tes) aiguille(s) n'affectent pas ton THS de quelque manière que ce soit; ça va seulement impacter le confort et la facilité d'injection.

\subsection{C'est quoi les aiguilles/seringues "Luer lock" / " à insuline ?"}\label{5-13}

\textit{Les seringues Luer lock} ont une aiguille qui se sépare de la seringue, avec une aiguille de prélèvement, et une autre pour injecter. Les \textit{seringues à insuline} ont une aiguille fixée, ce qui veut dire que tu vas utiliser la même seringue pour prélever et pour injecter. Note : tu ne veux pas utiliser de seringues "Luer slip" vu qu'elles mènent souvent à des erreurs. Quand c'est possible, on préfère utiliser des seringues à insuline pour minimiser l'espace mort (voir la Question \ref{5-26}).

\textbf{Recommandations de sécurité : remettre le bouchon d'une aiguille n'est généralement pas recommandé par risque de te piquer, mais si jamais tu dois le faire (par exemple quand tu change d'aiguille après avoir prélevé ton œstrogène), ne mets JAMAIS de force avec ta main sur le chemin de l'aiguille.}

Il est possible que le bouchon se casse et que tu te fasses mal si jamais tu place le bouchon incorrectement. Mets ton bouchon sur une suface horizontale, utilise l'aiguille pour le ramasser gentiment et appuie le bouchon sur un mur ou tire le sur les côtés pour le fermer. Il n'y a pas de risque d'infection mirobiologique en faisant une auto-injection, donc prends cette mise en garde comme tu veux, mais remettre un bouchon est une action très sérieuse si tu injectes quelqu'un d'autre. Pour la gestion des déchets, va voir la Question \ref{5-27}.

\subsection{Avec quel calibre dois-je prélever le fluide ?}

Si tu utilises des seringues Luer lock, il vaut mieux utiliser un calibre plus bas que celui que tu vas utiliser pour l'injection pour minimiser le temps de prélèvement du fluide de la fiole. Un calibre trop peu élevé peut mener à la fragmentation du sceau de la fiole (voir Question \ref{5-23}), donc je recommande de prendre au moins une aiguille 21-23G. Si tu es patient.e et as moins de volume à injecter, alors il vaut mieux utiliser un calibre plus élevé pour éviter au maximum une fragmentation du sceau de la fiole. A noter que l'aiguille ne devient pas vraiment moins perforante après avoir percé la fiole. La question ne se pose pas avec les seringues à insuline vu que tu ne peux pas changer l'aiguille.

\subsection{Quelle longueur d'aiguille dois-je utiliser pour le prélèvement ?}

Si tu utilises des seringues Luer lock, la longueur de l'aiguille de prélèvement n'importe pas sauf pour le fait qu'une aiguille trop longue peut être compliquée. Autrement dit, pas besoin de s'embêter avec ça. La question ne se pose pas avec les seringues à insuline vu que tu ne peux pas changer l'aiguille.

\subsection{Quel calibre dois-je utiliser pour l'injection ?}\label{5-16}

C'est une question difficile à répondre, avec une réponse très subjective, qui va dépendre de 4 facteurs principaux : 1) l'huile solvante que tu t'injectes; 2) si la fiole contient un cosolvant; 3) ta patience relative au fait d'avoir une aiguille dans ta jambe plus longtemps; 4) ta volonté/possibilité de pousser plus fort sur le piston de la seringue. C'ests une question de confort. Une huile plus épaisse demande plus de temps et d'efforts si tu utilise un calibre plus élevé, mais un calibre élevé fait significativement moins mal. \textbf{Une aiguille 25G devrait être le calibre minimum à utiliser pour minimiser l'inconfort lié à l'injection.} La plupart des huiles solvantes peuvent être utilisées confortablement avec des aiguille jusqu'à 27G, et la l'huile MCT en particulier peut totalement être utilisée avec des aiguilles de 30G (voir la Question \ref{6-16}).

\subsection{Quelle longueur d'aiguille dois-je utiliser pour l'injection ?}

\textbf{Je recommande des aiguilles de 12,5mm à 25mm (0,5” à 1”), en fonction du calibre que tu as choisi.} En dessous de 12,5mm (0,5"), tu augmentes les risques de fuites. Des aiguilles de 6,5mm (0,25") peuvent suffire si tu as la bonne technique et si le fluide à injecter le permet, mais une aiguillle de 12,5 mm (0,5") est un pari plus sûr. Une aiguille plus grande que 25mm (1") est plus douloueurse et intimidante sans contrepartie.

\subsection{Est-ce que la taille de la seringue compte ?}

\textbf{Oui, la taille compte.} Il y a deux raisons à cela. 1) Des seringues de gros volumes ont tendance à être moins précises, ce qui peut mener à un dosage incorrect, et 2) la physique fait que les seringues de gros volumes sont plus difficiles à utiliser pour l'injection. Pour garder une bonne précision sur ton dosage, il ne faut pas utiliser une seringue beaucoup plus large que le volume que tu injectes (i.e., pour des injections de moins de 0.1mL, prends des seringues de moins de 1mL). \textbf{Les seringues de 3mL sont à éviter si possible.} Evidemment si c'est tout ce que tu as utilise-les, mais elles ne sont pas vraiment faites pour ce genre de choses. Me demande pas pourquoi les pharmacien.nes ont l'air de ne donner que celles-ci. Une blague cruelle peut-être. 

\subsection{Où est-ce que je peux acheter des seringues et des aiguilles ?}

Ca dépend de ta juridiction locale, vu que certains endroits interdisent la vente d'aiguille et de seringues aux public pour criminaliser les gens qui prennent des drogues. Sinon, les fournisseurs vétérinaires, de pharmacie, ou des producteurs autorisés en vendent, et sont des endroits à regader. \textbf{Amazon n'est pas un vendeur recommandé}, parce que la qualité de leur seringue ne peut pas être garantie.

En France, les pharmacies et parapharmacies en vendent (et en donnent gratuitement dans le cadre de programmes de réduction des risques), et tu as le droit d'en acheter. Si tu préfères gérer ça sans aller physiquement quelque part, beaucoup de pharmacies en ligne en proposent pour un prix raisonnable.
%todo: localisation

\subsection{Est-ce que je peux réutiliser des aiguilles ou des seringues ?}

\textbf{Non. Ne réutilise jamais d'aiguille ni de seringues.} Ne les partage pas non plus. Pour plus de clarté : ça vaut autant pour les seringues, qu'elles aient des aiguilles amovibles ou non, que pour les aiguilles. Tu le sais déjà probablement, mais je le rappelle ici parce que c'est vraiment dangereux !

\subsection{Comment faire si je veux faire des injections mais que j'ai du mal à le faire moi-même ?}\label{5-21}

Tu aimeras peut-être les auto-injecteurs. Comme le nom l'indique, les auto-injecteurs font l'injection pour toi. Les auto-injecteurs comme le \href{https://unionmedico.com/90-super-grip/}{\textit{UnionMedico 45/90 Super Grip}} peuvent prendre des seringues de 1mL, ce qui peut rendre l'injection moins difficile (mais il faudra toujours appuyer sur le piston), alors que des auto-injecteurs comme le \href{https://www.youtube.com/watch?v=hPbhEpUN43Y}{\textit{Owen Mumford Autoject 2}} cachent complètement l'aiguillle de la seringue à insuline, et poussent automatiquement le piston. Il y a aussi des modèles 3D imprimables disponibles en ligne. Je n'ai utilisé aucun de ces produits, et je ne peux m'avancer sur leur qualités respectives.

\subsection*{Le B-A BA d'une fiole}
\addcontentsline{toc}{subsection}{\textemdash{} Le B-A BA d'une fiole}

\subsection{Qu'est-ce que je dois regarder quand j'inspecte une fiole ?}

A part pour des signes de fragmentation du sceau (voir plus bas), cherche des signes de décoloration, de séparation, de contamination, de cristallisation, de fissures dans le verres, de fibres ou de cheveux dans la fiole, etc. Une fiole bien faite ne doit pas être différente par rapport à d'habitude. \textbf{Inspecte toujours une fiole avant usage. N'utilise jamais une fiole qui n'a pas l'air OK.} 

\subsection{Qu'est-ce que le "fragmentation du sceau"}\label{5-23}

\textit{Le fragmentation du sceau} arrive quand un bout du sceau en caoutchouc se casse et tombe dans la fiole. Ca peut arriver à cause de l'usage d'un calibre trop peu élevé, des ponctions répétées exactement au même endroit, ou à cause de trop de ponctions (i.e. une injection de faible volume avec une fiole très volumnineuse). \textbf{Une fiole dont le sceau est fragmenté doit être jetée immédiatement.} \href{https://www.youtube.com/watch?v=w5F0SLoMjC8}{\textit{La technique du 45-90°}} peut aussi être utilisée pour éviter de fragmenter le sceau de ta fiole.

Le risque lié à cela est de t'injecter des bouts de caoutchouc. Si tu vois des gros morceaux, il y a probablement des plus petits que tu ne vois pas. Le but du sceau est de protéger le contenu de la fiole des éléments extérieurs, donc une fiole avec un trou est plus à risque d'oxydation et d'infection bactérienne. \textbf{Note: Fais attention à retirer la capuche en métal ou en plastique du dessus de la fiole.} Ca peut paraître évident, mais certains design de fioles peuvent porter à confusion.

\subsection{J'ai combien de temps avant que ma fiole arrive à expiration ?}

Une fiole scellée peut tenir des années sans soucis si elle est stockée à température constante loin de la lumière. Le plus gros soucis lié à l'âge de ta fiole est l'oxydation de l'huile solvante, en supposant que la fiole ait été stérilisée comme il faut. Une fiole ouverte qui a des conservateurs (voir Question \ref{6-17}) devrait tenir le temps que tu la finisse complètement. Le message "jeter après 28 jours" qui est marqué sur les fioles vendues en pharmacie répondent à un minimum garanti par les producteurs sur la stérilité du produit, pas sa durée de vie maximum.

\subsection{Comment est-ce que je stocke ma fiole ?}\label{5-25}

Dans un endroit loin d'une source de lumière et à température stable. Les chaleurs élevées et les UV peuvent dégrader l'huile solvante, et les températures bassent peuvent causer une cristallisation. Les cristaux peuvent être dissous et réincorporés, mais ça peut causer des irritations s'ils ne sont pas complètement dissous. Le conseil vaut pour les fioles ouvertes et les fioles scellées.

Additionnellement: Tu vas toujours vouloir remplacer le volume de fluide que tu vas prélever par un volume d'air équivalent pour éviter de créer un vide dans la fiole au cours du temps, ce qui rendrait le prélèvement plus difficile. Fais ça en remplissant ta seringue avec de l'air (va jusqu'au même marquage que ce que tu prévois de prélever), injecte cet air dans la fiole, et prélève le fluide comme d'habitude. Cette étape t'évitera de futurs maux de tête.

\subsection{C'est quoi "l'espace mort"}\label{5-26}

\textit{L'espace mort} est le petit volume de fluide qu'on perd quand on fait une injection. C'est du fluide qui est bloqué dans la seringue ou dans l'aiguille. Avec une seringue Luer lock standard, ça peut être jusqu'à 0.1mL, alors que pour une aiguille à insuline on est plutôt autour des 0.003mL. Il vaut mieux réduire l'espace mort au plus possible pour raisons économiques, mais aussi parce qu'au bout du compte, ça fait beaucoup d'œstrogène gaché. \href{https://hrtcafe.net/Calc/}{Ce calculateur} peut être utile pour estimer combien tu gaches d'œstrogène en fonction de l'équipement utilisé.

A noter que si tu changes d'aiguille entre le prélèvement et l'injection, il faut un peu tirer sur le piston avant d'enlever l'aiguille de prélèvement pour que le fluide bloqué dans l'aiguille ne soit pas gaché. C'est pas grand-chose, mais ça peut faire une différence. \textbf{A noter aussi que les marquages sur la seringue prennent déjà en compte l'espace mort, donc pas la peine de l'ajouter dans ton volume d'injection.} Va voir la Question \ref{7-7} pour une autre stratégie si l'espace mort t'inquiète.

\subsection{Qu'est-ce que je fais de mes seringues / aiguilles usagées ?}\label{5-27}

Place tout ton matériel d'injection dans un conteneur à seringues (soit un conteneur DASRI fait pour ça ou en réutilisant un pot en plastique dur comme un pot à protéine en poudre ou de détergent). Quand le pot est plein aux trois-quarts, scelle le de façon à ce qu'on ne puisse pas l'ouvrir accidentellement. Marque clairement "Seringues usagées" dessus et jette-le en accord avec la juridiction de l'endroit où tu vis. \textbf{A noter que les aiguilles / seringues ne doivent pas être jetées dans la poubelle ou la poubelle de recyclage.} Ta ville/région/pays a probablement un site web dédié qui explique comment se débarrasser des matières dangereuses.

Note : en France, tu peux demander à une pharmacie de te remettre gratuitement une boîte à aiguilles DASRI, que tu leur rendras une fois remplie et scellée pour être jetée ensuite. Il est d'ailleurs officiellement interdit d'utiliser autre chose qu'une boîte fournie par une pharmacie pour jeter des seringues usagées.
%todo: localisation

\section{OU TROUVER DES INJECTABLES}\label{sv}

\subsection{Où est-ce que je peux obtenir des fioles pour mes injections?}

En gros, tu as deux options : \textit{en pahramacie} ou \textit{via le DIY}. Pour en demander en pharmacie, il te faudra généralement une ordonnance, vu que dans la plupart des pays, on ne peut pas obtenir d'œstrogènes sans ordonnance et, quand c'est possible, ça ne concerne pas les injections. Le \textit{DIY}, c'est toutes les autres manières d'en obtenir.

Rappel pour la France : Pour nous, c'est le DIY ou rien. 
%todo: localisation

\subsection{Est-ce que je devrais plutôt aller voir en pharmaice ou directement récupérer mes fioles en DIY ?}

Le choix t'appartient, mais souvent tu n'as pas vraiment le choix. Il y a des avantages et des inconvénients dans les deux cas. Evidemment, rien ne t'empêche de récupérer des œstrogènes de plusieurs sources différentes pour cumuler les avantages. Dans la plupart des cas, c'est même recommandé.

\subsection*{En pharmacie}
\addcontentsline{toc}{subsection}{\textemdash{} En pharmacie}

\subsection{Quels sont les avantages d'aller en pharmacie ?}

\begin{itemize}
  \item Tu peux généralment faire confiance aux processus de contrôle de qualité et aux certifications;
  \item C'est couvert par l'assurance maladie;
  \item Ca peut être plus pratique, en fonction de ta chance avec les docteurs;
  \item Le produit a plus de chance de respecter ses spécifications;
  \item \textbf{Donner au moins l'illusion d'utiliser les routes prévues par les institutions, qui sont souvent demandées pour avoir des remboursements de tes chirurgies.}
\end{itemize}

\subsection{Quels sont les inconvénients d'aller à la pharmacie ?}

\begin{itemize}
  \item Peu (ou pas) de choix d'esters;
  \item Possiblement des temps d'attente très longs (des mois, voire des années);
  \item Il te faudra probablement une ordonnance (selon les pays);
  \item L'assurance maladie de ton pays ne couvre peut-être pas tous les coûts;
  \item Dans certains cas, tu ne peux pas t'en procurer dans ton pays;
  \item On peut refuser de te faire une odronnance de façon arbitraire;
  \item On peut refuser de te renouveller ton ordonnance de façon arbitraire;
  \item Il peut y avoir des ruptures de stocks;
  \item Il est possible que tu sois obligé.e de suivre les recommandations de la WPATH, voire pire;
  \item Plus difficile de faire des stocks;
  \item Ton accès est assujetti aux caprices de la situation politique de ton pays, ce qui veut dire que ton dossier médical contiendra probablement l'information que tu es trans
\end{itemize}

\subsection*{En DIY}
\addcontentsline{toc}{subsection}{\textemdash{} En DIY}

\subsection{C'est quoi les avantages de récupérer ses fioles en DIY}

\begin{itemize}
\item Généralement beaucoup moins cher dans la plupart des pays;
\item Disponible partout dans le monde;
\item En obtenir te fait attendre des mois voire des années de moins si tu es sur une liste d'attente (les seuls délais sont liés à la production et l'envoi);
\item Facile de faire un stock;
\item Large choix d'esters;
\item Pas besoin de s'embêter avec le système médical;
\item C'est probalement fait avec amour.
\end{itemize}

\subsection{Quels sont les inconvénients du DIY ?}

\begin{itemize}
  \item Ne sont certainement pas faits dans une pièce certifiée propre;
  \item La qualité dépend de la source;
  \item Des fois peu pratique, en fonction de la source;
  \item Nécessite de faire confiance à ta source;
  \item Nécessite de trouver une source;
  \item Les sources ont plus de chance de disparaître que ta pharmacie locale;
  \item Les temps de livraisons de produits peuvent varier;
  \item Très probable que tu aies besoin de passer par des cryptomonnaies, ce qui est ennuyant;
  \item Tu ne peux pas te faire rembourser pas l'assurance maladie.
\end{itemize}
De plus, comme dit plus haut, si tu as besoin de l'assurance pour te faire rembourser tes chirurgies, il te faudra probablement un temps minimum avec une prescription de THS. Ce qui peut (ou non) être un problème pour toi.

\subsection{Quels sont les types d'œstrogènes injectable qu'on peut uniquement obtenir en DIY ?}

Principalement \textit{l'enanthate d'estradiol} et \textit{l'undecylate d'estradiol}. Le secteur médical va quasiment toujours te prescrire du \textit{valerate d'estradiol}, mais pas toujours concentré à 40mg/mL. On prescrit des fois du \textit{cypionate d'estradiol}, mais rarement plus concentré que 5mg/mL ou 10mg/mL, ce qui est embêtant pour faire un dosage correct. Les avantages de l'utilisation de \textit{l'enanthate d'estradiol} sont à eux seuls des bonnes raisons de considérer le DIY, mais tu peux obtenir n'importe quel ester à 40mg/mL via le DIY. \textit{l'undecylate d'estradiol} est lui aussi seulement disponible en DIY pour les personnes qui veulent expérimenter, mais comme dit plus haut, je ne le recommande pas si tu ne sais pas ce que tu fais.

\subsection{C'est quoi exactement les sources en DIY ?}

Les sources de DIY sont généralement des fabricants professionnels, des groupes d'entraide, tes ami.es, ou toi-même si tu te sens l'esprit entrepreneurial !

\subsection{Comment je fais pour obtenir des fioles en DIY ?}

T'es un flic ? Je vais pas te dire ça. C'est pas le but de ce guide. Il y a d'autre ressources pour obtenir ce genre d'informations. Reste concentré.e.

\subsection{Comment ça se fait que le DIY puisse être moins cher que d'aller à la pharmacie ?}

Une fiole coûte en moyenne 10€ à produire, en comptant les amortissement et le salaire, en comptant large. La plupart des coûts des sources de DIY "commerciales" sont liés à tout ce qu'il faut pour envoyer les fioles en gardant son anonymité. Les sources non-commerciales n'ont sûrement pas de tels problèmes. Pour les laboratoires pharmaceutiques, iels n'ont juste pas de raison de vendre moins cher qu'iels le font aujourd'hui.  

\subsection{Est-ce que le DIY est légal ?}\label{6-11}

Dans beaucoup d'endroits, l'œstrogène n'est pas une substance interdite, et la testostérone peut ou non être criminalisée. Il est cependant très rare d'être inculpé.e dans des affaires criminelles autour de la possession d'hormones. \textbf{Je n'endosse aucune responsabilité à ce propos.} 

Note: en France, les deux substances sont considérées comme des médicaments, et sont donc interdites.
% TODO: localisation

\subsection{Est-ce que le DIY est dangereux ?}

Le "DIY" en tant que catégorie large de sources n'est ni sûr ni dangereux, mais toutes les sources ne se valent pas. Quand on parle de s'injecter des choses de manière sûre, la vraie question à poser est : est-ce que tu as confiance dans la personne qui a produit cette fiole pour avoir suivi les bonnes techniques d'aseptisation, de façon à ce que la fiole ne contienne que ce que tu veux et rien d'autre ? Quand tu achètes à un labo pharmaceutique, tu fais confiance aux réglementations et aux lois qui les surveillent. A défaut de ça, pour les sources en DIY, la confiance est obtenue en expliquant / montrant les processus, de si des tests ont été faits par des labos indépendants pour vérifier la pureté et la concentration, et via la réputation communautaire.  

\subsection{Qu'est-ce que je dois regarder pour savoir si une source de DIY est digne de confiance ?}

Fais confiance à ton cerveau et ton instinct.

\begin{itemize}
  \item Est-ce que la personne est prêt.e à parler de son processus / est-il listé quelque part ? (e.g. est-ce que la poussière est filtrée ? La réponse devrait être oui !!!)
  \item Est-ce qu'iel a l'air compétent.e ?
  \item Le produit a-t-il été testé ? 
  \item Est-ce que cette personne est réputé.e comme digne de confiance dans la communauté ?
  \item Est-ce que quelqu'un / une institution en qui tu as confiance peut attester de la qualité ? (i.e. inspections, témoignages, retours, etc.)
  \item Les erreurs arrivent, mais est-ce qu'iel en prend la responsabilité quand elles arrivent ou est-ce qu'iel essaie de taire les reproches ? 
  \item Pour les soures commerciales, comment se résolvent les soucis de commandes ?
  \item Pour les sources commerciales, est-ce que tu payes pour un produit qui n'a pas encore été fabriqué sans qu'il soit indiqué que c'est une précommande ? (Tu ne devrais jamais précommander !)
  \item Est-ce que les fioles ont des conservateurs ?
  \item Iel produit depuis combien de temps ? (Iel pourrait ne pas te le dire pour de bonnes raisons !)
  \item Combien de fioles sont produites ? (Iel pourrait ne pas te le dire pour de bonnes raisons !)
  \item Est-ce que tu ne le sens juste \textit{pas} ?
\end{itemize}

C'est juste une partie de la variété de questions que tu peux poser pour savoir si tu peux avoir confiance dans le fait que ta source s'intéresse autant que toi à la qualité du produit.

\subsection{Est-ce que je devrais avoir des standards différents selon mes sources de DIY ?}

Probablement, oui. Les fabricants commerciaux devraient être tenus à un standard plus haut, vu que tu leur donnes de l'argent en échange du produit, donc iels peuvent se permettre de bien le faire. Tu ne peux probablement pas te permettre d'être aussi sévère avec un projet d'entraide qui distribue des fioles gratuitement, ce qui ne veut pas dire que le produit sera meilleur ou moins bien. A toi de voir ! 

\subsection*{Anatomie d'une Fiole}
\addcontentsline{toc}{subsection}{\textemdash{} Anatomie d'une Fiole}

\subsection{Qu'est-ce que je devrais chercher dans une fiole ?}

Les ingrédients d'une fioles peuvent être classés dans deux catégories : les \textit{"principes actifs"} et les \textit{"excipients"}. Le \textit{principe actif} dans notre cas est l'ester d'œstrogène, et les \textit{excipients} tout le reste. Il y a généralement trois ou quatre ingrédients : 1) l'ester d'œstrogène; 2) l'huile solvante; 3) le conservateur; et optionnellement 4) un ou des cosolvant(s). On a déjà fait un paragraphe sur les esters dans la Section \ref{td} "TYPES ET DOSAGES". Les fioles de laboratoires pharmaceutiques ont presque toujours ces quatre ingrédients.

\subsection{Quelle huile solvante devrais-je avoir dans ma fiole ?}\label{6-16}

C'est une question de préférence, de tolérance personnelle, et possiblement d'allergies. \textbf{La variable principale qui t'intéresse est la viscosité qui affecte le confort d'injection et la praticité de celle-ci.} Comme discuté (dans la Question \ref{5-16}), les huiles moins épaisses peuvent te permettre d'utiliser des calibres d'aiguilles plus hauts sans plus de difficulté au prélèvement et à l'injection. \textbf{Les huiles solvantes les plus utilisées pour un THS sont l'huile MCT et l'huile de ricin.} L'huile de ricin est l'huile la plus épaisse couramment utilisée, mais elle tend à causer le moins d'irritation donc les labos pharmaceutiques utilisent en général celle-ci. L'huile MCT est moins épaisse, mais certaines personnes la trouvent plus irritante que les autres et elle n'est disponible qu'en DIY. Les huiles de graines de coton ou de pépins de raisins sont occasionnellement utilisées, mais généralement pas par des fabricants de THS. D'autres huiles comme l'huile de tournesol ou de sésame sont utilisées des fois, mais ne sont généralement pas recommandées. Selon tes circonstances, cette question peut ne pas en être vraiment une puisque tu n'as peut-être pas le choix.

\subsection{Quels conservateurs devrais-je avoir dans ma fiole ?}\label{6-17}

Le conservateur le plus utilisé dans les fioles d'injectables est \textit{l'alcool benzilique} (BA) en faible concentration. C'est nécessaire, et il n'y a pas de débat là-dessus. \textbf{Tu ne dois jamais utiliser une fiole qui n'a pas de conservateurs.} Pour les gens qui y sont allergiques, le \textit{cholobutanol} est un conservateur alternatif utilisé communément, mais presque jamais par les sources de DIY, ce qui veut dire que tu vas devoir aller chercher le truc vendu par un labo qui marche pour toi. 

\subsection{Quels cosolvents devrais-je avoir dans ma fiole ?}

Le cosolvant principalement utilisé est \textit{le benzoate de benzyle} (BB) qui réduit la viscosité de la solution. C'est techniquement optionnel, mais il est généralement recommandé pour avoir une variance moins élevée entre les lots, et des fois même nécessaire en fonction de l'huile solvante utilisée et de la concentration désirée. Certaines personnes trouvent ce composant irritant, mais c'est pas le cas pour tout le monde.
 

\section{GESTION DES PROBLEMES}

\subsection*{Incertitude sur les Dosages}
\addcontentsline{toc}{subsection}{\textemdash{} Incertitude sur les Dosages}

\subsection{Mes niveaux ne sont pas aussi hauts/bas que je pensais, pourquoi ?}

Il y a plusieurs possibilités. Rappelle-toi d'abord que les estimations sont des modèles qui ne prennent pas en compte une pléthore de facteurs qui peuvent causer des déviations. Rappelle-toi aussi qu'il faut plusieurs injections avant d'arriver à une stabilité, donc si tu viens tout juste de changer ton dosage, c'est peut-être pour ça. Vérifie plusieurs fois avec un.e ami.e que tu injectes autant que ce que tu penses. C'est plus commun qu'on ne pourrait le penser, mais pour les fioles obtenues en DIY il est possible que la concentration soit plus basse qu'annoncée à cause d'équipement imprécis ou dû au manque d'expérience du fabricant. Dans ce cas, injecte un petit peu plus si tu prends depuis cette fiole. \textbf{Mais rappelle-toi, le plus important c'est comment tu te sens, pas tes niveaux.} Note aussi que même les labos pharmaceutiques produisent des fois des fioles pourries qui ne sont pas repérées par les contrôle de qualité, même si c'est rare ! 

\subsection{Est-ce que je peux comparer mes niveaux à travers plusieurs tests si je ne les ai pas fait le dernier jour du cycle ?}

\textbf{Non.} Pas de façon précise en tout cas. C'est une des raisons pour lesquelles tu dois toujours faire tes tests au dernier jour du cycle, quelques heures avant ta prochaine injection; c'est ça que tu veux. Retirer le plus de variables possible est ce qui rend les données plus utiles pour toi. Si c'est la seule chose que tu gardes de ce guide, c'est ça : teste au dernier jour du cycle.

\subsection{Je me sens vraiment pas bien sur mon dernier jour de cycle, qu'est-ce que je dois faire ?}\label{7-3}

Dans la plupart des cas, ton dosage est trop bas ou ta fréquence d'injection trop basse. C'est particulièrement vrai dans le cas du \textit{valerate d'estradiol} et du \textit{cypionate d'estradiol}. Ajuste ton dosage dans la fourchette donnée plus haut et/ou ajuste ta fréquence d'injection. Trouve ce qui marche pour toi. C'est aussi possible avec \textit{le valerate d'estradiol} en particulier que ton dosage soit trop *haut* plutôt que trop bas et que les variations tout au long du cycle soient la raison de cette sensation de crash. En bref : prends de \textit{l'enanthate d'estradiol} si tu peux.

\subsection*{Difficultés au Moment de l'Injection}
\addcontentsline{toc}{subsection}{\textemdash{} Difficultés au Moment de l'Injection}

\subsection{Mon injection est plus dure à faire quand il fait froid, qu'est-ce qu'il faut que je fasse ?}

Réchauffe la fiole avant de prélever, et réchauffe la seringue avant d'injecter. Fais tourner la fiole dans tes mains pour réchauffer le fluide, et fais pareil avec le cylindre de la seringue. Prends cette habitude pour toutes tes injections pour garantir une meilleur régularité dans tes injections.  

\subsection{Mon injection fait plus mal quand il fait froid, qu'est-ce que je fais ?}

Réchauffe ta jambe avant d'injecter. Relaxe tes muscles avec un massage ou une douche bien chaude (spécifiquement : réchauffe ta jambe avec de l'eau chaude juste avant de sortir de la douche). Ca va aider.

\subsection{J'ai saigné après mon injection. Est-ce que je vais mourir ?}

Non. Ca veut juste dire que tu as touché un capillaire ou une veine, ça arrive de temps en temps. Tu vas probablement avoir un bleu ou des démangeaisons. Ca passera plus vite avec un pansement mignon.

\subsection{Il y avait un peu d'air dans ma seringue. Est-ce que je vais mourir ?}\label{7-7}

Non plus. Même si évidemment tu ne veux pas injecter que de l'air, et que ça peut affecter ton dosage s'il y'a trop d'air dans la seringue, un petit peu d'air (moins de 0.1mL) ne va pas causer de problèmes. Ca peut même être recommandé dans certains cas. Par exemple, la technique de l'\textit{air lock} (une technique standard pour injecter des fluides irritants ou qui tâchent, pas une connaissance super importante pour le THS) nécessite d'injecter entre 0.1mL et 0.3mL d'air, donc pas besoin de s'inquiéter. Tu n'es pas en train de faire une injection intraveineuse.

\subsection{Un peu de fluide est sorti du site d'injection. Est-ce que j'ai raté mon injection/Est-ce que je vais mourir ?}

Non. Une fuite peut arriver pour plein de raisons et c'est rarement un volume suffisant pour être significatif, donc pas besoin de faire une nouvelle injection. Pour la prochaine fois, pense à laisser l'aiguille 5 à 10 secondes avant de la retirer et applique une pression après avoir retiré l'aiguille. Si les fuites t'inquiètent, tu peux utiliser la technique de l'air lock mentionnée plus tôt, ou \href{https://www.nurse.com/nursing-resources/definitions/what-is-z-track-method/}{la méthode d'injection par voie Z (lien en anglais)} si les fuites t'inquiètent particulièrement.

\subsection{Des fois, j'ai mal après une injection. Est-ce que je vais mourir ?}

Non. En supposant que tu as suivi les recommandations de ce guide, des fois tu injectes juste dans un endroit inconfortable pour une raison ou une autre. Tu auras plus de chance la prochaine fois. \textbf{Assure-toi d'alterner les sites d'injection !} Tu ne veux pas que le tissu cicatriciel s'accumule sur le long terme, et si un endroit te fait déjà mal, tu ne veux pas empirer la chose.

\subsection{Le site de ma dernière injection est irrité / gratte. Est-ce que je vais mourir ?}

Probalement pas. Il y plein de raisons possible à ça. La plus inquiétante est l'infection, mais elle est peu probable. \textbf{Va voir un.e docteur.e immédiatement si tu as des symptômes de fièvre, de douleurs importantes, de douleurs musculaires, du pus, des marques rouges ou autres marques d'infection.} Dans la plupart des cas cependant, des rougeurs, un léger gonflement, etc. sont dûs au fait que l'huile et l'œstrogène se sont dissociés. Plus d'informations plus bas. C'est aussi possible que tu aies juste une réaction à l'huile solvante, mais si jamais tu as des soucis après plusieurs injections sans aucun problèmes, c'est probablement que le contenu de la fiole s'est dissocié.

\subsection{Il y a des cristaux dans ma fiole, est-ce que je peux toujours l'utiliser ?}

C'est probablement que ta fiole est trop froide. Réchauffe la un peu dans tes mains et secoue-la doucement pour récincorporer les cristaux. Si les cristaux ne disparaissent pas, alors il est possible que le contenu de la fiole se soit dissocié. Avec beaucoup de chaleur et de temps à mélanger, les cristaux peuvent se réincorporer, mais c'est plus simple et plus sûr de changer de fiole si tu peux.
 

\section{PROGESTERONE}

\subsection{Est-ce que je veux prendre de la progresterone ?}

\textbf{Probablement.} C'est une question à l'origine de pas mal de controverses pour une raison que j'ignore. Celle.ux qui sont contre (les docteur.es pour les nommer) te diront qu'il n'y a pas d'études pour montrer que cette hormone joue un rôle pour la féminisation, et qu'elle ne doit donc pas être prise. Sans compter le fait que tous les sujets touchant à la transféminité sont gravement sous-étudiés, heuristiquement parlant, la progestérone est une hormone féminine clé qui joue un rôle important dans le cerveau et dans beaucoup de fonctions de ton corps. Même sans s'intéresser à l'effet physiquement féminisant, c'est une hormone importante pour être en bonne santé et qui ne devrait pas passer à la trappe. 

\subsection{C'est quoi la différence entre “progestérone” et les "progestatifs” / "progestrogènes"”?}

La classe d'hormones, naturelles et synthétiques, qui activent le récepteur de progestérone sont les "proges\textbf{tatifs}". La version natuelle, bioidentique et le plus importants des progestatifs est la "proges\textbf{térone}". Les progestatifs synthétiques sont les "proges\textbf{tines}" (souvent appelés progestatifs de synthèse). Ces trois termes sont utilisés de manière interchangeable dans la littérature scientifique et dans les milieux cliniques, ce qui porte à confusion sur le rôle de la progestérone dans un THS, malgré le fait que ces hormones ne soient \textbf{pas} équivalentes.

\subsection{Est-ce que je veux de la progestérone ou une progestine ?}

Tu veux de la progestérone.

\subsection{C'est quoi le soucis avec les progestines ?}

Les progestines, plus particulièrement \textit{le médroxyprogestérone}, \textit{l'acétate de médroxyprogestérone} et \textit{le lévonorgestrel} sont généralement associé à des effets secondaires et des risques à long terme (cancer du sein, thromboses, dépression, etc.) qui sont faussement attribués à la progestérone. Ces hormones ne sont pas bioidentiques, ce qui veut dire qu'elles ne se comportent pas de la même manière que la progestérone et ne peuvent pas être directement comparées.

\subsection{Qu'est-ce que la progestérone fait pour la féminisation ?}

On pense que la progestérone joue un rôle dans le développement mammaire et de la libido en particulier, mais comme mentionné plus tôt c'est une hormone importante même en dehors de ses aspects plus visibles. Elle a aussi des effets antigonadotropiques (i.e. elle contribue à la suppression de la testostérone) qui peuvent être intéressants des fois. 

\subsection{Est-ce que je dois attendre avant de commencer la progestérone ?}

On ne sait pas trop. Certaines personnes pensent que commencer trop tôt peut heurter le développement mammaire à long terme, mais c'est une croyance purement théorique et anecdotalement, on observe l'inverse, ce qui rend difficile la réponse à cette question. L'estimation la moins risquée est d'attendre un an de THS (soit le stage de Tanner 3 ou 4) dans l'éventualité où la progestérone ait un impact. 

\subsection{Comment est-ce qu'on prend la progestérone ?}

A part pour les application topiques, la forme principale est une pilule. Elle est prescrite comme une pilule par voie orale, mais elle est la plus efficace lorsque prise comme un suppositoire. Les sprays et crèmes fonctionnent aussi très bien. 

\subsection{T'es sérieuse ? La progestérone doit être prise en suppositoire ?}

La progestérone se métabolise totalement différemment quand elle est prise oralement plutôt que rectalement parce qu'elle passe par le foie lorsque prise oralement. La progestérone orale se convertit principalement en \textit{alloprégnanolone} qui peut causer des fortes somnolences, alors que la progesterone prise rectalement reste de la progesterone dans ton corps, ce qui est ce qu'on cherche (même si une partie se convertit quand même). Certaines personnes prennent de la progesterone orale pour aider à dormir, mais note quand même que trop de \textit{alloprégnanolone} peut avoir des effets négatifs sur ta santé mentale.

\subsection{Comment est-ce que je prends de la progestérone en suppositoire ?}

Mets un peu d'eau sur la pilule, ça devrait suffire, puis sèche-toi et lave tes mains. Evidemment, ne va pas aux toilettes pendant une bonne heure, donc le mieux est de le faire juste avant d'aller dormir. Si la capsule ne se dissout pas bien, tu peux essayer de percer la capsule, mais tu ne devrais pas avoir de soucis. Fais attention au fait que les grands suppositoires fait maison utilisent de l'huile de coco, et que ce volume d'huile de coco ne va pas vouloir rester en toi.  

\subsection{Quelle quantité de progestérone devrais-je prendre ?}

Pour les pilules, la dose standard est de 100 à 200mg par jour, le soir. C'est un dosage assez arbitraire; le plus que les docteurs acceptent de prescrire est 200mg. Certaines personnes prennent plus de 200mg de temps en temps, mais prends note que faire monter tes niveaux comme ça peut mener à un crash déplaisant. (voir la question en-dessous).

Pour une application topique, personne ne peut vraiment savoir étant donné la haute variance de la méthode de délivrance, et il n'y a pas non plus de guide clair sur les niveaux qu'on cherche, ou même la fréquence (journalière sûrement), vu que la progesterone est totalement sous-étudiée. Je te conseillerais de titrer ton dosage pour que tu voies comment la progestérone t'affecte.

\subsection{Est-ce qu'il y a un intérêt à faire des "cycles" de progestérone ?}\label{8-11}

Non. Certaines personnes le font pour mimer un cycle menstruel de femme cis, mais il n'y a pas de raison de penser que ça puisse avoir des effets bénéfiques, et ça peut causer des symptômes de PMS. La seule exception est si tu as de bonnes raisons de suspecter que tu as une condition intersexe qui concerne un utérus. Je ne le conseille pas sinon. Va voir la question \ref{11-10} pour plus d'informations. 

\subsection{Je dois prendre de la progestérone pendant combien de temps ?}

Aussi longtemps que tu prends des œstrogènes, et aussi longtemps que tu veux, donc probablement toute ta vie.

Des fois les gens (ou les médecins) disent de prendre de la progestérone pendant X années de manière arbitraire. Il n'y a aucune raison théorique ou empirique pour suggérer que c'est un bon conseil. C'est aussi cohérent que si quelqu'un (ou un médecin) demandait pendant combien de temps une personne trans comptait prendre son THS\textemdash{}Ah non en fait j'ai rien dit iels demandent déjà ça. 

\subsection{Est-ce que la progestérone peut se convertir en \textit{dihydrotestostérone} (DHT) ?}

Non. En fait théoriquement si, mais en vrai non. C'est en grande partie un mythe, même si \href{https://whsah.co/posts/rethinking-progesterone-and-androgens/}{Comme mentionné en détail par alix dans cet article (en anglais)} dans le cas des personnes qui ont une \textit{hyperplasie congénitale des surrénales non-classique} (ncCAH) la progestérone peut avoir des effets secondaires négatifs qui augmentent l'activité androgénique. Dans ce cas-là il faut arrêter la prise de progestérone et chercher un diagnostic/traitement formel pour un potentiel trouble surrénal.

\subsection{Est-ce qu'il y a un intérêt à une application topique de progestérone en plus des pilules ?}

Possible. C'est une alternative aux pilules, notamment dans le cas où tu as une allergie aux arachides puisque le fabricant principal de pilules utilise de l'huile d'arachides, mais on ne sait pas clairement établir un dosage. Certaines personnes trouvent amusant de prendre plus de progestérone, même si ça fait rien d'autre. Fais ce qui t'amuse tant que ça n'impacte pas ta santé.

Pour plus de clarté : applique la crème à l'intérieur des cuisse (ou autre part si on t'en donne l'instruction), ou éventuellement sur la peau du scrotum (elle est fine et très vascularisée) pour les sprays. Et non, mettre de la progestérone directement sur ta poitrine ne va pas la faire grandir plus que si tu l'applique autre part.

\subsection{Est-ce que je peux sniffer la poudre de progestérone ?}

Je t'en prie, ne fais pas ça. Ca défonce tes sinus. C'est pas super compliqué de faire ton propre spray de progestérone et il y a des guides pour ça. C'est ce que je recommande de faire plutôt. C'est significativement plus efficace, plus régulier et plus sûr.

\subsection{Où est-ce que je peux me procurer de la progestérone ?}

La progestérone est généralement plus chère en DIY au vu du volume d'hormones nécessaire qui est plus grand, donc idéalement via l'assurance maladie. Il y a aussi un marché gris de pharmacies à l'étranger, mais c'est un peu le parcours d'obstacles pour commander. Les crèmes de progestérone sont disponibles sans ordonnance dans certains pays, même si c'est pas toujours le moyen le plus abordable selon la concentration.

\subsection{J'aimerais en savoir plus à propos de la progestérone dans le cadre d'un THS. Quelles ressources puis-je aller lire ?}\label{8-17}

Au départ, j'avais mis un lien vers un document ici, mais j'ai choisi de le retirer dû à des erreurs dedans qui peuvent être trompeuses. Le problème avec la progestérone est que littéralement personne n'est d'accord sur quoi que ce soit dessus. Je ne connais pas une seule source ou étude sur laquelle tout le monde est d'accord pour dire qu'elle est bien. La plupart des gens ne sont même pas d'accord sur le fait que le mot commence par la lettre "P". \textbf{La chose cruciale à savoir est que la progestérone n'est pas strictement nécessaire pour une féminisation complète ou un bon développement mammaire, mais si ce n'est pas contre-indiqué pour toi, alors il vaut probablement mieux en prendre.}

Il faut noter que pour la catégorie entière des progestatifs, il y une quantité innombrable de mythes et de mensonges inventés de toutes pièces par les défenseurs et par les opposants à leur utilisation, ce qui n'aide pas à discerner la vérité dans des études à la qualité discutable. Les revendications des propriétés magiques de la progestérone comme les paniques morales sur ses prétendus risques aident aussi peu l'un que l'autre, même si le deuxième est pire à mon avis quand ça vient d'une institution médicale, que ça soit par incompétence ou par malveillance.

\subsection{Est-ce que la progestérone interagit avec d'autres médicaments liés au THS ?}\label{8-18}

Si tu prends des inibitheurs de 5$\alpha$-Réductase comme la \textit{finasteride} et la \textit{dutastéride} (voir la section \ref{AA} "ANTIANDROGENES", ou continue à lire), ils peuvent influencer la façon dont la progestérone se transforme natuellement en \textit{alloprégnanolone}, ce qui peut avoir des effets néfastes sur l'humeur de certaines personnes, peu importe comment tu prends ta progestérone. Ce n'est pas vraiement clair comment la voie d'administration des inhibiteurs de 5$\alpha$-Réductase (i.e. topique vs orale) a un impact, mais le fait que la voie topique ait une absorption systémique plus basse peut aider à en atténuer les effets. Ca n'arrive pas tout le temps, mais si tu es affecté.e par ces interactions, il vaut mieux ne prendre aucun des deux. A noter que tu peux avoir des symptômes dépressifs jusqu'à un mois après l'arrêt de ces 2 médicaments.

\section{TESTOSTERONE}\label{T}

\subsection{Pourquoi est-ce qu'on ne veut pas avoir zero testostérone ?}

La testostérone est une hormone sexuelle essentielle qui joue un rôle important dans ta bonne santé physique et mentale. On veut la diminuer pour une féminisation, mais une testostérone proche de 0 (moins que 10ng/dL ou 0,35 nmol/L) peut causer des problèmes comme une libido affaiblie, un manque d'énergie, un affaiblissement musculaire (produit d'une fatigue anormale plutôt qu'à la perte musculaire normale dûe au THS), un manque de concentration, des troubles du sommeil, etc. Ce qui sont notablement des soucis assez similaires à ceux dû à un niveau trop bas d'œstrogènes. Les femmes cis ont aussi plus que zero testostérone, donc ça ne devrait pas t'effrayer. \textbf{Avoir des niveaux d'hormones adéquats est important !}

\subsection{Est-ce qu'il y a des cas où je devrais me supplémenter en testostérone ?}\label{9-2}

Oui, si tu expérimentes les symptômes cités ci-dessus et que tes niveaux d'œstrogènes sont dans la norme, il est possible que tu veuilles te supplémenter avec une microdose de testostérone. Si tu veux améliorer ta fonction érectile, éviter l'atrohpie avant une chirurgie génitale, ou si tu veux juste expérimenter avec les hormones pour voir ce qui te convient le mieux, alors tu as une bonne raison de vouloir explorer avec la testostérone dans un contexte différent, que j'espère que tu appécieras plus, par rapport à avant ton THS.

\subsection{Si je veux me supplémenter en testostérone, comment est-ce que je peux faire ?}

Il y a quelques possibilités. La testostérone existe en injections et en gels / crèmes topiques, de la même manière que l'œstrogène, comme dit plus tôt. Tu vas probablement te faire prescrire une forme topique. Les applications topiques ont les inconvénients qu'on a déjà mentionné pour l'œstrogène, mais c'est moins embêtant quand on ne cherche pas autant à avoir des niveaux précis comme c'est le cas ici.

\subsection{Quelles sont les formes topiques de testostérone ?}

Elle existe sous forme de gel et de crème. On va probablement te prescrire le gel, mais certaines pharmacies qui font des préparations sur-mesure peuvent faire des crèmes peu pénétrantes si tu veux une application topique sur les parties génitales. Cette dernière cependant est plus difficile à obtenir et plus chère.

\subsection{Est-ce que l'emplacement d'application a une importance ?}

Ca dépend de si tu utilise un gel ou une crème. Si tu utilise une crème locale comme mentionné au-dessus, tu peux l'appliquer directement comme mentionné. Sinon, tu peux l'appliquer sur le haut de tes bras ou sur tes épaules. Fais bien attention à ne rien toucher jusqu'à ce que le gel ait séché depuis longtemps !

\subsection{Quelle quantité et à quelle fréquence dois-je appliquer de la testostérone ?}

Fais à ta sauce. Ca dépend principalement de comment tu te sens. Si tu en mets trop, tu vas ressentir les effets secondaires de la testostérone (e.g. peau grasse et poils), mais tu es la seule personne qui peut dire ce qui est le mieux pour toi. Une injection hebdomadaire de 5 à 10mg de \textit{cypionate de testostérone} peut marcher pour toi, mais pour les gels topiques à 1\% qui sont souvent distribués en paquets de 25/50mg, il y a plus de variabilité. Même la moitié d'un paquet sera probablement trop, et sûrement pas tous les jours. Je recommande de commencer avec une dose beaucoup moins élevée que ce que tu penses, et regarde comment tu te sens.

\subsection{Comment est-ce que j'obtiens de la testostérone ?}

Il te faudra une ordonnance, ou aller demander aux mecs dans le BasicFit (ou autre salle de sport si c'est plus pratique) du coin. Pour raisons légales, c'est une blague. Voir la Question \ref{6-11}

\subsection{Est-ce que les autres stéroïdes sont équivalents à la testostérone dans le contexte du THS ?}

Les stéroïdes androgéniques anabolisants, i.e. les composés qui sont structurellement similaires à la testostérone, ne sont pas tous équivalents. Les stéroïdes facilement trouvables sur les marchés noirs comme \textit{l'acétate de trenbolone} ont toute une liste d'effets secondaires indésirables, mais d'autres stéroïdes comme le \textit{décanoate de nandrolone} sont des fois utilisés pour les femmes cis ménopausées parce que cette molécule a relativement peu de propriétés androgéniques, ce qui le rend très intéressant pour les personnes transféminines. Cependant, il est peu probable que tu te fasse prescrire autre chose que de la testostérone, dans le cas tu arrives à obtenir une ordonnance.

\subsection{C'est quoi la relation entre la testostérone et la \textit{dihydrotestostérone} (DHT) ?}

La \textit{dihydrotestostérone} est principalement synthétisée à partir de testostérone via l'enzyme 5$\alpha$-Réductase, où environ 5\% de la testostérone de ton corps est convertie. En général, si tes niveaux de testostérone sont diminués (ou si tu as eu une chirurgie génitale), alors il ne devrait plus avoir grand-chose à convertir, mais les niveaux ne vont pas être à 0 parce que toujours produits localement. En fonction de ton corps, ça pourrait être la raison principale de vouloir se supplémenter en antiandrogènes inhibiteurs de 5$\alpha$-Réductase, comme on le détaille dans la section suivante. Je rappelle que la \textit{dihydrotestostérone} est l'hormone responsable de la poussée des poils et de la chute de cheveux.

\textbf{Pour toutes les personnes transmasculines qui lisent ceci,} je vais faire un bref détour pour noter qu'au moment d'écrire ce guide, il n'est pas clair quel rôle joue cette hormone pour la croissance du dicklit en ce qui concerne la taille ou la vitesse de croissance, et son lien avec l'inhibition de 5$\alpha$-Reductase. Tout ça pour dire : on sait que la \textit{dihydrotestostérone }joue un rôle primaire dans le dévelopement pénien, mais ce n'est pas clair à quel point son manque affecte directement une personne transmasculine. Si on applique les connaissances de traitements du micropénis, on sait qu'une crème topique est plus efficace que des injections exogènes, particulièrement en sachant à quel point une crème de \textit{dihydrotestostérone} est utile quand un.e patient.e ne répond pas bien à la testostérone (en particulier dans le cas de déficience de 5$\alpha$-Reductase). Ca donne matière à penser. Que quelqu'un appelle Oliver Longdick pour gérer le reste du problème. 

\section{ANTIANDROGENES}\label{AA}

\subsection{C'est quoi les "antiandrogènes" ?}

Les \textit{antiandrogènes}, aussi communément référés comme "bloqueurs de T" ou "bloqueurs", comme leur nom l'indique, empêchent les androgènes (c'est ce qu'est la testostérone) d'agir dans ton corps. Il y a plein de types d'antiandrogènes et ils sont prescrits communément dans le cadre d'un THS. Ils sont nécessaires pour quelqu'un qui produit toujours de la testostérone et qui ne suit pas une forme de THS qui permet la monothérapie comme les injections, mais ils ne sont généralement pas désirables. A noter que les (ou la plupart des) antiandrogènes ne réduisent pas les niveaux de testostérone d'aucune manière qui compte, mais juste réduisent / annulent ses effets sur le corps. C'est important quand on regarde les résultats de prise de sang par exemple.

\subsection{Pouquoi est-ce que je ne voudrais pas d'antiandrogènes ?}

Le souci principal avec les antiandrogènes est qu'ils ont généralement des effets secondaires très indésirables qui sont superflus si la testostérone est déjà supprimée par le fait d'avoir suffisamment d'œstrogènes, donc ces effets sont subis alors que\textemdash{}dans la plupart des cas au moins\textemdash{}rendus non nécessaires par une dose raisonnable d'un traitement en monothérapie. Aussi, n'importe quelle type de chirurgie génitale rend les antiandrogènes inutiles dans la plupart des cas. 

\subsection{Dans quelle situation voudrais-je des antiandrogènes ?}

Si tu n'es pas dans la plupart des cas, si tu veux t'assurer une tranquilité d'esprit, ou si tu as besoin de prouver que tu utilises ton ordonnance pour te faire rembourser une procédure, alors tu peux vouloir des antiandrogènes. Les médicaments utilisés comme antiandrogènes peuvent avoir d'autres effets qui sont désirables même en dehors de leurs propriétés antiandrogéniques en fonction de ton état de santé. Aussi, si tu te supplémentes en androgènes, tu peux vouloir prendre un bloqueur de \textit{dihydrotestostérone} (DHT) pour minimiser les effets secondaires liés aux poils ou à la chute de cheveux, mais rappelle-toi que ça ne te concerne peut-être pas si tu n'utilises pas de la testostérone bioidentique (e.g. \textit{décanoate de nandrolone}) parce que tous les androgènes ne se comportent pas de la même façon. 

\textbf{A noter qu'utiliser des antiandrogènes temporairement au début d'un THS si tu prévois de faire une monothérapie n'est ni nécessaire ni recommandé.} Il y a une période d'ajustement que tu vas subir quoi qu'il arrive pendant que ton corps s'adapte au changement d'hormone primaire, donc pas besoin de rendre plus compliqué ce que tu fais déjà. Pas besoin de s'inquiéter.

\subsection{Quels sont les différents types d'androgènes ?}

La plupart des médicaments pris en tant que bloqueurs de testostérone dans le contexte d'un THS sont la \textit{spironolactone}, le \textit{bicalutamide} et l'\textit{acétate de cyprotérone}. La plupart des médicaments pris pour bloquer la conversion de testostérone en \textit{dihydrotestostérone} (DHT), appelés les “inhibiteurs de 5$\alpha$-Réductase” (5-ARI) sont la \textit{finastéride} et la \textit{duastéride}. Il y a aussi les antagonistes de GnRH comme le \textit{leuprolide} et la \textit{triptoreline}, qui sont aussi utilisés en tant que bloqueurs de pubertés pour les mineurs. 

\subsection{Dans quel cas est-ce que je voudrais prendre de la \textit{spironolactone} ?}

Dû aux dosages héroïques et à ses effets secondaires néfastes significatifs nécessaires pour que ça fonctionne comme un antiandrogène dans la plupart des cas, la seule raison pour laquelle je recommanderais de prendre de la \textit{spironolactone} serait de si tu bénéficies de ses autres effets, par exemple le fait que ça soit un antiminéralocorticoide (i.e., qui bloque la \textit{aldostérone}), une propriété qui la lie à la gestion de la pression sanguine ou des œdèmes. \textbf{Si tu insistes pour prendre de la \textit{spironolactone}, s'il te plaît n'en prends pas plus que 100mg par jour.} Ca a mauvaise réputation pour une raison. "Le Diable", comme on dit. 

Si jamais les effets secondaires néfastes ne te sont pas familiers, en voici une liste non exhaustive : brouillard cérébral, léthargie, problèmes de mémoire, augmentation de la fréquence urinaire, pression sanguine basse, carence de sodium / d'électrolytes, etc. En d'autres termes, la \textit{spironolactone} est un diurétique qui réduit la pression sanguine et un antiandrogène médiocre, qui est typiquement prescrit à très hautes doses pour une population en bonne santé pour une utilisation non prévue avec une efficacité douteuse. Dans n'importe quel autre contexte de santé, son utilisation serait (ou DEVRAIT ETRE !) déconseillée au vu des effets secondaires indésirables et de la quantité d'alternatives disponibles qui existent, mais bon j'imagine que c'est représentatif de l'état actuel de la santé pour les personnes trans.  

\subsection{Dans quel cas est-ce que je voudrais prendre de la \textit{bicalutamide} ?}

Si tu dois prendre un antiandrogène, il faut probablement choisir la \textit{bicalutamide}. Elle est plutôt bien tolérée, à part pour les 1\% des cas de dysfonctionnements du foie ou de faux positifs dans les résultats de prise de sang, mais à part ça elle fait son travail avec relativement peu d'effets secondaires. \textbf{Si tu prends de la \textit{bicalutamide}, fais des tests de fonctionnement du foie régulièmement pour t'assurer que tout va bien.} Les risques liés au foie sont dépendants de ton corps plutôt que de facteurs extérieurs, donc tout problème devrait se voir dans la première année du traitement. Mais il ne devrait pas y avoir de problème.

Note: en France je pense qu'il vaut mieux prendre de la triptoréline que de la bicalutamide si tu en a l'occasion, au moins pour la facilité à lire les résultats de prise de sang, vu que la production de testostérone est arrêtée. Le fait que c'est une seule injection tous les 3 mois plutôt qu'une pilule quotidienne est aussi un gros argument en faveur de la triptoréline.
% TODO: localisation

\subsection{Dans quel cas est-ce que je voudrais prendre de l'\textit{acétate de cyprotérone} ?}

Probablement jamais. Prends de la \textit{bicalutamide} plutôt.

Le profil des risques à long terme n'est pas terrible, et il n'y a pas vraiment de cas de figure qui me vient à l'esprit dans lequel je pourrais le recommander comme solution alternative. Tu peux faire tout ce que fait l'\textit{acétate de cyprotérone} en prenant plus d'œstrogènes et en ajoutant de la progestérone à ton THS.

\subsection{Dans quel cas est-ce que je voudrais prendre de la \textit{dutastéride}?}

Si la possibilité de la chute de cheveux te préoccupe et/ou si tu veux maximiser ta probabilité de repousse de cheveux, tu peux vouloir prendre de la \textit{dutastéride}. Si ta testostérone est supprimée, alors théoriquement il ne devrait pas avoir d'avantages à en prendre, vu que ta \textit{dihydrotestostérone} est probablement à des niveaux bas, mais les corps peuvent se comporter de manière étrange, donc ça peut avoir un intérêt pour toi. Va voir la Question \ref{11-14}.

A noter que la \textit{dutasteride} peut causer des sautes d'humeurs chez certaines personnes, et dans ce cas il est fortement recommandé d'arrêter le traitement. A noter que les effets dépressifs peuvent être ressentis jusqu'à un mois après l'arrêt de la prise du médicament.

\subsection{Dans quel cas est-ce que je voudrais prendre de la \textit{finastéride}?}

Si la \textit{duastéride} ne t'est pas prescrite ou si tu as une prescription spécifiquement pour de la \textit{finastéride}. Sinon, il vaut mieux prendre de la \textit{dutastéride} vu qu'elle est plus efficace et mieux tolérée.

A noter que la \textit{finastéride} peut causer des sautes d'humeurs chez certaines personnes, et dans ce cas il est fortement recommandé d'arrêter le traitement. A noter que les effets dépressifs peuvent être ressentis jusqu'à un mois après l'arrêt de la prise du médicament.

\subsection{Où est-ce que je peux obtenir des antiandrogènes ?}

A part en te faisant prescire par ton docteur ou si c'est possible directement en pharmacie, il y a toujours l'option des marchés gris que sont les pharmacies étrangères. C'est simplement des pharmacies d'autres pays, même si souvent ça demande quelques étapes pour acheter là bas. La \textit{Dutastéride} et la \textit{finastéride} sont généralement plus faciles à obtenir sans ordonnance au vu de leur fréquence d'utilisation comme médicament contre la chute de cheveux.
 

\section{MYTHES ET DIVERS}\label{MM}


\subsection*{Questions Fréquentes}
\addcontentsline{toc}{subsection}{\textemdash{} Questions Fréquentes}

\subsection{Est-ce-que je dois m'inquiéter des caillots sanguins ?}\label{11-1}

Oui et non. C'est vrai qu'il y a une corrélation entre les taux d'œstrogènes et le risque de caillot sanguin, cependant le lien est principalement lié à la méthode d'administration et au type d'œstrogènes. Les œstrogènes synthétiques sont la raison très justifiée de ce mépris et du risque de caillots sanguin mais les œstrogènes bioidentiques ne pas sont aussi inquiétants. La méthode d'administration fait un grande partie de la différence. Les œstrogènes biodentiques par voie orale font un premier passage par le foie, ce qui est la cause de l'augmentation du risque de caillots sanguin. Les injections évitent ce passage et il n'y a aucune preuve ou raison de penser que les injections d'estrogènes bioidentiques créent une quelconque augmentation significative du risque au-delà de la différence innée entre la testostérone et l'œstrogène. La perpétuation de cette panique contre toutes les formes d'œstrogène continue depuis des dizaines d'années malgré ces différences importantes.

\textbf{Si tu vas faire une opération chirurgicale, sache que mettre en pause ton THS par peur de causer des caillots sanguins n'est plus recommandé par la WPATH.} Beaucoup de chirurgiens le demandent dans leurs instructions de pré-chirurgie par peur de caillots, mais c'est une torture qui a été prouvée inutile et même la WPATH ne la recommande plus. Incroyable, je sais. D'après \href{https://www.tandfonline.com/doi/pdf/10.1080/26895269.2022.2100644}{l'élément 12.19 du SOC 8 de la WPATH (en anglais)}, \blockquote{After careful examination, investigators have found no perioperative increase in the rate of VTE [KT: \textit{venous thrombœmbolism}, i.e. a blood clot] among transgender individuals undergoing surgery, while being maintained on sex steroid treatment throughout when compared with that among patients whose sex steroid treatment was discontinued preoperatively (Gaither et al., 2018; Hembree et al., 2009; Kozato et al., 2021; Prince \& Safer, 2020).}. Je devrais mettre ça dans une autre question, mais pour ne pas casser les liens, il faudrait que je le mette à la fin d'une section et je pense que c'est trop important pour le mettre là bas, donc je laisse la note ici. Une clarification importante que j'aurais du faire plus tôt.

\subsection{Est-ce que c'est ok de prendre de la nicotine sous THS ?}\label{11-2}

C'est lié à la question au-dessus. \textbf{L'utilisation de nicotine sous THS, surtout si tu es sous pilules, multiplie les risques de caillots sanguins en plus des autres raisons qui font que la nicotine n'est pas bonne pour ton corps.} Ca vaut pour toutes les manières de consommer de la nicotine, mais évidemment c'est pire si tu la fumes. Tu ne veux vraiment pas avoir de caillots sanguins. Même si tu n'es pas sous pilules, la nicotine empêche les œstrogènes d'être métabolisées correctement, ce qui peut mener à des effets féminisants réduits. Cet aspect est très peu étudié mais il y a beaucoup d'informations anecdotales à ce sujet. C'est pas facile d'arrêter, mais je crois en toi. Il y a de bonnes ressources et de bonnes stratégies qui marchent vraiment, comme en prendre moins en compensant avec des patches. Tu peux le faire.

Cependant, pour être parfaitement claire, \textbf{ça ne veut pas dire que tu ne dois pas prendre d'œstrogènes. Les inconvénients de ne pas prendre d'œstrogènes sont bien pires que ceux de consommer de la nicotine sous THS.} Cette section cherche juste à mettre en lumière les risques et le ralentissement potentiel de féminisation, pour te donner une bonne raison d'arrêter. Une étape à la fois.

\subsection{Est-ce qu'il y a des avantages à commencer à un dosage faible par rapport à un dosage fort ?}

Avec toutes les informations à ma disposition, non. Les hormones sexuelles ne sont pas comme d'autres médicaments qui doivent être titrés pour éviter des effets secondaires et on sait quels sont les dosages qui fonctionnent pour la majorité des gens, donc je vois les "doses de départ" et les régimes "antiandrogènes d'abord" comme une forme de maltraitance médicale. Certaines personnes croient qu'imiter la puberté est le mieux (malgré le fait qu'il ya bien plus de choses qui arrivent que juste des changements de niveaux d'œstrogènes), mais il n'y a aucune preuve de ça. Si ça se trouve, le mieux serait de faire une orchiectomie au premier jour, mais qui est-ce qui va faire ça au moment de commencer son THS ? 

En le disant d'une manière différente : \textbf{il n'y a pas de raisons de croire que "commencer doucement" avec un dosage en dessous des niveaux typiques est avantageux ou préférable en ce qui concerne la féminisation.} On ne peut pas vraiment aller "trop vite" avec ça. Les médecins comme les femmes trans ont l'air d'inventer des nouveaux mythes en permanence. 

\subsection{Est-ce que le poids de corps affecte le dosage ?}

Non. Comme il n'y a pas de niveaux "optimaux" d'œstrogènes dans le sang, et que la fourchette de niveaux acceptable est grande, la masse corporelle n'affecte pas de manière significative le dosage pour un THS. C'est pour la même raison que des petites variations dans le dosage ne vont probablement pas changer comment tu te sens. C'est pas possible d'être "trop léger.e" ou "trop lourd.e" pour un THS.

\textbf{Pour la même raison, un ajutsement de dosage de 0,1mg (note que j'ai marqué milligrammes, et pas millilitres) est une différence qui ne devrait pas être perceptible pour la simple et bonne raison que nos corps ne sont pas suffisamment sensibles pour réagir à une si faible différence} , même sans compter la probabilité élevée d'imprécision en faisant une injection, ce qui amoindrit encore la certitude de ta mesure. En bref, il vaut mieux privilégier la régularité dans ton calendrier d'injections à la précision d'une dose en particulier.

\subsection{Est-ce que c'est possible de commencer son THS trop tard ?}

\textbf{Non.} Peu importe quand tu commences, les œstrogènes sont capables de beaucoup et un bon régime permettra d'obtenir des résultats impressionants. Les hormones sexuelles font partie des hormones qui affectent le plus l'apparence de nos corps. Tout le monde aimerait avoir commencé plus tôt, mais ce n'est pas une raison de ne pas commencer maintenant. Même si tu as été sous œstrogènes pendant des années, il y a des avantages à améliorer la qualité de ton régime.

\subsection{Est-ce que la féminisation / le développement mammaire s'arrête après X années ?}\label{11-6}

\textbf{Non.} Il n'y a pas de date arbitraire à partir de laquelle les œstrogènes arrêtent de fonctionner. On entend parler de pas mal de dates ça et là, mais elles sont souvent soit 1) inventées de toutes pièces, soit 2) pointent vers une étdue qui a été menée pendant X années. En particulier, les docteurs aiment dire aux femmes trans de ne pas s'attendre à avoir mieux que des bonnet B (la mesure de seins ne fonctionne même pas comme ça, mais je m'écarte du sujet), ou de ne pas espérer de croissance mammaire après 2 ans, mais c'est juste totalement faux. On voit beaucoup de gens qui recommencent les œstrogènes après avoir stoppé quelques années qui voient une nouvelle poussée de croissance. 

\subsection{Je n'ai vu aucun changement malgré des années sous injections. Est-ce que passer aux pilules ferait une différence ?}

Peut-être, ou peut-être pas. Il y a quelques anecdotes de gens qui passent des injections aux pilules (ou qui prennent des pilules en plus des injections) qui expérimentent une nouvelle poussée de croissance mammaire après que ça se soit arrêté, mais le mécanisme n'est pas clair. On spécule que le ratio entre E1 et E2 (\textit{estrone} et \textit{estradiol}) penche plus du côté de l'E1 pour les pilules que dans le cas des injections, et que ça cause une différence pour certaines personnes, même si l'\textit{estrone} n'est pas typiquement associée à la féminisation. Il y a probablement d'autres facteurs, et tu peux expérimenter autant que tu veux. Les données sont limitées. 

\subsection{Est-ce que le fait d'avoir une libido faible ou une énergie faible est normal pendant un THS ?}

Généralement, non. Comment ta libido s'exprime change au début mais, dans la majorité des cas, si tu as une libido anormalement basse, c'est parce que tu as un souci avec tes hormones. Pareil pour le manque d'énergie. Règle tous les soucis possibles avec tes hormones, et si c'est pas ça, vérifie ta diète / tes vitamines. Vérifie que tu n'as pas des niveaux de vitamine D critiquement bas ou quelque chose du genre. Ca arrive plus souvent qu'on ne le pense.

\subsection{J'ai entendu parler de [stratégie / médicament nébuleux] et j'ai des potes qui me disent que ça aide à la féminisation. Est-ce que ça aide vraiment ?}

Peut-être, mais probablement pas. Il y a beaucoup de spéculation à propos des moyens d'atteindre une meilleure féminisation, mais beaucoup sont des faux remèdes qui peuvent avoir des effets délétères sur ta santé. Tu as le droit à une autonomie complète sur ton corps et je ne peux pas t'arrêter, mais je t'encourage à bien réfléchir à ce que tu fais. Au plus tu vas loin dans les fanges de la biologie autour de la transition, au plus les eaux deviennent troubles et la qualité des données s'amoindrit. Le désespoir peut t'amener à des décisions regrettables ou dangereuses. Donc fais attention.

\subsection{Est-ce qu'on veut imiter le cyclé œstrogénique des femmes cis ?}\label{11-10}

Pas vraiment, non. C'est une opinion controversée, mais je pense que nous (enfin, la plupart d'entre nous) n'avons pas d'utérus et de cycle menstruel correspondant synchronisé à nos niveaux hormonaux, donc il n'y a pas de raison à copier ce comportement. C'est un exemple de la loi de Hume, pour moi. Le souci principal en ce qui concerne les hormones pour la plupart des femmes trans est la suppression de testostérone, qui nécessite un niveau minimum d'œstrogènes en permanence (à part si tu as déjà fait une chirurgie génitale, dans ce cas il n'y a plus de testostérone à supprimer), donc des fluctuations importantes, ou des niveaux relativement bas vont probablement causer une détresse non nécessaire. Tu peux expérimenter sur toi-même, évidemment. Surtout si la testostérone n'est plus un soucis pour toi. Voir la question \ref{8-11}, et voir plus bas.

\subsection{Est-ce que les femmes trans peuvent avoir des règles ?}

Similairement à la dernière question, il est important de comprendre ce qu'il se passe. La courbe produite par le croisement de ton ester, dosage et fréquence peut causer des changements dans ton humeur vu que tes niveaux d'œstrogènes oscillent entre les injections. Certaines femmes trans vont rapprocher ce phénomène à celui des règles, mais la cause sous-jacente pour ces changements physiologiques est différente et c'est souvent un signe que ton régime à besoin d'un petit changement pour que tu te sentes au mieux, vu que la souffrance n'est pas une vertu féminine. Tu n'as pas besoin de souffrir ou d'être inconfortable pour être une femme, et on ne devrait pas s'affirmer sur la base d'arguments bioessentialistes. L'exception étant pour les femmes trans intersexes qui ont un utérus et qui ont littéralement des règles et dans ce cas : oui évidemment. Voir la question \ref{11-35}.

\subsection{Est-ce qu'un excès d'œstrogènes peut se convertir en testostérone ?}

\textbf{Non.} L'aromatase est l'enzyme responsable de la conversion de testostérone en œstrogène, mais il n'y pas de mécanisme pour convertir l'œstrogène en testostérone. Ca ne peut pas arriver. C'est une affabulation et tu devrais te méfier immédiatement du niveau de connaissances de qui que ce soit qui te sort ça. Malheureusement, c'est souvent les docteurs qui disent ça.

\subsection{Est-ce qu'une chirurgie génitale peut causer une augmentation de testostérone ?}

Non, ce n'est pas possible. Il n'y a pas de mécanisme magique qui soudainement cause une augmentation de la testostérone au moment où on retire les testicules. Même si la magie était stockée dans les boules, c'est tout simplement pas comme ça que la production d'hormones fonctionne. "Enfin tes glandes surrénales.." elles ne fonctionnent pas comme ça non plus. La seule exception, rare, serait une hyperandrogénie surrénalienne non diagnostiquée, qui était supprimée par un antiandrogène comme la \textit{spironolactone} avant la chirurgie et qui peut se dévoiler après l'arrêt d'antiandrogènes. Arrêtez de propager ce mythe s'il vous plaît. 

\subsection{Comment est-ce que j'évite/inverse la chute de cheveux ?}\label{11-14}

Mécaniquement, c'est très simple. Un régime de THS standard à lui seul est à la limite de la magie (ne demande pas où est stockée la magie) en ce qui concerne la chute de cheveux, mais l'inclusion d'inhibiteurs de 5$\alpha$-Réductase comme dit dans la Section \ref{AA} "ANTIANDROGENES" est recommandée dans les cas les plus extrêmes pour arrêter toute chute. Le minoxodil 5\% sous forme topique est la seule chose qui fonctionne pour améliorer la ligne de cheveux au-delà des hormones seules, mais garde en tête que à part dans des cas miracles, tu ne sauves que des follicules dormants/mourants. Les follicules morts ne reviennent pas.   

Si ça ne te suffit pas, la technologie en matière de transplantation de cheveux s'est significativement améliorée. La procédure d'extraction d'unités folliculaires peut valoir le coup d'œil. Je rajouterai plus tard un lien ici écrit par une experte sur comment se faire rembourser pour cette procédure, dès qu'elle a fini de l'écrire. Je lui mets la pression. Garde un œil sur ce paragraphe. 

\subsection{Est-ce que le sport affecte la féminisation ?}

Probablement. Le THS cause une recomposition de ton corps de manière graduelle, donc tu peux encourager ton corps à changer avec de l'exercice. Garde en tête que ce processus est TRES LENT, donc il est crucial que tu manges assez pour te donner l'énergie d'avoir la patience nécessaire. Les hormones de croissance créées par la stimulation de tes muscles jouent un rôle dans le développement mammaire, donc c'est probablement une bonne chose même sans compter les bienfaits évidents du sport.

Ce n'est pas juste le fantasme de l'autrice mal déguisé; un entraînement musculaire est important pour ta santé ! Je mentionne ça parce que beaucoup de femmes trans croient que toucher une haltère vont les faire ressembler à Hulk. Je comprends, mais si tu n'as pas de testostérone et que tu n'es pas sous stéroïdes, tu ne vas pas ressember à ça. Et c'est sans parler du temps, de l'effort ou de l'application nécessaire pour s'approcher de ce genre de physique.

\subsection{Comment est-ce que je dois m'entraîner alors ?}\label{11-16}

Le cardio est utile pour vivre, ce qui est important. Les exercices de bas de corps vont élargir tes hanches et tes fessiers pour accentuer ta figure. Les exercices de haut de corps vont améliorer ta posture et créer un support pour ta poitrine, ce qui va rendre plus proéminente. En d'autres mots, un peu tout. Tu es sous œstrogènes. Tu as vu des femmes cis athlètes ? L'entraînement va te féminiser.

\href{https://docs.google.com/document/d/1-NyE5EY5TTaRRMhk7HlTbKJ7HifjEsA4jlDO1qKQVl0/edit?tab=t.0}{On m'a partagé ce guide (en anglais)} \textcolor{red}{(Attention: Lien Google Docs)} et ç'à l'air d'être un bon point de départ. Je note qu'il n'y a pas particulièrement d'exercice qui te féminisent / masculinisent vu que les corps ne fonctionnent pas comme ça, mais tu va probablement vouloir te concentrer plus sur les exercices de bas de corps et ceux de souplesse que la normale.  

\subsection{Est-ce que je vais rétrécir sous œstrogènes ?}

Oui. C'est possible que ça soit lié aux changements de contenu d'eau dans tes tendons et ligaments, mais ce n'est pas quelque chose qui à été étudié donc la cause est entièrement spéculée. Pour les scientifiques : voilà une idée d'étude ! 

\subsection{Est-ce que l'œstrogène peut causer un rétrécissement des pieds ?}

Oui, voir plus haut.

\subsection{Est-ce que l'œstrogène peut causer d'autres genres de rétrécissements ?}

Eh bien, "utilise-la ou perds-la", comme disent les jeunes.

\subsection*{Santé sexuelle}
\addcontentsline{toc}{subsection}{\textemdash{} Santé Sexuelle}

\subsection{Comment est-ce que j'améliore ma fonction érectile sous THS ?}\label{11-20}

Mis à part le fait de l'utiliser régulièrement, les manières d'amliorer sa fonction érectile incluent : 1) Améliorer ta condition et ta santé physique, en particulier ta capacité cardiovasculaire; 2) prendre du \textit{tadalafil} ou du \textit{sildenafil}; 3) prendre des suppléments de testostérone (voir Section \ref{T} "TESOTSTERONE"). 

Si tu veux lire une explication plus approfondie du mécanisme de la fonction érectile, \href{https://stainedglasswoman.substack.com/p/how-to-maintain-your-penis-function}{cet article substack (en anglais)} donne une bonne vue sur le sujet.

\subsection{Comment est-ce que j'améliore le volume de liquide séminal sous THS ?}

Ne sois pas embarrassé.e, c'est une question commune. Tu peux essayer la lécithine de tournesol et le pygeum, en plus d'une supplémentation en testostérone comme mentionné plus tôt. C'à l'air d'aussi faire une différence pour la lubrification vaginale et l'excitation pour celle.ux qui ont eu une chirurgie génitale, mais les données et les anecdotes sont limitées, donc c'est difficile à dire. A part ça, assure-toi de boire suffisamment et de te nourrir correctement.

\subsection{Est-ce que je peux produire du lait sous THS ?}

Oui. Domperidone, fenugrec, léchitine de tournesol, beaucoup d'œstrogènes, beaucoup de progesterone. Utilise une pompe, et éclate-toi.

A noter que la domperidone a des effets secondaires et des risques associés à son utilisation, et que la capacité de lactation n'influe pas sur le développement mammaire. Si tu veux plus de détails, tu vas vouloir regarder le protocole Newman-Goldfarb.

\subsection{Est-ce que le THS peut changer tes perceptions / sens, comme par exemple l'odorat ?}

Tu étais probablement dans un état de dissociation et de dépression avant de commencer ton THS. Le monde est plus vibrant maintenant et tu n'es plus en dissociation 24h/24. Les miracles de la médecine moderne !

Ca peut, cependant, changer ta vue. Ca fait partie des choses qui arrivent.

\subsection{Est-ce que le THS peut changer ma sexualité ?}

Similairement avec ce que je disais à propos de la dissociation au dessus, le fait d'être sous THS t'amène à une certaine ouverture d'esprit et d'acceptance de toi-même, ce qui peut causer des changements dans la façon dont ta sexualité se présente. Le fait que le changement soit comportemental ou chimique n'est au fond qu'une affaire de sémantique. Une question de perspective.

\subsection{Est-ce que je devrais être sous PrEP ?}

\textbf{Oui.} \href{https://fr.wikipedia.org/wiki/Prophylaxie_pr%C3%A9-exposition}{\textit{la prophylaxie pré-exposition} (PrEP)} est une catégorie de médicaments antiviraux qui a pour but déviter de contracter le VIH/SIDA. C'est pas vraiment lié au THS, mais c'est courant pour les femmes trans d'être sujet.tes à des risques élevés de contracter le VIH/SIDA. En connaissant l'histoire de la pandémie du SIDA, le PrEP est un miracle de la médecine moderne qui devrait t'intéresser. \textbf{Note: Aucun des médicament de la PrEP n'affecte le THS, donc tu devrais être sous PrEP.}

Si tu es sexuellement actif.ve, tu devrais vraiment considérer d'être sous PrEP. Même si tu n'es pas sexuellement actif.ve, les femmes trans sont sujet.tes à des risques élevés de violences sexuelles, donc tu devrais vraiment considérer le fait d'être sous PrEP. Dans la plupart des endroits, tu te fera prescrire de la \textit{Truvada} en pilule journalière, mais si ça te cause des nausées, tu vas probablement passer sous \textit{Descovy}, sans changement d'efficacité. Le nouveau médicament \textit{lenacapavir}, prescrit comme une injection tous les 6 mois, va probablement rendre la PrEP significativement plus accessible quand elle sera disponible dans plus d'endroits, si les options actuelles sont prohibitives pour toi.

\subsection*{Maltraitance médicale}
\addcontentsline{toc}{subsection}{\textemdash{} Maltraitance médicale}

\subsection{J'ai entendu dire que les injections sont moins stables parce que tu fais moins souvent. Est-ce que c'est vrai ?}

Seulement si tu suis le guide SOC 8 de merde de la WPATH, qui liste un régime recommmandé de \textit{valerate d'estradiol} ou de \textit{cypionate d'estradiol} de 5 à 30mg toutes les deux semaines, ce qui, pour être parfaitement claire, est quelque chose à ne jamais faire, sous aucun prétexte. "Je m'abstiendrai de tout mal", mon cul.

\subsection{Mais mon docteur dit-?}

Le docteur moyen n'a pas vraiment de formation sur quoi que ce soit de lié à la santé des personnes trans, et \href{https://www.endocrine.org/news-and-advocacy/news-room/2017/endocrinologists-want-training-in-transgender-care }{4 endocrinologues sur 5 aux Etats-Unis n'ont jamais reçu de formation pour la santé des personnes trans}. C'est très probable que tu sois sa.on premièr.e patient.e trans et qu'iel soit totalement inexpérimenté.e dans la gestion d'un.e patient.e trans. Même pour les médecins qui font attention, iels sont souvent limité.es par les standards de soin conservateurs qu'iels sont forcé.es de suivre et qui ne s'alignent pas toujours avec ce qui est bon pour toi. Voir au-dessus.

Aussi, rappelles-toi du "syndrome du bras cassé trans", aka la tendance des docteurs de faire porter le chapeau de la totalité de tes problèmes au THS. Si ton bras est cassé, c'est probablement pas "à cause de ces satanées hormones" ! 

Et je devrais le mettre dans une question à part, mais je ne veux pas casser le formatage : en lien avec les mauvaises pratiques médicales, il n'y a pas de situation dans laquelle il est raisonnable pour un.e médecin de te demander de voir ou de toucher ta poitrine pour "surveiller leur croissance" ou pour quelque raison que ce soit. C'est beaucoup moins commun dernièrement, heureusement, mais c'est une agression sexuelle est c'est totalment inacceptable.

\subsection{Ma.on docteur ne veut pas me prescrie d'injection. Qu'est-ce que je fais ?}

Essaie de la.e convaincre, d'en changer, ou passe en DIY. Ne laisse pas une instiution médicale t'empêcher de recevoir un traitement approprié que tu mérites. \textbf{L'aspect le plus crucial quand tu interagis avec le système médical est que tu dois te défendre.} Ca se multiplie quand tu est handi, racisé.e et d'autres afflictions qui effraient les docteurs comme le fait d'être une femme.

\subsection{Comment est-ce que le THS pour les femmes cis ménopausées se rapproche du THS pour les femmes trans ?}\label{11-29}

Même si nous avons des objectifs différents et crucialement des dosages très différents, il y un recouvrement immense dans l'expérience entre les femmes trans et les femmes cis ménopausées. Le misogynie médicale dans la forme d'une incompétence, d'un mépris, d'un antagonisme, et/ou d'une désinformation est quelque chose que malheureusement les 2 catégories vivent. C'est pour cette raison qu'il est très important de construire une solidarité sur ce front. Pour donner un exemple de ce que je veux dire, \href{https://www.youtube.com/watch?v=W0XW6av2wLQ}{Les premières 30-40 minutes de cette interview (en anglais)} vont te sembler extrêmement familières, si jamais tu veux augmenter ta pression sanguine. La personne interviewée note la connexion aussi ! La WHI a ruiné la vie de femmes innombrables.

\subsection*{Intersexualité et Comorbidités}
\addcontentsline{toc}{subsection}{\textemdash{} Intersexualité et Comorbidités}

\subsection{C'est quoi le truc avec le Syndrome d'Ehlers-Danlos?}

C'est un trouble du tissu conjonctif qui n'est pas lié au THS mais que beaucoup de personnes trans ont, donc félicitations dans le cas où tu as appris que c'est le cas pour toi aussi. A part pour les problèmes cardiovasculaires à long terme que tu devrais probablement surveiller, fais juste attention à bien garder un bon entraînement musculaire pour que tes articulations fonctionnent bien. Mais va plutôt voir ailleurs pour ça. Voir la Question \ref{11-16}.

\subsection{Quelles sont les trucs liés aux conditions intersexes que je dois garder en tête ?}

Au cours de ce guide, j'ai mentionné vaguement les conditions intersexes. Au-dessous figure une courte liste qui peut être utile à savoir pour toi ou pour tes proches.

\subsection{C'est quoi le truc avec le Syndrome de Klinefelter ?}

C'est une condition liée à l'intersexualité plutôt commune (au regard des mutations chromosomiques) que certaines femmes trans pourraient ne pas savoir qu'elles ont vu que ces deux ont des recouvrements. Généralement, ça se voit à un niveau de testostérone faible au début de la puberté. C'est bon de savoir le nom, juste au cas où.

\subsection{C'est quoi le truc avec le Syndrome des canaux de Müller persistants (PMDS)?}

Une nouvelle entrée pour la catégorie "Je mets ça ici parce que c'est probablement la première fois que tu entends le terme". C'est une condition liée à l'intersexualité qui peut affecter certaines femmes trans, même si on ne sait pas à quelle fréquence étant donné qu'on n'a pas de chiffres à ce propos. La présence possible d'un utérus sous-développé peut mener à des complications et des bizarreries. Tu veux probablement prendre de la progestérone pour éviter des risques de cancer de l'utérus.

\subsection{C'est quoi le truc avec le Trouble Ovotesticulaire ?}

Cette condition intersexe en particulier peut causer des fluctuation au départ du THS, ce qui peut mener à des résultats de prise de sang confus dû à la présence de tissus ovariens et testiculaires, soit séparés ou combinés en \textit{ovotestis}. Cela se présente sous différentes formes avec lequelles le THS peut interagir, au moment où tu commence à diminuer les niveaux \textit{d'hormone lutéinisante} (LH). Un utérus peut être présent ou non, plusieurs ensembles de gonades peuvent être présents, même si la condition peut ne pas être visible extérieurement.

\subsection{Quelle est la différence entre des crampes intestinales et des crampes utérines ?}\label{11-35}

Elles sont communément attribuées à tord au début de la transition comme étant un symptôme de condition intersexe. Les crampes intestinales sont étendues et se diffusent partout sur ton abdomen, alors que les crampes utérines sont très concentrées en un seul endroit juste au dessous du nombril et sont des contractions qui font comme des coups de poignard et qui arrivent en successions rapides. Comme si l'intérieur de ton corps était utilisé comme une boule à stress. Rien à voir !

\subsection{Et à propos des autres conditions intersexes ?}

J'ai parlé de quelques unes des plus notables, mais il y a bien plus d'expressions et de manières de les tester, qui dépassent de loin l'objectif de ce guide. Anecdotalement, la prévalence de ces conditions est plus élevée parmi les personnes trans, donc une familiarité de base avec ça est toujours utile.

\subsection*{Questions excentriques}
\addcontentsline{toc}{subsection}{\textemdash{} Questions excentriques}

\subsection{Beaucoup de sources de DIY utilisent de la crypto. Est-ce que c'est nécessaire ? Comment ça marche ?}

Il y a d'autres guides qui couvrent mieux ce sujet et plus en profondeur que ce dont je suis capable, et certains vendeurs ont leur propres guides. Mais oui, l'utilisation des crypto est souvent nécessaire pour plein de raisons. "Crypto" est un gros mot, mais son utilisation en tant que monnaie était le but au départ. C'est juste que c'est chiant à utiliser. Monero (XMR) est pas mal. 

\subsection{Et à propos des Modulateurs sélectifs des récepteurs aux œstrogènes (SERM) pour les régimes non-binaires ?}

Certaines personnes utilisent des SERMs pour leur transition pour ne pas se féminiser autant et avoir un look plus androgyne, mais c'est plus ou moins une terre inconnue aujourd'hui, et c'est pourquoi je ne les ai pas mentionné autre part dans ce guide. Tu dois te débrouiller si c'est quelque chose que tu veux explorer, donc fais preuve de prudence. Personnellement, je ne pense pas qu'ils soit si intéressants, étant donné que je n'ai rien vu qui suggère que les SERMs font ce que les gens qui en prennent veulent qu'ils fassent, en tout cas pas sans inconvénients, mais évidemment beaucoup de gens les aiment bien. C'est juste pas quelque chose que je pourrais confortablement recommander.

Les nombreux régimes non-binaires qui existent sont souvent très personnalisés parce qu'ils sont spécifiques au but de chacun.e. Tous les THS devraient être personnalisés jusqu'à un certain point, mais il y a souvent plus de variations dans les buts recherchés lorsqu'on traite d'androgynie. Hormonalement, ce n'est pas trivial. Tout ce qui est mis dans ce guide doit être traité seulement comme un point de départ si tu cherches à expérimenter avec quelque chose de plus compliqué, mais rappelles-toi qu'il y a bien plus pour aller vers tes objectifs de transition que les hormones seules.

\subsection{Est-ce que les choses commme les "hormones végétales" et les "phytoœstrogènes" sont des vrais trucs ?}

\textbf{Non.} Si quelqu'un te dit avoir des "hormones végétales", alors cette personne te dit avoir un remède de charlatan. La seule chose qui va te féminiser est l'œstrogène, et pas des œstrogènes de plantes. Aucune quantité de produits "naturels" ne peut remplacer les œstrogènes. Ce n'est pas une arnaque commune, et tu la connais probablement déjà, mais juste au cas où tu la vois quelque part, au moins maintenant tu as la certitude que c'est bien une arnaque. Si ç'à l'air d'être de la merde, c'est que c'en est probablement. A part si on parle de stéroïdes d'insectes, qui sont eux super cool. Mais aucune chance que ça fasse quoi que ce soit pour la féminisation par contre.

\subsection{Est-ce que le médecin de reddit dont tout le monde parle est compétent ?}

Non.

\subsection{J'ai entendu dire que les œstrogènes en DIY sont faites dans une baignoire, est-ce que c'est vrai ?}

Non. Je n'ai honnêtement aucune idée d'où vient cette blague ou de pourquoi certaines personnes la prennent au sérieux, mais il n'y a aucune étape de fabrication ni aucun procédé pour lequel on pourrait même considérer l'idée d'utiliser une baignoire. Ne crois pas tout ce qu'on dit sur internet. Je ne sais même pas qu'est-ce qu'on pourrait théoriquement faire avec une baignoire, à part si tu penses que les fioles sont pleines d'eau du bain de femmes trans. Je vois pas trop pourquoi on penserait ça ceci dit, c'est évident qu'elles sont faites par éjaculation.

\subsection{Comment est-ce que le THS affecte la fertilité ?}\label{11-42}

Il est très important de comprendre que c'est un sujet vraiment sous-étudié donc il est impossible de sortir des faits exacts et, étant donné l'impact d'une grossesse, je te supplie de te protéger et de prendre le moins de risques possible. Le THS en soit peut, et va probablement te rendre infertile au bout d'un moment, mais seulement avec un blocage complet de l'axe HPG (voir la Question \ref{2-3}) sur le temps long. En d'autres mots, si tu n'as pas eu de chirurgie génitale quelle qu'elle soit, et que tu es sous un régime de THS qui est moins capable de suppression de l'axe HPG (comme par exemple les pilules), alors il faut faire attention.

\textbf{Si l'axe HPG n'est pas inactif, alors il est totalment possible de féconder quelqu'un}, et la chronologie de maturation du sperme est suffisamment longue pour que ça reste vrai même après que l'axe HPG soit inactif depuis \textbf{plusieurs mois}. S'il te plaît, prends vraiment ça au sérieux. La désactivation de l'axe HPG depuis au moins six mois, voire plutôt un an en exerçant un maximum de prudence, est recommandée. 

\subsection{L'infertilité dûe au THS est-elle définitive ?}\label{11-43}

C'est théoriquement possible d'inverser l'infertilité causée par le THS, en supposant que tu n'étais pas déjà infertile avant de commencer (une supposition présomptueuse !), mais il n'y a pas beaucoup de cas documentés de ce phénomène, donc l'efficacité de la restoration de fertilité après un THS qui a commencé depuis longtemps n'est pas connue. Ce procédé nécessite de redémarrer l'axe HPG avec une variété de médicaments, tout en arrêtant ton THS complètement, ce qui veut dire une détransition hormonale pour au moins six mois probablement, et même comme ça la qualité de sperme n'est pas garantie. Ce n'est pas quelque chose que tu devrais prévoir, donc il serait sage d'avoir d'autres solutions. Je recommanderais une banque de sperme avant de commencer ou au départ de ton THS, si ta situation te le permet, si tu considères comme une priorité d'avoir des enfants biologiques et si une relation future où il serait possible/désirable d'en avoir est probable.  

Note: en France, il est possible de stocker ses gamètes au CECOS le plus proche jsuqu'à ses 60 ans gratuitement et, même si la loi est aujourd'hui nébuleuse sur la possibilité d'utiliser tes gamètes une fois que tu as changé ta mention de sexe à l'état civil, tu peux toujours demander à déplacer tes échantillons en Belgiques pour faire une procédure d'assistance à la procréation là bas. Tu peux contacter le CECOS le plus proche pour prendre rendez-vous à ce propos. Attention, dans certains cas, le parcours peut nécessiter un rendez-vous avec un.e psychologue qui va évaluer ta transidentité. Même si cette personne n'aura pas d'influence sur ton THS, l'expérience peut être compliquée.
%todo: localisation

\section{CRÉATINE}

\subsection{Qu'est-ce que c'est la créatine ?}

La créatine est un composé organique dans les muscles et le cerveau. Elle recycle l'ADP en ATP, ce qui est important pour la production d'énergie dans le corps, en particulier lors d'efforts intenses initiaux avant que d'autres systèmes énergétiques ne prennent le relais. 

\subsection{C'est pas genre un stéroïde ou un truc que les bodybuilders utilisent ?}

Non. Les bodybuilders et les athlètes l'aiment parce qu'avoir plus d'énergie permet de faire plus de choses avant d'être fatigué.e. Ce ne sont pas les seul.e.s à l'utiliser parce qu'en fait c'est le 1\ts{er} supplément vraiment efficace en terme d'effet et qui sont vraiment soutenu par la recherche.

\subsection{En quoi la créatine est liée au THS ?}

Il n'y a aucun lien ! Par contre c'est un sujet sur lequel je radote souvent parce que je pense que c'est une bonne chose et je suis fatiguée de me répéter parce que les gens continuent de me poser des questions à ce sujet et tu lis déjà ça de toutes façon non ? J'adore un public captif. Ma routine de stand-up est au sol.

\subsection{Okay d'accord pourquoi est-ce que je devrais prendre de la créatine alors ?}

Quelle splendide question ! C'est bon pour ton cerveau et tes muscles. La créatine se retrouve souvent en concentration relativement faible pour la plupart des gens suivant leur régime alimentaire, surtout pour les personnes qui ne mangent pas de viande. Il y des éléments de recherche convaincants qui lient diverses conditions de fatigue chronique ou d'état post-infection virale (COVID long en particulier) avec des réserves vides de créatine dans le cerveau, pour cela certaines personnes observent des améliorations des fonctions congnitives en se supplémentant. Ce n'est pas de la magie mais c'est vraiment pas cher donc ça vaut le coup d'essayer à mon avis.

\subsection{Sous quelle forme ?}

Tu veux juste de la poudre de \textit{créatine monohydrate}. Les pilules ont en général des dosage faibles et se font payer plus de toutes façon, de leur côté les bonbons gélifiés détruisent souvent la créatine lors de leur création. Beaucoup de marques ont de la créatine dans divers mélanges mais la version pure est souvent moins chère.

\subsection{Comment je la prends du coup ?}

La recommendation habituelle est de prendre 5-10g chaque jour en la dissolvant dans un liquide. Elle se dissout mieux dans les choses qui ne sont pas que de l'eau. C'est quasiment sans goût donc tu peux juste mettre une portion ou deux dans ton café ou ton smoothie. Il se peut de qu'en buvant il y ait une texture sableuse ou granuleuse suivant la quantité de poudre et le liquide utilisée. 

\subsection{Est-ce-que c'est important quand je la prends ?}

Pas vraiment. Il n'y a pas d'effet immédiat qui pourrait le justifier, ce qui rend le fait que ce soit micro-dosé dans des mélanges pré-entraînement assez marrant. Prends-la quand c'est le plus pratique pour toi.

\subsection{Comment ça marche alors ?}

Cela s'accumule dans ton corps jusqu'à atteindre un niveau maximum de saturation après une ou deux semaines. Il suffit ensuite de maintenir cet état et profiter des bénéfices (de peut-être te sentir mieux).

\subsection{Est-ce-que j'ai besoin de faire une phase de "charge" où j'en prend plus au début ?}

Probablement pas non. Si tu n'es pas dans une période d'entraînement intense ou un truc du genre ça ne changera rien. Contente toi de la prendre quand c'est le plus pratique pour toi de manière régulière.

\subsection{Quels sont les effets secondaires ?}

Il se peut que tu observes une légère augmentation de poids à cause de l'augmentation du poids de l'eau dans tes muscles (ce qui, soyons clair, est une bonne chose donc pas besoin de s'alarmer). Si tu ne la prends pas dans de l'eau ou si tu en prends trop d'un coup tu auras peut être des maux d'estomac. Ouille.

\subsection{Qui ne devrait pas en prendre ?}

Les gens avec des problèmes de foie. Pas parce que ça en cause mais parce que la créatinine (l'orthographe est différente ! la créatine devient de la créatinine) est utilisé dans les résultats d'analyses comme marqueur pour un éventail de problèmes du foie et prendre ce supplément peut créer des faux positifs. Important à garder en tête.

\subsection{Est-ce-que tu as une marque que tu recommandes ?}

Non. Celle que tu prends ne devrait pas avoir d'importance. Prends un truc qui semble avoir une bonne réputation avec un prix raisonnable. Je pourrais recommander celle que j'aime bien mais quand j'ai demandé à la marque un lien d'affiliation elle n'a pas voulu, tant pis pour eux ! Pas de publicité gratuite. 

\subsection{T'as vraiment mis la créatine dans ce document hein ?}

Ouais c'est assez marrant. C'est pas ma faute si j'en ai parlé comme ça et que des gens m'ont dit que ça les avait vraiment aidé.e.s, donc maintenant je me sens obligée de continuer d'en parler !!!

 

\section{OBSERVATIONS FINALES}

Si n'importe quelle des affirmations ci-dessous est toujours vraie :

\begin{itemize}
\item tu m'en veux toujours malgré les avertissements;

\item tu as remarqué un problème ou une faute de frappe;

\item tu as une question dont la clarification devrait être mise dans le texte;

\item tu as une objection qui je l'espère n'est pas une Euh Alors En Fait;

\item tu veux chanter mes louanges;

\item tu veux me jurer loyauté; 

\item tu veux m'envoyer une dîme;
\end{itemize}

Alors tu peux me contacter et je verrai ce que je peux faire. Bluesky est l'endroit où ce sera le plus facile, et tu peux me MP pour mon Signal. Sinon, merci d'avoir lu et j'espère que cela a été utile.

\textbf{Si tu veux donner pour soutenir ce projet,} \href{https://cash.app/Katitties}{CashApp}, \href{https://ko-fi.com/katitties}{Ko-Fi}, et \href{https://account.venmo.com/u/katitties}{Venmo} sont disponibles. Merci beaucoup !

Et finalement : \textbf{la chose la plus importante que tu puisses faire en tant que personne trans c'est de vivre.} Ce document est tout autant un manuel qu'il est un message à toi en tant que personne trans pour te dire que ton existence est un don pour ce monde, ta présence est une bénédiction pour celleux autour de toi, et tu mérites d'être traîté.e avec respect. Même si tu ne fais rien d'autre ta vie est une prouesse digne d'éloges. Merci.



\section*{AMIS DE PGHRT}\label{FOPGHRT}
\addcontentsline{toc}{section}{AMIS DE PGHRT}

Tout au long de ce document sont éparpillés des liens vers d'autres guides et ressources. La liste ci-dessous est une consolidation de ces liens avec quelques autres vers des ressources externes qui seront ajoutées au fur et à mesure, idéalement par d'autres personnes trans. Pour les personnes sensibles à leur vie privée ou qui veulent absolument éviter : il y a des liens Google Docs dans la liste.

\begin{enumerate}
  \item \href{https://startwith4mgestradiolenanthateweeklyandtestatonetothreemonths.com/}{SW4EEWATAOTTM} - TL;DR de PGHRT
  \item \href{https://hrtcafe.net/}{HRT Cafe} - Aggrégateur de Ressources THS
  \item \href{https://transfemscience.org/}{Transfeminine Science} - Ressource d'information pour la littérature médicale trans
  \item \href{http://estrannai.se}{Estrannai.se} - Bac à sable de Pharmacocinétique de l'Estradiol
  \item \href{https://globoho.mœ/}{Globoho.mœ} - Guide de Tourisme Médical pour l'Orchiectomie en Thaïlande
  \item Guide sur le FUE de Julia - ARRIVE BIENTÔT, JE L'EMBÊTE POUR QU'ELLE ÉCRIVE POUR PLUS VITE
  \item \href{https://docs.google.com/document/d/1-NyE5EY5TTaRRMhk7HlTbKJ7HifjEsA4jlDO1qKQVl0/edit?tab=t.0}{Sky's Feminine Figure Beginner Program} - Un programme d'exercices physiques à destination des trans fems
  \item \href{https://docs.google.com/document/d/114sztSw1aVWM2pXLDl9NrHklyvewz3EmFiHiisjM71k/edit?tab=t.0}{Sky's Diet 101} - Un guide pour ajuster son poids d'une manière saine
  \item \href{https://stainedglasswoman.substack.com/p/how-to-maintain-your-penis-function}{How to Maintain Erectile Function on HRT} - Une explication plus détaillée du phénomène "utilise-la ou perds-la"
  \item \href{hhttps://docs.google.com/document/d/1DXFxzN0XTudPZez_SO61fpqncRLPH_Be_QG_8Pcz9LU/edit?pli=1&tab=t.0}{Biohax Guide Googleslop Edition} - Guide de DIY Trans Masc
  \item \href{https://wikitrans.co/}{Wiki Trans français} - Ressources pour la transition (mais pas que) en France
\end{enumerate}

\section*{À PROPOS DE L'AUTRICE}
\addcontentsline{toc}{section}{À PROPOS DE L'AUTRICE}

Katie Tightpussy est une autrice primée et une femme trans professionnelle avec près d'une décennie d'expérience dans le champs de transgenre. Ses succès incluent transifier son sexe par le biais d'une technique novatrice d'injections d'hormones transexuelles, être physiquement incapable de la fermer, et utiliser une ensemble très pratique d'hyperfixations dans leur relation avec la transbobulation des humeurs. Elle passe ses journée dans la campagne rurale idyllique de Los Angeles devisant de nouvelles manière d'atteindre la domination mondiale et se plait à faire du vélo. Les demandes de renseignements des média peuvent s'addresser à son agent via \href{http://katietightpussy.com}{katietightpussy.com}. 


 

\section*{DIVULGATIONS}
\addcontentsline{toc}{section}{DIVULGATIONS}

Aucune meuf robot n'a été blessée ou endommagée lors de la réalisation de ce document, cela inclut toute utilisation de modèles de langage génératif (LLM). L'autrice n'approuve aucune reproduction qui serait faite sans attribution ainsi que tout grattage de ce contenu. Laissez ces pauvres robots tranquille.

L'autrice déclare une attraction envers les femmes et reconnaît un possible conflit d'intérêt vis-à-vis de l'existence d'un plus grand nombre de femmes trans magnifiques dans le monde.

 

\section*{REMERCIEMENTS}
\addcontentsline{toc}{section}{REMERCIEMENTS}

Bien que ce texte soit mien il serait loin d'être aussi bon sans les contributions, retours et suggestions d'autres personnes impliquées tout au long de la rédaction. Il s'agit d'un rappel oh combien important qu'une transition n'est que meilleure quand on est accompagné.e.

Merci beaucoup à Q, R, RM et S dans l'ordre alphabétique pour leurs revues précises et pour être des nerds sympa avec qui parler plus généralement ; je vous adore. Remerciements très spéciaux à CB et J pour leurs revues détaillées qui ont inspirés des passages marrants. Merci à KG pour les informations supplémentaires sur les personnes intersexes. Merci à w [sic] pour des ressources supplémentaires sur l'injection. Merci collectif à BIR pour la pléthore de pinaillages de nerd très importants. Appréciations pour les revues générales de C, JTP, K, S et V. Merci à toutes les personnes sur Bluesky qui m'ont encouragée à écrire ce guide et à tout le monde qui ont pu partager leurs connaissances au travers des années. Évidemment pour finir : merci aux nerds du THS, même quand nous avons des désaccords, car ce que nous voulons par dessus tout c'est le meilleur pour notre communauté trop souvent abandonnée. Continuons comme ça.

Dédicace à mon professeur d'IB Chemistry HL il y bien longtemps qui doutait de manière très raisonnable de mon assiduité alors que je mets en application une très grande partie de ces connaissances au service de l'art de la transexualité ; va savoir pourquoi.

Remerciements pour la traduction française : merci à N pour ton aide à la traduction et à la relecture et merci au SRTIP et au FLIRT pour avoir fait des super guides d'injections !

\section*{MISES À JOUR}
\addcontentsline{toc}{section}{MISES À JOUR}

\noindent \href{https://github.com/Juicysteak117/pghrt/}{Code source disponible sur ce dépot Github.}

\noindent Date de la dernière génération : \DTMnow

\noindent(Il n'y a pas de binding LaTeXML pour \texttt{datetime2}, \texttt{hanging}, ou \texttt{hyphenat}, donc le formatage est un peu moche. Si tu veux vraiment m'aider j'adorerais que tu écrives ces bindings !!!)

\noindent 2026-01-2X: Publication initiale en français. 24.3k mots.

\end{document}