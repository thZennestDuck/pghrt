\documentclass{article}
\usepackage{hyperref}
\usepackage{float}
\usepackage{csquotes}
\usepackage[style=iso]{datetime2}
\usepackage[usenames,dvipsnames]{color}
\usepackage{booktabs}
  \setlength\heavyrulewidth{0.20ex}
  \setlength\cmidrulewidth{0.10ex}
  \setlength\lightrulewidth{0.10ex}

\usepackage[font=normalsize,labelfont={bf}]{caption}
  \captionsetup[table]{aboveskip=3pt}

\hypersetup{
    colorlinks=true,
    linkcolor=blue,
    filecolor=magenta,      
    urlcolor=magenta,
 }
 % ZU ERLEDIGEN
 % Eine A5-PDF-Version hinzufügen
 % Entscheiden, wie Übersetzungen und Übersetzungshinweise gezeigt werden sollen
 % Eine automatische Lokalisierungsfunktion für verfügbare Lokalisierungen beim Einrichten der Lokalisierung hinzufügen
 
\usepackage{graphicx}
\graphicspath{{img/}}
\renewcommand{\abstractname}{DISCLAIMER}
\title{EINE PRAKTISCHE ANLEITUNG ZUR FEMINISIERENDEN HRT}
\author{\href{https://katea.gay/}{Katie Tightpussy}}
\date{\today}
\setcounter{section}{-1}
\setcounter{figure}{-1}
\urlstyle{same}

\begin{document}


\maketitle
\tableofcontents

\begin{abstract}
    Ich bin keine Ärztin. Ich arbeite nicht im medizinischen Bereich. Ich bin in keiner Weise medizinisch ausgebildet. Ich bin eine Nichtfachfrau, die nichtfachmännische Meinungen anbietet, auf der Basis und im Umfang meiner eigenen Erfahrungen. Alle folgenden Informationen und Behauptungen sollten dementsprechend als bloße Meinungsäußerungen verstanden werden, nicht als Fakten oder medizinische Ratschläge. Diese Anleitung gibt der in der Community erarbeiteten moralische Wahrheit den Vorzug, wo die Wissenschaft noch nicht soweit ist. Kurzum: Sei nicht böse auf mich. 
\end{abstract}


\section*{LANGUAGES}
\addcontentsline{toc}{section}{LANGUAGES}

Dieses Dokument ist derzeit in diesen Sprachen verfügbar:

\href{https://en.pghrt.diy}{🇺🇸 English}, \href{https://de.pghrt.diy}{🇩🇪 Deutsch}

\noindent\textbf{Wenn du Interesse hast, eine Übersetzung oder eine andere Version zu machen, melde dich bitte bei Katie!}



\section{VORWORT}

\subsection*{Vorwort zur Übersetzung}

Im deutschsprachigen Raum war bisher keine umfassende, aktuelle und vor allem verständlich geschriebene Informationssammlung zur feminisierenden Hormonersatztherapie verfügbar. Diese Übersetzung des hervorragenden Werks von Katie Tightpussy ist ein Versuch, diesen Mangel auszugleichen. Nicht alle sind in der Lage, englischsprachige Anleitungen und Übersichten zu verstehen. KI-gestützte Übersetzungen - trotz ihrer vermeintlichen Vorzüge hinsichtlich der leichten Handhabung und Schnelligkeit - sind für diese Aufgabe ebenfalls ungeeignet, da viele der Nuancen und Feinheiten des Textinhalts dabei verloren gehen und wir sowieso beim Thema Gesundheit absolut niemals einer dieser Maschinen vertrauen sollten; insbesondere da sie transphoben Arschlöchern gehören. Diese in liebevoller Handarbeit entstandene Übersetzung soll es allen deutschsprachigen Transpersonen (und  interessierten cis Menschen) ermöglichen, sich selbst zu bilden und informierte Entscheidungen treffen zu können ohne dabei das Augenzwinkern des Originals zu verlieren. 

Im Text wird, wenn notwendig, durch die Abkürzung "A.d.Ü." (Anmerkung der Übersetzerinnen) auf besondere Entscheidungen bei der Übersetzung, sprachliche Entscheidungen und Ergänzungen oder spezifische Informationen zur Situation in Deutschland verwiesen. Leider wissen wir beide nicht viel darüber wie es in anderen deutschsprachigen Regionen abläuft - wenn es da Ergänzungen zu Österreich oder der Schweiz gibt sind wir für Feedback dankbar!

Wir haben uns auch bewusst dafür entschieden gängige Terminologien (z.B. Vial, Coring oder auch HRT anstatt HET) aus der Community beizubehalten - schließlich geht es darum, dass lesende Menschen weiterführende Informationen für \textit{ihre persönliche HRT finden und sich beim drüber austauschen untereinander verstehen.} Medizinisches Fachpersonal bitten wir darüber hinweg zu sehen. Wir freuen uns natürlich wenn auch ihr diese Anleitung lest, aber unser Augenmerk gilt unseren Schwestern und nicht-binären Geschwistern. 

Ebenfalls ist es wichtig, dass der allgemeine Disclaimer auch für uns gilt: Wir sind keine medizinischen Fachfrauen sondern übersetzen nur einen Text der uns persönlich geholfen hat. Entsprechend ist dies eben \textit{keine medizinische Beratung oder Fachliteratur} sondern muss als journalistisches Erzeugnis verstanden werden.

Für etwaige (Tipp-)Fehler entschuldigen sich die Übersetzerinnen im Voraus. Kritik, Lobgesänge und Hinweise aller Art werden auf Bluesky \href{https://bsky.app/profile/camille.gay}{@camille.gay} und \href{https://bsky.app/profile/synthiecat.bsky.social}{@synthiecat.bsky.social} angenommen.

Virgo Sine Ruga, a.k.a. Camille, Oktober 2025 \& synthie aka. Lina, Dez. 2025/Jan. 2026

\subsection*{Katies Vorwort}
Der Zweck dieses Dokuments ist die Katalogisierung meiner Gedanken und Meinungen zur feminisierenden Hormonersatztherapie (A.d.Ü.: zur besseren Übersichtlichkeit ab jetzt mit "HRT" abgekürzt), da die verschiedenen Wikis der Community meiner Meinung nach ungeeignet sind. Diese sind zwar wertvolle Ressourcen, aus meiner Sicht jedoch für Personen ungeeignet die eine klare, handlungsorientierte Anleitung suchen, statt umfassende Information und Diagramme zu allen biologischen Vorgängen. Mein Ziel ist es, eine umfassende und schnell erfassbare Anleitung mit nachvollziehbaren Antworten zu den meisten Fragen zur Durchführung von HRT verfügbar zu machen. Diese lebensrettende Medizin soll sowohl für Menschen, die HRT in Betracht ziehen, als auch für erfahrene Transpersonen entmystifiziert werden. Somit setze ich eine gewisse Vertrautheit mit den Effekten von HRT voraus. Falls du nicht damit vertraut bist: HRT ist sehr effektiv und macht wahrscheinlich mehr als du denkst. Es ist eine tolle Erfahrung! \textbf{Dein Gender zu ändern ist echt cool und macht Spaß. Ich kann's nur empfehlen.} Du verdienst eine gute medizinische Versorgung deiner Transition und kannst die besten Entscheidungen für Dich treffen. Ich hoffe, dass dieses Dokument eine nützliche Hilfe in deinem Entscheidungsprozess sein kann, und, falls es Dich interessiert, als Startpunkt für weitere Recherche dient.

Und halte Dich fern von den transbezogenen Subreddits. Vertrau mir einfach, okay? Vermeide zumindest /r/mtf (A.d.Ü. Und /r/trans\_de!), da dieses besonders übel ist. Die Subs sind keine gesunden, unterstützenden Ressourcen und dort ist einfach nur viel Unsinn zu finden. Du wirst dir jahrelang verfaulte Gehirnwürmer herausziehen müssen um die Fehlinformationen aus deinem Kopf zu bekommen. Wirklich der beste Rat, den ich dir geben kann! (A.d.Ü. r/germantrans ist ganz okay und für unsere kleinere, zum Teil zersplitterte Community schon wichtig und hilfreich. Trotzdem nur in kleineren Mengen genießen.)

Für die Männer und transmasc enbies da draußen: teile dieses Dokuments sind auch für euch relevant, aber es gibt offensichtliche und entscheidende Unterschiede in den Zielen und gewünschten Effekten. \href{https://docs.google.com/document/d/1DXFxzN0XTudPZez\_SO61fpqncRLPH\_Be\_QG\_8Pcz9LU/edit?tab=t.0}{Diese Anleitung für eine maskulinisierende HRT} \textcolor{red}{(Warnung: Google Docs link)} (A.d.Ü. und auf Englisch) sieht ziemlich gut aus, ich habe sie aber nicht vollumfänglich durchgearbeitet, setze also auf Deinen Verstand und lass Vorsicht walten. Ihr solltet euch sowieso eine maskuline Katie Tightpussy einfallen lassen. Oliver Longdick oder so. Vielleicht Xavier. 

\textbf{Wenn du dieses Projekt durch eine Spende unterstützen möchtest,} \href{https://cash.app/Katitties}{CashApp}, \href{https://ko-fi.com/katitties}{Ko-Fi}, und \href{https://account.venmo.com/u/katitties}{Venmo} gehen alle. Vielen Dank!

\subsection*{Wie dieses Dokument benutzt wird}

Das Dokument hat eine lineare Struktur und ist als eine Reihe von Fragen und Antworten aufgebaut, sodass in der Regel jede Frage und Sektion fließend in die nächste übergeht. Ich empfehle, von oben bis unten alles durchzulesen, um hoffentlich wie in einer Unterhaltung alle mögliche Fragen zu beantworten (sogar diejenige von denen du nicht wusstest, dass du sie hast). Nimm Dir Zeit und lese dieses Dokument Stück für Stück - es ist lang aber ausführlich.

Du kannst das Inhaltsverzeichnis benutzen, um zu einer bestimmten Frage oder Sektion zu springen, vor allem beim Wiederlesen oder Nachschlagen. Ich empfehle, diese Seite/dieses Dokument zu speichern, um es immer zur Hand zu haben wenn du Fragen zu deiner HRT hast. Es sind sehr viele Informationen auf einmal, es ist okay wenn du es  Dir nicht alles sofort merkst! Es gibt keinen Grund dich zu hetzen. 

\noindent\textbf{\href{https://raw.githubusercontent.com/Juicysteak117/pghrt/refs/heads/main/pdfs/pghrt.pdf}{Hier kann das Dokument als PDF heruntergeladen werden. Mach das bitte.}}

\noindent\href{pghrtgretchensversion.txt}{Hier kannst du es alternativ als 90er-Style Textdatei lesen, wenn Dir das Spaß macht.} Diese Version wird nicht aktualisiert. (A.d.Ü. und ist nur auf Englisch verfügbar!)

\section*{WIDMUNG}
\addcontentsline{toc}{section}{WIDMUNG}

Dieses Dokument ist all unseren Schwestern gewidmet, die es nicht geschafft haben. Mögen wir gemeinsam ihr Licht in einen neuen Tag tragen.
 

\section{EINFÜHRUNG}

\subsection{Ist eine Östrogen-HRT sicher?}

Mit modernen, bioidentischen Hormonen ist HRT so sicher wie noch nie. Du ersetzt einfach den Saft, den dein Körper in erster Linie benutzt, durch den anderen, und veränderst die Gewichtung der Hormone, die sich schon in deinem Körper befinden. Obwohl die Details der Optimierung komplex sein können, ist der Prozess ziemlich fehlertolerant. Der Körper ist flexibel und du wirst es schaffen, deine HRT so zu justieren dass sie sich gut für Dich anfühlt.

\subsection{Welche Medikamentenform sollte ich wählen?}

Injektionen. Sie sind insgesamt die effektivste, einfachste, einheitlichste, sicherste und preiswerteste Form von HRT. Für manche werden Injektionen zu einem willkommenen Ritual, oder machen sogar Spaß. (A.d.Ü. Injektionen sind in Deutschland fast immer DIY. Es gibt einige wenige Apotheken die Estradiol Valerate Injektionen anbieten, eine Übersicht findet ihr \href{https://transdb.de/search?type=pharmacy&offers=eInjection}{hier}. Ihr braucht dann allerdings ein Rezept, entsprechend eine Indikation und Glück das euer Endo/Gyn/Uro mitspielt.)

\noindent\underline{\textbf{Am wichtigsten bleibt aber: jede Form von Östrogen ist besser als kein Östrogen.}}

\subsection{Warum keine Pillen, Gels oder Pflaster?}

Weil sie alle gegenüber Injektionen große Nachteile haben. Es ist nicht so, dass sie nichts bringen, aber du hast es nicht verdient, diese Nachteile aushalten zu müssen. Ich wiederhole: \textbf{alle Formen von HRT können gute Ergebnisse erreichen}, aber das heißt nicht, dass sie alle gleichwertig oder gut sind.

\subsection{Ist die Östrogen-Dosis die gleiche zwischen den verschiedenen Verabreichungsformen?}

Nein. Und das ist wichtig genug, um es hier zu erwähnen und nicht erst in der Sektion \ref{MM} “FAQ, MYTHEN UND IRRTÜMER”. \textbf{Die Dosierung von Östrogen kann zwischen verschiedenen Verabreichungsformen nicht direkt verglichen oder umgerechnet werden.} 1mg von der einen "Sorte" ist nicht 1mg von der anderen. Unterschiedliche Formen des gleichen Wirkstoffes haben unterschiedliche Eigenschaften, welche die Aufnahme des Östrogens durch den Körper beeinflussen (\textit{die sogenannte “Bioverfügbarkeit”}), sowie unterschiedliche Geschwindigkeiten der Aufnahme und die daraus resultierende Halbwertszeit des Wirkstoffes.

\subsection{Was bedeutet "Halbwertszeit"?}

Vereinfacht ausgedrückt, ist die \textit{Halbwertszeit} einer Substanz die Zeit, die vergeht, bis die Hälfte dieser Substanz ausgeschieden wurde. Im Kontext von HRT ist dies die entscheidende Angabe dafür, wie lang eine Dosis in deinem System aktiv bleibt, und somit dafür, wie oft du eine weitere Dosis verabreichen musst. Das ist dann dein Hormonzyklus, und der bildet eine Kurve. Der Östrogenspiegel steigt nach der Einnahme, erreicht einen Höhepunkt, und sinkt dann wieder ab. Die Eigenschaften dieser Hormonwertkurve (wie sich der Östrogenspiegel über die Zeit verändert) sind wichtig.

\subsection{Was spricht gegen Pillen?}

Das größte Problem mit Pillen sind das erhöhte Risiko für Blutgerinnsel und Leberstörungen bei langfristiger Anwendung. Dieses Risiko kann abgeschwächt werden, indem man die Pillen sublingual oder bukkal (also indem die Pille unter der Zunge oder in der Wange aufgelöst wird) nutzt statt sie oral einzunehmen (die Pille einfach schlucken), um so den sogenannten First-Pass-Effekt in der Leber zu vermeiden. Es ist jedoch anzunehmen, dass sogar bei sublingualer oder bukkaler Anwendung etwas von der Pille verschluckt wird, sodass die Risiken nicht ganz weg sind. Das absolute Risiko immer noch sehr niedrig (z.B. ist \textit{Paracetamol} um eine Größenordnung gefährlicher für die Leber als Östrogen), jedoch \textbf{wird dieses Risiko weiter in Kombination mit nikotinbedingtem Östrogen-Wechselwirkungen verstärkt.} Siehe Frage \ref{11-2} dazu.

Darüber hinaus sprechen zwei weitere Eigenschaften von Pillen gegen ihre Anwendung: 1) ihre sehr kurze Halbwertszeit und schlechte Bioverfügbarkeit, 2) der in Kombination mit Pillen oft notwendige Einsatz von Antiandrogenen. Die erste Punkt bedeutet, dass Pillen für eine Monotherapie (wird weiter unten genauer erklärt) im Vergleich mit Injektionen fast immer ungeeignet sind. Der zweite Punkt bedeutet oft eine Menge an unerwünschten Nebenwirkungen, je nachdem welches Antiandrogen verwendet wird (siehe Sektion \ref{AA} “ANTIANDROGENE”). Zusammen führen diese Eigenschaften zu mehr Variabilität, die schwierigen Abläufen und unerwünschten Nebenwirkungen (wie z. B. niedrige Energie/Libido und langsamere Ergebnisse) wahrscheinlicher machen als andere Verabreichungsformen. Es ist auch schwieriger, mit Pillen einen Vorrat anzulegen, und mancherorts sind sie teurer als Injektionen. Merke auch, dass die Einfuhr von Pillen aus dem Ausland in großen Mengen den Zoll alarmieren kann, was zu finanziellen Verlusten, zum Verlust der Pillen und/oder möglicherweise zu rechtlichen Problemen je nach deinem lokalen Gesetz führen kann. \textbf{Falls irgendjemand fragt, weißt du nicht, wer diese Pillen bestellt hat.}

\textbf{Falls du aus welchem Grund auch immer Pillen nimmst, nimm bitte 4-8mg sublingual über den Tag verteilt. Unter 4mg ist fast nie eine ausreichende Dosis.}

\subsection{Was spricht gegen Pflaster?}

\begin{itemize}
  \item Relativ teuer (meistens sogar teurer als Pillen);
  \item Schwieriger über DIY zu bekommen (nur über den sogenannten "grauen Markt");
  \item Benötigen meistens ein Antiandrogen (siehe Sektion \ref{AA} “ANTIANDROGENE”);
  \item Kann zu Hautirritationen führen;
  \item Müssen 24/7 dran bleiben;
  \item Neigen dazu sich abzulösen;
  \item Die Aufnahme ist nicht immer gleichmäßig (verändert sich z.B. durch Hitze);
  \item Schwierig, einen Vorrat anzulegen (schwer, sie in größeren Mengen zu bekommen);
  \item Schaffen es meistens nicht, Östrogen-Werte über Wechseljahren-Niveau anzuheben, sogar mit mehreren Pflastern auf einmal.
\end{itemize}

\subsection{Was spricht gegen Gel?}

\begin{itemize}
  \item Schwer, es genau zu dosieren, was zu ungleichmäßigen Werten führt;
  \item Es muss regelmäßig aufgetragen werden, da die Halbwertszeit relativ niedrig ist;
  \item Anwendung kann unangenehm sein (schleimig) (A.d.Ü. und stark riechend);
  \item Risiko der passiven Exposition von anderen durch Hautkontakt (A.d.Ü. Denkt auch an eure Haustiere);
  \item Benötigt meistens ein Antiandrogen (siehe Sektion \ref{AA} “ANTIANDROGENE”).
\end{itemize}

Es muss jedoch angemerkt werden, dass Gel mit relativ wenig Aufwand selbst hergestellt werden kann, was in machen Fällen ein Segen sein kann.

\subsection{Was ist mit Pellets (Implantaten)?}

\begin{itemize}
  \item Meistens viel teurer als andere Optionen;
  \item Sehr selten verfügbar;
  \item Die Dosis kann nur über einen langen Zeitraum justiert werden;
  \item Fehlerhafte Pellets können zu schlechten Werten führen;
  \item Gebrochene/zerquetschte Pellets können zu unerwartet erhöhten Werten führen;
  \item Generell nicht als DIY Option verfügbar.
\end{itemize}

Der letzte Punkt bedeutet, dass es nur über wenige, wahrscheinlich teure Anbieter möglich ist, überhaupt an Pellets zu kommen. Vielleicht hast du hier zum ersten Mal davon gehört. Siehst du das Problem? (A.d.Ü. Bisher gibt es keine Anbieter\_innen für Implantate in der DACH Region.)

\subsection{Was ist mit Sprays?}

Diese sind noch recht experimentell, daher kann nicht so viel darüber gesagt werden, aber sie teilen sich Vor- und Nachteile mit Gel. Sie werden hier erwähnt damit du weißt, dass es sie gibt.

\subsection{Ist der Unterschied wirklich so relevant?}

\textbf{Ja.}So relevant, dass ich das alles im Detail aufzuschreiben, damit ich mich weniger wiederholen muss indem ich diesen Punkt einfach verlinke. Ein gut ausgearbeiteter Injektionsrythmus ist zum Erreichen einer Östrogen-HRT-Monotherapie-Werten der beste Weg.

\subsection{Ist diese Tabelle korrekt?}\label{1-12}

\begin{figure}[H]
	\centering
	\includegraphics[width=1\linewidth]{STUPID_CHART_evil_bad_bad_destroy_evil_bad.png}
	\caption{Diese Tabelle ist scheiße.}
	\label{fig:scebbdeb}
\end{figure}

\textbf{Nein.} Während es nicht die Aufgabe dieses Leitfadens ist, jede Bullshit-Information direkt zu widerlegen die in sozialen Netzwerken als auch ärztlichen Kontexten aufkommt rechtfertigt die weite Verbreitung dieser Tabelle das hier zu tun. Die schiere Menge an Schaden, die dieses Bild angerichtet hat rechtfertigt mehr als nur eine Erwähnung in \ref{11-6}.

\textbf{Der Inhalt dieser Tabelle ist kategorisch falsch.} Nahezu jeder Aspekt ist in irgendeiner Weise irreführend, mit der einzigen Ausnahme, dass es völlig richtig ist, dass Östrogen keine Stimmveränderungen verursacht. Die Zeitspannen, die für das “erwartete Einsetzen“ angegeben werden, sind schlichtweg falsch, und die Idee einer zeitlichen Begrenzung für die “maximale Wirkung“ ist so irreführend, dass sie fast als kriminell zu bezeichnen ist. Eine Vielzahl von Veränderungen wird nicht aufgeführt (z. B. psychische Wirkungen) oder ist offensichtlicher Unfug (die Tabelle widerspricht sich selbst in Bezug auf erektile Dysfunktion). Auch auf die Art der Einnahme wird nicht eingegangen.

\textbf{Die wichtigsten Punkte der Grafik werden wir uns in diesem Leitfaden genauer anschauen. Der eine wichtige Punkt den man sich merken sollte ist der Fakt, dass Veränderungen unterschiedlich schnell passieren.} Diese Tabelle weckt falsche Erwartungen, vermittelt ein ungenaues Bild von der Wirkung einer HRT und führt Transmenschen in die Irre, indem sie ein falsches Verständnis von Hormonen erzeugt. Bitte ignoriere sie einfach komplett. Nach diesem kleinen Schwenker machen wir jetzt weiter im Programm.

\section{WARUM INJEKTIONEN!?}

\subsection{Was macht Injektionen so gut?}

\textbf{Gleichmäßigkeit.} Bei HRT kommt es auf Gleichmäßigkeit an. Gleichbleibende Hormonenwerte bedeuten Stabilität, und Stabilität ist gut. Sogar die am wenigsten stabile Art von Injektionen (kommen wir gleich zu) kann einen einheitlicheren Hormonzyklus als alle anderen Verabreichungsformen ermöglichen.

\subsection{Sind Antiandrogene bei Injektionen notwendig?}

Meistens nicht. Ein richtig dosierter und zeitlich abgestimmter Injektionszyklus, der kontinuierliche, ausreichend hohe Östrogen-Werte erreicht, kann die Testosteron-Produktion auf natürliche Weise unterdrücken. Somit wird kein Antiandrogen benötigt, was fas immer der bevorzugte Weg sein sollte. Dieses Vorgehen wird \textit{“Monotherapie”} genannt.

\subsection{Wie funktioniert Monotherapie?}\label{2-3}

Ganz vereinfacht gesagt interessiert es das Gehirn nicht, welches dominante Sexualhormon es hat, solange es genug davon hat. Wenn ständig genug Hormone in deinem Körper vorhanden sind, werden keine weitere produziert. Diese Beständigkeit in den Hormonwerten ist das, was Injektionen ermöglichen und andere Verabreichungsformen oft nicht schaffen. Zum Beispiel ist Monotherapie mit Pillen in den meisten Fällen so gut wie unmöglich. Genauer gesagt und im Bezug zur Hypothalamus-Hypophysen-Nebennierenrinden-Achse \href{https://en.wikipedia.org/wiki/Hypothalamic-pituitary-gonadal_axis}{(HHN-Achse) (A.d.Ü. Englischer Link)} werden durch einen ausreichend hohen \textit{Estradiol-Serumspiegel (E2)} das \textit{luteinisierende Hormon} (LH) und \textit{follikelstimulierende Hormon} (FSH) unterdrückt, was wiederum die Produktion vom Gonadotropin-Releasing-Hormon (GnRH) unterdrückt und damit die Produktion von Testosteron in den Testikeln “ausschaltet”.

\subsection{Inwiefern sind Injektionen sicherer als andere Formen?}

Da Antiandrogene meistens nicht gebraucht werden, werden die damit verbundenen Langzeitrisiken vermieden(siehe Sektion \ref{AA} “ANTIANDROGENE”). Bioidentisches Östrogen, das gar nicht durch die Leber verarbeitet wird (siehe Frage \ref{11-1}), kommt einer natürlichen Östrogen-Produktion am nächsten und verringert somit weiter die Risiken.

\subsection{Aber ist das Injizieren nicht an sich schon gefährlich?}

Ja, aber durch eine einfache Anleitung (siehe Sektion \ref{ts} “TECHNIK UND MATERIALIEN”) lässt sich die Gefahr bis auf den einen oder anderen blauen Fleck minimieren. Es ist ein bisschen wie Fahrrad fahren: sobald du weißt, wie es geht, müsstest du dich schon SEHR anstrengen, es so richtig falsch zu machen.

\subsection{Inwieweit sind Injektionen leichter?}

Sobald du dich eingestellt hast, bist du startklar. Injektionen müssen nicht oft gemacht werden (z. B. eine wöchentliche Injektion im Vergleich zu mehreren Pillen täglich), es besteht kein großes Risiko einer falschen Dosierung (im Vergleich zu Gelen), sie können sich nicht mitten im Zyklus von der Haut ablösen (im Vergleich zu Pflastern) und du musst nicht extra zu einem Ärzty fahren (im Vergleich zu Pellets).

\subsection{Inwieweit sind Injektionen preiswert?}

In einfachen Worten, es wird weniger Östrogen gebraucht. Eine 10ml Ampulle (A.d.Ü. ab jetzt “Vial” genannt), das dich über ein Jahr lang mit Östrogen versorgen kann, beinhaltet nur 400-500mg Estradiol, während das äquivalente Minimum an Pillen (4mg * 365 Tage = 1460 mg) wesentlich mehr benötigt. Das lässt sich nicht direkt vergleichen, aber es macht den Unterschied im Maßstab jedoch deutlich. Als weiterer lustiger Vergleich kann man 1 bis 2 Jahre Östrogen-Vials in einer handelsüblichen Pillen Packung eines drei Monats Vorrats verstauen. (A.d.Ü. Die Werte hier sind beim Übersetzen auf gängige Größen in Europa angepasst worden.)

\subsection{Aber ich habe keine Krankenversichrung / meine Versicherung zahlt es nicht / durch meine Versicherung sind Pillen billiger als Injektionen / Injektionen gibt es in meinem Land nicht / mein Ärzty verschreibt mir keine Injektionen?}

Warte bis wir zur Sektion \ref{sv} “VIALS BEZIEHEN” kommen. Du wirst vermutlich überrascht sein und, sehr wahrscheinlich, radikalisiert werden.

\subsection{Ist es gut, auch nach Jahren der HRT zu Injektionen zu wechseln?}

\textbf{Ja.} Nichts ist jemals sicher, aber viele Leute erleben wesentliche und deutliche Verbesserungen nachdem sie zu Injektionen gewechselt sind, selbst nach Jahren der Hormonersatztherapie. Diese Verbesserungen betreffen oft Brustwachstum, das psychischen Wohlbefinden, das Verschwinden von Nebenwirkungen durch Antiandrogene oder anderen Östrogen-Formen, bis hin zu allgemein besserer Stimmung etc. Der Wechsel lohnt sich.

\textbf{Aber keine Sorge wenn ihr bereits eine HRT mit anderen Medikamenten angefangen habt, die HRT-Zeit vor den Injektionen ist nicht “verschwendet” und es gibt auch kein begrenztes Zeitfenster, in dem Östrogen wirkt.} Das gute an Injektionen ist einfach das die Werte gleichmäßig sind. Sieh dir die Fragen \ref{11-5} und \ref{11-6} an.

\subsection{Aber Injektionen machen mir Angst.}
Ja, das geht allen am Anfang so. Niemand mag Nadeln  (A.d.Ü. Wir sollten eigentlich den technischen Begriff "Kanüle" verwenden, da dieser genauer ist. Nadeln interessieren uns hier eigentlich nicht, da sie streng gesehen zum Nähen, Stricken oder zur Akupunktur benutzt werden. Wir verwenden hier jedoch “Nadel” im Sinne einer einfachen Sprache weiter), und man hat eine natürliche Angst davor sich selbst zu stechen, aber mit ein bisschen Übung und dem richtigen Zubehör merkt man es kaum. Es gibt zahlreiche Menschen, die Anfangs eine Nadelphobie hatten und sich schnell so sehr dran gewöhnt haben, dass sie ihre Injektionen jetzt nur noch langweilig finden. Deine Angst ist normal und weit verbreitet, aber sie kann überwunden werden, und es lohnt sich. “Oh, das war ja gar nicht so schlimm” ist ein Satz, den man oft hört. Wie Seneca einst schrieb: “Trau dich, sei mutig! Kein Übel ist so schlimm wie die Angst davor”. Du wirst es schaffen.

\subsection{Sind Injektionen so schlimm wie eine Blutabnahme oder eine Impfung?}

Nein. Zur Blutabnahme werden viel größere Nadeln/Kanülen verwendet, die an einem empfindlicheren Ort eingestochen werden - und dazu wird dir dann auch noch Blut abgenommen, was unangenehm ist. Impfungen hingegen beinhalten Impfstoffe, die eine Immunreaktionen auslöst, weil das die Aufgabe von Impfstoffen ist. HRT-Injektionen bringen ein bisschen Hormone in einem Trägeröl in deinen Körper ein, wodurch du dich wohler fühlst, weil du dann die richtigen Hormone in dir hast. Ich hoffe du siehst den Unterschied. Es kann auch einfacher sein, sich selbst eine Injektion zu verabreichen, als wenn eine fremde Person dich sticht, je nachdem wie du drauf bist.

\subsection{Gibt es barrierefreies Zubehör für Injektionen?}

Ja. Sogenannte Auto-Injektoren können zum Beispiel bei Problemen mit Feinmotorik sinnvoll sein. Siehe bitte Frage \ref{5-21}, oder lese einfach weiter.

\subsection{Aber ich \textit{bin} anders und kann nichts injizieren weil ich Glasknochen habe und meine Haut aus Papier ist und \textemdash{}!?}

Ich verstehe die Angst, aber wenn du wirklich unter keinen Umständen eine Injektionen machen willst (oder eine legitime Kontraindikation wie Hämophilie hast), dann mach es nicht. Das kannst du einfach sagen. Es ist okay. Wenn du deine Meinung änderst wird diese Anleitung immer noch da sein. Und wenn nicht, dann nicht. 

 

\section{ESTERTYPEN UND DOSIERUNGEN}\label{td}

\subsection*{Wichtige Begriffe}
\addcontentsline{toc}{subsection}{\textemdash{} Wichtige Begriffe}

\subsection{Was sind die verschiedenen “Arten” von injizierbarem Östrogen?}

Die vier wichtigsten Arten, die für HRT benutzt werden, sind \textit{Estradiol Valerate} (EV), \textit{Estradiol Cipionate} (EC), \textit{Estradiol Enantate} (EEn), und \textit{Estradiol Undecylate} (EUn). Diese "Arten" bezeichnet man als \textit{Estradiol}-Ester und sie werden im Körper zu \textit{Estradiol} (E2)  umgewandelt. 

Es ist wichtig anzumerken, dass in manchen Ländern Pillen mit dem Namen \textit{Estradiol Valerate} verkauft werden, was verwirrend ist, aber das ist etwas anderes, wenn wir von EV sprechen geht es um den Ester für Injektionen.

(A.d.Ü. Um das Finden von weiterführenden Informationen zu erleichtern verwenden wir die englische Schreibweise, also z.B. Estradiol Enantate anstatt Estradiolenantat. Wir sind uns dessen bewusst - allerdings denken wir auch hier das der praktischen Wert darin besteht zu finden wonach man sucht - und nicht darin der deutschen Sprache gerecht zu werden. Möge Schopenhauer im Grab rotieren.)

\subsection{Was sind die Unterschiede zwischen den verschiedenen injizierbaren Estern?}

Der wichtige Unterschied zwischen Estern ist, dass jeder eine andere Halbwertszeit hat und die resultierende Hormonspiegelkurve unterschiedlich ist, was wiederum die Dosierung und Häufigkeit beeinflusst.

\subsection{Ist ein bestimmter Ester “besser” als die anderen?}

\textbf{Nein.} Die Unterschiede beziehen sich auf die Dosis und notwendige Häufigkeit der Injektion. Das ist jedoch nur ein qualitativer Unterschied in der Anwendung, und man kann sich da den eigenen Wünschen entsprechend etwas aussuchen. Alle vier Ester haben eine gleichwertige feminisierende Wirkung und die vorher erwähnten Vorteile von Injektionen gegenüber andere Methoden. 

\subsection{Welche Art von injizierbarem Östrogen sollte ich wählen, wenn ich es mir aussuchen kann?}

Wenn du die Wahl hast, empfehlen die meisten \textit{Estradiol Enantate}, da es sehr stabile Werte ermöglicht. Es muss jedoch angemerkt werden, dass es in den meisten Ländern nur bei DIY zur Verfügung steht (siehe Sektion \ref{sv} “VIALS BEZIEHEN”). Über die ärztliche Versorgung kann es sein, dass dir\textit{Estradiol Cipionate} angeboten wird (A.d.Ü. In Deutschland soweit mir bekannt nicht. Wir wollen hier trotzdem den originalen US Text in der Übersetzung erhalten), aber dann meistens in niedrigkonzentrierter Form, was zu vergleichsweise großen Injektionsmengen führt und somit etwas unangenehmer macht. Die am häufigsten verschriebene Art (insbesondere in den USA (A.d.Ü. Und einzige verschriebene in Deutschland)), \textit{Estradiol Valerate}, kann richtig gute Ergebnisse ermöglichen, hat aber ihre Macken (insbesondere bei DIY), weshalb sie nicht die erste Wahl sein sollte. Lies weiter.

\subsection{Was bedeutet “Konzentration”?}

Östrogen-Vials beinhalten Östrogen, das in Öl aufgelöst ist. Die \textit{Konzentration} einer Ampulle sagt wie viel Östrogen in dieser Öllösung insgesamt enthalten ist. Dies wird als ein Verhältnis von Gewicht und Volumen für das Vial ausgewiesen. Einfach gesagt: Für jeden Milliliter Öl (Volumen), gibt es so und so viel Milligramm Östrogen (Gewicht). \textbf{Bitte denk dran, dass die Konzentration allein nicht die Dosierung ist.} Oft wird die Konzentration für das Gesamtvolumen des Vials angegeben (z.B. 500mg / 10ml), es ist aber fast immer besser, dieses Verhältnis vereinfacht auszudrücken (also in diesem Fall 50 mg/ml). \textbf{Typische Konzentrationen sind 5 mg/ml, 10 mg/ml, 20 mg/ml, 40 mg/ml, und 50 mg/ml.}

\subsection{Was ist mit “Dosis und Häufigkeit der Anwendung” gemeint?}

\textit{Dosierung} und \textit{Häufigkeit} sind die zwei Faktoren, die deinen Hormonzyklus bestimmen. Die \textit{Dosierung} ist die Menge an Östrogen, die du zu dir nimmst (in mg), und die \textit{Häufigkeit} ist wie oft du es zu dir nimmst (gemessen in Tagen oder Wochen). Manchmal wird für die Kombination aus beiden Faktoren das Wort "regimen" benutzt, das dann beschreibt, wie oft und wieviel HRT-bezogenen Sachen du zu dir nimmst. (A.d.Ü. 
 Das ist am ehesten als "Behandlungsplan" zu übersetzen - was aber selten innerhalb der Community als Vokabel verwendet wird.)

\subsection{Wie kann ich meine Dosis errechnen?}

Deine Dosierung ist die Konzentration deines Vial mal das Volumen, das du injizierst. \[Konzentration (mg/ml) * Injektionsvolumen (ml) = Dosierung (mg)\] \textbf{Bitte verstehe, dass das Volumen allein keine Aussage zur Dosierung macht.} Es ist wie beim Backen: Du kannst nicht einfach sagen "45 Minuten im Ofen backen", ohne auch die Temperatur zu nennen. 

\subsection{Gibt es Beispiel-Dosierungen?}

Die Berechnung ist ganz einfach, versprochen! Hier ist eine kleine Referenztabelle die Konzentration und Volumen für einige gängige Dosierungen vergleicht. Es wird immer nur auf zwei Kommastellen aufgerundet, denn du wirst keine Spritzen verwenden, die z.B. genau 0.153ml messen \textit{könnten. }Das liegt innerhalb der Rundungsfehler-Toleranz und macht im Fall einer HRT keinen relevanten Unterschied. Du kannst auch \href{https://hrtcafe.net/Calc/}{diese Webseite zum berechnen} als Referenz nehmen.

\begin{table}
	\centering
	\caption{Beispiel-Dosierungen für gängige Konzentrationen nach Volumen}
	\label{tab:concentrations}
	\begin{tabular}{@{}llllll@{}}
		\toprule
		\multicolumn{1}{c}{} & \multicolumn{5}{c}{Konzentration (mg/ml)} \\
		\cmidrule(rl){2-6}
		& 5    & 10  & 20 & 40 & 50   \\
		\cmidrule(rl){2-6}
		Dosierung (mg) & \multicolumn{5}{c}{Volumen (mL)}  \\
		\cmidrule(r){1-1} \cmidrule(lr){2-6}
		4        & 0.8  & 0.4 & 0.20  & 0.10 & 0.08\\
		5        & 1.0    & 0.5 & 0.25 & 0.13 &  0.10\\
		6        & 1.2  & 0.6   & 0.30  & 0.15 & 0.12   \\
		7        & 1.4  & 0.7 & 0.35  & 0.18 & 0.14\\
		8        & 1.6  & 0.8   & 0.40  & 0.20 & 0.16  \\
		9        & 1.8  & 0.9 & 0.45  & 0.23 & 0.18\\
		10       & 2.0    & 1.0   & 0.50  & 0.25 & 0.20 \\
		\bottomrule
	\end{tabular}
\end{table}

\textbf{Wie diese Tabelle gelesen wird:} Links findest du deine gewünschte Dosierung, rechts das entsprechende Volumen und in der oberen Zeile die jeweilige Konzentration. Du wirst merken, dass das benötigte Volumen für eine vernünftige Dosierung bei einer Konzentration von 5 mg/ml schwierig ist. Das liegt daran, dass Vial mit einer Konzentration von 5 mg/ml nicht sonderlich praktisch sind.

\subsection{Kann man Dosierungen zwischen den verschiedenen Estern umrechnen?}

\textbf{Nein, kannst du nicht.} Da die Ester sich jeweils anders verhalten gibt es keine “Konvertierung” im direkten Sinne. Falls du zu einem anderen Ester wechselst, solltest du einfach eine typische Dosierung für dieses Ester nutzen und schauen, wie es sich bei dir auswirkt. Es ist möglich, Hormonwertkurven und Dosierungen zwischen Estern zu vergleichen, aber es gibt keine Methode für eine direkte Konvertierung.

\subsection{Wie kann ich Hormonwertkurven und Dosierungen zwischen Estern vergleichen?}

Falls du etwas nerdy bist, kann ich \href{http://estrannai.se}{estrannai.se} sehr empfehlen. Solche Simulatoren zu benutzen ist nicht verpflichtend, aber es ist ein gutes Werkzeug für grobe Vergleiche. \href{https://estrannai.se/\#i0__cu,7,7,1-cu,5,7,3-cu,5,7,2}{Hier ist ein beispielhefter Vergleich zwischen typischen wöchentlichen Dosierungen} die wir uns jetzt einzeln anschauen werden.

\textbf{Beachte, dass bei von-bis Werten die niedrigen aufgeführten Dosierungen meist ausreichend sind.} Fang mit einer geringen Menge an und nimm nur mehr, wenn du es brauchst um auf die Zielwerte zu kommen. Mehr ist nicht zwangsläufig besser, aber darüber reden wir später. Die Herkunft/Hersteller deines Vials hat keinen Einfluss auf diese Dosierungsempfehlungen.

\subsection*{Die unterschiedlichen Ester im Detail}
\addcontentsline{toc}{subsection}{\textemdash{} Die unterschiedlichen Ester im Detail}

\subsection{Wie dosiere ich \textit{Estradiol Valerate}?}

Für gute Werte mit \textit{Estradiol Valerate} empfehle ich entweder eine niedrige Dosis zwei mal pro Woche, oder eine höhere Dosis einmal die Woche. Es ist eine Frage der Bequemlichkeit und Verträglichkeit. Die typische Faustregel ist etwa 1 mg pro Tag in einem Zyklus von 3 bis 7 Tagen. \textbf{Ich empfehle eine wöchentliche Dosis von 6-8mg}, aber 4-5mg in  5 Tagen ist auch üblich. \textbf{Bei EV sollte immer mindestens wöchentlich injiziert werden (also nie länger als 7 Tagen zwischen Injektionen).} Ein wöchentlicher Rythmus ist schon an der Grenze dessen, was das Ester leisten kann. Mehr Zeit zwischen Injektionen wird ausdrücklich nicht empfohlen, aufgrund der Varianz-bedingten Nebenwirkungen (Siehe Frage \ref{7-3}).

Bemerkenswerterweise werden in manchen Regionen Pillen mit dem Namen \textit{Estradiol Valerate} verkauft, was verwirrend ist, aber hier geht es nur um die Form für Injektionen.

\subsection{Wie sieht die Hormonwertkurve für \textit{Estradiol Valerate} aus?}

\textit{Estradiol Valerate} ist ziemlich vertrackt. Es geht schnell hoch, mit einem Gipfelwert wenige Tage nach der Injektion, geht aber genauso schnell wieder runter. Diese relative Instabilität kann (A.d.Ü. unter anderem emotional) anstrengend sein, dies lässt sich aber durch die richtige Einstellung von Dosierung und Frequenz teilweise abfedern.

\begin{figure}[H]
	\centering
	\includegraphics[width=1\linewidth]{ev.png}
	\caption{Serum-Östradiol (pg/ml) von Estradiol Valerate im Vergleich zur Zeit (Tage))}
	\label{fig:ev}
\end{figure}

\subsection{Wie dosiere ich \textit{Estradiol Cypionate}?}

\textit{Estradiolcypionat} kann problemlos wöchentlich angewendet werden. \textbf{Eine wöchentliche Dosierung von 5-7mg ist typisch.} Eine niedrigere Frequenz (z.B. alle 10 Tage) wird nicht empfohlen, da sie weniger Effizient ist und für gute Werte höhere Dosierungen verlangt. Jede Verlängerung über 7 Tage hinaus erhöht die Gefahr von Varianz-bedingten Nebenwirkungen (Siehe Frage \ref{7-3}).

\subsection{Wie sieht die Hormonwertkurve für \textit{Estradiol Cypionate} aus?}

\textit{Estradiol Cypionate} ist weniger extrem als \textit{Estradiol Valerate}. Die Kurve geht nicht so schnell hoch und runter, es kommt aber über eine Woche hinweg immer noch gewissen Schwankungen.

\begin{figure}[H]
	\centering
	\includegraphics[width=1\linewidth]{ec.png}
	\caption{Estradiol-Serumspiegel (pg / ml) von Estradiol Cipionate vs Zeit (Tage)}
	\label{fig:ec}
\end{figure}

\subsection{Wie dosiere ich \textit{Estradiol Enanthate}?}

\textit{Estradiol Enanthate} kann leicht wöchentlich eingesetzt werden und kann (notfalls) auch eine Frequenz von 10 Tagen ermöglichen. Ein noch längerer Abstand zwischen den Injektionen ist theoretisch möglich, jedoch aufgrund der entstehenden Varianz der Werte nicht zu empfehlen. \textbf{Eine wöchentliche Dosiserung von 4-6mg ist empfohlen}, bei 10 Tagen wird eher zu 5-7mg geraten. Eine Verlängerung auf 10 Tage hat jedoch keine Vorteile zu bieten, sodass die wochentliche Anwendung empfohlen wird. (A.d.Ü. Alle 7 Tage ist ein einfacher Rhythmus. Macht es euch nicht unnötig kompliziert.)

\subsection{Wie sieht die Hormonwertkurve für \textit{Estradiol Enanthate} aus?}

\textit{Estradiol Enanthate} ist bei injizierbarem Östrogen der Goldstandard. Die Kurve ist über die Dauer der typischen wöchentlichen Anwendung sehr flach (also: wenig Varianz). Dadurch werden sehr stabile Werte ermöglicht, was die Gefahr von Varianz-bedingten Nebenwirkungen sehr gering hält (Siehe Frage \ref{7-3}).

\begin{figure}[H]
	\centering
	\includegraphics[width=1\linewidth]{een.png}
	\caption{Estradiol-Serumspiegel (pg / ml) von Estradiolenanthat vs Zeit (Tage)}
	\label{fig:een}
\end{figure}

\subsection{Wie dosiere ich \textit{Estradiol Undecylate}?}

\textit{Estradiol Undecylate} kann die Frequenz potentiell weit über das wöchentliche bis ins monatliche strecken. Die dafür empfohlene Dosierung ist jedoch nicht standardisiert oder erforscht. Anders als bei den anderen Estern sind die Faktoren, welche die Aufnahme des Östrogens bedingen (\textit{“Pharmakokinetik”}) bei \textit{Estradiol Undecylate} sehr wichtig aber unerforscht. Dementsprechend ist die Anwendung als experimentell zu verstehen und wird in dieser Anleitung nicht besprochen oder empfohlen. Frag am besten mal im Esoterikladen nach dem Mondkalender, um dich bei jedem Vollmond zu injizieren. (A.d.Ü. Steht so im Original.)

\subsection{Wie sieht die Hormonwertkurve für \textit{Estradiol Undecylate} aus?}

Dazu kann keine Aussage getroffen werden. Es gibt zu wenige Datenpunkte und zu viele Variablen, um ein genaues Bild davon zu haben. Wenn es dich interessiert müsstest du leider an dir selbst experimentieren, sei dir jedoch der Risiken bewusst. Ich empfehle es nicht, wenn du nicht ohnehin weißt, was du tust.

(A.d.Ü. Wir möchten nochmal ganz deutlich sagen, dass wir keine\_r dazu raten, medizinische Experimente an sich selbst durchzuführen. EUn \textit{ist aber nun einmal in DIY Kreisen verfügbar und diskutiert} - die Erwähnung hier dient der Harm Reduction, der Hinweis auf den Mond(kalender) ist als Witz zu verstehen.)

\begin{figure}[H]
	\centering
	\includegraphics[width=1\linewidth]{moon.png}
	\caption{Der Mond}
	\label{fig:moon}
\end{figure}

 

\section{BLUTTESTS UND HORMONWERTE}

\subsection*{Werte bekommen}
\addcontentsline{toc}{subsection}{\textemdash{} Werte bekommen}

\subsection{Wie oft sollte ich meine Werte testen?}

In der Einstellphase musst du häufiger testen. Nach jeder Veränderung (also in der Dosierung oder Frequenz) solltest du ein bis zwei Monate warten, damit deine Werte sich stabilisieren, und dann testen.

\subsection{Muss ich meine Werte vor Beginn der HRT testen?}

Eigentlich nicht, da das Testosteron zu hoch und das Östrogen zu niedrig sein werden, was erwartbar ist, aber ein allgemeiner Blutwerte-Check-Up (Leber, Lipide etc.) kann deiner Gesundheit nicht schaden. Bei Verdacht einer Intergeschlechtigkeit, welche die HRT beeinflussen könnte, kann ein vorbereitender Bluttest für den Ausschluss empfehlenswert sein. (A.d.Ü. Es gibt einige Endos die Gentests machen wollen. Die sind langfristig hilfreich, verzögern aber den Beginn der Behandlung meist um 1-3 Monate.)

\subsection{Muss ich meine Werte testen, wenn ich über einen längeren Zeitraum nichts verändert habe?}

Nicht unbedingt - es ist nicht zu erwarten, dass dein Körper plötzlich ganz anders auf Estradiol reagiert. Es kann aber Sicherheit geben, wenn andere Aspekte deiner Routine sich verändert haben wie wenn du den Hersteller gewechselt hast oder Ärztys oder deine Versicherung das verlangen. Falls du hingegen mit \textit{Estradiol Undecylate} experimentierst solltest du mindestens vierteljährlich testen.

\subsection{Ich bin nicht versichert oder habe keinen Arzt, wie kriege ich ein Bluttest?}

Suche nach privaten Labors in deiner Region, je nachdem, ob es legal ist. In vielen Regionen können private Bluttests gekauft werden, diese sind aber oft nicht billig. Online-Angebote sind manchmal billiger, das ist aber regional unterschiedlich (A.d.Ü.: Im deutschsprachigen Raum gibt es viele verschiedene Anbieter für private Bluttests. Google einfach, was es in deiner Region/dein Preissegment gibt. \href{https://www.meindirektlabor.de/}{Mein Direktlabor} ist z.B. ein bekannter überregionaler Anbieter.)

\subsection{Ich bekomme keinen Bluttest/kann mir keinen leisten. Kann ich trotzdem HRT machen?}

Es ist immer besser, Gewissheit über die Hormonwerte zu haben, aber bei typischen Dosierungen ist eine HRT in der Regel sehr sicher. Entsprechend solltest und musst du dich dann mehr auf dich selbst und deine Beobachtungen  deiner selbst verlassen.

(A.d.Ü. Wir wollen niemandem dazu raten einen Blindflug zu machen. Das ganze ist generalisiert zu verstehen und auf absolute Notfälle bezogen.)

\subsection{Welche Werte soll ich testen lassen?}

Mindestens \textit{Estradiol} (E2) und \textit{Testosteron gesamt} (T) da diese Werte uns am meisten interessieren. \textit{Sexualhormon-bindendes-Globulin} (SHBG), \textit{Dihydrotestosteron} (DHT), \textit{Estrone} (E1), und \textit{Prolaktin }(PRL) zu testen kann auch sinnvoll sein, insbesondere wenn du Schwierigkeiten hast und bei der Klärung helfen. \textit{Follikel stimulierendes Hormon} (FSH) und \textit{luteinisierendes Hormon} (LH) können dir sagen, ob deine Hypothalamus-Hypophysen-Nebennierenrinden-Achse deaktiviert ist, was die Basis für Monotherapie bildet (Siehe Frage \ref{2-3}). Aber ich wiederhole: \textbf{\textit{Estradiol} (E2) und \textit{Testosteron gesamt} sind das Wichtigste. }

\subsection{Zu welchem Zeitpunkt in meinem Hormonzyklus sollte ich testen?}

Am Ende des Zyklus (\textit{ dem “Talwert”}). Du willst so nah wie möglich am tiefsten Wert messen, da diese Information Vergleichbarkeit bietet. Man könnte sogar meinen, dass dies die einzige relevante Information ist, da gleichmäßige Talwerte das Wichtigste sind. Zum Beipiel: Wenn du immer Donnerstag Nachmittag injizierst, solltest dein Bluttest am Vormittag oder frühen Nachmittag des nächsten Donnerstags machen. (A.d.Ü. So wird sichergestellt, dass du nie in bedenkliche E2 / T Werte rutschst die deiner HRT und Gesundheit abträglich sind.)

\subsection{Mein Arzt sagt, ich soll den Höchstwert / den Mittelwert testen. Soll ich?}

\textbf{Nein.} Der Höchstwert sagt gar nichts aus und zeigt nur, welchen Ester du benutzt. Wohlwollend interpretiert deutet solch eine Vorgehensweise auf Inkompetenz hin, die auf veralteten, konservativen Versorgungsstandards beruht. Etwas weniger wohlwollend interpretiert ist es böswillig, für Östrogenwerte zu sorgen, die zu schlechten Ergebnissen oder gar gesundheitlichen Schäden führen können. \textbf{Ich empfehle trotzdem den Talwert zu messen.}

\subsection*{Bluttest auswerten}
\addcontentsline{toc}{subsection}{\textemdash{} Bluttest auswerten}

\subsection{Welche Östrogenwerte will ich erreichen?}

In der Transition ist dies vielleicht die umstrittenste Frage. Die ausführliche Antwort ist: Hoch genug, dass du dich wohl fühlst und dein Testosteron so weit unterdrückt wird, wie du es brauchst. Hohe Werte sind im besten Fall verschwenderisch, im schlimmsten kontraproduktiv. Das ist jedoch ein breites Spektrum und durch die vielen Variablen gibt es viele individuelle Schwankungen. Einfach gesagt: \textit{Du willst so viel Östrogen, dass \textbf{du dich wohl} fühlst.}

\subsection{Führen höhere Östrogen-Werte zu schnelleren oder besseren Ergebnissen?}

\textbf{Nein.} Östrogen-Werte, die über das Notwendige hinausgehen, werden von manchen aus persönlichen Gründen bevorzugt, aber sie verbessern die Feminisierung nicht. Tatsächlich können zu hohe Werte zu unerwünschten Nebenwirkungen führen, wie zum Beispiel Stimmungsschwankungen. \textbf{Für die Feminisierung ist es viel wichtiger, das Testosteron zu einem ausreichend niedrigen Wert zu unterdrücken und genug Estradiol im Serum zu haben.}

\subsection{Okay, aber welche Zahl will ich bei Östrogen auf meinem Befund sehen?}

Mit dem Verständnis, dass die genaue Zahl keine Rolle spielt, dass die Zahl immer etwas höher sein wird als das, was beim Talwert gemessen wird, und dass die Zahl in einem Spektrum von Möglichkeiten liegt, die auf einer Reihe von Faktoren basieren, \textbf{empfehle ich einen Talwert von mindestens 200 pg/ml (730 pmol/L).} Diese Empfehlung ist etwas konservativ, da die Unterdrückung der HHN-Achse schon früher erfolgt, aber sie bietet Spielraum. Für die meisten funktioniert es in diesem Bereich ganz gut, aber manche mögen es etwas tiefer oder höher. Ich denke nicht, dass diese Zahl voll im Fokus stehen sollte, da sie immer variabel ist und das wichtigste dein Gefühl ist, \textbf{aber} Werte von \textbf{über 300pg/ml (1100 pmol/L) im Talwert ist höher als es sein muss oder sollte.} Es gibt Ausnahmen, aber du bist wahrscheinlich nicht die Ausnahme. Mach aber einfach, was sich für Dich gut anfühlt. Darüber hinaus siehe Frage \ref{11-1}.

\subsection{Was für einen Testosteron-Wert will ich?}

Die Unterdrückung des Testosterons ist die Bedingung für eine Feminisierung, also reicht meistens ein Wert der geringer 50 ng/dL (1.7 nmol/L) aber über Null ist. \textbf{Ein Testosteronwert von (nahe) Null ist nicht gut.} Siehe Sektion \ref{T} “TESTOSTERON”.

\subsection{Ich habe von Natur aus hohes/niedriges T. Muss ich meine Dosierung anpassen?}

Wahrscheinlich nicht. Die Testosteronwerte, die typischerweise vor der HRT bestehen, sind für die Feminisierung sowieso höher als erwünscht (Siehe Frage \ref{2-3}). Die Ausnahme wäre irgendeine Form von Intergeschlechtigkeit, welche ein Grund für eine genauere Justierung der Dosierung sein könnte. Dies führt jedoch über die Grenzen dieser Anleitung hinaus. Du musst die Empfehlungen wahrscheinlich nicht anpassen, aber vielleicht fühlst du dich besser, wenn du es tust. Letztendlich sollst du das machen, was sich gut anfühlt. Siehe Frage \ref{9-2}.

\subsection{Ich hatte eine Geschlechtsangleichende Operation (GaOP). Sollen meine Östrogen-Werte anders sein?}

Da die Unterdrückung von Testosteron für dich kein Problem mehr ist, könntest du dich wahrscheinlich mit niedrigeren Östrogen-Werten gut fühlen, \textbf{aber du brauchst immer noch Östrogen.} Da du jetzt keine eigenen Sexualhormone mehr produzierst ist es extrem wichtig, dass du deinen Körper mit ausreichend Hormonen versorgst. Keine oder zu wenige Hormone führen zu Wechseljahresbeschwerden, wie sie ältere cis Frauen erleben können. Passe es so an, wie es dir passt.

Genauer gesagt: \textbf{Ein Minimum von ca 100 pg/ml (350 pmol/L) ist notwendig, um Probleme mit der Knochenmineraldichte und anderes zu vermeiden.} Wenn deine Feminisierung großenteils schon erfolgt ist, ist dein Hormonprofil in vieler Hinsicht mit dem einer menopausalen cis Frau vergleichbar und relevante Studien können hilfreich sein. (Siehe Frage \ref{11-29}). Ein zusätzliche Anwendung von Testosteron kann in Fällen von Antriebslosigkeit oder Müdigkeit sinnvoll sein (Siehe Frage \ref{9-2}).

\subsection{Kann ein Bluttest ungenau sein?}

Je nachdem, wie das Blutserum gemessen wird (\textit{Also je nach Assay}), können Nahrungsergänzungsmittel mit Biotin den \textit{Estradiol} (E2) Wert (und andere, aber uns geht es \textit{Estradiol}) unerwartet hoch erscheinen lassen. Es ist nicht immer möglich zu wissen, welche Methode zur Auswertung angewandt wird und so ist es einfacher, kein Biotin für einige Tage vor dem Test zu nehmen. Es ist auch immer möglich, dass bei der Entnahme Blutprobe für den Test ein Fehler vorkommt, das ist aber sehr unwahrscheinlich.

(A.d.Ü. Da viele in Deutschland Gel einsetzen, könnte das häufiger vorkommen - stell sicher, dass deine Blutabnahmestelle nicht aus versehen mit Gel kontaminiert ist.)

\subsection{Ist durch einen Bluttest erkennbar welchen Ester oder Verabreichungsform ich nutze?}\label{4-16}

Nein. Durch ein Bluttest kann nicht erkannt werden, welche Östrogen-HRT eine Person nutzt. Die verschiedenen HRT Medikamente werden alle im Körper zu \textit{Estradiol (E2)} umgewandelt, genau wie wir es wollen. Man kann nicht unterscheiden ob jemand Pillen, Pflaster, Gel, Sprays oder Injektionen verwendet. Am Ende des Tages ist es alles Estradiol.

 

\section{TECHNIK UND MATERIALIEN} \label{ts}

\subsection*{Injektionsstellen \& Sicherheit}
\addcontentsline{toc}{subsection}{\textemdash{} Injektionsstellen \& Sicherheit}

\subsection{Wie führe ich eine Injektion sicher durch?}

Ich empfehle diese beiden Videos:

\begin{enumerate}
  \item \href{https://www.youtube.com/watch?v=cBabaGC2Dok}{\textit{“How to perform an intramuscular (IM) self-injection”}(A.d.Ü. Originaltitel beibehalten!)}
  \item \href{https://www.youtube.com/watch?v=YfNlAZLxLyw}{\textit{“Painless (for me so far) IM Injection Technique”}(A.d.Ü. Hier auch)}
\end{enumerate}

(A.d.Ü.: Die deutschsprachigen Videos, die ich zum Thema finden konnte, sind alle recht lang und für Pflegepersonal gedacht und somit für unsere Zwecke mit überwältigenden Infos versehen. Im wesentlichen können die hier im original verlinkten Videos auch ohne Englischkenntnisse verfolgt werden; die wichtigen Punkte werden weiter unten besprochen. Du kannst dir die deutschen Videos auf YouTube natürlich gern anschauen.) 

Mithilfe dieser zwei Videos solltest du ausreichend vorbereitet sein, um eine schmerzlose Injektion bei dir vorzunehmen. Ich empfehle dir, sie dir aufmerksam anzuschauen und bei Bedarf zu wiederholen. \textbf{Um Schmerzen zu verhindern ist ein Punkt besonders wichtig: Den abgeschrägten Teil der Nadelspitze (die Fase) beim einstechen nach oben zu halten).} Anders gesagt: Die Nadel hat eine klar definierte Spitze, und diese soll deine Haut als erste berühren. Du willst schön gerade einstechen. Du kannst dir gern vorstellen, wie deine Hand bzw. dein Handgelenk sich dabei bewegen sollen, wenn es dir hilft, aber am Ende ist es Übungssache, und wird irgendwann intuitiv.

\textbf{Merke: Injizieren ist eine Fertigkeit, die gelernt werden will!} Du wirst mit der Zeit besser. Keine Sorge: Bekommste schon hin.

\subsection{Muss ich genau so wie in den Videos injizieren?}

Nun, Variationen sind erlaubt. Da es am Ende bloß darum geht, dich sicher selbst zu pieksen, gibt es viele Wege, wie man es machen kann. Finde den Weg, der für dich am besten ist. Ein schnelles, direktes Einstechen funktioniert für viele gut, aber wenn du es langsamer angehen willst ist das auch okay, solange du dich damit wohlfühlst und bei einer Herangehensweise bleibst (um so mit der Zeit sicherer zu werden).

\subsection{Wie kann ich die Angst vor dem Injizieren überwinden?}

Ich empfehle daraus ein Ritual zu machen. Wenn du dir eine Routine aufbaust, wird es irgendwann ganz natürlich. Wenn du dich mit Musik, Gesprächen, Fernsehen oder was immer für dich gut ist ablenken kannst und dann irgendwann dein Muskelgedächtnis übernimmt, toll! Finde heraus, was zu dir hilft. Es kann z.B. helfen, wenn eine befreundete Person oder ein Partner die ersten Injektionen übernimmt, denn für die meisten ist die erste die aufregendste. Die häufigste Reaktion danach ist sowieso ein “Wie, das war's?” Es ist nie so schlimm, wie du es erwartest. Versprochen.

\subsection{Ist es wichtig, wo am Körper ich die Injektion mache?}

Ja und nein. Es ist wichtig, in sicheren Körperbereichen zu injizieren, aber sonst hängt es davon ab, wie gelenkig du bist, was für eine Menge du injizierst, was für eine Spritze du benutzst, und dein eigenes Gefühl. Es ist aber wichtig, \textbf{Injektionsstellen abzuwechseln.} Zum Beispiel kannst du jede Woche auf eine andere Körperseite injizieren, z.B. mal ins linke Bein, mal ins rechte Bein. So wird das Risiko einer langfristigen Narbenbildung verringert.

\subsection{Welche Körperbereiche sind sicher?}

Darüber streiten sich die Gelehrten, aber es kommt vor allem auf deinen Körperbau an. Ich persönlich empfehle die Beine, so wie es in den Video(s) zu sehen ist die ich oben verlinkt habe, da es für die meisten gut erreichbar ist und du das gut üben kannst, aber machen bevorzugen die Pobacke oder den Bauch. \href{https://vertisis.com/articles/how-to-self-administer-a-subcutaneous-injection}{Dieses Video auf dieser Webseite} zeigt andere Injektionsstellen, die je nach deinem Körperbau in Frage kommen können. (A.d.Ü.: Eine entsprechende Deutsche Version mit einer detaillierten Anleitung als PDF findest du \href{https://www.bk-trier.de/media-bkt/docs/PIZ_HZ_Subkutane_Injektion_2021.pdf}{hier}.) Finde raus, was für dich am besten ist.

\subsection{Was bedeuten “intramuskulär” (IM) und “subkutan” (SubQ/SC)?}

Im Kontext von Injektionen bedeutet \textit{Intramuskulär}, dass in das Muskelgewebe injiziert wird, \textit{subkutan} dass in das Unterhautfettgewebe injiziert wird.

\subsection{Was ist der Unterschied zwischen intramuskulären (IM) und subkutanen (SubQ/SC) Injektionen?}

\textbf{Im Kontext von HRT gibt es keine wesentliche Unterschiede zwischen subkutanen und intramuskulären Injektionen.} Subkutane Injektionen werden zwar etwas langsamer aufgenommen als intramuskuläre, der Unterschied ist aber generell nicht groß genug um für die Dosierung relevant zu sein. Darüber hinaus wird bei jeder Injektion nicht immer eindeutig \textit{nur } das Unterhautfettgewebe oder \textit{nur } das Muskelgewebe getroffen, sodass der Unterschied in der Praxis weiter verschwimmt.

Kleine Randbemerkung: Pharmazeutische Hersteller von Estradiol-Vials geben in der Regel an, dass diese nur für intramuskuläre Injektionen bestimmt sind. Das liegt daran, dass diese nur für diesen Zweck offiziell zugelassen sind, aber es gibt da keinen praktischen Unterschied. Die tatsächlichen Inhalte der Vials sind identisch und darauf kommt es an. Lies einfach weiter.

\subsection{Soll ich intramuskuläre (IM) oder subkutane Injektionen (SubQ/SC) durchführen?}

\textbf{Das ist die falsche Frage.} \textbf{Eine Injektion ist eine Injektion.} Subkutane Injektionen werden oft empfohlen, da viele davon aus ausgehen, dass sie weniger schmerzhaft sind, es gibt aber keinen wesentlichen Unterschied in der Durchführung. \textbf{Die Vorteile, von denen ausgegangen wird, haben weniger mit der Art der Injektion zu tun als mit Faktoren, die den möglichen Injektionsschmerz beeinflussen.} Die bessere Frage wäre "wie kann ich Injektionsschmerzen minimieren?", aber zuerst zwei weitere Fragen. (A.d.Ü. Ebenfalls wichtig ist die Frage: Womit fühlst du dich wohler? Bei SubQ sind die Nadeln kürzer weshalb ich mich für SubQ entschieden habe.)

\subsection{Spielt mein Injektionswinkel und/oder meine bevorzugte Injektionsmethode eine Rolle?}

Nein. Ich wiederhole: Der wichtigste Aspekt einer Injektion ist, dass du eine Nadel in deinem Körper reinsteckst und eine Flüssigkeit reinspritzt. Solange die Flüssigkeit nicht wieder rauskommt (oder zumindest nicht viel davon) und es nicht weh tut (oder zumindest nicht doll) hast du einen fantastischen Job gemacht. \textbf{Ich kann nicht genug betonen, dass die "Wahl" zwischen subkutaner und intramuskulärer sehr egal ist und für die Wirksamkeit von injizierbaren Estradiol keine Rolle spielt.} Einzig \textit{Estradiol Undecylate} ist der Fall, in dem die Art der Injektion möglicherweise einen Unterschied macht, aber die Details sind noch unklar. Der Punkt ist: Es gibt sinnvollere Dinge um die du dich Sorgen kannst.

\subsection{Muss ich beim Injizieren auf die Aspiration achten?}

Nein. “Aspiration” meint das kurzzeitige Zurückziehen des Spritzenstempels beim Injizieren um festzustellen, ob versehentlich ein Blutgefäß getroffen wurde. Über die allgemeine Notwendigkeit wird gestritten, aber im Kontext von Hormoninjektionen ist die Gefahr ein Blutgefäß innerhalb der empfohlenen Injektionsareale mit kurzen Nadelspitzen zu treffen ohnehin verschwindend gering, und gibt es keine wesentliche Vorteile. Bei der Injektion ins Gewebe wird die Aspiration von den meisten medizinischen Fachstellen auch nicht mehr empfohlen.

\subsection{Wie kann ich den Schmerz beim Injizieren minimieren?}

Du kannst üben und deine Technik verbessern, aber darüber hinaus ist deine Spritze- und Nadelkombination der wichtigste Faktor. \textbf{Um Beschwerden zu minimieren, sollte die niedrigst mögliche Nadelstärke ("Gauge", siehe nächste Frage) verwendet werden, die zum Trägeröl in deinem Vial passt, zusammen mit einer passenden Spritze und Nadellänge. }Die richtige Frage ist also "Was ist die geringste Nadelstärke und -länge die ich nutzen kann?". Um das herauszufinden, lass uns darüber reden wie Nadeln bzw. Kanülen funktionieren.

\subsection*{Nadelkunde}
\addcontentsline{toc}{subsection}{\textemdash{} Nadelkunde}

\subsection{Was bedeutet "Gauge" bei Nadelgrößen?}

\textit{Gauge }ist die Einheit für die Stärke bzw. Dicke der Nadel. Wenig intuitiv ist: Je höher die Zahl, desto dünner die Nadel. (A.d.Ü. Wir nutzen hier kurzzeitig das Wort "Kanüle" anstatt der umgangssprachlichen "Nadel".) Eine 25G-Kanüle ist dünner als eine 20G-Kanüle. Dünnere Kanülen sind meist kürzer, da sie sich leichter verbiegen können. Es ist nicht überraschend, dass dünnere Kanülen weniger Chancen haben weh zu tun. Die Stärke der Kanüle hat wohl bemerkt keinen Einfluss auf die HRT selbst, es geht nur darum, wie angenehm (oder unangenehm) die Injektion wird.

\subsection{Was ist der Unterschied zwischen Luer Lock- beziehungsweise Insulin-Spritzen?}\label{5-13}

\textit{Luer lock-Spritzen} ermöglichen es, die Nadel zwischen dem Aufziehen und der Injektion zu wechseln. \textit{Insulinspritzen} habe eine feste Nadel, sodass sie sowohl beim Aufziehen als auch beim Injizieren verwendet wird. Wenn möglich werden Insulinspritzen bevorzugt, da sie einfacher sind und einen sehr geringen Hohlraum/Deadspace haben (Siehe Frage \ref{5-26}) “Luer slip" Nadel werden nicht empfohlen, da sie fehleranfällig sind.

\textbf{Sicherheitshinweis: Das Wiederaufsetzen der Schutzkappe auf Nadeln wird generell nicht empfohlen, da die Gefahr besteht, dich zu stechen. Solltest du es dennoch tun (z. B. beim Auswechseln einer Ziehnadel), drücke NIE mit deiner Hand in Richtung der Nadel.} Die Kappe könnte brechen und du könntest dich verletzen, wenn du sie nicht richtig aufsetzt. 

Es wird empfohlen, die Kappe sanft auf einer ebenen Fläche mit der Nadel "aufzufangen”. Du kannst sie z.B. gegen eine Wand drücken oder, falls wirklich notwendig, die Kappe mit den Fingern an der Seite halten. Bei Injektionen am eigenen Körper gibt es zumindest keine Gefahr einer Krankheitsübertragung, sodass diese Warnung in diesem Fall nicht so streng genommen werden muss, aber bei Injektionen an anderen Personen ist dies SEHR wichtig. Zur Entsorgung der Spritzen siehe Frage \ref{5-27}.

\subsection{Welche Nadel-/Kanülenstärke sollte ich beim Aufziehen verwenden?}

Bei Luer Lock-Spritzen empfiehlt sich, eine niedrigere Gauge (= eine stärkere Nadel) beim Aufziehen zu verwenden als beim Injizieren. Allerdings kann eine zu niedrige Gauge zum Ausstanzen/Beschädigung, also dem sogenannten “coring” (A.d.Ü. Wir nutzen “coring” in der Übersetzung, wollten es aber hier wenigstens definieren) deines Stopfens führen (Siehe Frage \ref{5-23}), sodass mindestens 21-23G empfohlen wird. Wenn du geduldig bist und kein großes Injektionsvolumen hast muss wird eine hohe Gauge empfohlen, um die Gefahr des corings zu reduzieren. 

Nein, die Nadel wird durch das Einstechen in den Gummistopper nicht stumpf. Diese Frage ist bei Insulinspritzen sowieso nicht relevant, da die Nadel in diesem Fall nicht austauschbar ist.

\subsection{Welche Nadel-/Kanülenlänge sollte ich beim Aufziehen verwenden?}

Bei Luer Lock-Spritzen ist die Nadellänge beim Aufziehen nicht so wichtig, zu lange Nadeln können jedoch unpraktisch sein. Anders gesagt gibt es keinen Grund wählerisch zu sein. Diese Frage ist bei Insulinspritzen nicht relevant, da die Nadel in diesem Fall ohnehin nicht austauschbar ist.

\subsection{Mit welcher Nadelstärke sollte ich injizeren?}\label{5-16}

Diese Frage ist nicht ganz einfach und eher subjektiv, die Antwort hängt im Wesentlichen von 4 Faktoren: 1) das Trägeröl, was du injizierst; 2) ob das Vial ein zusätzliches Lösemittel beinhaltet; 3) ob du die Geduld hast, die Nadel länger in deinem Körper zu haben; and 4) deine Bereitschaft/Fähigkeit, den Spritzenstempel stärker herunterzudrücken. Es ist eine Frage deines Komforts. Dickflüssigere Öle können mit hohen Gauge länger brauchen und mehr Druck benötigen, aber dafür haben sie weniger Risiko schmerzen zu verursachen. \textbf{In der Regel ist 25G das Minimum, um keine Schmerzen zu verursachen. }Die meisten Öle gehen bis 27G einfach rein, während MCT-Öl bemerkenswerterweise bis 30G gut geht (Siehe Frage \ref{6-16}).

\subsection{Mit welcher Nadellänge sollte ich injizieren?}

\textbf{Ich empfehle zwischen 12.5mm und 25mm, je nach Gauge.} Unter 12.5mm erhöht wird Ausfluss wahrscheinlicher. 6.5mm kann funktionieren, je nach deiner Technik und dem Öl, den du injizierst, aber 12.5mm ist die sichere Wahl. Alles über 25mm ist unnötig lang, beängstigend und schmerzhaft, ohne irgendeinen Mehrwert zu bieten. (A.d.Ü. Auch 8mm funktionieren sehr gut und sind relativ gebräuchlich in Deutschland.)

\subsection{Ist die Spritzengröße wichtig?}

\textbf{Ja, die Größe ist wichtig.} Dafür gibt es zwei Gründen. 1) Größere Spritzen mit mehr Volumen sind gröber skaliert; und 2) größere Spritzen mit mehr Volumen sind rein physikalisch schwieriger zu benutzen. Um genau zu dosieren willst du eine Spritze benutzen, die nicht viel größer ist als das Volumen was du injizierst (z.B. sollten für Injektionen von weniger als 0.1ml Spritzen benutzt werden, die kleiner sind als 1ml). \textbf{Vermeide 3ml-Spritzen gänzlich, soweit du kannst.} Du kannst sie natürlich benutzen, wenn du keine anderen Spritzen hast, aber warum diese am häufigsten in Apotheken ausgegeben werden, erschließt sich mir nicht. Vielleicht ein schlechter Scherz. (A.d.Ü. Das bezieht sich auf die Ausgabe im US amerikanischen Raum. Im deutschsprachigen Regionen sollte es kein Problem sein entsprechende Spritzen zu bekommen. Nadeln und Spritzen sind frei verkäuflich und in einer breiten Auswahl verfügbar. Beispielsweise nutzt eine der Übersetzerinnen eine 0,3ml 30G Insulinspritze mit einer Nadellänge von 8mm.)

\subsection{Wo kaufe ich Spritzen und Nadeln/Kanülen?}

Es hängt von deinem Wohnort ab, da der Verkauf von Nadeln und Spritzen mancherorts als Strafe gegen drogenabhängige Personen eingeschränkt ist. Sonst sind medizinische und Veterinär-Versorgungsgeschäfte gute Quellen, oder direkt bei den Herstellern. \textbf{Amazon wird nicht empfohlen} da die Qualität dort oft unsicher ist. (A.d.Ü.: Siehe oben. Im deutschsprachigen Raum geht jede Apotheke deiner Wahl, auch online.)

\subsection{Ist es okay, Nadeln oder Spritzen wiederzuverwenden?}

\textbf{Nein. Benutze Nadeln und Spritzen nur einmal. }Und teile sie auch nicht mit anderen. In aller Deutlichkeit: Alle Spritzen, auch die mit austauschbaren Nadeln, sind nur für den einmaligen Gebrauch gedacht und dürfen unter keinen Umständen mehrfach verwendet werden. Das weißt du wahrscheinlich schon, aber es ist wichtig, dass zu wiederholen.

\subsection{Was wenn ich Injektionen machen will, es aber schwierig finde, mich selbst zu injizieren?}\label{5-21}

Du könntest ein Autoinjektor probieren. Wie es im Name steht führt der Autoinjektor die Injektion für dich durch. Ein Autoinjektor wie der \href{https://unionmedico.com/90-super-grip/}{\textit{UnionMedico 45/90 Super Grip}} (A.d.Ü.: auf Deutsch hier \href{https://www.b12-injektion.de/}) kann 1ml-Spritzen aufnehmen und das Injizieren vereinfachen (aber du musst immer noch selbst draufdrücken), während Autonijektoren wie der \href{https://www.owenmumford.com/us/medical-devices/autoject-2}{\textit{Owen Mumford Autoject 2}} (A.d.Ü.: auf Deutsch hier \href{https://shop.owen-mumford.de/Sonstiges/Autoinjektor-Autoject-2-mit-austauschbarer-Nadel.html}) die Nadel einer Insulinspritze ganz verstecken und von selbst herunterdrücken. Es gibt auch verschiedene 3D-gedruckte Modelle, die online verfügbar sind. Ich habe keine dieser Produkte getestet und dies ist keine Empfehlung.

\subsection*{Vials: Eine kleine Ampullenkunde}
\addcontentsline{toc}{subsection}{\textemdash{} Vials: Eine kleine Ampullenkunde}

\subsection{Worauf sollte ich bei einem Vial achten?}

(A.d.Ü. Ja, "Vial" ist kein deutsches Wort - aber eben gängig. Du liest das hier ja auch nicht auf einem Klugtelefon.) Abgesehen von einem klaren Coring (siehe unten) solltest du auf Anzeichen wie dunkle (Orange bis Braun) Verfärbung beim Öl, Separation der Lösung, Kristallisierung, Brüche im Glas, Staub, Haare oder andere Kontaminierungen innerhalb des Vials achten. Ein gut hergestelltes Vial sollte sich visuell von anderen aus der gleichen Herstellung nicht unterscheiden lassen. \textbf{Untersuche dein Vial immer gewissenhaft, bevor du es benutzst. Injiziere nicht aus einem Vial, das bzw dessen Inhalt nicht gut aussieht.}

\subsection{Was ist mit Coring gemeint?}\label{5-23}

Jedes Vial hat ein Gummistopfen, der oben auf der Ampulle sitzt und die Lösung schützt. Das \textit{Coring} (deutsch: Ausstanzen) passiert, wenn ein Stück vom Gummi herausgeschnitten wird/herausbricht und in die Lösung gelangt. Das kann passieren wenn eine zu große Nadel beim aufziehen verwendet wird, wenn immer wieder an der gleichen Stelle eingestochen wird oder einfach zu häufig in die Vial gestochen wird (z.B. wenn du immer sehr geringen Mengen aus einem recht großen Vial entnehmen würdest). \textbf{Ein Vial mit ausgestanzten Core sollte weggeworfen werden. }Mit der \href{https://www.youtube.com/watch?v=w5F0SLoMjC8}{\textit{45-90° Technik}} kann diese Gefahr verringert werden. (A.d.Ü.: Dies betrifft nur Luer Lock-Spritzen. Insulinspritzen sind meist zu kurz dafür haben aber auch eine geringe Stärke, so das 90° Einstiche weniger bedenklich sind.)

Einfach gesagt: Du willst keine Gummistückchen in deinen Körper injizieren und wenn du größere Gummistücke in der Lösung siehst, könnte es auch kleinere, unsichtbare Stücke geben. Darüber hinaus soll der Gummistopfen die Lösung vor Luft und Bakterien schützen; wenn es einen Loch gibt erhöht sich die Gefahr einer Kontaminierung und Oxidierung. \textbf{Nebenbei: Bitte entferne den mittleren Teil der Metallabdeckung bzw. Plastikkappen oben am Vial, bevor du es benutzt. }Das mag selbstverständlich erscheinen, aber bei manchen Vials kann das verwirrend sein. 

\subsection{Wie lang ist eine Vial haltbar?}\label{5-24}

Ein geschlossenes Vial könnte sich jahrelang halten, wenn es bei stabilen Temperaturen und in Dunkelheit gelagert wird. Bei der Haltbarkeit geht es vor allem um die Gefahr der Oxidierung oder Verlust der Sterilität. Ein angefangenes Vial, das einen Konservierungsstoff beinhaltet (siehe Frage \ref{6-17}), sollte mindestens ein Jahr halten, oder wie immer lang es braucht bis es aufgebraucht ist. Oft steht auf Vial "28 Tage haltbar", das ist aber nur das Minimum, was von Herstellern verlangt wird, nicht die tatsächliche maximale Haltbarkeit. (A.d.Ü. Dies ist eigentlich nur der Fall bei pharmazeutisch hergestellten Vials. Diese haben in der Regel auch weniger Konservierungsstoffe (BA), siehe unten, und sollten ggf. kürzer verwendet werden.)

\subsection{Wie lagere ich eine Vial richtig?}\label{5-25}

Bei stabiler Raumtemperatur und dunkel. Hitze und Sonnenlicht können dem Trägeröl schaden, und Kälte kann dazu führen, dass das Estradiol auskristallisiert. Kristalle lösen sich wieder auf, wenn die Lösung erwärmt wird, aber wenn dies nicht vollständig passiert kann es zu Irritationen beim Injizieren führen. Das trifft sowohl auf angefangene und neue Vials zu.

Außerdem solltest du immer die Flüssigkeit im Vial, das du entnimmst, durch die gleiche Menge Luft ersetzen, damit sich mit der Zeit kein Vakuum im Vial bildet, was das Entnehmen deutlich erschweren kann. Dazu füllst du deine Spritze teilweise mit Luft (bis zur Markierung der Menge an Injektionsflüssigkeit die, die du entnehmen möchtest), spritzt diese Luft in das Vial und entnimmst dann die Flüssigkeit wie gewohnt. Dieser einfache Schritt erspart dir später viel Ärger.

\subsection{Was genau meint man mit Hohlraum/Deadspace?}\label{5-26}

Der \textit{Hohlraum/Deadspace} ist die Menge an Flüssigkeit, die bei einer Injektion verschwendet wird. Diese Flüssigkeit bleibt baubedingt in der Nadel bzw. in der Spritze zurück. Bei einer Luer Lock-Spritze kann dies bis zu 1mL betragen, bei einer Insulinspritze kann ist es oft viel weniger bis zu nur 0.003mL. Es lohnt sich eine Spritze mit möglichst wenig deadspace zu verwenden, da am Ende ganz schön viel Östrogen dadurch verloren geht. \href{https://hrtcafe.net/Calc/}{Dieser Rechner} kann helfen, die verschwendete Menge je nach Spritzenart einzuschätzen.

Wenn du die Nadel/Kanüle zwischen dem Aufziehen und dem Injizieren wechselst, dann solltest du den Spritzenstempel leicht zurück ziehen bevor du die Aufziehnadel abnimmst, damit die darin enthaltene Flüssigkeit nicht verlorengeht. Es ist zwar nicht viel, aber es kann einen Unterschied machen. \textbf{Beachte auch, dass die Markierungen auf der Spritze den Hohlraum schon berücksichtigen, du also bei deiner Injektion nicht manuell mehr hinzufügen solltest.} Siehe Frage \ref{7-7} für eine andere mögliche Herangehensweise wenn die Verschwendung durch den dead space ein Problem ist.

\subsection{Was mache ich mit meinen benutzten Nadeln und Spritzen?}\label{5-27}

Werfe sie alle (vorsichtig, mit der Spitze nach unten) in eine Entsorgungsbox (entweder eine richtige medizinische Entsorgungsbox für medizinischen Sondermüll oder einen stich- und bruchfesten Abfallbehälter, z.B. eine Blechdose vom Kaffee). Wenn der Behälter dreiviertel-voll ist, mach ihn fest zu, sodass es sich nicht von selbst öffnen kann. Schreibe "BENUTZTE SPRITZEN" drauf und entsorge es nach den bei dir geltenden Richtlinien. \textbf{Du darfst es NICHT in den Restmüll tun.} (A.d.Ü In Deutschland ist dies je nach regionalem Entsorgungsunternehmen unterschiedlich. Am besten ist es, den Behälter in der Apotheke oder bei irgendeinem Arztbesuch abzugeben, damit es fachgerecht und sicher entsorgt werden kann oder lokal nachzufragen.)

\section{VIALS BEZIEHEN}\label{sv}

\subsection{Wo bekomme ich Estradiol zum Injizieren?}

Generell gibt es zwei Möglichkeiten: \textit{pharmazeutische Quellen} und \textit{DIY-Quellen}. Für \textit{pharmazeutische Quellen} brauchst du meistens ein Rezept vom Ärzty, da HRT in den meisten Ländern nicht rezeptfrei ist (oder zumindest nicht in injizierbarer Form). \textit{DIY-Quellen} umfassen alles andere. 

(A.d.Ü. Während es ein paar Apotheken in Deutschland gibt, die Injektionen herstellen, gibt es dafür leider eine Unmenge an Voraussetzungen: du benötigst eine 1) Indikation 2) einen Ärzty der gewillt ist dir Injektionen zu verschreiben und 3) Geld. Während es möglich sein sollte, die Injektionen auf Kassenrezept zu bekommen, stellen sich erfahrungsgemäß die meisten Ärztys quer. Pharmazeutische Bezugsquellen brauchen spezifische Rezepte - diese regulierten Quellen finden sich in der \href{https://transdb.de/search?type=pharmacy&offers=eInjection}{trans\*DB}. Wenn dein Ärzty auf einem Privatrezept besteht können diese Injektionen sehr teuer sein (~180€/Quartal).)

\subsection{Sollte ich pharmazeutische oder DIY Quellen bevorzugen?}

Theoretisch ist das deine Entscheidung, aber manchmal gibt es gar keine Entscheidungsmöglichkeiten. Es gibt bei jeder Variante Vor- und Nachteile. Es hält dich natürlich nichts davon ab, Östrogen aus verschiedenen Quellen zu beziehen, um die Vorteile beider Varianten zu genießen. In vielen Situationen kann das empfehlenswert sein.

\subsection*{Pharmazeutische Quellen}
\addcontentsline{toc}{subsection}{\textemdash{} Pharmazeutische Quellen}

\subsection{Was sind die Vorteile von pharmazeutischen Quellen?}

\begin{itemize}
  \item Vertrauen in der Qualitätskontrolle und Zertifizierung;
  \item Die Krankenversicherung kann die Kosten gänzlich oder teilweise übernehmen;
  \item Kann praktischer sein (A.d.Ü. je nachdem wie viel Glück du mit deinen Ärztys hast);
  \item Das Produkt wird sehr wahrscheinlich von verlässlicher Qualität sein;
  \item \textbf{Für die Übernahme von Operationen durch die Krankenkasse kann zumindest der Anschein von einer Benutzung von pharmazeutischen Quellen notwendig sein.} (A.d.Ü. Wird in Deutschland nicht benötigt)
\end{itemize}

\subsection{Was sind die Nachteile von pharmazeutischen Quellen?}

\begin{itemize}
  \item Kleiner (oder gar keine) Auswahl zwischen Estern; (A.d.Ü. In Deutschland NUR Estradiol Valerate)
  \item Möglicherweise lange Wartezeit (Monate oder Jahre); 
  \item Nur auf Rezept (je nach Land);
  \item Die Krankenkasse übernimmt es vielleicht nicht;
  \item Wird möglicherweise in deinem Land gar nicht verschrieben;
  \item Dein Ärzty kann sich einfach weigern, es dir zu verschreiben;
  \item Dein Ärzty kann sich einfach weigern, dir eine neues Rezept zu geben;
  \item Lieferengpässe könnten dein Rezept nutzlos machen;
  \item Du wirst wahrscheinlich unter den stringenten WPATH-Richtlinien behandelt, oder was schlimmeres;
  \item Schwierig, ein Vorrat anzulegen;
  \item Dein Zugang zu HRT hängt stark von der politischen Stimmung deines Landes ab und deine Trans-Diagnose wird sehr wahrscheinlich in der medizinischen Akte vermerkt. 
\end{itemize}

\subsection*{DIY-Quellen}
\addcontentsline{toc}{subsection}{\textemdash{} DIY-Quellen}

\subsection{Was sind die Vorteile von DIY-Quellen?}

\begin{itemize}
\item In den meisten Regionen viel billiger;
\item Überall auf der Welt verfügbar;
\item Der Zugang kann viel schneller sein (Wartezeit nur für Produktion und Versand);
\item Leicht, ein Vorrat anzulegen;
\item Volle Auswahl an Estern;
\item Keine Notwendigkeit, dich mit dem Gesundheitssystem herumzuschlagen; (A.d.Ü sollte eine cis Person hier noch mitlesen: Ja, die Formulierung ist bewusst gewählt)
\item Es ist wahrscheinlich mit Liebe gemacht.
\end{itemize}

\subsection{Was sind die Nachteile von DIY-Quellen?}

\begin{itemize}
  \item Sehr wahrscheinlich nicht in einem zertifizierten Reinraum hergestellt;
  \item Produktionsqualität kann je nach Quelle variieren;
  \item Je nach Quelle kann der Bezug umständlich sein;
  \item Setzt Vertrauen in die Quelle voraus;
  \item Es muss eine Quelle gefunden werden;
  \item DIY-Quellen können, nun, einfach verschwinden;
  \item Versandzeiten können variieren, insbesondere wenn du im Ausland bestellst;
  \item Anbieter müssen mit Kryptowährungen arbeiten, was nervig ist;
  \item Wir nicht von der Krankenkasse übernommen.
\end{itemize}

Darüber hinaus wird, wie bereits erwähnt, für die meisten OPs eine gewisse Zeit mit einer "offiziellen" Hormonbehandlung vorausgesetzt. Je nach deinen Plänen kann das relevant sein. (A.d.Ü. Irrelevant für Deutschland)

\subsection{Welche injizierbaren Estradiol Ester gibt es nur bei DIY-Quellen?}

Vor allem \textit{Estradiol Enanthate} und \textit{Estradiol Undecylate}. Bei pharmazeutischen Quellen gibt es fast immer \textit{Estradiol Valerate}, in einer Konzentration von <40 mg/ml. \textit{Estradiol Cypionate} wird manchmal (A.d.Ü. nicht in Deutschland) verschrieben, aber selten mit Konzentrationen über 5 mg/ml oder 10 mg/ml, was die Dosierung schwierig macht. Allein die Vorteile von \textit{Estradiol Enanthate} sind ein guter Grund, DIY in Betracht zu ziehen. Es ist möglich die meisten Ester mit einer Konzentration von 40mg/ml oder mehr aus DIY-Quellen zu beziehen. \textit{Estradiol Undecylate} ist ebenfalls nur als DIY möglich, aber wie bereits erwähnt würde ich es nur experimentierfreudigen Menschen empfehlen.

\subsection{Was für Menschen stehen hinter DIY-Quellen \textit{wirklich}?}

Kommerzielle Hersteller, solidarische Projekte, deine Freunde, und vielleicht sogar du, wenn du einen unternehmerischen Geist hast!

\subsection{Wo kann ich DIY-Vials bekommen?}

Bist du ein Cop oder was? Darum geht es in dieser Anleitung nicht. Dafür gibt es andere Quellen. Bleib fokussiert.

\subsection{Wie kann es sein, dass DIY-Quellen billiger sind als pharmazeutische Quellen?}

Der Herstellungspreis für ein Vial ist ca. \$10, inklusive der Personalkosten und den amortisierten Investitionskosten. Und das ist vermutlich sogar noch eine Überschätzung. Für die kommerziellen DIY-Quellen entstehen die meisten Kosten durch die benötigte Anonymität und den Versand. Nicht-kommerzielle DIY-Quellen habe keine solchen Kosten. Pharmazeutische Quellen haben meist keine Motivation, ihre Preise zu senken. 

\subsection{Ist DIY legal?}\label{6-11}

In den meisten Ländern, inklusive der USA, wird Östrogen nicht streng reguliert, während der Besitz von Testosteron-Präparaten mal strafbar sein kann, manchmal nicht. Die USA sind in der Hinsicht eine Ausnahme, da der Besitz von Testosteron Medikamenten dort kriminalisiert wird, die meisten anderen Länder den Besitz jedoch nicht kriminalisieren. Aufgrund der Verbreitung von Steroiden ist eine strafrechtliche Verfolgung aber selten. \textbf{Dieser Leitfaden stellt keine Rechtsberatung dar.} 
(A.d.Ü. Nach dem Verständnis der Übersetzerin bringt man sich in Deutschland rechtlich nicht in Gefahr, wenn man Östrogen zu DIY-Zwecken bestellt. Im schlimmsten Fall kann z.B. Östrogen aus dem Nicht-EU Ausland vom Zoll eingezogen werden. Endverbraucher\_innen werden (bisher) nicht verantwortlich gemacht. Der kommerzielle Vertrieb ist wahrscheinlich eine andere Frage, aber darum geht es hier nicht).

\subsection{Ist DIY sicher?}

“DIY” als Konzept ist weder sicher noch unsicher, aber nicht alle DIY-Quellen sind gleich. Wenn es darum geht, eine Substanz in deinen Körper zu injizieren, sind die wichtigen Fragen: Hast du genug Vertrauen in die Person, die das Vial produziert hat? Vertraust du, dass sie richtig und sauber gearbeitet hat, dass das Vial steril ist und nur das beinhaltet, was drauf steht? Bei den pharmazeutischen Quellen wird dieses Vertrauen aufgrund der geltenden Gesetzen und Regeln angenommen. Bei DIY-Quellen muss dieses Vertrauen erst aufgebaut werden, durch Information zum Herstellungsprozess, unabhängige Prüfungen und Feedback innerhalb der Community.

\subsection{Worauf sollte ich achten um rauszufinden, ob eine DIY-Quelle vertrauenswürdig ist?}

Verlasse dich auf dein Bauchgefühl und deinen Kopf. 

\begin{itemize}
  \item Reden sie offen über ihren Produktionsprozess oder machen Angaben dazu? (Wird zum Beispiel Staub gefiltert? Die Antwort sollte ja sein!!!)
  \item Machen sie einen kompetenten Eindruck?
  \item Haben sie ihr Produkt testen lassen? 
  \item Werden sie in der Community als vertrauenswürdig eingeschätzt?
  \item Werden sie von anderen, dir vertrauten Menschen aus der Community empfohlen? Gibt es Rezensionen oder Erfahrungsberichte? 
  \item Fehler können passieren, aber wie wird damit umgegangen? Wird offen kommuniziert oder werden negative Berichte unterdrückt?
  \item Werden bei kommerziellen Quellen Probleme mit Bestellungen aufgeklärt?
  \item Werden bei kommerziellen Quellen Bestellungen angenommen, obwohl das Produkt noch gar nicht vorliegt? (Du solltest niemals etwas bestellen, was noch nicht produziert wurde!)
  \item Beinhaltet das Vial ein Konservierungsmittel? (Sollte es!)
  \item Wie lange sind produzieren sie schon? (Diese Frage wird aus guten Gründen nicht immer beantwortet!)
  \item Wie viel produzieren sie? (Diese Frage wird aus guten Gründen nicht immer beantwortet!)
  \item Fühlt sich irgendwas einfach \textit{falsch} an?
\end{itemize}

Diese Beispielfragen können dir helfen herauszufinden, ob die Quelle vertrauenswürdig ist, und ob ihr die Qualität des Produkts so wichtig ist wie dir.

\subsection{Sollte ich bei verschiedenen DIY-Quellen verschiedene Standards anwenden?}

Wahrscheinlich ja. Kommerzielle Anbieter, die dein Geld nehmen, sollten hohe Standards erfüllen können, da sie es sich leisten können. Bei Produkten die solidarisch und kostenlos verteilt verteilt werden, kann man nicht ganz so pingelig sein, was aber nicht heißt, dass sie unbedingt besser oder schlechter sind. Bei Freunden oder bei deiner eigenen Produktion kannst nur du es einschätzen!

\subsection*{Aufbau eines Vials}
\addcontentsline{toc}{subsection}{\textemdash{} Aufbau eines Vials}

\subsection{Was sollte in einem Vial enthalten sein?}

Der Inhalt setzt sich aus einem \textit{“aktiven”} Stoff und einem \textit{“Träger”} oder Hilfsstoff zusammen. Der \textit{aktive} Stoff ist in unserem Fall das Östrogen-Ester, während der \textit{Träger} und weitere Hilfsstoffe den Rest ausmachen. Es gibt generell insgesamt drei oder vier Inhaltsstoffe im Vial: 1) Das Estradiol-Ester; 2) das Trägeröl; 3) ein Konservierungsmittel; und manchmal 4) weitere Lösemittel. Die Ester haben wir schon in der Sektion \ref{td} “ARTEN UND DOSIERUNGEN” besprochen. Vials aus pharmazeutischen Quellen beinhalten fast immer alle vier Inhaltsstoffe.

\subsection{Welches Trägeröl ist am besten?}\label{6-16}

Die Antwort zu dieser Frage ist subjektiv und hängt in erster Linie von der persönlichen Verträglichkeit und möglichen Allergien ab. \textbf{Der relevanteste Aspekt ist die Viskosität des Öls, da die Injektion dadurch einfacher oder schwieriger werden kann.} Dünnere, flüssigere Öle sind, wie bereits besprochen (Siehe Frage \ref{5-16}), für das Aufziehen und Injizieren mit dünneren Kanülen von Vorteil. \textbf{Global werden MCT-Öl und Rizinusöl für HRT-Injektionen am häufigsten genutzt. }Rizinusöl ist ziemlich dickflüssig, führt aber selten zu Irritationen und wird typischerweise von pharmazeutischen Herstellern benutzt. MCT-Öl ist dünnflüssiger aber hat ein höheres Irritationsrisiko - es kommt eher im DIY-Umfeld zum Einsatz. Baumwollsamenöl und Traubenkernöl werden ebenfalls manchmal eingesetzt. Andere Öle wie z.B. Sonnenblumenkernöl oder Sesamöl werden manchmal eingesetzt, sind aber nicht empfohlen. Je nach deinen Umständen kann diese Frage irrelevant oder sehr wichtig sein, oder hast einfach keine Wahl. 

(A.d.Ü. Einige pharmazeutische Hersteller in Deutschland nutzen Sesamöl, eine bekannte DIY Quelle bietet Traubenkernöl an.)

\subsection{Welches Konservierungsmittel sollten enthalten sein?}\label{6-17}

Für Injektionen wird niedrig konzentrierter \textit{Benzylalkohol} (BA) am häufigsten verwendet. Das muss sein und darf nicht fehlen. \textbf{Injiziere niemals etwas aus einem Vial ohne Konservierungsmittel. }Für Menschen mit der seltenen Allergie dagegen wird meist \textit{Chlorobutanol }als Alternative verwendet, was aber bei DIY-Quellen eher selten der Fall ist, da sie den Stoff dafür extra auftreiben müssten.

\subsection{Welche Lösungsmittel sind gängig?}

Am häufigsten wird \textit{Benzylbenzoat} (BB) verwendet, was die Lösung flüssiger macht. Das ist technisch gesehen optional, wird aber im Allgemeinen für eine gleichbleibende Chargenkonsistenz empfohlen und ist in vielen Fällen und je nach Trägeröl und gewünschter Konzentration notwendig. Manche Leute reagieren darauf mit Reizungen, andere nicht.

\section{HÄUFIGE FEHLER UND LÖSUNGEN}

\subsection*{Unsicherheit bei der Dosierung}
\addcontentsline{toc}{subsection}{\textemdash{} Unsicherheit bei der Dosierung}

\subsection{Meine Hormonwerte sind nicht so, wie ich es erwartet habe. Warum?}

Dafür kann es verschiedene Ursachen geben. Zuerst gilt es zu verstehen woher deine Erwartungen kommen. Modelle und Simulationen wie von estrannai.se sind nicht in der Lage alle möglichen persönlichen Faktoren einzubeziehen und haben deswegen immer eine gewisse Unschärfe. Auch muss bedacht werden, dass es mehrere Injektionen braucht, bis deine Werte stabil und somit aussagekräftig sind. Wenn du also kurz vor deinem Bluttest die Dosierung oder den Ester geändert hast, könnten die unerwarteten Ergebnisse daran liegen. Wenn du dir unsicher bist, ob du bei den Injektionen einen Fehler machst, frag am besten eine befreundete Person, ob sie dir beim Aufziehen zuschauen kann, um sicher zu sein, dass du tatsächlich so viel injizierst, wie du planst. Keine Sorge: Das kommt öfters vor als man denken würde. Bei DIY-Quellen könnte auch aufgrund von Unerfahrenheit oder ungenauem Equipment die Konzentration niedriger sein als erwartet. In diesem Fall könntest du bei diesem spezifischen Vial etwas mehr injizieren. (A.d.Ü. Wobei wir nicht dazu raten.) \textbf{Aber am wichtigsten bleibt immer wie du dich fühlst, und nicht irgendwelche Werte an sich!. }Beachte auch, dass sogar in einer Apotheke professionell hergestellte Injektionslösungen manchmal eine niedrigere oder abweichende Konzentration bieten können, auch wenn das (hoffentlich) sehr selten vorkommt!

\subsection{Kann ich Hormonwerte von Tests vergleichen wenn ich nicht den Talwert gemessen habe?}

\textbf{Nein. } Genau deshalb sollte immer der Talwert gemessen werden, am besten so kurz vor deiner nächsten Injektion wie möglich, bevorzugt ein paar Stunden davor. Die Daten werden viel nützlicher, wenn du so viele Variablen wie möglich ausschließt. Wenn du dir sonst nichts aus dieser Anleitung merkst, dann bitte das: Messe immer nur den Talwert. (A.d.Ü. Neben der Vergleichbarkeit ist der Talwert eben auch die einzige relevante Größe die ausschließt, dass du dich unter dosierst. Die Varianz in einem Zyklus ist eben durch die Ester bedingt; wir wollen mit dem Talwert nur überprüfen, dass die Testosteron-Produktion zuverlässig und dauerhaft unterdrückt ist.)

\subsection{An den Tagen um den Talwert herum fühle ich mich richtig schlecht. Was soll ich tun?}\label{7-3}

In den meisten Fällen ist entweder die Dosierung zu niedrig oder der Abstand zwischen den Injektionen zu groß. Das ist bei \textit{Estradiol Valerate} und \textit{Estradiol Cypionate} besonders relevant. Passe die Dosierung (innerhalb der empfohlenen Werte) oder deinen Rhythmus an. Lass dir Zeit und finde heraus, was sich für dich gut anfühlt. Insbesondere bei \textit{Estradiol Valerate} kann es sein, dass deine Dosierung tatsächlich zu \textbf{hoch} ist und nicht zu tief, da die großen Schwankungen innerhalb des Zyklus Stimmungstiefs auslösen können. Kurz gesagt: Wechsle zu \textit{Estradiol Enanthate}, wenn du kannst.

\subsection*{Injektionsprobleme}
\addcontentsline{toc}{subsection}{\textemdash{} Injektionsprobleme}

\subsection{Die Injektion schwieriger, wenn es kalt ist. Wie soll ich vorgehen?}

Erwärme das Vial vor dem Aufziehen der Injektionslösung und erwärme die Spritze vor dem Injizieren leicht, indem du es vorsichtig zwischen deinen Handflächen rollst oder einfach hältst. Es ist eine gute Angewohnheit das in die Injektionsroutine aufzunehmen.

\subsection{Die Injektion schmerzt wenn es kalt ist. Was soll ich tun?}

Wärme die Injektionsstelle vorher auf. Sowohl bei IM als auch SubQ Injektionen kannst du den Bereich massieren oder mit einer warmen Dusche entspannen. Wenn du beispielsweise in dein Bein injizierst, kannst du die Stelle gezielt mit einem heißen Wasserstrahl anwärmen.

\subsection{Ich habe nach der Injektion leicht geblutet. Werde ich sterben?}

Nein. Das bedeutet nur, dass du ein kleines Blutgefäß getroffen hast, das kann passieren. Vielleicht kriegst du einen blauen Fleck oder es tut später etwas weh. Wenn du ein süßes Dino-Pflaster drauf klebst, wird es schneller heilen.

\subsection{Da war eine kleine Luftblase in meiner Spritze. Werde ich sterben!?}\label{7-7}

Nein. Natürlich willst du nicht einfach nur Luft injizieren - und zu viel Luft in der Spritze würde deine Dosierung sehr ungenau machen - aber alles unter 0.1ml ist wahrscheinlich egal. (A.d.Ü. Katie macht hier einen kleinen Exkurs, kein wichtiges Wissen für die HRT:) Bei manchen Substanzen kann es sogar empfohlen sein etwas Luft zu injizieren. Zum Beispiel wird bei der \textit{"Air-Lock-Technik"} eine Standardtechnik zum Spritzen von Flüssigkeiten, die Reizungen verursachen oder Flecken hinterlassen können normalerweise 0,1–0,3 ml Luft gespritzt. Du musst dir also wirklich keine Sorgen machen. Luft ist bei intravenösen Injektionen relevanter, also nicht für deine HRT. 

(A.d.Ü. Laut \href{https://flexikon.doccheck.com/de/Gasembolie#PathogeneseURL}{doccheck} werden unter 100ml Gasansammlungen einfach resorbiert. Es gibt wirklich keinen Grund zur Besorgnis. :3)

\subsection{Hilfe! Ein Teil der Flüssigkeit ist aus der Injektionsstelle ausgelaufen. Habe ich meine Injektion verschwendet und/oder werde ich jetzt vielleicht doch sterben?}

Nein. Es kann aus vielen Gründen dazu kommen, dass ein Bisschen ausläuft, aber es ist nur selten genug um einen Unterschied bei der Dosis zu machen. Die Injektion musst du nicht wiederholen. Versuche in der Zukunft, die Nadel 5-10 Sekunden drin zu lassen und danach auf die Stelle zu drücken. Wenn du dir besonders viele Sorgen um das Auslaufen machst, kannst du versuchen, die vorher erwähnte Air-lock-technik oder \href{https://www.nurse.com/nursing-resources/definitions/what-is-z-track-method/}{die Z-Track-Methode} (A.d.Ü. Englischer Link) anzuwenden. 

\subsection{Manchmal tut es nach einer Injektion richtig weh. Werde ich sterben?}

Nein. Auch wenn du allen Anweisungen sorgfältig gefolgt bist, kann es trotzdem vorkommen, dass die Flüssigkeit sich an einer besonders unangenehmen Stelle sammelt. Das wird beim nächsten mal besser! \textbf{Pass darauf auf deine Injektionsstellen zu durchzuwechseln / zu rotieren!} Du willst nicht, dass sich durch das wiederholte injizieren an der gleichen Stelle eine Narbe bildet. Außerdem willst du nicht eine Injektionsstelle, die bereits weh tut, noch schmerzhafter machen.

\subsection{Ich hab nach der Injektion echt starken Juckreiz und Reizungen an der Stelle. Werde ich sterben?}

Wahrscheinlich nicht. Dafür kann es verschiedene Gründe geben. Am schlimmsten wäre eine Infektion, aber das ist in den meisten Fällen sehr unwahrscheinlich. \textbf{Geh sofort zu einem Ärzty wenn du Fieber oder starke Schmerzen hast. Auch Muskelschmerzen, eitriger Ausfluss, sowie sich ausbreitende Rötungen oder andere Zeichen einer Infektion erfordern ärztliche Hilfe. } Meistens sind aber Sachen wie Jucken, Rötungen, leichte Schwellungen oder ein Wärmegefühl das Ergebnis von Vials, bei denen sich das Estradiol und das Öl voneinander gelöst haben (also das Östrogen kristallisiert ist). Siehe die nächste Frage. Es ist auch möglich, dass du eine allergische Reaktion auf das Trägeröl (oder einen anderen Inhaltsstoff) erlebst, aber wenn dies plötzlich auftaucht und die vorigen Injektionen problemlos waren, wird es eher an einer Separation der Lösung liegen.

\subsection{In meinem Vial sind kleine Kristalle sichtbar. Kann ich es trotzdem benutzen?}

Das liegt sehr wahrscheinlich daran, dass das Vial zu kalt geworden ist. Wärme es in deiner Hand leicht auf und schüttel es, um die Lösung wieder zu mischen. Wenn die Kristalle nicht verschwinden, kann es sein, dass die Lösung sich gänzlich getrennt hat. Mit viel mehr Wärme und Durchmischung könnten die Kristalle vielleicht wieder verschwinden, aber, wenn es möglich ist, solltest du einfach auf ein neues Vial ausweichen.

 

\section{PROGESTERON}

\subsection{Will ich Progesteron nehmen?}

\textbf{Wahrscheinlich.} Aus was für einem Grund auch immer ist diese Frage kontrovers. (A.d.Ü. Die Gründe sind hauptsächlich historischer Natur und auf eine alte WPATH Studie zurück zu führen, die jedoch kein bioidentisches Progesteron untersucht hat; siehe unten) Gegner (vor allem Ärztys) argumentieren, dass eine feminisierende Wirkung durch keine Studien belegt wird und es dementsprechend nicht genommen werden sollte. Abgesehen davon, dass transfeminine Themen chronisch untererforscht sind, ist Progesteron heuristisch gesehen ein zentrales weibliches Sexualhormon, das viele wichtige Funktionen im Körper und dem Gehirn erfüllt. Ungeachtet der äußerlichen Feminisierung ist es ein wichtiges Hormon, das sowohl für körperliche und mentale Gesundheit sorgt und nicht leichtfertig übergangen werden sollte.

\subsection{Was ist der Unterschied zwischen “Progesteron” und “Progestine” / ”Progestagene”?}

Die Hormone, welche auf die Progesteron-Rezeptoren wirken, heißen “Gestagene” und können sowohl bioidentisch/natürlich als auch synthetisch sein. Das wichtigste, natürliche und biodentische Hormon darunter ist “Proges\textbf{teron}”. Synthetische Gestagene werden manchmal als “Proges\textbf{tine}” bezeichnet. Die Bezeichnung sind sich alle recht ähnlich und werden oft untereinander verwechselt, obwohl sie \textbf{nicht }gleichwertig sind. (A.d.Ü. Die deutschen Namen weichen in diesem Kontext relativ stark von den englischen ab, das Problem bleibt aber ein Stück weit bestehen.)

\subsection{Will ich Progesteron oder ein Progestin/Gestagen?}

Progesteron. Du willst \textbf{NUR bioidentisches Progesteron}.

\subsection{Was ist das Problem mit Progestine/Gestagene?}

Progestine, darunter typischerweise \textit{Medroxyprogesteron}, \textit{Medroxyprogesteronacetat }, oder \textit{Levonorgestrel}, haben oft extrem schädliche Nebenwirkungen (Brustkrebs, Blutgerinnsel, Depressionen, etc.), die fälschlicherweise Progesteron zugeschrieben wurden. Sie sind nicht bioidentisch und verhalten sich deshalb nicht wie Progesteron und können nicht damit verglichen werden.

\subsection{Wie trägt Progesteron zur Feminisierung bei?}

Es wird angenommen, dass Progesteron vor allem eine wichtige Rolle beim Brustwachstum und bei der Libido spielt, aber es ist wie gesagt auch davon abgesehen ein wichtiges Hormon. Es ist auch ein Antigonadotropin (das heißt, es trägt zur Unterdrückung von Testosteron bei), was manchmal relevant sein kann.

\subsection{Ist es wichtig, wann ich mit Progesteron anfange?}

Das wissen wir nicht genau. Es wird manchmal behauptet, dass eine frühe Einnahme das langfristige Brustwachstum stören kann, aber das ist rein theoretisch und dazu gibt es nur anekdotische Berichte. Somit ist die Antwort unklar. Eine konservative Empfehlung wäre ungefähr ein Jahr nach Anfang der HRT (also bis Tanner Stadium 3 oder 4) zu warten. Also für den Fall, dass es einen Unterschied macht.

\subsection{Wie wird Progesteron normalerweise eingenommen?}

Abgesehen von örtlichen Anwendungen wird es meist als Pille verabreicht. Es wird als Pille verschrieben, ist aber als Zäpfchen effektiver. Sprays und Salben zur örtlichen Anwendung funktionieren auch gut.

\subsection{Meinst du es ernst, dass das Progesteron rektal eingenommen werden soll!?}

Progesteron wird rektal ganz anders im Körper verarbeitet als wenn du es schluckst. Oral geht es erstmal durch die Leber und wird primär zu \textit{Allopregnanolon} verarbeitet, das zu starker Müdigkeit führen kann, während rektales eingenommenes Progesteron als solches aufgenommen wird - also genau das was wir wollen (ein wenig davon wird trotzdem zu anderen Stoffen verstoffwechselt). Manche Menschen nehmen extra orales Progesteron ein, um besser schlafen zu können, aber zu viel \textit{Allopregnanolon }kann manchmal auch zu negativen psychischen Nebenwirkungen führen.

\subsection{Wie nehme ich Progesteron rektal ein?}

Etwas Wasser auf der Pille sollte reichen, dann trocknen und Hände Waschen (A.d.Ü. Generell macht vorher Waschen auch sinn und mit Gleitgel ist es viel angenehmer.). Du solltest dann innerhalb der nächsten Stunde oder so nicht auf Klo gehen, daher empfiehlt sich die Einnahme direkt vor dem Schlafen gehen. Falls du Probleme damit hast, dass die Kapsel sich nicht auflöst, kannst du probieren, sie vorher anzustechen, aber das kommt sehr selten vor. Bei großen, hausgemachten Zäpchen mit Kokosöl solltest du dir bewusst sein, dass das Öl nicht in dir bleiben will (A.d.Ü. und zu einem späteren Zeitpunkt ausgeschieden wird).

\subsection{Wie viel Progesteron sollte ich nehmen?}

Bei Pillen ist eine tägliche 100-200mg vor dem Schlafengehen normal. Das ist eine etwas arbiträre Dosis; 200mg ist das Maximum, das von Ärztys verschrieben wird. Manche Menschen nehmen mehr als 200mg, aber eine kurzfristige Erhöhung der Werte kann zu einem unangenehmen Crash führen, siehe die nächste Frage.

Bei transdermaler Anwendung weiß es niemand so genau, da diese Verabreichungsform sehr variabel ist und es keine Leitlinien zu den gewünschten Werten oder zur Frequenz gibt (wobei es aufgrund der Halbwertszeit täglich sein muss), da Progesteron in der trans-HRT nicht genug untersucht ist. Deshalb würde ich empfehlen, deine Dosis langsam anzupassen um zu beobachten, wie Progesteron bei dir wirkt.

\subsection{Bringt es Vorteile, Progesteron "zyklisch" einzunehmen?}\label{8-11}

Nein. Manche Menschen tun es, um den Zyklus einer cis Frau zu imitieren, aber es gibt keine Gründen dafür. Es kommt sogar zu PMS-ähnlichen Symptomen. Die einzige Ausnahme wäre beim Verdacht einer Intergeschlechtigkeit, ansonsten empfehle ich es nicht. Siehe Frage \ref{11-10}.

\subsection{Wie lange soll ich Progesteron einnehmen?}

So lange wie du Östrogen einnimmst und so lange wie du willst. Also wahrscheinlich für immer.

Manchmal sagen Leute (oder Ärztys), man sollte Progesteron nur für X Jahren einnehmen. Es gibt null theoretische oder empirische Gründe für diese arbiträre Empfehlung. Es macht genauso viel Sinn wie wenn eine cis Person (oder, spezifischer: ein Ärtzy) eine trans Person fragen würde, "wie lange sie denn vorhat, HRT zu nehmen???" \textemdash{} oh, mist warte - das fragen die manchmal echt!

\subsection{Kann Progesteron in \textit{Dihydrotestosteron} (DHT) umgewandelt werden?}

Nein. Naja, ganz genau gesehen schon, aber auch wieder nicht. Es handelt sich dabei größtenteils um einen Mythos \href{https://whsah.co/posts/rethinking-progesterone-and-androgens/}{wie von alix in diesem Artikel ausführlich dargelegt}, jedoch kann es bei Menschen mit \textit{nicht-klassischem adrenogenitalen Syndrom} zu negativen Nebenwirkungen durch erhöhte Androgenaktivität aufgrund der Einnahme von Progesteron kommen. In solchen Fällen sollte Progesteron nicht eingenommen und eine passende Diagnose und Therapie für Nebennierenerkrankungen gesucht werden.

\subsection{Gibt es neben der Einnahme von Tabletten noch weitere Vorteile bei der topischen Anwendung von Progesteron?}

Vielleicht. Es kann eine Alternative zu Pillen sein, insbesondere wenn eine Erdnuss-Allergie besteht (die meisten Pillen beinhalten Erdnussöl), aber auch hier ist die Dosierung unklar. Manche Menschen finden mehr Progesteron besser. Pass auf dich auf und hab Spaß. (A.d.Ü. Die gängigen Progesteron Weichkapsel in Deutschland (siehe unten)) enthalten \textit{kein Erdnussöl}.)

Zur Klarheit: Salben z.B. können auf den Schenkelinnenseiten (oder wo immer angewiesen) aufgetragen werden, oder optional skrotal (dort ist die Haut dünn und besonders gut durchblutet). Dies kann insbesondere bei Sprays helfen. Und nein, das Progesteron direkt auf den Brüsten aufzutragen wird sie nicht größer oder schneller wachsen lassen. 

\subsection{Kann ich Progesteronpulver schnupfen?}

Bitte nicht. Es ist ziemlich schlecht für die Nebenhöhlen. Es ist nicht schwer ein Spray selbst herzustellen dazu gibt es Anleitungen. Mach das lieber. Das ist viel effektiver, konsistenter und sicherer.

\subsection{Wo bekomme ich Progesteron?}

Progesteron ist bei DIY-Quellen oft teurer, da im Vergleich zu Estradiol größere Mengen des Hormons benötigt werden, daher solltest du es idealerweise von pharmazeutischen Quellen über die Krankenversicherung beziehen. Es gibt auch sogenannte "graue" Apotheken aus dem Ausland, aber dort zu bestellen ist oft schwieriger. Salben zur örtlichen Anwendung sind in manchen Ländern auch rezeptfrei erhältlich, aber je nach Konzentration machen sie vielleicht wirtschaftlich keinen Sinn.  (A.d.Ü. Ärztliche Verordnung braucht eine Indikation; gängige auch rektal verwendbare Marken sind Utrogest, Famentia und Progestan.)

\subsection{Ich würde gern mehr zu Progesteron im Kontext von HRT lesen. Welche Quellen bieten sich an?}\label{8-17}

Ursprünglich war hier ein Dokument verlinkt, das ich aber entfernt habe, da es teilweise irreführend war. Das Problem mit Progesteron ist, dass sich niemand auch nur über einen einzigen Aspekt einig zu sein scheint. Ich kenne keine einzige Quelle, die von allen als gut akzeptiert wird. Ich sag's dir: es können sich nicht mal alle darauf einigen, dass es mit "P" beginnt. \textbf{Das wichtigste ist, dass Progesteron für die volle Feminisierung oder für Brustwachstum nicht zwingend notwendig ist. Wenn es keine Kontraindikationen gibt, lohnt es sich wahrscheinlich trotzdem es zu nehmen.}

Es ist anzumerken, dass es im Zusammenhang mit Gestagenen unzählige Mythen und teilweise Lügen gibt, die von Befürwortern wie Kritikern gleichermaßen erfunden wurden. Dies erschwert es zusätzlich die Wahrheit aus der ohnehin lückenhaften Forschung herauszufiltern. Fantastische Behauptungen über magische Vorteile treffen auf Panikmache über angebliche aber haltlose Risiken und sind gleichermaßen kontraproduktiv. Wobei Letzteres meiner Meinung nach noch schlimmer ist, wenn es von einer medizinischen Autorität stammt \textemdash sei es aus Nachlässigkeit oder böswilliger Absicht.

\subsection{Interagiert Progesteron mit anderen HRT-bezogenen Medikamenten?}\label{8-18}

Wenn du 5$\alpha$-Reduktasehemmer wie \textit{Finasterid} und \textit{Dutasterid} einnimmst (Siehe Sektion \ref{AA} “ANTIANDROGENE”, oder lies weiter) können diese die Verarbeitung von Progesteron zu \textit{Allopregnanolon} beeinflussen, was wiederum bei Menschen zu depressiven Verstimmungen führen kann, egal wie sie das Progesteron einnehmen. Es ist nicht ganz klar, inwieweit die Art der Einnahme der 5$\alpha$-Reduktasehemmer (örtlich oder oral) eine Rolle spielt, aber eine niedrigere systemische Absorption durch örtliche Anwendung \textit{könnte }diese negative Nebenwirkungen reduzieren. Es wird empfohlen, die Hemmer nicht einzunehmen, wenn diese Nebenwirkungen bei dir auftreten, aber sie treten nicht in allen Fällen auf. Merke, dass diese depressive Verstimmungen bis zu einem Monat nach dem Absetzen anhalten können. 

\section{TESTOSTERON}\label{T}

\subsection{Sollten trans Frauen \textbf{gar kein} Testosteron im Serum haben?}

Testosteron ist ein lebenswichtiges Hormon, das eine Schlüsselrolle in deiner Gesundheit und deinem Wohlbefinden spielt. Wir wollen es für die Feminisierung so weit unterdrücken wie nötig, aber extrem niedriges Testosteron (weniger als 10 ng/dl, oder 0.35 nmol/L) kann sich negativ auswirken: Verlust der Libido, Antriebslosigkeit, körperliche Schwäche (über den Muskelverlust aufgrund der HRT hinaus), Konzentrationsschwierigkeiten, Schlaflosigkeit, etc. Diese Symptome ähneln wohlgemerkt denen von cis Frauen in den Wechseljahren. Cis Frauen haben auch Testosteron in ihrem Körper, also mach dir da keine Sorgen. \textbf{Gute Hormonwerte sind wichtig!}

\subsection{Gibt es sogar Situationen, in denen ich Testosteron zusätzlich einnehmen möchte?}\label{9-2}

Ja. Wenn du die eben beschriebenen Symptome bemerkst aber deine Östrogenwerte sonst gut sind, könntest du überlegen, eine geringe Menge Testosteron zu nehmen. Das hilft beispielsweise die Erektionsfähigkeit verbessern, oder der Atrophie des Penis - auch und insbesondere im Vorfeld einer GaOP - entgegenwirken. Oder vielleicht willst du einfach experimentieren, welche Hormon-Kombination sich für dich am besten anfühlt. Alles gute Gründe, Testosteron in einem anderen Kontext als vor der HRT zu erkunden.

\subsection{Falls ich Testosteron zusätzlich einnehmen will, wie könnte ich das tun?}

Es gibt einige Möglichkeiten. Testosteron gibt es als Injektionslösungen aber auch als Sprays, Salben und Gels, wie beim Östrogen. Ein örtliche Verabreichungsform wie Gel wird am häufigsten verschrieben. Gels haben die gleichen Nachteile, die wir schon bei Östrogen-Gels besprochen haben, aber in diesem Fall ist zumindest die genaue Dosierung weniger wichtig.

\subsection{Was für Testosteron Medikamente kann man auf der Haut anwenden?}

Sowohl Gel als auch Salben. Meistens wird Gel verschrieben, aber manche Apotheken können eine Salbe mit geringer Penetrationsrate herstellen, falls es nur um eine örtliche Anwendung auf den Genitalien geht. Das ist aber schwierig zu bekommen und oft teurer. (A.d.Ü. Uns ist keine solche Salbe in Deutschland bekannt, wohl aber T-Gels. Das wird allerdings eher auf Privatrezept verschrieben, kann also zwischen 60-80€ kosten.)

\subsection{Ist bei Testosteron der Ort der Anwendung wichtig?}

Es hängt davon ab, ob es sich um Gel oder eine Salbe handelt. Eine wie oben beschriebenen Salbe wird direkt auf den Genitalien aufgebracht, Gel wird auf den Oberarmen oder Schultern aufgetragen. Bitte achte darauf nichts anzufassen bis die Salbe / das Gel wirklich vollständig eingezogen ist. (A.d.Ü. Denkt auch z.B. an Haustiere!)

\subsection{Wie viel und wie häufig sollte ich Testosteron nehmen?}

Je nach Bedarf. Es hängt vor allem davon ab, wie du dich fühlst. Wenn du zu viel Testosteron zu dir nimmst, können Nebenwirkungen auftreten. Dazu zählen zum Beispiel ölige Haut oder vermehrte Körperbehaarung. Aber nur du weißt, was sich gut für dich anfühlt. Eine wöchentliche Injektion von 5 bis 10mg \textit{Testosteroncypionat} könnte für dich funktionieren.  1-prozentige Gels, die oft in 25/50mg Packungen kommen, können ähnlich wie E-Gels Schwankungen im Hormonhaushalt herbeiführen. Außerdem ist eine halbe Packung fast immer zu viel, vor allem für eine tägliche Anwendung. Ich würde dir raten, mit viel weniger anzufangen, als du denkst zu brauchen, und dich langsam heran zu tasten. (A.d.Ü. Gels in Deutschland (z.B. Testogel von Besins Healthcare) kommen meistens in einem Pumpspender. Anekdotisch reichen 2-3 Hübe. Ich kenne persönlich keine Menschys die mit einer E-Monotherapie unter gesunde T-Werte fallen, die meisten anwendenden Personen nutzen es eher für den Erhalt der Funktionsfähigkeit. Das Gel muss über ein paar Tage skrotal angewendet werden bis die "Funktionsverbesserung" eintritt.)

\subsection{Wo kann ich Testosteron bekommen?}

Du könntest in deinem lokalen Fitnessstudio nach den muskulösesten Bodybuilder Ausschau halten und dann höflich fragen. Achtung: Das war ein Witz. Siehe Frage \ref{6-11} “Ist DIY legal?” (A.d.Ü. Ärztys fragen geht halt auch immer. ;) )

\subsection{Können andere Steroide Testosteron im HRT-Kontext ersetzen?}

Anabol-androgene Steroide, also Stoffe, die in ihrer Struktur Testosteron ähneln, sind nicht alle gleichwertig. Oft benutzte Schwarzmarkt-Steroide wie \textit{Trenboloneazetat} haben viele negative Nebeneffekte, aber Steroide wie \textit{Nandrolondecanoat }werden manchmal bei postmenopausalen cis Frauen eingesetzt, da sie relativ niedrige androgene Eigenschaften haben. Das macht sie für transfeminine Menschen auch interessant. Nichtsdestotrotz ist es unwahrscheinlich, dass dir etwas anderes als Testosteron verschrieben wird. (A.d.Ü. Im Original wird hier explizit auf die US-Lage hingewiesen. Wie es im deutschsprachigen Raum läuft und inwiefern hierzulande Alternativen verschrieben werden, wissen wir nicht. Schreib uns, wenn du es weißt!)

\subsection{Wie hängen Testosteron und \textit{Dihydrotestosteron} (DHT) zusammen?}

\textit{Dihydrotestosteron} wird im Körper auf der Basis von Testosteron durch das Enzym 5$\alpha$-Reduktase gebildet, dabei wird ca. 5\% des Testosterons im Körper umgewandelt. Grob gesagt: Wenn der Testosteronwert richtig unterdrückt ist (oder wenn du eine geschlechtsangleichende OP hattest), dann sollte es nicht viel Testosteron zum Umwandeln geben. Der systemische Wert sollte aber nicht null sein, da einiges immer noch lokal produziert wird. Je nachdem, wie es bei deinem Körper läuft, könnte dies ein Grund sein, ein 5$\alpha$-Reduktase-Hemmer zu supplementieren, wie in der nächsten Kapitel besprochen wird. Zur Erinnerung, \textit{Dihydrotestosteron }sorgt für Körperbehaarung und androgenen Haarverlust.

\textbf{Für Transmascs, die hier mitlesen} will ich hier kurz besprechen, dass es bisher nicht bekannt ist, inwieweit dieses Hormon beim bottom growth eine Rolle spielt; sei es bei der Geschwindigkeit oder Gesamtgröße im Zusammenhang mit 5$\alpha$-Reduktase. Es ist allerdings klar, dass \textit{Dihydrotestosteron }eine primäre Rolle bei der Penis-Entwicklung spielt, aber es ist unklar, inwieweit ein Fehlen transmaskuline Personen in der genitalen Entwicklung einschränkt. Wir wissen, dass eine lokale Behandlung mit \textit{Dihydrotestosteron }-Salbe bei Mikropenissen effektiver ist als Injektionen, insbesondere bei Patient\_en die auf Testosteron nicht reagieren (wie bei einer 5$\alpha$-Reduktase-Defizienz). Aber wie so häufig: Das müsste besser erforscht werden. Jemand sollte Oliver Longdick motivieren sich das genauer anzuschauen!

 

\section{ANTIANDROGENE}\label{AA}

\subsection{Was sind "Antiandrogene"?}

\textit{Antiandrogene, }oft "Testoblocker", "T-Blocker"  oder nur "Blocker" genannt, verhindern die Wirkung von Androgenen (also Testosteron) im Körper, und deshalb heißen sie auch so. Es gibt viele verschiedene Antiandrogene und sie werden oft als Teil der HRT verschrieben. Sie werden gebraucht, wenn die Person noch Testosteron produziert und eine Form der HRT macht, die sich für Monotherapie nicht eignet, aber an sich sind sie nicht wünschenswert. Es muss auch angemerkt werden, dass (die meisten) Antiandrogene die tatsächlichen Testosteronwerte nicht bedeutend unterdrücken, sondern die Effekte vom Testosteron im Körper reduzieren/vermeiden. Das ist bei der Auswertung von Blutwerten wichtig.

\subsection{Warum sollte ich keine Antiandrogene nehmen wollen?}

Das größte Problem mit den meisten Antiandrogenen ist, dass sie oft unerwünschte Nebenwirkungen haben und gar nicht notwendig wären, wenn das Testosteron durch genug Östrogen unterdrückt wird. Diese Nebenwirkungen könnten also (in den meisten Fällen) durch eine wohldosierte Monotherapie vermieden werden. Eine geschlechtsangleichende OP macht sie ebenfalls (in den meisten Fällen) auch überflüssig.

\subsection{Unter welchen Umständen könnte ich Antiandrogene in Betracht ziehen?}

Du könntest Antiandrogene gebrauchen wenn du einerseits eben nicht "die meisten Fälle" bist, wenn es zu deiner Beruhigung beitragen würde, oder wenn deine Krankenversicherung das zur Voraussetzung für andere Prozeduren macht. Die Stoffe, die als Antiandrogene verwendet werden, können auch andere Effekte haben, die für deine Gesundheit hilfreich sein können. Und falls du Androgene zu dir nimmst (A.d.Ü. wie das erwähnte Testogel), könntest du dir einen \textit{Dihydrotestosteron }-Blocker wünschen, um die Nebenwirkungen (Körperbehaarung und Haarverlust) zu minimieren. Das hängt aber auch davon ab, ob du bioidentisches Testosteron einnimmst(z.B. \textit{Nandrolondecanoat}), da nicht alle Androgene sich gleich verhalten.

\textbf{Es muss angemerkt werden, dass die Anwendung von Antiandrogenen am Beginn einer geplanten Monotherapie weder notwendig noch empfohlen ist.} So oder so kommt es zu einer Anpassungszeit, wenn der Körper sich auf die neuen Hormone einstellt. Es gibt also keinen Grund, die Sache komplizierter zu machen. Mach dir keine Sorgen.

\subsection{Was für Antiandrogene gibt es?}

Im Kontext von HRT werden \textit{Spironolacton}, \textit{Bicalutamid} und \textit{Cyproteronacetat} verwendet. Gängige Medikamente zur Unterdrückung der Umwandlung von Testosteron in \textit{Dihydrotestosteron} (DHT), also "5$\alpha$-Reduktase-Hemmer”,  sind \textit{Finasterid} und \textit{Dutasterid}. GnRH-Analoge wie \textit{Leuprorelin} und \textit{Triptorelin} werden als Pubertätsblocker verwendet, aber in manchen europäischen Ländern werden sie auch bei Erwachsenen angewendet. (A.d.Ü. Zumindest Leuprorelin wird, anekdotisch, manchmal in Deutschland auch bei erwachsenen als Injektion verschrieben. Wie immer: Muss man halt Glück mit dem Ärzty haben.)

\subsection{Unter welchen Umständen sollte ich \textit{Spironolacton} nehmen?}

Da für die antiandrogene Wirkung sehr hohe Dosen mit signifikanten Nebenwirkungen erforderlich sind, würde ich \textit{Spironolacton} nur dann empfehlen, wenn du von den anderen Effekten etwas hättest, z.B. der Wirkung als  Aldosteron-Antagonist im Kontext von Blutdruckproblemen oder Schwellungen. \textbf{Wenn unbedingt \textit{Spironolacton} einnehmen willst, dann bitte nicht mehr als 100mg täglich.} Der schlechte Ruf von Spiro ist begründet, es ist quasi der Teufel.

Nur damit du gewarnt bist , gängige Nebenwirkungen sind: "Gehirnnebel" (A.d.Ü. "brain fog" ist da gängiger in der Diskussion), Lethargie, verschlechterte Gedächtnisleistung, häufiger Harndrang, niedriger Blutdruck, niedriger Natriumspiegel/Elektrolytungleichgewicht usw. Mit anderen Worten: \textit{Spironolacton} ist ein blutdrucksenkendes Diuretikum, das als mittelmäßiges Antiandrogen wirkt und normalerweise in hohen Dosen bei ansonsten gesunden Leuten für eine fragwürdige Off-Label-Anwendung verschrieben wird. In wirklich jedem anderen medizinischen Kontext würde kein Medikament verschrieben werden, dass so viele unerwünschte Nebenwirkungen hat und zu dem es bessere Alternativen gibt - aber das ist einfach der Stand der Dinge in Sachen Trans-Gesundheitsversorgung. (A.d.Ü. Glücklicherweise ist Spiro sehr selten in Deutschland. Cyproteronacetat (CPA )ist aber halt auch nicht besser.)

\subsection{Warum sollte ich \textit{Bicalutamid} in Betracht ziehen?}

Wenn du ein Antiandrogen nehmen willst, ist \textit{Bicalutamid} wahrscheinlich das Mittel der Wahl. Es ist im Allgemeinen gut verträglich - abgesehen von 1\% der Fälle, in denen abnormale Leberwerte und Symptome einer Leberfunktionsstörung auftreten - aber ansonsten hat es relativ geringe Nebenwirkungen. Wenn du Bicalutamid nimmst, solltest du regelmäßig deine Leberwerte überprüfen lassen, um sicherzustellen, dass sie im Normbereich liegen. Risiken für die Leber hängen eher von deinem Körper ab als von der kumulativen Wirkung des Medikaments, sodass sich eventuelle Probleme wahrscheinlich innerhalb des ersten Jahres zeigen würden. Ansonsten sollte es vermutlich keine Probleme geben.

\subsection{Wann sollte ich \textit{Cyproteronacetat} in Betracht ziehen?}

Nie. Nimm lieber \textit{Bicalutamid}.

Das langfristige Risikoprofil von CPA ist furchtbar und ich kann mir keine Situation vorstellen, in der ich dieses Medikament einem anderen vorziehen würde. Du kannst alles erreichen, was \textit{Cyproteronacetat} kann, indem du einfach mehr Östrogen nimmst und Progesteron zu deiner Behandlung hinzufügst. (A.d.Ü. Ein besonderes "yey!" für die deutsche Trans-Gesundheitsversorgung in der CPA regulär verschrieben wird. Solltest du CPA nehmen: 5mg am Tag reichen, 12.5mg alle 2 Tage sollten es ebenfalls tun. Nimm auf \textbf{KEINEN FALL MEHR }.)

\subsection{Wann sollte ich \textit{Dutasterid} in Betracht ziehen?}

Wenn du dir große Sorgen wegen Haarausfall machst und/oder deine Chancen auf Haarwachstum maximieren willst, solltest du vielleicht \textit{Dutasterid} nehmen. Wenn dein Testosteronspiegel ansonsten unterdrückt ist, sollte es theoretisch nicht viel bringen, da dein \textit{Dihydrotestosteronspiegel} (DHT) relativ niedrig sein sollte, aber der Körper kann kompliziert sein, sodass es für dich vielleicht trotzdem interessant sein könnte. Siehe auch Frage \ref{11-14}.

Es ist zu beachten, dass \textit{Dutasterid} bei manchen Menschen unerwünschte Auswirkungen auf die Stimmung haben kann. In diesem Fall wird dringend empfohlen, die Einnahme abzubrechen. Beachte auch, dass diese depressiven Wirkungen bis zu einem Monat nach Absetzen des Medikaments spürbar sein können. (A.d.Ü. Wir reden hier von Depressionen und Suizidgedanken. Ihr solltet das ernst nehmen; konkrete suizidale Gedanken sind ein legitimer Grund einen Notarzt zu rufen. Wir wollen auch darauf hinweisen, dass Bundesministerium für Arzneimittel und Medizinprodukte eine \href{https://www.bfarm.de/SharedDocs/Risikoinformationen/Pharmakovigilanz/DE/RHB/2025/rhb-finasterid.pdf}{Rote-Hand-Brief }\textbf{explizite Warnung} im September 2025 dazu herausgegeben hat. Bitte achte gut auf dich! $\heartsuit$)

\subsection{Wann sollte ich \textit{Finasterid} in Betracht ziehen?}

Wenn dir \textit{Dutasterid} nicht verschrieben wird oder deine Versicherung speziell \textit{Finasterid} für die Haarbehandlung vorschreibt. Ansonsten ist \textit{Dutasterid} eigentlich die bessere Wahl, weil es besser wirkt und verträglicher ist.

Beachte, dass Finasterid bei manchen Leuten depressive Verstimmungen auslösen kann. In diesem Fall solltest du es besser absetzen. Denk auch daran, dass diese depressiven Effekte bis zu einem Monat nach dem Absetzen noch spürbar sein können. (A.d.Ü. Hier gilt die gleiche Warnung, \textbf{siehe die vorherige Anmerkung!})

																																																																																 

\subsection{Wo kann ich Antiandrogene bekommen?}

Abgesehen davon, dass sie dir von deinem Ärzty verschrieben werden oder vielleicht rezeptfrei erhältlich sind (A.d.Ü. Nicht in Deutschland), gibt es auch die Möglichkeit, sie über ausländische Apotheken auf dem Graumarkt zu kaufen. Das sind Apotheken in einem anderen Land, wobei der Kauf dort oft mit einigen Hürden verbunden ist. \textit{Dutasterid} und \textit{Finasterid} sind aufgrund ihrer Verbreitung als Medikamente gegen Haarausfall in der Regel am einfachsten rezeptfrei zu bekommen. (A.d.Ü. Ebenfalls nicht in Deutschland; Beides ist Rezeptpflichtig.)

 

\section{FAQ, MYTHEN UND IRRTÜMER}\label{MM}

\subsection*{Häufig gestellte Fragen}
\addcontentsline{toc}{subsection}{\textemdash{} Häufig gestellte Fragen}

\subsection{Sollte ich mir wegen eines erhöhten Risikos für Blutgerinnseln Sorgen machen?}\label{11-1}

Ja und nein. Es stimmt, dass es einen Zusammenhang zwischen der Östrogendosis/-konzentration und dem Risiko für Blutgerinnsel gibt, aber das hängt vor allem davon ab, wie das Östrogen verabreicht wird und um welche Art von Östrogen es geht. Synthetische Östrogene sind ein Grund zur Sorge und erhöhen das Risiko für Blutgerinnsel deutlich, aber bioidentische Östrogene sind nicht bedenklich. Vor allem die Art der Einnahme macht einen großen Unterschied. Oral eingenommenes, bioidentisches Östrogen geht durch die Leber, was für das erhöhte Risiko für Blutgerinnsel sorgt. Injektionen umgehen die Leber, und es gibt keine Hinweise oder Gründe zu der Annahme, dass Injektionen von bioidentischem Östrogen ein signifikant erhöhtes Risiko darstellen, das über die natürlichen Unterschiede zwischen Testosteron und Östrogen dominanten endokrinen Systemen hinausgeht. Trotz dieser Unterschiede hält sich die weit verbreitete Panikmache gegenüber der Östrogen Einnahme seit Jahrzehnten hartnäckig.

\textbf{Wenn du dich bald Operation unterziehst, solltest du wissen, dass WPATH es nicht mehr empfiehlt, die HRT wegen der Gefahr von Blutgerinnseln zu unterbrechen.} Viele Chirurgen schreiben es immer noch in ihre Vorbereitungsrichtlinien, weil sie sich Sorgen wegen Blutgerinnseln machen, aber das ist einfach nur Quälerei, die widerlegt wurde. Sogar die WPATH (A.d.Ü. Die Abkürzung bedeutet World Professional Association for Transgender Health, also der Weltverband für Transgender-Gesundheit. Die haben sich deutlich gebessert aber eine \textit{wirklich wilde Geschichte}) empfiehlt es nicht mehr. Krass, ich weiß. Per \href{https://www.tandfonline.com/doi/pdf/10.1080/26895269.2022.2100644}{WPATH SOC 8 Statement 12.19}: \blockquote{Nach genauer Untersuchung haben die Forscher keinen perioperativen Anstieg der VTE-Rate [Anmerkung: \textit{venöse Thromboembolie}, also ein Blutgerinnsel] bei Transgender-Personen festgestellt, die sich einer Operation unterzogen haben und währenddessen die Sexualhormonbehandlung fortgesetzt haben, im Vergleich zu Patienten, bei denen die Sexualhormonbehandlung vor der Operation abgebrochen wurde.(Gaither et al., 2018; Hembree et al., 2009; Kozato et al., 2021; Prince \& Safer, 2020).} (A.d.Ü. Englisches Dokument.) Ich sollte das eigentlich in einer separaten Frage beantworten, aber um die Links nicht zu zerstören, müsste das am Ende eines Abschnitts stehen, und ich finde, das ist zu wichtig dafür, also schreibe ich es hier rein. Eine echt wichtige Klarstellung, die ich schon früher hätte machen sollen.

\subsection{Ist es okay, während der HRT Nikotin zu konsumieren?}\label{11-2}

Das hängt mit der vorigen Frage zusammen. \textbf{Nikotinkonsum während der HRT, vor allem wenn du Pillen nimmst, erhöht - zusätzlich zu all den anderen Gründen, warum Nikotin nicht gut ist - das Risiko für Blutgerinnsel.} Das gilt für alle Arten von Nikotinkonsum, aber Rauchen ist natürlich mit Abstand das Schlimmste. Du willst echt kein Blutgerinnsel. Selbst wenn du keine Pillen nimmst, stört Nikotin den Östrogenstoffwechsel und kann zu einer deutlichen Verringerung der Feminisierungseffekte führen. Dieser Aspekt ist noch nicht ausreichend erforscht, aber es gibt viele Erfahrungsberichte aus der Community. Es ist nicht einfach, aufzuhören, aber ich glaube an dich. Es gibt gute Hilfsmittel und Strategien wie die schrittweise Reduzierung mit Nikotinpflastern, die echt funktionieren. Du schaffst das.

Nur um das klar zu sagen: \textbf{Das heißt nicht, dass du kein Östrogen nehmen kannst oder solltest. Die Nachteile, wenn du gar kein Östrogen nimmst, sind viel schlimmer als die, die im Zusammenhang mit Nikotin entstehen.} Dieser Abschnitt soll dich nur auf die erhöhten Risiken und die möglicherweise langsamere Transition aufmerksam machen, um klar zu Empfehlen (und dich zu ermutigen), aufzuhören. Aber: Ein Schritt nach dem anderen.

\subsection{Ist es besser, mit einer niedrigen Dosis anzufangen als mit einer hohen?}

Soweit ich weiß, nein. Sexualhormone verhalten sich nicht wie andere Medikamente, bei denen man die Dosis anpassen muss, um Nebenwirkungen zu vermeiden. Wir kennen gute Dosierungen, die bei den meisten Menschen funktionieren. Deshalb empfinde ich persönlich “Anfangsdosen” und “Antiandrogen zuerst”-Behandlungen als medizinische Folter. Manche Leute denken, dass es am besten ist, den langsamen Verlauf der Pubertät nachzuahmen (auch wenn bei der viel mehr passiert als nur der Anstieg des Östrogenspiegels), aber dafür gibt es keine Belege. Eine Orchiektomie am ersten Tag wäre vermutlich die sinnvollste Lösung, aber wer möchte das schon, wenn man gerade erst verstanden hat, dass man trans ist bzw. sich für eine Hormonersatztherapie entschieden hat?

Anders gesagt: \textbf{Es gibt keinen Grund zu glauben, dass es für die Feminisierung besser ist, mit einer Dosis unter dem normalen Bereich “langsam anzufangen”.} Man muss sich keine Sorgen machen, dass man “zu schnell" vorgeht oder so. Sowohl Ärztys als auch andere Transfrauen erfinden jeden Tag neue Mythen.

\subsection{Hat das Körpergewicht einen Einfluss auf die Dosierung?}

Nein. Weil es keinen “optimalen“ Blutspiegel für Östrogen gibt und weil der therapeutische Bereich akzeptabler Werte so breit ist, hat das Körpergewicht keinen nennenswerten Einfluss auf die Dosierung der Hormone in einer Hormonersatztherapie. Aus dem gleichen Grund ist es unwahrscheinlich, dass geringfügige Abweichungen in der Dosierung dein Befinden beeinflussen. Allgemein gibt es auch kein zu niedriges zu hohes Gewicht um eine HRT durchzuführen.

\textbf{Die Anpassung deiner Dosis in Schritten von 0,1 mg (Milligramm, nicht Milliliter!) wäre ein Unterschied, den du wahrscheinlich nicht spüren könntest, weil unser Körper für eine so differenzierte Wahrnehmung einfach nicht empfindlich genug ist}, ganz zu schweigen von der hohen Wahrscheinlichkeit von Ungenauigkeiten bei der Injektion, da Spritzen eine so genaue Abmessung der Dosierung unmöglich machen. Anders ausgedrückt: Wenn man bei der Injektionsmenge zu genau sein möchte, ist das nicht besonders hilfreich, da man dann mehr Präzision fordert, als eigentlich nötig ist.  

\subsection{Kann es zu spät sein mit Östrogen anzufangen?}\label{11-5}

\textbf{Nein.} Egal, wann du anfängst, Östrogen kann echt viel bewirken und mit der richtigen Behandlung kannst du super Ergebnisse erzielen. Sexualhormone gehören zu den effektivsten Hormonen in unserem Körper, wenn es um unser Aussehen geht. Fast alle von uns wünschten sich, früher angefangen zu haben, aber das ist kein Grund, nicht jetzt anzufangen. Und selbst wenn du schon seit Jahren Östrogen nimmst, lohnt es sich trotzdem, die Qualität deiner Behandlung zu verbessern.

\subsection{Hört die Feminisierung/Brustentwicklung nach X Jahren auf?}\label{11-6}

\textbf{Nein.} Es gibt keinen bestimmten Zeitpunkt, an dem Östrogen plötzlich nicht mehr wirkt. Werden dafür Zahlen genannt, sind die entweder 1) komplett erfunden sind oder 2) es wird auf eine Studie verweisen, die nur X Jahre lang lief. Vor allem Ärztys sagen Transfrauen gerne, dass sie nicht mehr als Körbchengröße B erwarten sollen (so funktionieren Brustgrößen nicht mal, aber ich schweife ab), oder dass nach zwei Jahren kein Wachstum mehr zu erwarten ist, aber das stimmt einfach nicht. Es gibt sogar Fälle, in denen Leute nach einer mehrjährigen Pause wieder mit Estradiol angefangen haben und trotzdem noch neues Wachstum hatten.

\subsection{Ich hab seit Jahren keine Veränderungen durch Injektionen bemerkt. Würde es was bringen, wieder auf Pillen umzusteigen?}

Vielleicht, vielleicht auch nicht. Es gibt anekdotische Berichte von Leuten, die von Injektionen zu Pillen zurück gewechselt sind (oder zusätzlich zu den Injektionen Pillen genommen haben) und nach einer “Stagnationsphase” ein stärkeres Brustwachstum festgestellt haben, aber der Mechanismus dahinter ist nicht klar. Es gibt Spekulationen, dass das E1:E2-Verhältnis (\textit{Estron} : \textit{Estradiol}) bei oralen Pillen im Vergleich zu E2 bei Injektionen stark in Richtung E1 verschoben ist, was bei manchen Menschen einen Unterschied machen könnte, obwohl \textit{Estron} normalerweise nicht mit Feminisierung in Verbindung gebracht wird. Wahrscheinlich spielen noch andere Faktoren eine Rolle, aber wenn du möchtest kannst du gerne experimentieren. Die Datenlage ist begrenzt.

\subsection{Ist es normal, dass man sich bei einer HRT schlapp fühlt und wenig Lust auf Sex hat?}

Im Allgemeinen: Nein. Wie sich die Libido äußert, ändert sich am Anfang der HRT, aber wenn du eine ungewöhnlich geringe Libido hast, liegt das eher daran, dass dein Hormonhaushalt nicht gut ist. Das Gleiche gilt für Energielosigkeit. Bring deine Hormone in Ordnung und schau dir als Nächstes deine Ernährung/Vitamine an. Stell sicher, dass du nicht zufällig einen kritisch niedrigen Vitamin-D-Spiegel oder ähnliches hast. Das kommt öfter vor, als du denkst. (A.d.Ü. Insbesondere in Deutschland. Vitamin D Präparate sind rezeptfrei sowohl im Supermarkt als auch in der Apotheke erhältlich. Veganer\_innen (wie ich) sollten auch auf B12 und Eisen achten.)

\subsection{Ich hab von [Mystery Medikament/einer Strategie] gehört, von dem eine befreundete Person sagt, dass es bei der Feminisierung hilft. Stimmt das wirklich?}

Vielleicht, aber wahrscheinlich nicht. Es gibt viele wilde Spekulationen darüber, wie man Feminisierungsziele erreichen kann, aber die meisten sind eher Schlangenöl oder haben potenziell große Risiken, die weit über die der HRT selbst hinausgehen. Du hast das Recht auf körperliche Selbstbestimmung und ich kann dich nicht davon abhalten wild Dinge zu probieren, aber ich kann dich ermutigen, klug zu handeln. Je mehr du dich mit den biologischen Details der Geschlechtsangleichung beschäftigst, desto unsicherer wird der Boden, da immer weniger verlässliche Daten verfügbar sind. Verzweiflung kann zu vielen unklugen und gefährlichen Entscheidungen führen. Mein Rat: Sei klug und gehe auf Nummer sicher.

\subsection{Sollten wir den Östrogenzyklus von cis Frauen nachahmen?}\label{11-10}

Wahrscheinlich nicht. Das ist zwar umstritten, aber da wir (naja, die meisten von uns) keine Gebärmutter und keinen Menstruationszyklus haben, der mit unseren Hormonen zusammenhängt, gibt es keinen Grund zu versuchen, dieses Verhalten nachzuahmen. Meiner Meinung nach ist das eine \textit{Sein-Sollen-Dichotomie}. Das größte hormonelle Problem für die meisten Transfrauen ist die Testosteronunterdrückung, die einen konstant hohen Östrogen-Spiegel erfordert (außer nach einer GaOP, nach der kein Testosteron unterdrückt werden muss), sodass starke Schwankungen und/oder relativ niedrige Spiegel wahrscheinlich unnötiges Leid verursachen. Du kannst natürlich gerne experimentieren. Vor allem, wenn das Unterdrücken von Testosteron für dich kein Problem mehr ist. Siehe Frage \ref{8-11} und folgend.

\subsection{Haben Transfrauen ihre Periode?}

Ähnlich wie bei der letzten Frage ist es wichtig zu verstehen, was da eigentlich passiert. Die einzige Hormonschwankung, die durch deinen speziellen Ester, deine Dosierung und deinen Injektionsrythmus entsteht, kann zu Stimmungsschwankungen führen. Manche Transfrauen vergleichen diese Erfahrung mit einer Periode, aber die Ursache dafür ist eine andere. Es ist meistens ein Zeichen dafür, dass deine Behandlung angepasst werden muss, damit du dich so gut wie möglich fühlst. Schmerzen und Unwohlsein sind keine Voraussetzungen für Weiblichkeit. Das ist ein Unfugsargument das von Bioessentialisten / TERFs gestreut wird. Eine Ausnahme bilden hier intersexuelle Transfrauen, die eine Gebärmutter haben und tatsächlich ihre Periode bekommen. Also in dem spezifischen Fall: Ja, klar, haben die eine Periode. Duh. Siehe Frage \ref{11-35}.

\subsection{Kann zu viel Östrogen in Testosteron umgewandelt werden?}

\textbf{Nein.} Es gibt das Enzym Aromatase, welches Testosteron in Östrogen umwandelt, aber es gibt keinen umgekehrten Mechanismus, der Östrogen in Testosteron umwandelt. Das kann einfach nicht passieren. Das ist ein  Mythos und du solltest sofort misstrauisch werden, wenn jemand sowas behauptet. Leider sind es gerade Ärztys, die diesen Mythos am häufigsten verbreiten.

\subsection{Führt eine GaOP zu einem Anstieg des Testosteronspiegels?}

Nein auch das ist Unfug. Es gibt keinen magischen Mechanismus, der plötzlich dafür sorgt, dass der Testosteronspiegel steigt sobald die Hoden entfernt wurden. Also selbst wenn deine Eier magische Kräfte hätten, funktioniert die Hormonproduktion einfach nicht so. “Aaaaber deine Nebennieren…” - die funktionieren auch nicht so. Die einzige mögliche Ausnahme wären nicht diagnostizierte Fälle von Nebennierenhyperandrogenismus, (A.d.Ü. Sag das drei mal schnell!) die vor der Operation mit einem Antiandrogen wie \textit{Spironolacton} behandelt wurden und nach Absetzen des Antiandrogens wieder auftreten könnten. Also bitte hört auf, diesen Mythos zu verbreiten.

\subsection{Wie kann ich Haarausfall verhindern/rückgängig machen?}\label{11-14}

Biologisch gesehen ist das ziemlich einfach. Eine Standard-HRT-Therapie allein ist in dieser Hinsicht schon fast wie Magie (frag nicht, wo \textit{hier} die Magie steckt), aber die Einbeziehung von 5$\alpha$-Reduktasehemmer (5-ARI), wie in Abschnitt \ref{AA} “ANTIANDROGENE” besprochen, wird in krassen Fällen empfohlen, um den Verlust komplett zu stoppen. 5\% Topisches Minoxidil ist das Einzige, was neben Hormonen hilft, deinen Haaransatz zu festigen. Aber denk daran, dass du im Regelfall nur sterbende/ruhende Haarfollikel retten kannst. Tote Follikel kommen nicht zurück.

Wenn das allein nicht reicht, solltest du wissen, dass sich die Haartransplantationstechnologie stark verbessert hat. Das FUE-Verfahren (Follicular Unit Extraction) ist genau das, was du dir anschauen solltest. Hier werde ich in Zukunft einen Leitfaden verlinken, den eine Expertin zum Thema der Kostenübernahme durch die Krankenkasse für diese Behandlung geschrieben hat, sobald er fertig ist. Das ich das jetzt schon erwähne ist Gruppenzwang. Schau hier regelmäßig vorbei. (A.d.Ü. Es ist unwahrscheinlich, dass das auch bei deutschen Krankenkassen funktionieren wird.)

\subsection{Beeinflusst Sport die Feminisierung?}

Wahrscheinlich. Die Hormonersatztherapie verändert langsam deinen Körper, mit Sport kannst du aktiv dazu beitragen. Denk dran, dass dieser Prozess SEHR LANGSAM ist. Es ist also wichtig, dass du genug isst, um die nötige Energie zu haben, und das du geduldig bleibst. Die Wachstumshormone, welche durch die Muskelstimulation beim Krafttraining freigesetzt werden, spielen auch eine Rolle bei der Brustentwicklung, also ist das wahrscheinlich eine gute Sache, abgesehen von den anderen offensichtlichen gesundheitlichen Vorteilen von Sport.

Das ist NICHT nur der kaum verhüllte Fetisch der Autorin; Krafttraining ist wichtig für deine Gesundheit! Ich erwähne das, weil viele Transfrauen denken, dass sie wie der Hulk aussehen werden, wenn sie eine Hantel auch nur anfassen. Ich verstehe das, aber wenn du kein Testosteron hast und keine Steroide nimmst, wirst du nicht so aussehen. Ganz zu schweigen von der Zeit, Mühe und Disziplin, die nötig sind, um auch nur annähernd so auszusehen.

\subsection{Was soll ich denn trainieren?}\label{11-16}

Cardio ist wichtig für den ganzen Organimus und Übungen für den Unterkörper machen deine Hüften und Gesäßmuskeln fester und betonen deine Figur. Übungen für den Oberkörper verbessern deine Haltung und stützen deine Brüste, sodass sie größer aussehen. Mit anderen Worten: alles. Du nimmst Östrogen. Hast du schon mal cis Sportler\_innen gesehen? Sport macht dich weiblicher.

\href{https://docs.google.com/document/d/1-NyE5EY5TTaRRMhk7HlTbKJ7HifjEsA4jlDO1qKQVl0/edit?tab=t.0}{Dieser Leitfaden wurde mir mal geschickt} \textcolor{red}{(Warnung: Google Docs)} (A.d.Ü. Auf Englisch!) und scheint ein guter Ausgangspunkt zu sein. Ich möchte darauf hinweisen, dass es keine speziellen Übungen gibt, die den Körper feminisieren oder maskulinisieren, da der Körper so nicht funktioniert. Du solltest dich jedoch vielleicht mehr auf Übungen für den Unterkörper und auf Flexibilität konzentrieren als typische Kraftsportler.

\subsection{Kann Östrogen wirklich dazu führen, dass man kleiner wird?}

Ja. Es könnte sein, dass es mit Veränderungen des Wassergehalts in den Sehnen und Bändern zusammenhängt, aber das wurde noch nicht untersucht und ist somit reine Spekulation. Wissenschaftler\_innen aufgepasst: eine coole Idee für eine Studie!

\subsection{Kann Östrogen wirklich dazu führen, dass die Füße schrumpfen?}

Ja. Siehe oben.

\subsection{Kann Östrogen wirklich... "andere Arten von Schrumpfung" verursachen?}

Nun, wie man so schön sagt: “Use it or lose it“ – nutze es oder verliere es.

\subsection*{Sexuelle Gesundheit}
\addcontentsline{toc}{subsection}{\textemdash{} Sexuelle Gesundheit}

\subsection{Wie kann ich bei einer Hormonersatztherapie meine Erektionsfähigkeit verbessern?}\label{11-20}

Neben der regelmäßigen Nutzung gibt es noch andere Möglichkeiten, die Erektionsfähigkeit zu verbessern: 1) Verbessere deine körperliche Fitness und Gesundheit, vor allem deinen Herz-Kreislauf; 2) denk über Medikamente wie \textit{Tadalafil} oder \textit{Sildenafil} nach; und 3) überleg dir, ob du Testosteron Medikamente nehmen willst (siehe Abschnitt \ref{T} “TESTOSTERON”).

Wenn du mehr darüber wissen willst, wie Erektionen funktionieren, findest du in \href{https://stainedglasswoman.substack.com/p/how-to-maintain-your-penis-function}{diesem (A.d.Ü. Englischen) Substack-Artikel} einen guten Überblick über das Thema.

\subsection{Wie kann ich während meiner HRT die Menge an Sperma/Vorflüssigkeit erhöhen?}

Mach dir keine Sorgen, das ist eine ganz normale Frage. Sonnenblumenlecithin und Pygeum können da helfen, zusätzlich zur oben erwähnten Testosteron-Supplementierung. Es scheint auch bei der vaginalen Feuchtigkeit und Erregung bei Frauen, die eine GaOP hatten, einen Unterschied zu machen, aber es gibt noch nicht genug Daten und Erfahrungsberichte, um das sicher zu sagen. Ansonsten solltest du einfach genug Wasser trinken und auf eine ausgewogene Ernährung achten.

\subsection{Kann ich durch die HRT laktieren?}

Ja. Domperidon, Bockshornklee, Sonnenblumenlecithin, reichlich Östrogen und reichlich Progesteron. Hol dir eine Pumpe. Hau rein.

Man sollte wissen, dass Domperidon Nebenwirkungen und Risiken hat und dass die Fähigkeit zu laktieren keinen Einfluss auf die Entwicklung der Brust hat. Die Newman-Goldfarb-Protokolle sind das, was du dir anschauen solltest wenn dir das als Erfahrung wichtig ist.

\subsection{Kann die HRT die Sinne und die Wahrnehmung, z. B. den Geruchssinn, verändern?}

Du hast vermutlich die Jahre vor Beginn der Hormonersatztherapie unter Dissoziation und Depressionen gelitten. Die Welt ist jetzt lebendiger, weil du nicht mehr rund um die Uhr dissoziierst. Ein Wunder der modernen Medizin!

Auf jeden Fall kann Estradiol aber deine Sehstärke direkt verändern.

\subsection{Kann die HRT meine Sexualität verändern?}

Ähnlich wie bei der oben beschriebenen Dissoziation führt die Hormonersatztherapie oft zu mehr Offenheit und Akzeptanz gegenüber sich selbst, was eine Veränderung in der Ausprägung der eigenen Sexualität bewirken kann. Ob es sich dabei um einen chemischen oder verhaltensbedingten Effekt handelt, ist weitgehend eine Frage der eignen Betrachtungsweise.

\subsection{Sollte ich eine PrEP machen?}\label{11-25}

\textbf{Ja.}\href{https://de.wikipedia.org/wiki/Pr%C3%A4expositionsprophylaxe}{\textit{Präexpositionsprophylaxe zur HIV-Prävention} (PrEP)} ist eine Kategorie von antiviralen Medikamenten, die HIV/AIDS vorbeugen sollen. Das hat zwar nicht direkt mit HRT zu tun, aber Transfrauen haben oft ein erhöhtes Risiko, sich mit HIV/AIDS anzustecken. Angesichts der Geschichte der AIDS-Pandemie ist PrEP ein Wunder der modernen Medizin, das du nehmen solltest. \textbf{Hinweis: PrEP-Medikamente haben keinen Einfluss auf HRT, daher kannst du PrEP als trans Person bedenkenlos eine solche Prophylaxe machen.}

Insbesondere wenn du sexuell aktiv bist, solltest du ernsthaft darüber nachdenken PrEP zu machen. Aber auch wenn du nicht sexuell aktiv bist, haben Transfrauen ein deutlich höheres Risiko: Wir sind häufiger Opfer sexueller Gewalt, daher solltest du trotzdem ernsthaft über eine PrEP nachdenken. In den meisten Fällen wird dir wahrscheinlich \textit{Truvada} als einmal täglich einzunehmende Tablette verschrieben, aber wenn du unter Übelkeit als Nebenwirkung leidest, kannst du wahrscheinlich ohne Verlust der Wirksamkeit auf \textit{Descovy} umsteigen. In den USA übernimmt die Krankenversicherung Descovy normalerweise nicht, es sei denn, du hast Truvada ausprobiert (oder du behauptest, dass du es ausprobiert hast). Das neue Medikament \textit{Lenacapavir,} das nur zweimal jährlich gespritzt wird, sollte die PrEP deutlich einfacher machen. Das ist zwar nicht günstig aber da es weniger häufig eingenommen werden muss kann es sein, dass es durch die niedrige Einnahmefrequenz trotzdem günstiger ist.

(A.d.Ü. Truvada und Descovy sollten von der Krankenkasse bezahlt werden, Lenacapavir. \href{https://www.aidshilfe.de/de/hiv-prep/prep-praxen-finden-kosten}{Weiterführende Informationen gibt es bei der deutschen AIDS-Hilfe}.)

\subsection*{Behandlungsfehler und ärztliche Ignoranz}
\addcontentsline{toc}{subsection}{\textemdash{} Behandlungsfehler und ärztliche Ignoranz}

\subsection{Ich habe gehört, dass Injektionen für weniger stabile Werte sorgen, weil man sie seltener macht. Stimmt das?}
																  

																														

Nur wenn du dich an die bescheuerten WPATH SOC 8-Richtlinien hältst, die eine empfohlene Dosierung von \textit{Estradiol Valerate} oder \textit{Estradiol Cypionate} im Bereich von 5-30 mg alle zwei Wochen vorschreiben, was du, um es ganz klar zu sagen, auf keinen Fall tun solltest. “Keinen Schaden anrichten“, my ass.

\subsection{Aber mein Ärzty meinte doch...?}

Die durchschnittliche Ärzty-Person hat keine Ausbildung in Sachen Trans-Gesundheitsversorgung, und \href{https://www.endocrine.org/news-and-advocacy/news-room/2017/endocrinologists-want-training-in-transgender-care }{4 von 5 Endokrinologen haben auch nie eine spezifische Vertiefung für Trans-Gesundheitsversorgung gemacht}. (A.d.Ü. Die Statistiken lassen sich natürlich nicht direkt auf Deutschland übertragen aber hier dürfte es auch nicht besser aussehen.) Es ist nicht unwahrscheinlich, dass du deren erste\_e Transpatient\_in bist und dass keine vorherigen Erfahrungen mit den Bedürfnissen und der Behandlung von trans Personen bestehen. Selbst Ärztys, die sich sehr engagieren, sind oft durch konservative Behandlungsstandards eingeschränkt, die sie befolgen müssen und die nicht immer mit der für dich besten Behandlung übereinstimmen. Siehe oben. (A.d.Ü. \href{https://schwulenberatungberlin.de/wp-content/uploads/2025/06/AWMF-online_S3_Richtlinie.pdf}{die S3-Leitlinien} sind in Deutschland maßgeblich und furchtbar konservativ.)

Pass auch auf das "Trans-Armbruch-Syndrom“ auf, also die Tendenz von Ärztys, alles auf die Hormonersatztherapie zu schieben. Wenn du dir den Arm gebrochen hast, liegt das wahrscheinlich nicht "an diesen Hormonen"

Ich sollte das eigentlich als separaten Punkt aufschreiben, aber ich will das Layout nicht durcheinanderbringen: Es gibt keine Situation, in der es okay ist, dass ein Arzt deine Brüste anschauen oder anfassen will, um “das Wachstum zu beobachten“ oder aus irgendeinem anderen Grund. Zum Glück kommt das heutzutage viel seltener vor, aber das ist nichts anderes als sexuelle Belästigung durch eine medizinische Person und absolut inakzeptabel.

\subsection{Mein Ärzty will mir keine Injektionen verschreiben. Was soll ich machen?}

Du kannst versuchen, die Ärtzy zu überzeugen, zu einer anderen Ärzty-Person zu wechseln oder zu einer DIY-Quelle wechseln. Lass dich nicht von einer medizinischen Einrichtung davon abhalten, eine angemessene und dir zustehende Gesundheitsversorgung zu bekommen. \textbf{Du musst dir bei jeder Interaktion mit dem medizinischen System als Transperson merken, dass du dich für dich selbst einsetzen musst. Geh ihnen auf die Nerven wenn du dich traust! } Wenn dazu noch Behinderungen hast oder nicht-weiß bist - die auf Ärztys ebenso abschreckend wie Weiblichkeit wirken - viel Glück! (A.d.Ü. Der Sarkasmus ist in der englischen Version viel deutlicher - aber seid versichert: Dies ist sarkastisch gemeint.)

\subsection{Wie verhält sich die HRT für cis Frauen in den Wechseljahren zur HRT für Transfrauen?}\label{11-29}

Auch wenn wir nicht zu 100\% die gleichen Ziele bei der HRT haben (Feminisierung / Allgemeiner Erhalt der Gesundheit) und vor allem ganz unterschiedliche Dosierungsbedürfnisse, gibt es doch viele Gemeinsamkeiten. Medizinische Frauenfeindlichkeit in Form von Inkompetenz, Abweisungen, Feindseligkeit und/oder Fehlinformationen ist leider etwas, das alle Frauen erleben. Deshalb ist es super wichtig hier Solidarität aufzubauen. \href{https://www.youtube.com/watch?v=W0XW6av2wLQ}{Schau dir die ersten 30–40 Minuten dieses Interviews an} - Die Probleme werden dir wahrscheinlich sehr bekannt vorkommen, und ist nützlich wenn du deinen Blutdruck in die Höhe treiben möchtest. Die Interviewte selbst weist auch auf diesen Zusammenhang hin! Die \href{https://en.wikipedia.org/wiki/Women's_Health_Initiative}{Frauengesundheitsinitiative in den USA} hat das Leben unzähliger Frauen ruiniert.

\subsection*{Intersexualität und Begleiterkrankungen}
\addcontentsline{toc}{subsection}{\textemdash{} Intersexualität und Begleiterkrankungen}

\subsection{Was ist denn mit dem Ehlers-Danlos-Syndrom los?}

Diese Bindegewebsstörung hat eigentlich nichts mit der Hormonersatztherapie zu tun, aber viele Transmenschen haben sie. Also, herzlichen Glückwunsch, falls du hierdurch erfahren hast, dass du auch davon betroffen bist. Abgesehen von den allgemeinen langfristigen Herz-Kreislauf-Problemen, die du vielleicht im Auge behalten solltest, mach weiter mit Krafttraining, damit deine Gelenke funktionieren. Siehe Frage \ref{11-16}.

\subsection{Was sollte ich bei Intersexualität beachten?}

Bisher habe ich Intersexualität nur am Rande erwähnt. Hier ist eine kurze Liste mit Sachen, die du schon mal gehört haben solltest - entweder für deine eigene Gesundheit oder auch für Freund\_innen.

\subsection{Was ist das Klinefelter-Syndrom?}

Das ist eine relativ häufige (wenn man bei seltenen Chromosomenanomalien überhaupt dieses Wort nutzen will) intersexuelle Variante, von der manche transfeminine Personen vielleicht nicht wissen, dass sie es haben, weil sich die beiden Zustände überschneiden können. Es zeigt sich normalerweise durch einen niedrigen Testosteronspiegel zu Beginn der (ersten) Pubertät. Es ist gut, den Namen zu kennen, nur für den Fall.

\subsection{Was ist das Müller-Gang-Persistenzsyndrom (PMDS)?}

Noch eine “Ich schreib das mal mit hier rein, weil du vielleicht so wenigstens zum ersten Mal davon hörst das es existiert”-Intersex-Konditionen, die manche transfeminine Personen betreffen kann, auch wenn wir nicht wissen wie viele/wenige es tatsächlich sind, weil wir keine Zahlen haben. Durch die Existenz der Müller'schen Gänge besteht die Möglichkeit das eine unterentwickelten Gebärmutter oder weibliches Gewebe existiert und das kann das zu Komplikationen und Besonderheiten führen. Du solltest wahrscheinlich extra Progesteron nehmen, um das Risiko für Gebärmutterkrebs zu vermeiden.

\subsection{Worum geht es bei dem ovotestikulären Syndrom?}

Insbesondere diese intersexuelle Kondition kann zu verwirrenden Schwankungen in den Hormonwerten führen. Dies geschieht aufgrund des Vorhandenseins von sowohl Eierstock- als auch Hodengewebe, das entweder getrennt oder kombiniert in einem \textit{Ovotestis} existiert. Dies äußert sich auf viele verschiedene Arten, die mit deiner HRT interagieren kann, wenn du mit der Unterdrückung des \textit{luteinisierenden Hormons} (LH) beginnst. Es kann vorkommen das eine Gebärmutter existiert (oder auch nicht), es können mehrere Gonaden vorhanden sein und/oder es kann äußerlich völlig typisch aussehen.

\subsection{Was ist der Unterschied zwischen Darmkrämpfen und Gebärmutterkrämpfen?}\label{11-35}

Darmkrämpfe werden in der frühen Transition häufig fälschlicherweise als Symptom einer Intersexualität angesehen. Darmkrämpfe sind weit verbreitet und betreffen den gesamten Bauchraum, während Gebärmutterkrämpfe stark auf einen Bereich unterhalb des Bauchnabels konzentriert sind und eher als scharfe Stiche/Kontraktionen in schneller Folge auftreten. Es fühlt sich an, als würde das Innere deines Körpers als Stressball benutzt werden. Ein großer Unterschied!

\subsection{Was ist mit anderen Intersex-Konditionen?}

Ich habe hier nur ein paar wichtige Beispiele aufgelistet, aber es gibt noch viel mehr Möglichkeiten, die weit über den Rahmen dieses Leitfadens hinausgehen. Anekdotisch heißt es, dass diese Anomalien bei Trans-Personen häufiger auftreten als bei cis Menschen, daher ist es gut, sich damit ein bisschen auszukennen.

\subsection*{Randfragen die für trans Personen wichtig sind aber nicht direkt die HRT betreffen}
\addcontentsline{toc}{subsection}{\textemdash{}Randfragen die für trans Personen wichtig sind aber nicht direkt die HRT betreffen }

\subsection{Viele DIY-Quellen akzeptieren nur Kryptowährungen. Ist das ein Muss? Wie läuft das ab?}

Es gibt andere Anleitungen, die das Thema der sicheren Nutzung von Kryptowährungen besser abdecken als ich es kann. Einige Anbieter bieten auch eigene Anleitungen an. Aber ja, Kryptowährungen sind oft aus vielen Gründen erforderlich. “Kryptowährung“ ist auch eben ein diffuser Begriff der vieles bedeuten kann, aber die Verwendung als Währung war schließlich der ursprüngliche Sinn. Meistens ist es einfach nur nervig. Monero (XMR) ist gut. (A.d.Ü. Für das Handeln mit Krypto wird in Deutschland ein Personalausweis benötigt.)

\subsection{Was ist mit SERM-Medikamenten für nicht binäre HRT?}

Manche Leute nehmen SERMs (selektive Östrogenrezeptormodulatoren) als Teil einer Transition, bei der es nicht so sehr um Feminisierung geht, sondern eher um einen androgynen Look, aber das ist noch ziemlich Neuland, weshalb ich in diesem Leitfaden nicht weiter darauf eingehe. Wenn du das ausprobieren willst, musst du das auf eigene Faust machen, also sei bitte vorsichtig. Ich persönlich halte nicht viel von ihnen, da ich nicht viele Hinweise darauf gesehen habe, dass sie so gut funktionieren, wie die Leute normalerweise denken oder wollen, dass sie funktionieren, zumindest nicht ohne viele weitere Vorbehalte, aber offensichtlich gibt es Leute, die sie mögen. Es ist einfach nichts, wofür ich gerne Empfehlungen aussprechen würde.

Die verschiedenen vorgeschlagenen nicht-binären Therapieformen sind oft sehr individuell, weil sie auf die speziellen Ziele einer Person zugeschnitten sind. Jede HRT sollte bis zu einem gewissen Grad individuell angepasst werden, aber oft gibt es größere Unterschiede bei den gewünschten Ergebnissen, wenn Menschen nach Androgynie suchen. Hormonell gesehen ist das nicht trivial. Alles, was in diesem Leitfaden steht, sollte nur als Ausgangspunkt betrachtet werden, wenn du mit etwas Komplizierterem experimentieren möchtest, aber denk daran, dass es viel mehr braucht, um Transitionsziele zu erreichen, als nur Hormone allein.

\subsection{Sind Sachen wie “pflanzliche Hormonersatztherapie“ oder “Phytoöstrogene“ echte Alternativen oder gut zur Unterstützung geeignet?}

\textbf{Nein.} Wenn dir jemand versucht eine “pflanzliche Hormonersatztherapie“ aufzuschwatzen, dann verkauft er dir nur Bullshit. Das Einzige, was dich weiblicher macht, ist bioidentisches Estradiol und keine pflanzliche Östrogene. Kein “natürliches” Produkt kann Estradiol ersetzen. Glücklicherweise ist das kein weitverbreiteter Scam und du wusstest vermutlich schon, das es reiner Betrug ist. Aber falls nicht, und du jetzt bei deiner Recherche darauf triffst, weißt du jetzt Bescheid. Wenn es nach Bullshit riecht, ist es wahrscheinlich auch Bullshit. Außer wir reden über Insektensteroide, die sind echt cool. Aber sie machen dich nicht weiblicher.

\subsection{Ist der Reddit-Arzt, über den alle reden, gut?}

Nein.

\subsection{Ich hab gehört, dass man Östrogen selbst in der Badewanne machen kann. Stimmt das?}

Nein. Ich hab echt keine Ahnung, woher dieser Witz ursprünglich stammt, der jetzt von viel zu vielen ernst genommen wird, aber es gibt keinen Schritt in irgendeinem Prozess in der HRT Herstellung, bei dem eine Badewanne überhaupt in Betracht gezogen werden könnte. Glaub nicht alles, was du online liest. Ich weiß nicht mal, was man theoretisch mit einer Badewanne machen will, es sei denn, du denkst, dass Östrogen-Vials mit dem Badewasser von Transfrauen gefüllt sind. Ich weiß allerdings nicht, warum du das denken solltest. Es ist offensichtlich Sperma. (A.d.Ü. Schaut mich nicht so an, ich Übersetz das nur. :p)

\subsection{Wie wirkt sich die HRT auf die Fortpflanzungsfähigkeit aus?}\label{11-42}

Es ist wichtig zu wissen, dass das Thema noch nicht ausreichend erforscht ist, sodass wir keine Quantitativen aussagen treffen können. In Anbetracht der signifikanten Konsequenzen einer ungewollten Schwangerschaft wird dringend empfohlen, immer auf Safer Sex zu achten undvorsichtig zu sein. Die HRT selbst kann und wird wahrscheinlich irgendwann zu Unfruchtbarkeit führen, allerdings nur durch eine vollständige Unterdrückung der HHN-Achse (siehe Frage \ref{2-3}) über einen langen Zeitraum hinweg. Mit anderen Worten: Wenn du noch keine GaOP hattest und eine HRT Form nimmst, welche die HHN-Achse weniger zuverlässig ausschaltet (wie Pillen), dann solltest du das bedenken.

\textbf{Wenn die HHN-Achse nicht überbrückt wird, kann man jemanden durchaus schwängern,} und Spermien brauchen zur Reifung genug Zeit, dass das sogar für einige Monate \textbf{Beginn von ausreichend hohen Estradiolwerten der HHN-Achse} noch gilt. Nimm das bitte ernst. Es wird empfohlen, die HHN-Achse mindestens sechs Monate lang komplett zu unterdrücken, vielleicht sogar eher ein Jahr lang, nur um ganz sicher zu gehen.

\subsection{Ist Unfruchtbarkeit durch HRT umkehrbar?}\label{11-43}

Theoretisch kann man eine durch HRT verursachte Unfruchtbarkeit rückgängig machen - vorausgesetzt, man war vor der HRT nicht bereits unfruchtbar (eine gewagte Annahme!) -  aber es gibt nicht viele dokumentierte Fälle, sodass die volle Wirksamkeit der Wiedererlangung der Fruchtbarkeit nach einer langfristigen HRT unbekannt ist. Zum wiedererlangen der Zeugungsfähigkeit ist die Wiederaufnahme der Funktion der HHN-Achse mit Medikamenten notwendig sowie die vollständige Absetzung der HRT, was im Wesentlichen einer hormonellen Detransition von mindestens sechs Monaten entsprechen würde. Selbst dann wäre die Zeugungsfähigkeit aufgrund der Spermienqualität nicht sicher oder garantiert. Ehrlich gesagt: Es macht keinen Sinn so vorzugehen oder zu planen - geh am besten davon aus, dass du unfruchtbar sein wirst. Also ist das Einfrieren einer Samenspende wäre vor oder frühzeitig zu Beginn der HRT empfehlenswert (sofern es finanziell möglich ist), wenn potenzielle leibliche Kinder Priorität haben und wenn eine zukünftige Beziehung, in der dies möglich/erwünscht ist, wahrscheinlich ist.



\section{KREATIN}

\subsection{Was ist Kreatin?}

Kreatin ist eine organische Verbindung, die in deinen Muskeln und deinem Gehirn vorkommt. Es wandelt ADP wieder in ATP um, was für die Energieproduktion in deinem Körper wichtig ist, vor allem bei anfänglichen hohen Belastungen, bevor andere Energiesysteme übernehmen.

\subsection{Ist das nicht so was wie Steroide oder so, die Bodybuilder nehmen?}

Nein. Bodybuilder und Sportler mögen es, weil mehr Energie bedeutet, dass sie länger aktiv sein können, bevor sie müde werden. Beide sind aber nicht die einzigen Personengruppen, die es nehmen, denn es ist im Grunde das beliebteste Nahrungsergänzungsmittel dessen Wirksamkeit echt nützlich und auch erforscht ist.

\subsection{Was hat Kreatin mit HRT zu tun?}

Nichts! Aber ich rede darüber, weil ich es für gut halte und es leid bin, mich ständig zu wiederholen, weil die Leute immer wieder danach fragen und du das hier sowieso liest, oder? Ich liebe ein Publikum das sich mir ausliefert. Meine Stand-up-Routine folgt. (A.d.Ü. Katie doing Katie things. Aber sie hat recht, also lest weiter.)

\subsection{Okay, warum sollte ich dann Kreatin nehmen?}

Was für eine tolle Frage! Es ist gut für dein Gehirn und deine Muskeln. Kreatin ist bei vielen Leuten, je nach ihrer Ernährung, oft in relativ geringen Mengen vorhanden, vor allem bei Vegetarier\_innen (A.d.Ü. und Veganer\_innen). Es gibt überzeugende Forschungsergebnisse, die zeigen, dass verschiedene chronische Erschöpfungszustände und postvirale Erkrankungen (insbesondere Long COVID) mit erschöpften Kreatinreserven im Gehirn zusammenhängen, sodass manche Leute durch die Einnahme von Kreatin-Präparaten kognitive Vorteile feststellen. Es ist keine Zauberei, aber es ist spottbillig, sodass es meiner Meinung nach einen Versuch wert ist.

\subsection{Was für Arten von Kreatin gibt es?}

Du brauchst einfach nur \textit{Kreatinmonohydrat-Pulver.} Pillen haben meistens eine niedrige Dosierung und sind trotzdem teuer, während bei Gummibärchen das Kreatin im Herstellungsprozess oft zerstört wird. Viele Hersteller packen Kreatin auch in irgendwelche Performance-Mischungen, aber das reine Zeug ist meistens billiger.

\subsection{Wie soll ich es denn nehmen?}

Die allgemeine Empfehlung ist 5-10g täglich, aufgelöst in irgendeiner Flüssigkeit. Es löst sich am besten in Sachen auf, die nicht nur aus Wasser bestehen. Es ist meistens geschmacksneutral, also einfach ein oder zwei Messlöffel in deinen Kaffee oder Smoothie geben und fertig. Je nach Menge und Flüssigkeit kann es ein bisschen kreidig oder körnig sein.

\subsection{Spielt es eine Rolle, wann ich es nehme?}

Nicht wirklich. Es hat keine sofortige Wirkung, deshalb ist es albern, es in Mikrodosen in Pre-Workout-Mixes zu haben. Nimm es einfach, wann immer es dir passt.

\subsection{Was genau macht es denn dann?}

Es sammelt sich in deinem Körper über ein oder zwei Wochen bis zu einem maximalen Sättigungsgrad an. Dann musst du nur noch diesen Stand halten und kannst die Vorteile genießen (vielleicht fühlst du dich besser).

\subsection{Sollte ich eine "Startphase" haben in der ich mehr nehme?}

Eher nicht. Wenn du nicht gerade unter großem Zeitdruck stehst oder so, ist das wahrscheinlich total egal. Nimm einfach regelmäßig das, was dir am besten passt.

\subsection{Gibt es Nebenwirkungen?}

Eine leichte Gewichtszunahme kann durch mehr Wasser in deinen Muskeln passieren (was gut ist, also mach dir keine Sorgen). Wenn du es nicht mit Wasser nimmst oder zu viel auf einmal nimmst, kannst du Bauchschmerzen kriegen. Autsch.

\subsection{Gibt es Menschen die keins nehmen sollten?}

Leute mit Nierenproblemen. Nicht, weil es die Probleme verursacht, sondern weil Kreatinin (andere Schreibweise! Kreatin wird zu Kreatinin) in Labortests als Marker für verschiedene Nierenprobleme verwendet wird und eine Einnahme so zu einem falsch positiven Ergebnis führen könnte. Denk einfach daran.

\subsection{Hast du irgendwelche Marken-Empfehlungen?}

Nein. Das sollte auch eigentlich egal sein. Nimm einfach das, was seriös wirkt und einen vernünftigen Preis hat. Ich würde ja das Produkt empfehlen, das mir gefällt, aber als ich die Marke um einen Affiliate-Link gebeten habe, haben sie abgelehnt – ihr Pech! Keine kostenlose Werbung.

\subsection{Du hast echt Kreatin in dieses Dokument mit reingeschrieben, was?}

Ja, das ist echt witzig. Ich kann nichts dafür, dass ich darüber Witze gemacht habe und die Leute mir gesagt haben, dass es ihnen echt geholfen hat, denn jetzt fühle ich mich verpflichtet, weiter darüber zu reden!!!

\section{ABSCHLIESSENDE BEMERKUNGEN}

Wenn einer der folgenden Punkte zutrifft:

\begin{itemize}
\item du bist sauer auf mich, obwohl ich dich gewarnt habe;

\item du hast einen inhaltlichen Fehler oder Tippfehler entdeckt;

\item du hast eine klärende Frage, die in den Text aufgenommen werden sollte;

\item du hast einen Einwand, der hoffentlich kein “Ähm, eigentlich...“ ist;

\item du möchtest mich lobpreisen;

\item du möchtest mir deine Treue schwören; 

\item du möchtest mir einen Obolus entrichten.;
\end{itemize}

Dann melde dich einfach bei mir und ich schaue, was wir machen können. Bluesky ist der einfachste Kontaktweg, und du kannst mir eine Direktnachricht schicken, um meine Signal-ID zu bekommen. Ansonsten danke fürs Lesen und ich hoffe, es hilft dir weiter.

\textbf{Wenn du dieses Projekt durch eine Spende unterstützen möchtest,} \href{https://cash.app/Katitties}{CashApp}, \href{https://ko-fi.com/katitties}{Ko-Fi}, und \href{https://account.venmo.com/u/katitties}{Venmo} gehen alle. Vielen Dank!

Und zum Schluss: \textbf{Das Allerwichtigste, was du als Transperson tun kannst, ist einfach zu leben.} Dieses Dokument ist zwar eine Anleitung, aber genauso ist es eine Botschaft an dich als Transperson: Deine Existenz ist ein Geschenk für die Welt, deine Anwesenheit ein Segen für die Menschen um dich herum und du verdienst es, mit Respekt behandelt zu werden. Selbst wenn du sonst nichts tust, ist dein Leben eine Leistung, die es wert ist, gewürdigt zu werden. Danke.

\section*{FREUNDE VON PGHRT}\label{FOPGHRT}
\addcontentsline{toc}{section}{FREUNDE VON PGHRT}

In diesem Dokument findest du hier und da Links zu anderen Anleitungen und Ressourcen. Unten findest du eine Zusammenfassung davon, die im Laufe der Zeit auch mehr Links zu externen Ressourcen enthalten wird, idealerweise von anderen Trans-Personen. Für alle, die auf Datenschutz achten oder sich darüber Gedanken machen: Einige dieser Links sind Google Docs-Links.

(A.d.Ü. Die Links sind alle auf Englisch)

\begin{enumerate}
  \item \href{https://startwith4mgestradiolenanthateweeklyandtestatonetothreemonths.com/}{SW4EEWATAOTTM} - TL;DR von PGHRT
  \item \href{https://hrtcafe.net/}{HRT Cafe} - HRT-Ressourcen-Aggregator (A.d.Ü. Derzeit im Relaunch begriffen (Jan 2026))
  \item \href{https://transfemscience.org/}{Transfeminine Science} - Infos zu medizinischer Literatur für Transmenschen
  \item \href{http://estrannai.se}{Estrannai.se} - Estradiol-Pharmakokinetik-Spielplatz
  \item \href{https://globoho.moe/}{Globoho.moe} - Reiseführer für die Orchiektomie in Thailand im Rahmen des Medizintourismus
  \item Julias FUE-Ratgeber – KOMMT BALD, ICH DRÄNGLE SIE, SCHNELLER ZU SCHREIBEN
  \item \href{https://docs.google.com/document/d/1-NyE5EY5TTaRRMhk7HlTbKJ7HifjEsA4jlDO1qKQVl0/edit?tab=t.0}{Sky's Feminine Figure Beginner Program} – Ein Trainingsprogramm speziell für Transfrauen
  \item \href{https://docs.google.com/document/d/114sztSw1aVWM2pXLDl9NrHklyvewz3EmFiHiisjM71k/edit?tab=t.0}{Sky's Diät 101} – Ein Leitfaden für eine gesunde Gewichtsregulierung
  \item \href{https://stainedglasswoman.substack.com/p/how-to-maintain-your-penis-function}{Wie man die Erektionsfähigkeit bei HRT aufrechterhält} – Eine ausführlichere Erklärung zum Phänomen “Use it or lose it“ (Nutze es oder verliere es)
  \item \href{hhttps://docs.google.com/document/d/1DXFxzN0XTudPZez_SO61fpqncRLPH_Be_QG_8Pcz9LU/edit?pli=1&tab=t.0}{Biohax-Anleitung Googleslop Edition} – Trans-Masc-DIY-Anleitung
\end{enumerate}

\section*{ÜBER DIE AUTORIN}
\addcontentsline{toc}{section}{ÜBER DIE AUTORIN}

Katie Tightpussy ist eine preisgekrönte Autorin und professionelle Transfrau mit fast zehn Jahren Erfahrung im Bereich Transgender. Zu ihren Errungenschaften gehören die Umwandlung ihres Geschlechts durch die neuartige Technik der Cross-Sex-Hormoninjektionen, die physische Unfähigkeit, den Mund zu halten, und die Nutzung einer Reihe sehr glücklicher Hyperfixierungen im Zusammenhang mit der transbobulären Humoralmedizin. Sie verbringt ihre Tage in der idyllischen ländlichen Umgebung von Los Angeles, wo sie neue Wege zur Weltherrschaft ausheckt und gerne Fahrrad fährt. Medienanfragen können an ihren Agenten unter \href{http://katietightpussy.com}{katietightpussy.com} gerichtet werden.


\section*{OFFENLEGUNGEN}
\addcontentsline{toc}{section}{OFFENLEGUNGEN}

Bei der Erstellung dieses Dokuments wurden keine Roboter-Mädchen verletzt, auch nicht durch die Verwendung generativer Sprachmodelle. Die Autorin will nicht, dass dieses Werk ohne Quellenangabe kopiert oder gescrapt wird. Lasst die armen Roboter-Mädchen in Ruhe.

Die Autorin sagt ausdrücklich, dass sie sich zu Frauen hingezogen fühlt und dass es einen möglichen Interessenkonflikt gibt, weil es durch diesen Guide auf der Welt noch mehr schöne Transfrauen gibt.

 

\section*{DANKSAGUNGEN}
\addcontentsline{toc}{section}{DANKSAGUNGEN}

Auch wenn der Text hauptsächlich von mir stammt, wäre dieses Dokument ohne die Beiträge, Rückmeldungen und Vorschläge von anderen, die an jedem Schritt beteiligt waren, nicht mal halb so gut geworden. Das zeigt mal wieder, dass man eine Transition  am besten nicht alleine angeht.

Vielen Dank an Q, R, RM und S (in alphabetischer Reihung) für die gründliche Überprüfung und die unterhaltsamen Nerd-Gespräche; ich liebe euch alle. Besonderer Dank geht an CB und J für die gründliche Überprüfung, die auch zu einigen sehr guten Ideen geführt hat. Danke an KG für zusätzliche Infos zum Thema Intersexualität. Danke an w [sic] für zusätzliche Ressourcen zum Thema Injektionen. Danke an BIR insgesamt für eine Fülle wichtiger Nerd-Kleinigkeiten. Vielen Dank an c [sic], JTP, K, S und V für die allgemeine Überprüfung. Danke E und S für die Hilfe bei der Webentwicklung. Danke an an alle bei Bluesky, die mich überhaupt dazu ermutigt haben, das hier zu schreiben, und an alle, die über die Jahre ihr Wissen geteilt haben. Und natürlich: ein großes Dankeschön an alle HRT-Nerds, auch wenn wir mal nicht einer Meinung sind, denn wir versuchen alle, das Beste für unsere Community zu tun, wo wir sonst im Stich gelassen wurden. Macht weiter so, Leute.

Ein großes Dankeschön an meinen IB-Chemie-Lehrer von vor vielen Jahren, der meine Lernbereitschaft ziemlich zu Recht angezweifelt hat, obwohl ich jetzt einen Großteil dieses Wissens für die Kunst der Transsexualität nutze; wer hätte das gedacht.

\noindent{Camille dankt: } Antonia, Lina, Jessica, Anahita, Tessa, Krisho, Cassiopeia und Betty, in einer nur ihr bekannten Reihenfolge, sowie dem französischen Schulsystem, der HU-Berlin und foobar2000.

\noindent{Lina dankt }in alphabetischer Reihung: Ani, Camille, Cris, Deena, Emilia, Evelyn, Katie, Kim, Laura, Lilith, Melia, Melina, Nessa, Nia, Pasha, Sasha, Solstng, Zoopy. Ohne euch hätte ich es nicht geschafft. $\heartsuit$  Ruhe in Frieden, James.

\section*{ÄNDERUNGSPROTOKOLL}
\addcontentsline{toc}{section}{ÄNDERUNGSPROTOKOLL}

\noindent \href{https://github.com/Juicysteak117/pghrt/}{Der Quellcode ist hier auf GitHub verfügbar.} (A.d.Ü. Englische Version)

\noindent \href{https://github.com/synthie-cat/pghrt-de}{Der Quellcode der deutschen Version ist hier.}

\noindent Vollständige Zusammenstellung Datum/Uhrzeit: \DTMnow

\noindent(Es gibt keine LaTeXML-Bindungen für \texttt{datetime2}, \texttt{hanging} oder \texttt{hyphenat}, daher sieht die Formatierung etwas komisch aus. Wenn du mir wirklich helfen willst, schreib bitte diese Bindungen!!!) (A.d.Ü. Wenn ihr das macht bitte direkt an Katie wenden!)

\noindent 2025-08-20: Erste Veröffentlichung. 15,9 Tausend Wörter.

\noindent 2025-08-20: Viele Tippfehler und kleinere Änderungen am Wortlaut. Frage hinzugefügt. \ref{8-18}.

\noindent 2025-08-21: Tippfehler sind echt nervig. Frage hinzugefügt \ref{5-27}.

\noindent 2025-08-21: Noch ein paar Änderungen. Ich hab mich entschieden, “WARUM PROG“ aus der Frage zu streichen. \ref{8-17}. 17.0k Wörter.

\noindent 2025-08-22: Kleinigkeiten, Klarstellungen und Tippfehler. 17,2 Tausend Wörter.

\noindent 2025-08-24: Noch ein paar kleine Änderungen, sorry. 17,3 Tausend Wörter.

\noindent 2025-08-27: Wie lange dauert es noch, bis die restlichen Tippfehler peinlich werden? 17,3 Tausend Wörter.

\noindent 2025-08-28: Ein paar Sachen sind jetzt klarer. 17.400 Wörter.

\noindent 2025-08-29: Mehr Klarheit bei den Häufigkeiten in Abschnitt \ref{td}. 17,5 Tausend Wörter.

\noindent 2025-09-01: Sisyphus-Felsbrocken-Meme mit dem Titel “Tippfehler korrigieren“ dot png. 17,5k Wörter.

\noindent 2025-09-07: Spendenlinks auf Anfrage hinzugefügt. Das ist echt nett. 17,5 Tausend Wörter.

\noindent 2025-09-07: Noch ein paar kleine Änderungen. Ein weiteres gängiges Gestagen wurde genauer erklärt. 17,6 Tausend Wörter.

\noindent 2025-09-19: Frage \ref{4-16} hinzugefügt und ein paar Änderungen gemacht. 17,7k Wörter.

\noindent 2025-09-23: Viele verschiedene Erklärungen überall. 18,1 Tausend Wörter.

\noindent 2025-09-24: Ich hab 'ne wichtige Info über Operationen zu Frage \ref{11-1} hinzugefügt. 18,3k Wörter.

\noindent 2025-09-24: “Katie, mein Arzt hat mir gesagt ...“ Es hört einfach nicht auf. 18,5 Tausend Wörter.

\noindent 2025-09-26: Noch mal ein Durchgang mit Klarstellungen. Ja, ich sollte eigentlich ein Git-Diff haben. Tut mir leid, dass ich keins habe. Ich dachte, ich wäre jetzt schon fertig! 18,7 Tausend Wörter.

\noindent 2025-09-30: Ein paar Querverweise für mehr Klarheit eingefügt. 18,7k Wörter.

\noindent 2025-10-02: Mehr Querverweise. Ich werde wahrscheinlich noch mal drübergehen. 18,8 Tausend Wörter.

\noindent 2025-10-02: Hab 'ne fette Warnung über die Zusammenfassung zu Frage \ref{5-13} reingepackt, weil IRGENDJEMAND das Video nicht angeschaut hat, smh. 18,9k Wörter.

\noindent 2025-10-10: Frage \ref{11-42} und Frage \ref{11-43} wie gewünscht hinzugefügt. Ehrlich gesagt hatte ich einfach vergessen, dass Fruchtbarkeit ein Thema ist, lol. Außerdem den Abschnitt “Friends Of PGHRT“ am Ende hinzugefügt. 19,5k Wörter.

\noindent 2025-10-10: Gretchens Version (.txt) hinzugefügt und Formatierung korrigiert. 19,5k Wörter.

\noindent 2025-10-11: Ich hab überall Permalinks reingepackt, juhu! Und endlich ein Git-Repo erstellt. Ich bin jetzt echt erwachsen, wow. 19,5 Tausend Wörter.

\noindent 2025-10-11: Ich hab einen externen Link zu Frage \ref{11-20} hinzugefügt. 19,6k Wörter.

\noindent 2025-10-14: Reduced the pithiness and expanded the usefulness of Question \ref{11-25} per repeat request because I \textit{guess} there is public health utility to doing so. 19.8k words.

\noindent 2025-10-17: Dunkler Modus! Und eine Schriftart-Umschaltfunktion. Plus ein paar Links. 19,8 Tausend Wörter.

\noindent 2025-10-25: Ein paar kleine Änderungen und die Frage \ref{1-12} wurde hinzugefügt. 20,2k Wörter

\noindent 10.01.2026: Bei Frage \ref{5-25} wurde eine Klarstellung hinzugefügt, weil das irgendwo stehen sollte. 20,3k Wörter

\end{document}
