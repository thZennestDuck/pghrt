\documentclass{article}
\usepackage{hyperref}
\usepackage{float}
\usepackage{csquotes}
\usepackage[style=iso]{datetime2}
\usepackage[usenames,dvipsnames]{color}
\usepackage{booktabs}
  \setlength\heavyrulewidth{0.20ex}
  \setlength\cmidrulewidth{0.10ex}
  \setlength\lightrulewidth{0.10ex}

\usepackage[font=normalsize,labelfont={bf}]{caption}
  \captionsetup[table]{aboveskip=3pt}

\hypersetup{
    colorlinks=true,
    linkcolor=blue,
    filecolor=magenta,      
    urlcolor=magenta,
 }
 
\usepackage{graphicx}
\graphicspath{ {../img/} }
\renewcommand{\abstractname}{CLAUSE DE NON-RESPONSABILITE}
\title{UN GUIDE PRATIQUE POUR LE THS FEMINISANT}
\author{\href{https://katea.gay/}{Katie Tightpussy}}
\date{\today}
\setcounter{section}{-1}
\urlstyle{same}

\begin{document}
% TODO: regarder les changements sur le document principal et les intérger
% TODO: œ au lieu de oe
% TODO: relire le document pour le sens général des phrases
% TODO: relire le document pour les fautes d'orthographe
% TODO: vérifier que les titres sont traduits


\maketitle
\tableofcontents
\begin{abstract}
  Je ne suis pas médecin. Je ne travaille pas dans le domaine de la médecine. Je ne suis pas une professionnelle du domaine médical ou d'un domaine associé. Je suis une amateure qui offre des opinions d'amateure basées sur mon mes expérience et mes études. Toutes les informations et les affirmations ci-dessous doivent donc être traitées comme des opinions et non pas des vérités scientifiques établies ou des avis médicaux. Ce guide priorise les éléments découverts par la communauté, la qualité de la recherche scientifique sur ce sujet nous faisant défaut. En bref, c'est votre dos.
\end{abstract}


\section{AVANT-PROPOS}

Le but de ce document est de cataloguer mes pensées et mes opinions à propos du traitement hormonal de subsitution (THS) féminisant, car je pense que les différents wikis déjà existants manquent d'efficacité. Ce sont des ressources inestimables, mais de mon point de vue, ces wikis ne sont pas destinés aux personnes qui cherchent des conseils clairs et rapidement traduisibles ne action, mais plutôt à ceux qui cherchent à connaître des processus biologiques liés à la transition. Mon but ici est de fournir une référence exhaustive et rapide qui contient des réponses à toutes les questions que je vois souvent sur comment faire son THS de manière efficace et sûre, grâce à des connaissances que j'ai accummulé à travers les années, dans le but de démystifier ce processus qui sauve des vies, autant pour les personnes qui cherchent à savoir si le THS est pour eux que pour pour les transexuels vétérans. Je pars du principe que tu as déjà une idée de ce que fait le THS. Au cas où tu l'ignorais: le THS a beaucoup plus d'efftes que tu ne le penses. Ce qui est une bonne chose. \textbf{Changer son sexe, c'est fun et cool. Vraiment une expérience que je recommande.} Tu mérites des soins de transition de qualité, et tu es la personne la plus capable pour prendre les décisions qui te conviendront le mieux. J'espère que ce document t'aidera pour prendre tes décisions, et un jalon dans ton apprentissage si tu souhaite continuer à accumuler des connaissances dans le domaine.

Aussi, reste loin des sub-reddits trans. Fais-moi confiance là-dessus, ok? Au minimum évite /r/mtf, lui est vraiment horrible. Ce ne sont pas des endroits sains ou de sources de sagesse. Ces endroits te versent directement des vers pourris dans le cerveau, pendant des années. C'est le meilleur conseil que je puisse te donner.

Pour les mecs, certaines sections de ce documents sont tout de même très à propos, mais il y a évidememnt de grosses différences de but et de résultats. \href{https://docs.google.com/document/d/1DXFxzN0XTudPZez\_SO61fpqncRLPH\_Be\_QG\_8Pcz9LU/edit?tab=t.0}{Ce guide pour le THS masculinisant} \textcolor{red}{(Attention : lien Google Docs (et guide en anglais))} à l'air vraiment bien, mais je ne l'ai pas complètement examiné, donc reste vigilant. Ils devraient vraiment faire un mec trans comme Katie Tightpussy. Un type qui s'appelerait genre Oliver Longdick, ou Xavier pourquoi pas. 

\textbf{Si jamais tu veux aider ce projet,} \href{https://cash.app/Katitties}{CashApp}, \href{https://ko-fi.com/katitties}{Ko-Fi}, et \href{https://account.venmo.com/u/katitties}{Venmo} marchent tous. Merci beaucoup!

\subsection*{Comment utiliser ce document}

Ce document est organisé comme une série de questions/résponses, de telle sorte que chaque question et chaque section forment une suite logique. Je vous encourage fortement à le lire séquentiellement, ça se lit comme une conversation qui devrait répondre à la totalité de vos questions (incluant celles que vous ne saviez pas que vous vous posiez), même si c'est très long. Prends ton temps et lis le à ton rythme. 

You can use the table of contents to navigate to a particular section or question as needed, especially when re-visiting. I recommend saving this page / document so that you can refer back to it any time you have questions about your HRT. It is a lot to absorb up front, so it’s okay if it doesn’t! No rush on any of this.
Sinon, tu peux utiliser la table des matières pour naviguer directement vers une section ou une question particulière, ce qui est très utile si ce n'est pas ta première visite. Je recommande de garder ce document dans un coin pour y revenir à chaque fois que tu as des questions à propos du THS. C'est un peu lourd à absorber en un coup, donc pas de soucis si ça te prends quelques relectures pour tout intégrer, ne te presse pas. 

\noindent\textbf{\href{pghrt.pdf}{Ce document peut aussi être téléchargé en pdf en cliquant ici. Vraiment, télécharge-le et garde-le quelque part, on sait jamais.}}

\noindent\href{pghrtgretchensversion.txt}{Si tu préfères, tu peux lire une version de ce texte en .txt style années 90-2000.} Par contre je ne garantis pas qu'il soit à jour.

\noindent\textbf{If you are interested in doing a translation or any other alternate version, please get in touch!}



\section*{DEDICACE}
\addcontentsline{toc}{section}{DEDICACE}

Ce document est dédicacé à toutes nos soeurs qui ne sont plus. Puisse-t-on porter la lumière de leur torche dans un jour nouveau.
%  TODO: j'ai fait une traduction littérale de la deuxième phrase, je sais pas trop comment ça rend ceci dit

\section{INTRODUCTION}

\subsection{Est-ce que la prise d'oestrogènes est risquée ?}

Avec des hormones modernes, et donc bioidentiques, le THS est tout ce qu'il y a de plus sûr. Au final, tout ce que tu fais, c'est changer le carburant principal qui fais tourner ton corps pour obtenir un équilibre un peu différent avec des hormones qui étaient déjà dans ton corps. Même si c'est un peu plus complexe que ça quand on va chercher à optimiser le processus, l'idée de base de changer sa biologie permet pas mal de largeurs. Le corps est assez malléable et vas pouvoir ajuster ton traitement vers quelque chose qui te fais aller mieux.

\subsection{Quel voie d'administration prendre pour mon oestrogène ?}

Par injection. C'est le moyen le plus facile, prédicible, sûr, efficace et abordable de faire une transition hormonale. On dit même que pour certains, le moment de l'injection devient un rituel attendu, et que pour d'autres,  

\noindent\underline{\textbf{Rappel important : Peu importe la manière dont tu prends de l'oestrogène, c'est toujours mieux que de ne pas en prendre.}}

\subsection{Pourquoi est-ce que tu ne recommandes pas les pilules, les patches ou les gels ?}

La raison principale est que toutes les autres voies d'administrations ont des inconvénients que les injections n'ont pas. C'est pas qu'elles ne fonctionnent pas, mais tu mérite mieux que de devoir supporter des inconvénients parfois durs à vivre. Mais je le répète : \textbf{toutes les formes de THS permettent d'avoir des résultats satisfaisants}. Ca ne veut pour autant pas dire que toutes les formes de THS se valent.

\subsection{Est-ce que le dosage d'oestrogène est équivalent selon la voie d'adminstration choisie ?}

Non. Et c'est suffisamment important pour ne pas être relègué dans la Section \ref{MM} "Mythes et autres". \textbf{On ne peut pas comparer les dosages d'oestrogènes sans prendre en compte la voie d'administration choisie}. 1mg en pilule n'est pas la même chose qu'1mg en injections par exemple. Les différentes voies d'administration ont des propriétés différentes, ce qui affecte la quantité d'oestrogènes absorbée dans le corps (\text{"biodisponibilité"}), à quelle vitesse, et donc la demi-vie corresponante.

\subsection{C'est quoi une demi-vie ?}

Pour faire simple, la \textit{demi-vie} d'une substance est le temps qu'il faut pour que la moitié de cette substance soit éliminée. Dans le contexte du THS, c'est ce qui détermine coombien de temps un dosage reste actif dasn ton système, et donc la fréquence de prise d'hormones. C'est ce qu'on va appeler ton cycle hormonal, qui forme une courbe. Tes niveaux montent, arrivent à un pic, puis retombent. Les propriétés de cette courbe (comment tes niveaux d'oestrogène changent en fonction du temps) sont importants.

\subsection{C'est quoi le soucis avec les pilules ?}

Leur problème principal, c'est qu'elles sont associées à un risque plus élevé de caillots snaguins et de coagulation du foie. On peut diminuer ces risques en partie en prenant ces pilule de façon sublinguale ou buccalle (en faisant dissoudre la pilule sous ta langue ou entre ta gencive et ta joue, respectivement), plutôt qu'oralement (en avalant la pilule normalement), pour éviter de passer directement par le foie pour métaboliser l'oetrogène. Cependant, même avec les méthodes des voies sublinguales ou buccales, il est fréquent de tout de même avaler une partie du contenu de la pilule, donc on peut imaginer qu'il reste un risque résiduel. Attention, le risque reste très peu élevé, (e.g. \textit{le paracetamol} est plus risqué que l'oestrogène par un ordre de grandeur), mais \textbf{ces complications sont d'autant plus impactante qu'elles se composent avec les risques liés à la prise de nicotine sous oestrogène}. Pour en savoir plus, tu peux consulter la Question \ref{11-2}. 

A part ça, il y a d'autres soucis liés aux pilules qui viennent principalement de deux de ses caractéristiques : 1) elles ont une demi-vie très courte et une mauvaise biodisponibilité, et 2) elles nécessitent souvent de les accompagner avec des anti-androgènes. Le premier point fait que les pilules sont difficiles à utiliser en monothérapie (dont on dicutera plus en détail ensuite), comparé aux injections. Le deuxième fait qu'on se retrouve avec un assortimment d'effets secondaires liés aux antiandrogènes que tu prendrais avec (voir section \ref{AA} "ANTIANDROGENES"). Prises ensembles, ces caractéristiques et autres degrés de variabilités font qu'on a plus souvent avec les pilules des mauvais régimes remplis d'effets secondaires (comme par exemple un manque d'énergie/libido et des lenteurs à obtenir des résultats) par rapport à d'autres vois d'administration. Aussi, il est plus difficile de stocker des pilules et selon la façon dont tu les obtiens, plus chères que les fioles d'injections. A noter aussi que si tu essaye de faire importer de grandes quantités de pilules depuis des distributeurs étrangers pourrait faire tiquer les douanes et mener à une saisie des médicaments, ce qui t'expose à une perte financière, voire des risques de poursuites judicières. \textbf{Si on te demande, tu ne sais pas qui a commandé toutes ces pilules.}

\textbf{Si, pour quelque raison que ce soit, tu utilises des pilules, s'il te plaît prends 4-8mg en voie sublinguale en espaçant tes prises dans la journée.} Tout ce qui est en dessous de 4mg n'est probablement pas un dosage suffisant.

\subsection{Qu'est ce qui ne va pas avec les patches ?}

\begin{itemize}
  \item C'est cher (plus que les pilules);
  \item Plus difficile de s'en procurer en DIY (seulement via des marchés gris);
  \item Te demandent généralement un antiandrogène (voir la  Section \ref{AA} “ANTIANDROGENES”);
  \item Peuvent causer une irritation de la peau;
  \item Nécessistent d'être appliqués 24h/24;
  \item Se décollent souvent (si ça vous arrive, demandez des pansements qui couvrent le patch à la pharmacie);
  \item Ne permettent pas une absorbtion uniforme (à cause de la chaleur par exemple);
  \item Difficile de faire des stocks (difficile à acheter en grandes quantités);
  \item Permettent souvent pas de dépasser des niveaux équivalents à une ménopause même avec plusieurs patches en même temps.
\end{itemize}

\subsection{Et le gel alors ?}

\begin{itemize}
  \item Difficile à doser avec précision, ce qui mène à des niveaux non-uniformes;
  \item Requires regular application of goop due to a relatively short half-life;
  \item Requièrent une application régulière de truc poisseux à cause d'une demi-vie relativement courte;
  \item Peut-être sale (poisseux);
  \item Tu risques d'exposer les autres par contact
  \item Te demandent généralement un antiandrogène (voir la  Section \ref{AA} “ANTIANDROGENES”);
\end{itemize}

Il faut cependant noter que le gel nécessite peu d'éléments pour en produire soi-même ce qui peut-être avantageux selon les circonstances.

\subsection{Pourquoi pas les implants?}

\begin{itemize}
  \item Généralement beaucoup plus cher que n'importe quelle autre option;
  \item Peu de possibilités d'en obtenir;
  \item Les périodes d'ajustements de doses sont très espacées;
  \item Un implant déféctueux peut te faire avoir des niveaux insuffisants;
  \item Un implant cassé ou écrasé peut causer des niveaux trop hauts de façon inattendue;
  \item Impossible à obtenir en DIY.
\end{itemize}

Ce dernier point en particulier veut dire qu'il n'est quasiment pas possible de s'en procurer. C'est peut-être même la première fois que tu en entends parler. Tu vois le soucis avec ça ?

\subsection{Et pour les sprays ?}

Ils sont encore assez peu utilisés et on a peu de retours donc il n'y a pas grand-chose à dire à leur propos, mais ils ont pas mal d'avantages / inconvénients en commun avec le gel. Je note ça là essentiellement pour rappeler qu'ils existent.

\subsection{La différence est vraiment si énorme que ça ?}

\textbf{Oui.} C'est tellement le cas que j'ai écrit tout ce truc juste pour arrêter de me répéter tout le temps et envoyer un lien vers ici à la place. Un régime d'injection bien dosé est la meilleure forme d'oetrogène qu'on a pour obtenir des niveaux corrects en monothérapie.

 

\section{POURQUOI LES INJECTIONS}

\subsection{Qu'est ce qui fait que les injections, c'est si bien ?}

La régularité. La régularité est le truc important pour un THS. La régularité, ça veut dire que ton régime est stable, et la stabilité, c'est bien. Même les pires formes d'injections (on détaillera ça plus tard) te créront un cycle hormonal plus prédictible que toutes les autres voies d'administration, ce qui a beaucoup d'avantages.

\subsection{Est-ce qu'on a besoin d'antiandrogènes avec des injections ?}

En général, non. Si tu as un cycle d'injection qui est correctement dosé et espacé, qui donne des niveaux d'oetrogène suffisamment hauts tout le temps pour arrêter la production naturelle de testosterone, rendant les antiandrogènes inutiles, ce qui est préférable dans la plupart des cas. C'est ce qu'on appelle la \textit{“monotherapie”}.

\subsection{La monthérapie, ça marche comment ?}\label{2-3}

Pour faire simple, ton cerveau se fout un peu de savoir quel type d'hormone sexuelle est dans ton corps, du moment que tu en as en assez grande quantité. Si tu as tout le temps assez d'hormones dans ton corps, il ne voit pas d'intérêt à en produire plus. C'est sur cette "régularité" que les injections résussissent là où les autres voies d'administration sont plus en peine. Faire une monothérapie avec des pilules, par exemple, est quasiment impossible dans la plupart des situations. Si tu veux que je précise un peu, si on regarde l'axe HPG, la production \textit{d'hormone lutéinisante} (LH) et \textit{d'hormone folliculo-stimulante} (FSH) est supprimée par les niveaux élevées \textit{d'estradiol}, ce qui inhibe la production de GnRH et donc par extension de la production de testosterone dans les testicules. 

\subsection{Comment ça se fait que les injections soient moins risquées pour la santé ?}

Parce qu'on n'a pas besoin de les utiliser avec des antiandrogènes (voir la Section \ref{AA} “ANTIANDROGENES”), le risque à long terme associé à l'utilisation des antiandrogènes sur la santé est évité. Si tu utilises de l'oestrogène bioidentique et qu'elle ne passe pas par ton foie (voir la question \ref{11-1}), alors tu es au plus proche possible d'une prodcution naturelle d'oetrogènes par ton corps, ce qui enlève la plupart des risques pour ta santé

\subsection{Mais y'a pas des risques associés au fait de s'injecter une substance dans le corps ?}

En soi oui, mais avec un peu de technique, tout ce que tu risques c'est un bleu (see Section \ref{ts} “TECHNIQUES ET MATERIEL”). C'est un peu comme faire du vélo, une fois que tu sais faire, faut vraiment le vouloir pour faire n'importe quoi.
% TODO: vérifier que c'est le même nom de section

\subsection{D'accord, mais en quoi les injections sont plus simple que les autres voies d'administration ?}

Parce qu'une fois que tu as un régime stable, t'es tranquille. Tu t'en occupe moins souvent (e.g. une injection par semaine contre plusieurs pilules par jour), tu as des dosages précis, l'apport d'oestrogènes ne s'arrête pas en plein milieu de cycle à cause d'un patch qui se décolle, et tu n'as pas besoin de te déplacer pour aller voir un médecin souvent.

\subsection{Comment ça se fait que les injections soient aussi abordables ?}

Pour faire simple, c'est parce que tu as besoin de moins de principe actif. Une fiole de 5 mL contient à peu près un an de THS ne contient que 200mg d'oestrogènes, alors que l'équivalent en pilules par exemple (4mg * 365 jours = 1460 mg) en contient beaucoup plus. C'est une comparaison qui a ses limites, mais ça donne une idée. Une autre comparaison marrante : tu peux faire tenir une fiole qui contient 1 à 2 ans d'oestrogènes dans une boîte de 3 mois de pilules. 

\subsection{But I don’t have insurance / my insurance won’t cover it / pills are cheaper than injections with my insurance / injections are not available in my country / my doctor won’t prescribe injections?}
\subsection{Je ne peux pas en acheter / me faire rembourser en pharmacie, je fais comment ? (ou mon médecin ne veut pas m'en prescrire)}

Va voir la section \ref{sv} "OU TROUVER DES INJECTABLES". Ca va changer ta vie, et probablement te radicaliser dans la foulée.
% TODO: nom de section à regarder

\subsection{Est-ce que je peux passer aux injections même après quelques mois / années sous THS ?}

\textbf{Oui.} Je ne garantis rien, mais beaucoup de gens racontent avoir vu une différence notable après être passées aux injections même après quelques années sous THS. On parle d'un plus grand développement mammaire, une meilleure santé mentale, moins de soucis liés aux effets secondaires des anti-androgènes ou aux autres voies d'administration des oestrogènes. Vraiment, saute le pas, ça en vaut la peine.
% (Note de traduction: je fais partie de ces personnes qui ont vu une différence notable en passant aux injections)

\subsection{Oui mais j'ai peur des piqûres...}

On va pas se mentir, les piqûres dont peur, surtout au début. Beaucoup de gens n'aiment pas ça, parce qu'instintivement tu ne veux pas faire des trous dans ton corps, mais avec une bonne technique et de bons instruments, ça ne fait presque pas mal. Il y a énormément de gens qui avaient des cas sérieux de bélénophobie (peur des aiguilles) et qui maintenant trouvent l'expérience de l'injection ennuyante. C'est une peur commune et normale, mais c'est surmontable et ça vaut le coup de la surmonter. "Oh ce n'est pas si terrible que ça" est une réaction assez commune. Comme le dit le mantra: fais le en ayant peur, ça ira. 

\subsection{Est que la sensation des injections est semblables à une prise de sang ou à un vaccin ?}

Non. Une prise de sang utilise des aiguilles beaucoup plus larges et on pique dans un endroit beaucoup plus sensible, tout en te drainant du sang, ce qui est déplaisant en temps normal. Les vaccins contiennent des organismes qui causent des réactions immunitaires douloureuses parce que c'est des vaccins. Les injections d'oestrogènes ajoutent une petite quantité d'hormones dans ton corps qui te fait te sentir bien parce qu'après tu as des hormones dans ton corps. Je suis sûre que tu vois la différence. L'acte de s'injecter soi même peut aussi être plus facile que ce soit quelqu'un d'autre qui le fasse, selon ta personnalité.

\subsection{Est-ce qu'il y a des outils pour faciliter les injections ?}

Oui. Il existe des auto-injecteurs qui peuvent être très utiles si tu as des soucis moteurs par exemple. Pour plus d'informations, va voir la question \ref{5-21}, ou continue de lire.

\subsection{Tu comprends pas, \textit{je} suis spécial.e et je ne peux pas faire d'injections parce que j'ai des os en verre et une peau en papier et\textemdash{}?}

Je comprends que tu aies peur, mais si tu ne veux pas faire d'injections sous aucune circonstance et que tu n'as pas de contrindication comme le fait d'être hémophile, alors ne le fait pas. Tu peux juste dire ça, c'est ok. Quand tu changera d'avis, ce guide sera toujours là. Et si tu ne changes pas d'avis, pas de problème.
 

\section{TYPES ET DOSAGES}\label{td}

\subsection*{Vocabulaire clé}
\addcontentsline{toc}{subsection}{\textemdash{} Vocabulaire clé}

\subsection{Quels sont les différents types d'oestrogène injectable ?}

Les quatre types principaux utilisés pour le THS sont \textit{l'estradiol valerate} (EV), \textit{l'estradiol cypionate} (EC), \textit{l'estradiolm enanthate} (EEn) et \textit{l'estradiol undecylate} (EUn). Chacun de ces composés est un "ester" de \textit{l'estradiol} et va être converti en \textit{estradiol} dans ton corps.

A noter que dans certaines régions, les pilules sont vendues sous le nom \textit{estradiol valerate}, qui peut porter à confusion. Cette section ne se réfère qu'à sa forme injectable.

\subsection{Quelles sont les différences entre les types d'oetrogènes injectables ?}

La seule différence intéressante entre les esters est que chacun a une demi-vie différente, ce qui change leur courbe hormonale résultante affectant le dosage et sa fréquence.

\subsection{Un type d'oestrogène injectable est-il meillleur pour la féminisation qu'un autre ?}

\textbf{Non.} Les différences affectent uniquement le dosage et la fréquence de prise, ce qui change l'exprérience de manière qualitative. Ca peut rendre un ester préférable à un autre, mais les 4 types fonctionnent convenablement et partagent tous les avantages des injections.

\subsection{Quel type d'oestrogène choisir si j'ai le choix ?}

Si tu as le choix, \textit{l'estradiol enanthate} est préférable pour la plupart des gens au vu des niveaux exceptionnelement stables qu'il permet d'avoir, avec l'inconvénient que dans la plupart des pays ce choix est seulement possible si tu le fais en DIY (voir la section \ref{sv} "OBTENIR DES FIOLES"). Si tu passes par un docteur, tu auras probablement l'option de prendre de \textit{l'estradiol cypionate}, mais probablement seulement peu concentrée, ce qui peut être ennuyant en fonction de ta tolérance pour des gros volumes d'injections. L'injectible le plus prescrit (surtout aux USA), \textit{l'estradiol valerate}, permet toujours d'avoir des bons résultats, mais il est un peu ennuyant sur certains points qui font qu'il n'est pas préférable d'en prendre si tu as le choix (i.e. quand tu fais du DIY). Continue à lire.
% Du coup si je localise, il faudrait que je regarde selon les pays francophones comment ça fonctionne. En France, pas d'injections, Canada oui, Belgique ? et pour les autres pays francophones ?

\subsection{C'est quoi la concentration ?}

Les fioles d'oestrogènes sont faites à partir d'oestrogènes contenues dans une solution organique. La \textit{concentration} d'une fiole est la quantité d'oestrogène contenu dans cette solution. C'est un ratio de la masse par rapport au volume de la fiole. En d'autres termes : pour chaque mililitres d'huile (la mesure de volume), il y a tant de miligrammes d'oestrogènes (la mesure de masse). Tu va souvent voire des concentrations listées selon le volume total de la fiole (e.g. 200mg/5mL) mais il est toujours préférable de simplifier cette fraction (donc 40mg/mL) dans ce cas. \textbf{Les concentrations communes sont 5mg/mL, 10mg/mL, 20mg/mL, 40mg/mL, et occasionnellement 50mg/mL.}

\subsection{Qu'est ce qu'on veut dire par "dosage et fréquence ?"}

\textit{Le dosage} et la \textit{fréquence} sont les deux facteurs qui déterminent ton cycle hormonal. \textit{Le dosage} est la quantité d'oestrogènes que tu mets dans ton corps (mesuré en mg), et la \textit{fréquence} est à quelle fréquence tu mets de l'oestrogènes dans ton corps (mesuré en jours ou en semaines). Tu vas souvent entendre le mot "régime d'oestrogène", ce qui référence tout ce qui est lié au THS que tu prends et à quelle fréquence.

\subsection{Comment est-ce que je sais quel dose prendre ?}

Tu obtiens ton dosage en multipliant la concentration de ta fiole par le volume que tu es en train d'injecter \[Concentration (mg/mL) * volume (mL) = dosage (mg)\] \textbf{J'insiste sur le fait que le volume à lui seul ne suffit pas pour avoir un dosage correct.} Si tu veux une analogie, tu peux le voir avec la pâtisserie: tu ne peux pas dire à quelqu'un "fais cuire ça au four pendant 45 minutes", sans lui dire à quel température faire chauffer le four.

\subsection{Est-ce que je pourrais avoir un exemple de calcul de dosage ?}

Le calcul est super simple, promis ! J'ai mis une table de référence dessous qui compare en fonction de la concentration et du volume pour quelques dosages communs. Pas la peine d'être plus précis.e que 2 chiffres après la virgule. De toute façon, tu n'auras jamais une seringue assez précise pour doser 0.153mL par exemple. C'est une marge d'erreur acceptable qui ne fera aucune différence pour ce qu'on cherche à faire.

\begin{table}[]
\centering
\caption{Exemples de Dosages pour des couples Concentrations / Volume Fréqents}
\label{tab:concentrations}
\begin{tabular}{@{}lllll@{}}
    \toprule
    \multicolumn{1}{c}{} & \multicolumn{4}{c}{Concentrations (mg/ml)} \\
    \cmidrule(rl){2-5}
            & 5    & 10  & 20 & 40    \\
            \cmidrule(rl){2-5}
Dosage (mg) & \multicolumn{4}{c}{Volume (mL)}  \\
    \cmidrule(r){1-1} \cmidrule(lr){2-5} 
4        & 0.8  & 0.4 & 0.2  & 0.1      \\
5        & 1    & 0.5 & 0.25 & 0.13   \\
6        & 1.2  & 0.6   & 0.3  & 0.15     \\
7        & 1.4  & 0.7 & 0.35  & 0.18  \\
8        & 1.6  & 0.8   & 0.4  & 0.2    \\
9        & 1.8  & 0.9 & 0.45  & 0.23 \\
10       & 2    & 1   & 0.5  & 0.25   \\
    \bottomrule
\end{tabular}
\end{table}

\textbf{Comment lire ce graphe :} Commence par prendre la dose que tu veux sur la gauche, et en choisissant avec la colonne correspondant à la concentration de ta fiole, tu obtiens le volume correspondant sur la droite. Tu peux déjà remarquer que les volumes nécessaires pour les fioles concentrées à 5mg/mL ne sont pas terribles. C'est parce que les fioles concentrées à 5mg/mL ne sont pas terribles.

\subsection{Comment est-ce que je convertis les dosages entre différents esters ?}

\textbf{Tu ne le fais pas.} Comme ils se comportent de manière différentes, on ne peut pas faire de "conversion" de dosages. Si tu changes d'ester, tu dois (re)commencer à un dosage typique pour ce nouvel ester et ensuite retrouver ton cycle de croisière à partir de là. Tu peux faire des comparaison entre ester, mais on n'a aucune méthode pour convertir des doses.

\subsection{Comment est-ce que je peux comparer les différentes courbes en fonction du dosage et de l'ester choisi ?}

Si tu veux fouiller le sujet, je recommande \href{http://estrannai.se}{estrannai.se} que je trouve très bon. Garde en tête que ce n'est pas nécessaire, mais c'est un bon outil pour faire quelques comparaisons. \href{https://estrannai.se/\#i0__cu,7,7,1-cu,5,7,3-cu,5,7,2}{Voici un exemple de comparaison entre quelques dosage hebdomadaires typiques} qu'on va détailler un peu plus dès à présent.

\textbf{A noter que les dosages que je vais lister ci-dessous devraient suffire pour une dose minimum dans la plupart des cas.} Commence avec le dosage le plus bas, et augmente si besoin. Plus ne veut pas nécessairement dire mieux, mais on va voir ça plus en détail plus tard. Ces dosages fonctionneront peu importe comment tu as obtenu ta fiole.

\subsection*{A la rencontre de nos Esters}
\addcontentsline{toc}{subsection}{\textemdash{} A la rencontre de nos Esters}

\subsection{Comment est-ce que je dose mon \textit{estradiol valerate}?}

Tu peux soit predre une plus petite dose deux fois par semaine, ou une plus grosse dose une fois par semaine avec de \textit{l'estradiol valerate}. C'est en fonction de ta tolérance et de ton confort. En gros, il te faut 1mg pour chaque jour de ton cycle pour une injection tous les 3 à 7 jours. \textbf{Si tu fais une injection hebdomadaire, je recommande entre 6 et 8 mg}, mais 4 à 5mg tous les 5 jours est tout aussi bien. \textbf{Tu ne dois pas dépasser 7 jours entre deux injections.} Un cycle hebdomadaire pousse déjà suffisamment la demi-vie de l'ester. Toute fréquence plus basse est hautement déconseillée pour éviter des effets secondaires liés à la variance (voir la Question \ref{7-3}).
% TODO: je ne veux pas créer d'anxiété sur le ATTENTION NE PAS DEPASSER 7 JOURS SINON CONSEQUENCES, faudra peut-être rephraser

Je rappelle que dans certains endroits, les pilules sont vendues sous le nom \textit{estradiol valerate}, ce qui peut porter à confusion, cette section ne réfère qu'à la forme injectable du produit.

\subsection{Qu'est-ce qui caractérise la courbe hormonale de \textit{l'estradiol valerate} ?}

\textit{L'Estradiol valerate} est l'ester qui a le plus une courbe en forme de pointe. Il monte rapidement à un pic quelques jours après l'injection et retombe durement et rapidement. Cette relative instabilité peut être déplaisante en fonction de ta sensibilité, mais tu peux amoindrir les effets en ajustant le dosage et la fréquence.

 \begin{figure}[H]
     \centering
     \includegraphics[width=1\linewidth]{ev.png}
     \caption{Serum Estradiol (pg / ml) d'Estradiol Valerate en fonction du temps (jours) }
     \label{fig:ev}
 \end{figure}

\subsection{Comment est-ce que je dose mon \textit{estradiol cypionate}?}

\textit{L'Estradiol cypionate} peut être prise de façon hebdomadaire sans soucis. \textbf{On prend généralement une dose entre 5 et 7 mg.} Je ne recommande pas une fréquence plus basse qu'une fois tous les 7 jours (e.g. une fois tous les 10 jours) parce que l'oestrogène sera moins efficace que prise de façon hebdomadaire, vu qu'il faut une dose de plus en plus haute pour avoir des niveaux acceptables. Tout ce qui dépasse les 7 jours entre prise peut mener à des effets secondaires à cause de la variance (voir la Question \ref{7-3}). 

\subsection{Qu'est-ce qui caractérise la courbe hormonale de \textit{l'estradiol cypionate} ?}

\textit{L'Estradiol cypionate} pardonne un peu plus que \textit{l'estradiol valerate}. La courbe ne progesse pas aussi vite et varie beaucoup moins entre sa partie haute et sa partie basse, mais on remarque tout de même une montée et descente visibles sur un cycle hebdomadaire typique.

 \begin{figure}[H]
     \centering
     \includegraphics[width=1\linewidth]{ec.png}
     \caption{Serum Estradiol (pg / ml) d'Estradiol Cypionate en fonction du temps (jours) }
     \label{fig:ec}
 \end{figure}

\subsection{Comment est-ce que je dose mon \textit{estradiol enanthate}?}

\textit{L'Estradiol enanthate} peut se prendre sur cycle hebdomadaire, qui peut être étendu à une injection tous les 10 jours si ça t'arrange. Tu peux techniquement l'étendre plus que ça, mais je ne le recommande pas parce que tes niveaux seront de plus en plus instables. \textbf{On prend généralement une dose hebdomadaire entre 4 et 6mg}, ou 5 à 7mg pour une injection tous les 10 jours. Je préfère la solution hebdomaire de toute façons pour des raisons de facilité de planification, les extensions jusqu'à 10 jours n'apportant rien de particulier. 

\subsection{Qu'est-ce qui caractérise la courbe hormonale de \textit{l'estradiol enanthate} ?}

\textit{L'Estradiol enanthate} est le standard pour l'oestrogène injectable. Il a une courbe extrêmement plate (i.e. qui a peu de variance) sur le cycle hebdomadaire typique. Ca permet d'avoir des niveaux très réguliers sans avoir trop de soucis liés aux effets secondaires liés à la variance (voir la Question \ref{7-3}).

 \begin{figure}[H]
     \centering
     \includegraphics[width=1\linewidth]{een.png}
     \caption{Serum Estradiol (pg / ml) d'Estradiol Enanthate en fonction du temps (jours) }
     \label{fig:een}
 \end{figure}

\subsection{Comment est-ce que je dose mon \textit{estradiol undecylate}?}

\textit{L'Estradiol undecylate} est capable de dépasser de loin le cycle hebdomadaire pour aller vers le mensuel ou le trimestriel. Cependant, on n'a pas de standard ou de base pour déterminer une dose recommandée. Les éléments qui affectent la façon dont une injection d'oestrogène sont absorbées (\textit{"la pharmacocinétique"}), qui sont négligeables pour les autres esters sont plus influents pour \textit{l'estradiol undecylate}. A cause de ça, cet ester est toujours un terrain très expérimental qui dépasse le sujet de ce guide. Tu peux aller consulter un grimoire de sorcière pour savoir comment s'injecter sous la pleine lune.
% La traduction de la vanne est pas top ?

\subsection{Qu'est-ce qui caractérise la courbe hormonale de \textit{l'estradiol undecylate} ?}

On sait pas trop. Il y a trop peu de données pour avoir une vision claire, et il y a beaucoup de variables à prendre en considération. C'est quelque chose que tu peux rechercher et expérimenter si ça d'intéresse, mais c'est un saut dans l'inconnu et tu dois d'abord comprendre les risques induits par le fait d'être un rat de laboratoire, donc je ne recommande pas à moins de savoir ce que tu fais.

 \begin{figure}[H]
     \centering
     \includegraphics[width=1\linewidth]{moon.png}
     \caption{La lune}
     \label{fig:moon}
 \end{figure}

 

\section{PRISES DE SANG ET NIVEAUX}

\subsection*{Obtenir des résulats}
\addcontentsline{toc}{subsection}{\textemdash{} Obtenir des résulats}

\subsection{A quelle fréquence dois-je tester mes niveaux ?}

Tant que tu es en train d'affiner ton dosage, tu dois tester relativement fréquemment. Pour suivre chaque ajustement à ton régime, tu devrais laisser 1 ou 2 mois à tes niveaux pour se stabiliser, et tester dès que as atteint un rythme de croisière.

\subsection{Dois-je tester mes niveaux avant de commencer mon THS ?}

Ce n'est pas nécessaire, parce que tes niveaux de testosterone seront trop hauts et ceux d'oestrogène trop bas, donc ce n'est pas une donnée vraiment intéressante. Cependant, des prises de sang de routines (i.e. bilan lipidique et autres) sont recommandés pour ta santé de toute façon. L'exception étant si tu penses avoir une condition intersexe qui peut affecter ton régime de THS vu que c'est quelque chose qu'on peut des fois détecter dans une prise de sang préliminaire.

\subsection{Do I have to test my levels if I haven’t changed my dosage in a long time?}
\subsection{Est-ce que j'ai besoin de tester mes niveaux si je n'ai pas changé mon dosage depuis longtemps ?}

Il n'y a pas vraiment de raison de le faire, vu que tu n'as rien changé, rien n'a changé. Si tu veux avoir l'esprit tranquille ou si tu as changé des aspects de ta routine ou ta façon d'acheter ton THS, et de temps en temsp ton docteur va te le demander, mais tu ne devrais pas voir des différences énormes. Par contre, si tu es en train d'expérimenter avec \textit{l'estradiol undecylate}, tu devrais faire une prise de sang au moins tous les 3 mois quoi qu'il arrive. 

\subsection{Je ne peux pas avoir d'ordonnance pour une prise de sang, que faire ?}

Tu as quand même le droit de demander une prise de sang, qui ne sera par contre pas remboursée.
% TODO: localiser ici, je sais que c'est possible, mais ça serait top de dire comment
% Autre point intéressant, on pourrait parler du fait d'être en DIY et être suivi.e par un médecin, c'est possible 

\subsection{Je ne peux pas me permettre de payer une prise de sang, est-ce que je peux faire un THS quand même ?}

Même s'il est évident qu'il vaut mieux avoir l'information que de ne pas l'avoir, un THS est extrêmement sûr et pour des doses habituelles, ne posera pas de soucis. Tu devras juste plus te baser sur comment tu te sens et ce que tu observe de ton corps.

\subsection{Qu'est-ce que je dois tester ?}

\textit{L'estradiol} (E2) et \textit{testosterone totale} (T) au minimum parce que c'est les choses principales qui nous intéressent. Les niveaux de \textit{Sex hormone binding globulin} (SHBG), de \textit{dihydrotestosterone} (DHT), \textit{l'estrone} (E1), et de \textit{prolactine}(PRL) peuvent aussi être utiles si tu a des complications parce qu'elles peuvent être utiles pour établir un diagnostic. Les niveaux de \textit{d'hormone lutéinisante} (LH) et \textit{d'hormone folliculo-stimulante} (FSH) peuvent te dire si ton axe HPG est inactif, ce qui est la fondation de la monothérapie (voir la question Question \ref{2-3}). Mais je le répète : \textbf{\textit{L'estradiol} and \textit{Testosterone totale} sont les éléments principaux à regarder.} 
% TODO: traduire SHBG ?

\subsection{A quel moment de mon cycle dois-je faire ma prise de sang ?}

A la fin de ton cycle (\textit{"au bout"}). Tu veux être au plus bas possible parce que c'est l'information la plus utile. On peut même sire que c'est la seule information utile, vu que le fait d'avoir des niveaux régulier est notre problème principal. Par exemple : si tu fais une injection chaque jeudi après-midi, fais ta prise de sang le jeudi dans la matinée ou en début d'après-midi juste avant ta prochaine injection.

\subsection{Mon docteur m'a dit de faire mes prises de sang au milieu de mon cycle / juste après mon injection, est-ce que je fais ça ?}

\textbf{Non.} La mesure du taux d'oestrogène en milieu de cycle ne donne aucune information utile et ne permet que de savoir quel ester tu utilises. En restant gentille, ce genre de demandes est dûe à l'incompétence née de standards de soins datés et conservateurs. En l'étant moins, c'est une volonté malveillante d'asssurer que tu aies des niveaux d'oestrogène trop bas, ce va te donner des mauvais résultats, ou te rendre malade. \textbf{Je recommande de mesurer au dernier jour du cycle quand même.} 

\subsection*{Interpréter les Résultats}
\addcontentsline{toc}{subsection}{\textemdash{} Interpréter les Résultats}

\subsection{Quel est le taux d'oestrogène que je dois viser ?}

C'est probablement la question la plus controversée pour ce qui concerne la transition. Pour faire court, tu veux avoir des niveaux suffisants pour te sentir bien, et suffisants pour supprimer la production de testosterone si c'est ce que tu cherches, mais des niveaux plus hauts que ça sont au mieux du gâchis d'horomones, et au pire contreproductifs. C'est cependant une fourchette très large, et avec autant de variables il y a toujours un élément personnel. En d'autre mots, tu veux avoir suffisamment d'oestrogènes pour que tu te sentes bien, c'est à peu près tout.

\subsection{Est-ce que des niveaux d'oestrogène plus hauts me permettent de mieux me féminiser / de le faire plus rapidement ?}

\textbf{Non.} Des niveaux d'oestrogènes plus hauts que nécessaires sont préférés par certains parce qu'iels se sentent mieux comme ça, mais ça n'a aucun intérêt au niveau de la féminisation. En fait, des niveaux trop hauts peuvent causer des troubles de l'humeur ou d'autres effets secondaires non désirés. \textbf{Minimiser les niveaux de testostérone (jusqu'à un certain point) est bien plus important que de maximiser les niveaux d'oestrogène.}

\subsection{D'accord, mais en pratique je veux voir quel chiffre sur mon résultat de prise de sang ?}

En rappelant que le nombre exact n'est pas si important que ça, et que le chiffre sera toujours un peu plus important que ce que tu as dans ton corps au dernier jour du cyclé à cause de la latence, et que ce chiffre sera dans un nuage de possibilité basé sur un certain nombre de facteurs, \textbf{Je recommande 200pg/mL (730pmol/L) au minimum au dernier jour du cycle.} C'est une recommandation plutôt large, qui prévoit une bonne marge vu que la suppression de l'axe HPG arrive bien en-dessous de ce niveau. La plupart des gens préfèrent être autour de ce niveau, et certains préfèrent un peu plus ou moins. Je ne pense pas que ça soit un chiffre sur lequel trop se fixer, parce que c'est très dépendant de toi, et au final, le plus important c'est que tu te sens bien. \textbf{Cependant, au-delà de 300pg/mL (1100pmol/L) au dernier jour du cycle, tu as un niveau certainement trop haut pour tes besoins.} Il y a des exceptions à cela, mais tu n'en fais probablement pas partie. Mais bon, fais ce qui te fias sentir le mieux. Aussi, va voir la Question \ref{11-1}.

\subsection{Quel est le taux de testostérone que je dois viser ?}

La suppression de la testostrérone (T) est un prérequis pour une féminisation adéquate, donc descendre en-dessous de 50 ng/dL (1.7 nmol/L) es généralement suffisant. \textbf{A noter que des taux de testostérone approchant 0 ne sont pas désirés.} Voir la section \ref{T} "TESTOSTERONE" 

\subsection{J'ai naturellement des taux de T très hauts / très bas. Est-ce qu'il faut que je change quelque chose à mon dosage ?}

Probablement pas. les taux de testostérone qu'on voit avant le début d'un THS sont quasiment toujours plus hauts que ce qu'on cherche pour une féminisation et vont être réduits quoi qu'il arrive (voir la Question \ref{2-3}). L'exception est si tu as une condition intersexe de quelque variété que ce soit, ce qui va peut-être demander un ajustement plus fin que les préconisation listées dans ce guide et qui dépasse l'objet de celui-ci. Tu n'auras peut-être pas besoin de changer quoi ce soit, mais tu te sentira peut-être mieux si tu le fais. Voir la Question \ref{9-2}
% TODO: Pas sûr que supprimé soit le meilleur mot. 

\subsection{I have had bottom surgery. Do my estrogen levels need to be different?}
% TODO: A faire

Vu que la diminution des niveaux de testostérone n'est plus un problème pour toi, tu peux probablement t'en tirer avec des niveaux d'oestrogènes plus bas que pour les autres, mais \textbf{tu as toujours besoin d'oestrogène.} Puisque tu ne produis plus d'hormones sexuelles, il est crucial que tu maintienne des niveaux d'hormones suffisants pour rster en bonne santé. N'avoir plus aucune hormones ou presques va induire des symptômes de ménopause, raison pour laquelles les femmes cis plus âgées prennent aussi un THS parfois après la ménopause. Ajuste tes niveaux comme tu le sens.

\subsection{Est-ce qu'il existe des raisons pour lesquelles une prise de sang pourrait mener à des résultats imprécis ?}

Selon comment le test est conduit, des suppléments de biotine peuvent donner l'impression que les niveaux d'\textit{estradiol} (E2) (parmis d'autres, mais c'est l'\textit{estradiol} qui nous intéresse) sont anormalement hauts. Il n'est pas toujours possible de savoir avec quel méthode le test est fait, donc il vaut mieux arrêter de prendre tes suppléments de biotine quelque jours avant ta prise de sang. Il est possible aussi qu'il y ait eu une erreur avec l'équipement ou l'échantillon, même si c'est beaucoup moins probable 

\subsection{Est-ce qu'on voit quel ester / voit d'administration j'ai choisi sur mes résultats de prise de sang ?}\label{4-16}

Non. Il n'y a pas de moyen de savoir quel type d'oetrogène tu prends juste à partir d'une prise de sang. Tous les ester se transforment en \textit{estradiol} dans ton sang, ce qui est le but recherché, et c'est la même chose pour les pilules, gels, patches, sprays ou quoi que ce soit d'autre que tu pourrais utiliser. Au final, tout n'est qu'oetrogène. 
 

\section{TECHINQUES ET MATERIEL} \label{ts}

\subsection*{Sites d'injections \& Sécurité}
\addcontentsline{toc}{subsection}{\textemdash{} Sites d'injections \& Sécurité}

\subsection{How do I safely perform an injection?}
\subsection{Comment est-ce que je fais une injection en toute sécurité}

\href{https://rebellyon.info/Le-guide-d-auto-injection-sous-cutanee-du-25225}{\textit{Le FLIRT et STRIP ont créé un guide d'auto injection, que tu peux lire ou imprimer}}

Sinon, Je recommande ces deux vidéos (en anglais):

\begin{enumerate}
  \item \href{https://www.youtube.com/watch?v=cBabaGC2Dok}{\textit{"Comment s'auto injecter en intramusculaire (IM)”}}
  \item \href{https://www.youtube.com/watch?v=YfNlAZLxLyw}{\textit{“Une technique d'injection IM sans douleur (pour le moment pour moi)”}}
\end{enumerate}

Between these two videos, you should be fully equipped to properly inject with minimal pain. I suggest studying them and revisiting as needed. \textbf{One key thing to emphasize is to inject with the bevel facing up to reduce pain.} In other words: the needle has a clearly defined point, and you want that to be what touches your skin first. You want a nice straight line of travel. You can think about how your hand/wrist rotates if that helps you visualize the motion, but realistically it'll be intuitive muscle memory that you'll learn naturally.
Avec ce guide et ces deux vidéos, tu devrais avoir tout ce qu'il faut pour t'auto injecter sans trop avoir mal. Prends le temps d'étudier ça et reviens voir quand tu en a besoin. \textbf{Le truc le plus important quand tu t'injecte est d'avoir le biseau vers le haut pour réduire la douleur.} Je m'explique : l'aiguille se termine en un point, et tu veux que ça soit ça qui touche ta peau en premier. Tu veux une jolie ligne bien droite quand ton aiguille va se planter. Tu peux imaginer le mouvement de ta main si ça t'aide, mais en étant réaliste, c'est une mémoire musculaire que tu va apprendre avec l'expérience.
% TODO: Je pense que j'ai pas compris ce que j'ai traduit la 

\textbf{Rappelles-toi : s'auto injecter s'apprend ! } Tu vas t'améliorer avec le temps, et ça prendra pas longtemps avant que tu saches faire. Ca va le faire.

\subsection{Est-ce que je dois m'injecter exactement comme ça ?}

Non, tu peux personnaliser comme tu veux. Au final, quand il s'agit de se faire des trous, il y a plein de mainières d'y arriver. Trouve la manière qui marche le mieux pour toi. Faire un mouvement rapide de piqué marche le mieux habituellement, mais si tu préfère aller plus lentement, ça marche aussi. 

\subsection{Comment est-ce que je fais pour passer outre l'anxiété au moment de l'injection ?}

Je suggère de créer un rituel autour de l'injection. En fromant une routine, le processus va devenir une seconde nature pour toi. Si ça marche pour toi de te distraire en mettant de la musique, en ayant une conversation, en regardant une série, ou en faisant autre chose qui marche pour toi et qui laisse la mémoire musculaire prendre le dessus, c'est super ! Sinon, tu peux te faire aider par un.e ami.e ou un.e proche pour faire tes premières injections, ça aide aussi. La première injection est celle qui fait le plus peur. Habituellement, les gens disent "Ah c'est tout ?", parce que c'est jamais aussi terrible que ce qu'on croit.

\subsection{Est-ce que l'endoit où je m'injecte a une importance ?}

Oui et non. Il faut rester dans des zones sûres, mais à part ça, ça dépend surtout de ta mobilité, du volumes que tu injectes, de ton combo seringue/aiguille et de ce que tu préfères. En tout cas, \textbf{assure-toi de changer de site d'injection a chaque fois.} Alterne le côté de ton corps que tu choisis, par exemple, si tu injecte sur ta jambe droite une semaine, utilise ta jambe gauche la suivante. C'est pour éviter d'avoir des risques à long terme.

\subsection{What injection sites are safe?}

Opinions vary between medical authorities, but your body composition can also play a role. I recommend injecting on the side of the leg as shown in the video(s) because it is doable for most people and is capable of being very consistent which means consistently painless injections once your technique is practiced, but other people prefer their glute or their stomach. \href{https://vertisis.com/articles/how-to-self-administer-a-subcutaneous-injection}{This video and website} shows other injections sites that can be acceptable depending on the supplies you use. Figure out what works best for you.

\subsection{What do “intramuscular” (IM) and “subcutaneous” (SubQ/SC) mean?}

You will often hear these terms in the context of injections. \textit{Intramuscular} means injected into the muscle and \textit{subcutaneous} means injected into the fatty layer beneath your skin.

\subsection{What is the difference between intramuscular injections (IM) and subcutaneous injections (SubQ/SC)?}

\textbf{In the context of HRT, there is little to no difference between intramuscular and subcutaneous injections.} Subcutaneous injections are absorbed more slowly than intramuscular injections, however this is generally not significant enough of a difference to impact dosing. It should also be noted that an injection is rarely deposited fully in muscle or fully in the subcutaneous layer which blurs any difference together even further on an injection-by-injection basis.

As a side note, pharmaceutical sources for estradiol vials typically say they are for intramuscular injections only because that is what they are technically approved for. It does not matter though.

\subsection{Should I perform intramuscular injections (IM) or subcutaneous injections (SubQ/SC)?}

\textbf{This is the wrong question.} \textbf{An injection is an injection.} Subcutaneous injections are often recommended because people believe that they allow for less painful injections by virtue of being subcutaneous, but there is not a fundamental difference in how an injection is performed. \textbf{The advantages that people refer to are not inherent to the injection depot location; they are inherent to the factors that affect injection pain.} The better question would be “How do I minimize pain during injection?”, but two other questions first.

\subsection{Does my injection angle and/or preferred injection method matter?}

No. To reiterate, the most important part of performing an injection is that you pierce a needle through your skin and deposit fluid into your body. If the fluid doesn’t leak out (or at least, not much) and it doesn’t hurt (or at least, not much), then you have done a fantastic job. \textbf{I cannot stress enough that the intramuscular vs subcutaneous “divide” is nonexistent and that the question does not meaningfully impact the effectiveness of injectable estrogen.} \textit{Estradiol undecylate} is the only case where depot location seems to meaningfully affect absorption, but even then, we don’t fully understand the details. Point being: please be concerned about the things that matter and not the things that don’t matter.

\subsection{Do I have to aspirate?}

No. “Aspiration” refers to pulling the plunger back after puncturing the skin before injecting the fluid with the intent of ensuring a blood vessel is not being injected into. Its necessity is controversial, but for hormone injections following standard procedures, there are few benefits that outweigh the negatives. The standard injection sites have low risk of striking a blood vessel in the first place, lessened even further by shorter needle lengths, so this practice is not recommended anymore by most medical organizations.

\subsection{How do I minimize pain during injection?}

Aside from practicing your technique and improving your skill, the main factor for injection discomfort is the needle and syringe combination that you are using. \textbf{To minimize discomfort, the highest needle gauge that your vial’s carrier oil is capable of tolerating should be used along with an appropriately sized syringe and needle length. }You should ask “What needle gauge and length should I inject with?” To answer that, let’s talk about how needles work.

\subsection*{Knowing Your Needles}
\addcontentsline{toc}{subsection}{\textemdash{} Knowing Your Needles}

\subsection{What is “needle gauge”?}

\textit{Needle gauge }is a measure of needle thickness. The bigger the number, the thinner the needle. A 25G needle is thinner than a 20G needle, for instance. Higher gauge needles also tend to be shorter because longer needles become more prone to bending, so their length has a lower maximum. Unsurprisingly, thinner needles generally hurt less. It should be noted that the gauge of needle(s) used will not affect HRT in any way; it will only affect the ease and comfort of the injection itself.

\subsection{What are “Luer lock” and “insulin” syringe/needles?}\label{5-13}

\textit{Luer lock syringes} have separate syringes and needles so a separate needle can be used for drawing and injecting. \textit{Insulin syringes} have a needle fixed in place which means that the same needle will be used for drawing and injecting. Where possible, insulin syringes are preferred for comfort and for minimizing dead space (See Question \ref{5-26}).

\textbf{Safety Warning: Recapping needles is generally not recommended out of concern for sticking yourself, but if you do (such as when swapping out a drawing needle), NEVER apply force with your hand towards the needle.}

It is possible that the cap may break and you may injure yourself if you place the cap incorrectly. Gently “scooping” the cap onto the needle off of a horizontal surface and pressing the loosely capped needle against a wall or pulling the cap on the sides to fully seat the cap is preferred. There isn't a disease transfer risk when performing a self injection, so heed this warning at your own discretion, but resticking is a VERY serious concern when performing injections on others. For disposal, see Question \ref{5-27}.

\subsection{What needle gauge should I draw with?}

If you are using Luer lock syringes, it is recommended to use a lower gauge than what you inject with so that it takes less time to draw from the vial. Too low can lead to coring (See Question \ref{5-23}), so at least 21-23G is recommended. If you have patience and lower volumes to inject, then higher gauges are recommended for the aforementioned coring risk reduction. Please note that the needle does not meaningfully blunt on the stopper. This question is irrelevant with insulin syringes because the needle is not removable.

\subsection{What needle length should I draw with?}

If you are using Luer lock syringes, the length of the drawing needle does not matter too much outside of the inconvenience of having too long of a needle being unwieldy. In other words, no need to be picky. This question is irrelevant with insulin syringes because the needle is not removable.

\subsection{What needle gauge should I inject with?}\label{5-16}

This is a tricky and highly subjective question, and your answer will depend on 4 main factors: 1) the carrier oil for what you are injecting; 2) if the vial contains a cosolvent; 3) your patience to have a needle in your leg for longer; and 4) your willingness/ability to push harder on the syringe plunger. It’s a question of comfort. Thicker oils mean more time and more effort when using a higher gauge, but also higher gauges can be significantly less painful going in. \textbf{As a baseline, 25G is the minimum needle gauge that you should use to manage discomfort. }Most common carrier oils can generally do up to 27G comfortably, whereas MCT oil in particular is notable for being able to easily do 30G (See Question \ref{6-16}).

\subsection{What needle length should I inject with?}

\textbf{I recommend between 0.5” to 1” (12.5mm to 25mm) depending on your gauge.} Below 0.5” (12.5mm) increases the likelihood of leakage. 0.25” (6.5mm) length needles can be fine depending on your technique and the fluid you’re injecting, but 0.5” (12.5mm) is a safe bet. Beyond 1” (25mm) is unnecessarily daunting and painful without any added benefits.

\subsection{Does syringe size matter?}

\textbf{Yes, size matters.} There are two reasons for this. 1) Higher volume syringes tend to be less precise which leads to incorrect dosing, and 2) physics makes higher volume syringes more difficult to inject. For dosing accuracy, you do not want to use a syringe far larger than the volume that you are injecting (i.e., for injections less than 0.1ml, get smaller than 1ml syringes). \textbf{Avoid 3mL syringes entirely if you can.} Obviously use them if it’s all you have, but they’re really not meant for a task like this. Do not ask me why pharmacists seem to near-exclusively hand them out. A cruel joke, maybe.

\subsection{Where do I buy syringes and needles?}

It depends on your local jurisdiction as some localities ban the sale of needles and syringes to individuals as a punitive measure against drug users. Otherwise, medical and veterinary supply businesses or authorized manufacturer retailers should be good places to look. \textbf{Amazon is not recommended} because the quality is uncertain.

\subsection{Is it okay if I reuse needles or syringes?}

\textbf{No. Never reuse needles or syringes. }Or share either. You probably already know this but I’m just reminding you because it’s really not good or safe to do!

\subsection{What if I want to do injections but have difficulty performing it on myself?}\label{5-21}

You might like an auto-injector. As the name suggests, auto-injectors perform the injection for you. Auto-injectors like the \href{https://unionmedico.com/90-super-grip/}{\textit{UnionMedico 45/90 Super Grip}} can take 1ml syringes which can take the difficulty out of injecting (but you still manually press the plunger), whereas auto-injectors like the \href{https://www.owenmumford.com/us/medical-devices/autoject-2}{\textit{Owen Mumford Autoject 2}} entirely hide the needle of an insulin syringe and automatically push down the plunger. There are also a variety of 3D printable designs available online. I have used none of these products and these are not endorsements.

\subsection*{Basics of a Vial}
\addcontentsline{toc}{subsection}{\textemdash{} Basics of a Vial}

\subsection{What should I look for when inspecting vials?}

Aside from looking for signs of coring (see below), you should look for any signs of discoloration, separation, contamination, crystallization, cracks in the glass, fibers, hairs, etc. A properly made vial should not deviate too much from usual. \textbf{Always inspect your vials before use. Do not use a vial that does not seem right.}

\subsection{What is “coring”?}\label{5-23}

\textit{Coring} is when a piece of the rubber stopper breaks away and falls into the vial. This can occur with too large of drawing gauges, repeated punctures on the exactly same spot, or too many punctures (i.e., a very small volume injection with a very large volume vial). \textbf{A cored vial should be immediately discarded. }The \href{https://www.youtube.com/watch?v=w5F0SLoMjC8}{\textit{45-90° technique}} can also be used to help minimize coring.

The concern with coring is that you do not want to inject bits of rubber into you. If there are large bits of rubber, there might be smaller ones that you can't see. The purpose of the stopper is to protect the contents from the elements, so a vial with a hole in the top is more prone to oxidation and/or bacterial growth. \textbf{As a side note: Please ensure that you remove the metal or plastic cap off the top of a new vial. }This may seem obvious, but some vial designs can be confusing. 

\subsection{How long until a vial expires?}

A sealed vial could last for years without issue if it is stored at stable temperatures away from the light. Concerns with age are primarily carrier oil oxidation assuming that the vial was sterilized as it should be. A punctured vial that has a preservative in it (See Question \ref{6-17}) should last at least a year or whatever the life time of the vial is (i.e., how long until you use it all). The “discard after 28 days” listing on vials is simply the minimum requirement for how long manufacturers must guarantee sterility, not the maximum shelf life. 

\subsection{How should I store a vial?}

Stable room temperature and away from light. High heat and UV can cause degradation of the carrier oil, whereas low temperatures can cause crystallization. Crystals can be dissolved and reincorporated, but it’s a potential cause for irritation if they aren’t fully dissolved. This goes for both sealed and unsealed vials.

\subsection{What is “dead space”?}\label{5-26}

\textit{Dead space} refers to the amount of fluid that is wasted when performing an injection. This is fluid that is trapped in the syringe or in the needle. With a standard Luer lock needle/syringe this can be up to 0.1mL, whereas in an insulin needle can be as low as 0.003mL. Reducing dead space is recommended for economic reasons because it adds up to a lot of wasted estrogen. \href{https://hrtcafe.net/Calc/}{This calculator} can be useful for estimating how much estrogen is wasted depending on the supplies used.

One thing to note if you are swapping needles for drawing and injecting, then you should pull the plunger back slightly prior to taking off the drawing needle so that the fluid inside the drawing needle is not wasted. It is very minor, but it can make a difference. See Question \ref{7-7} for another possible strategy if concerned about high dead space.

\subsection{What do I do with my used syringes and needles?}\label{5-27}

Place all used injection supplies pointed down in a sharps container (either a dedicated biohazard container or reusing hard plastic tubs such as from protein powder or laundry detergent). When the container becomes three-quarters full, seal it closed so that it cannot be accidentally opened. Clearly label it “USED SHARPS” and then dispose of it according to your local jurisdiction's requirements. \textbf{Note that sharps should NOT be placed into trash or recycling containers.} Your city/state/region likely has a website somewhere describing how and where to dispose of household hazardous waste. For the US, \href{https://safeneedledisposal.org/}{you can go here.}



\section{SOURCING VIALS}\label{sv}

\subsection{Where do I get estrogen vials to inject?}

Broadly speaking, you have two options: \textit{pharmaceutical sources} and \textit{DIY sources}. \textit{Pharmaceutical sources} typically require a doctor's prescription because HRT is not available over-the-counter (or if it is, vials are not included) in most countries. \textit{DIY sources} encompass everything else.

\subsection{Should I use pharmaceutical sources or DIY sources?}

The choice is yours, but sometimes there is no choice at all. There are pros and cons to each. Of course, there is nothing stopping you from procuring estrogen from multiple sources to get the benefits of both. In many situations, it may be recommended.

\subsection*{Pharmaceutical Sourcing}
\addcontentsline{toc}{subsection}{\textemdash{} Pharmaceutical Sourcing}

\subsection{What are the pros of pharmaceutical sources?}

\begin{itemize}
  \item Can generally trust quality control processes and certifications;
  \item Insurance may cover it in part or in full;
  \item Can be more convenient depending on your luck with doctors;
  \item The product most likely will be consistent;
  \item \textbf{At least appearing to be using pharmaceutical sources may be required if you are seeking insurance approval for surgeries.}
\end{itemize}

\subsection{What are the cons of pharmaceutical sources?}

\begin{itemize}
  \item Reduced (or no) selection of esters;
  \item Possible lengthy wait time (months or years);
  \item May be required to have a prescription (depending on country);
  \item Insurance may not cover costs in part or in full;
  \item May not be prescribed at all in your country;
  \item Your doctor may arbitrarily refuse to prescribe it to you;
  \item Your doctor may arbitrarily withhold refilling a prescription;
  \item Shortages may prevent filling a prescription at all;
  \item Likely held to stringent WPATH requirements or worse;
  \item Harder to stockpile;
  \item Access is subject to the whims of your country’s political situation which also means that your transness will likely be included on your medical record. 
\end{itemize}

\subsection*{DIY Sourcing}
\addcontentsline{toc}{subsection}{\textemdash{} DIY Sourcing}

\subsection{What are the pros of DIY sources?}

\begin{itemize}
\item Generally much cheaper in most places;
\item Available anywhere in the world;
\item Obtaining it can take months or even years less time than waitlists (the only wait is shipping and production);
\item Easy to stockpile;
\item Full selection of esters;
\item No requirement of dealing with the medical system;
\item It’s probably made with love.
\end{itemize}

\subsection{What are the cons of DIY sources?}

\begin{itemize}
  \item Almost certainly not made in a certified clean room;
  \item Quality can vary depending on the source;
  \item Can be inconvenient depending on the source;
  \item Requires trusting the source;
  \item Requires finding a source;
  \item Sources are more likely to close than your local pharmacy;
  \item Product delivery times can vary;
  \item Most likely have to use cryptocurrency which is annoying;
  \item Cannot use insurance if that was an option for you.
\end{itemize}
Additionally as already stated, if you are seeking insurance approval for surgeries, they likely require a minimum amount of time with an HRT prescription. This may or may not be a concern for you.

\subsection{What types of injectable estrogen are DIY only?}

Chiefly, \textit{estradiol enanthate}. Pharmaceutical sources will almost always prescribe you \textit{estradiol valerate}, but not always at a 40 mg/ml concentration. \textit{Estradiol cypionate} may occasionally be prescribed, but rarely above 5 mg/ml or 10 mg/ml concentrations, which are annoying to dose. The benefits provided by \textit{estradiol enanthate} alone are very good reasons to consider DIY, but you can get any ester at 40 mg/ml from DIY sources.

\subsection{What actually \textit{are} DIY sources?}

DIY sources include commercial brewers, mutual aid projects, your friend, and yourself if you have an entrepreneurial spirit!

\subsection{Where can I get DIY vials?}

What are you, a cop? I’m not telling you that. That’s not the point of this guide anyways. There are other resources that have that information. Stay focused.

\subsection{How can DIY sources be cheaper than pharmaceutical sources?}

The cost to produce a vial is roughly around \$10, including labor and amortized capex cost. This is likely a high estimate. The bulk of the cost for commercial DIY sources are the layers of overhead and shipping involved in anonymity. Non-commercial DIY sources likely have no such overhead. Pharmaceutical sources generally do not have any incentive to be cheaper than what they are. 

\subsection{Is DIY legal?}\label{6-11}

In most locations including America, estrogen is not a scheduled substance, whereas testosterone may or may not be criminalized. The US is an anomaly for testosterone in this regard, as other countries don’t criminalize possession of testosterone, but prosecution is rare anyway given the wide availability of steroids. \textbf{This guide is not legal advice.}

\subsection{Is DIY safe?}

“DIY” as a broad category of sources is neither safe nor unsafe, but not all DIY sources are equal. When we are discussing the topic of safely injecting something into your body, the real question is: do you trust that the person who produced that vial properly followed aseptic techniques and procedures such that the vial contains what you want and nothing else? For pharmaceutical sources, that trust is innate on the assumption that laws and regulations exist. For DIY sources, that trust must be earned through demonstration/explanation of process, independent third-party testing for concentration/purity, and community reputation.

\subsection{What things should I look for to know if a DIY source is trustworthy?}

Use your gut and your brain. 

\begin{itemize}
  \item Are they open to talking to you about their process / have it listed somewhere? (e.g., do they filter for dust? The answer should be yes!!!)
  \item Do they seem competent in their ability?
  \item Have they had their product tested? 
  \item Are they a trusted member of the community?
  \item Have they been vetted or vouched for by other members of the community who you trust? (i.e., inspections, reviews, testimonials, etc)
  \item Mistakes happen, but do they take accountability or do they try to silence negativity?
  \item For commercials, do they resolve any issues with customer orders?
  \item For commercials, are they taking payment on product not yet produced without indicating that it is a backorder? (You should never backorder!)
  \item Do their vials contain preservatives?
  \item How long have they been producing? (For good reason, they may not tell you!)
  \item How much do they produce? (For good reason, they may not tell you!)
  \item Are the vibes just \textit{off}?
\end{itemize}

These are just some of the many questions that can be asked to know if you trust that they care as much as you do about the quality of their product.

\subsection{Should I hold different DIY sources to different standards?}

Likely, yes. Commercial brewers should also be held to a high standard if you are giving them money in exchange for product because they can afford to do it right. A mutual aid product on the other hand that is distributing vials for free might not be something that you can afford to be picky about, although that is not to say that the product is likely to be better or worse. As for a friend or yourself, only you can decide that!

\subsection*{Anatomy of a Vial}
\addcontentsline{toc}{subsection}{\textemdash{} Anatomy of a Vial}

\subsection{What should I look for in a vial?}

The ingredients inside of a vial can be categorized as \textit{“active”} and \textit{“excipient”}. The\textit{ active} is the estrogen ester in our case, and the \textit{excipients} are everything else. There are generally three or four ingredients: 1) the estrogen ester; 2) the carrier oil; 3) the preservative; and optionally, 4) any cosolvent(s). We have already covered the estrogen esters in Section \ref{td} “TYPES AND DOSAGES”. Pharmaceutical vials almost always have all four ingredients.

\subsection{What carrier oil should I look for in a vial?}\label{6-16}

This is a question of preference, personal tolerance, and possibly allergies. \textbf{The main variable relevant to you is viscosity because that affects injection comfort and convenience.} As discussed (See Question \ref{5-16}), thinner oils are able to more conveniently use higher gauge needles without difficulty when drawing and injecting. \textbf{The most commonly used carrier oils for HRT are castor oil and MCT oil. }Castor oil is the thickest oils commonly used, but it also tends to result in the least amount of irritation so pharmaceutical vials typically use it. MCT oil is the thinnest oil commonly used, but some people find it more irritating than other oils and it’s DIY only. Cottonseed oil and grapeseed oil occasionally find use, but usually not by HRT manufacturers. Other oils like sunflower or sesame or whatever else occasionally find use but aren’t generally recommended. Depending on your circumstances, this question might not matter to you, you might not have a choice, or it may be a strict requirement. 

\subsection{What preservatives should I look for in a vial?}\label{6-17}

The most common preservative used in injectable vials is \textit{benzyl alcohol} (BA) in low concentration. This is mandatory and not up to debate. \textbf{You should never use a vial without a preservative. }For people with the rare allergy, \textit{chlorobutanol }is an alternate commonly used preservative, but almost never by DIY sources which would necessitate hunting specific pharmaceutical formulas.

\subsection{What cosolvents should I look for in a vial?}

The main cosolvent used is \textit{benzyl benzoate} (BB) which reduces the viscosity of the resulting solution. This is technically optional, but it is generally recommended for batch consistency and in many cases is necessary depending on the carrier oil and the desired concentration. Some people find it irritating, but others don’t. 

 

\section{TROUBLESHOOTING}

\subsection*{Dosage Uncertainty}
\addcontentsline{toc}{subsection}{\textemdash{} Dosage Uncertainty}

\subsection{My levels aren’t what I expected them to be. Why not?}

There are a number of possibilities. Recall first that model estimations cannot take into account any plethora of factors which may cause some deviation. Recall as well that it takes multiple injections until you reach stability, so if you just changed your dosage that may be why. Quadruple check with a friend that you are injecting as much as you think you are. That is more commonly an issue than you might think, but for DIY sources it is also possible that the concentration is lower than advertised due to inexperience or less precise equipment. In that case, injecting you may just need to inject a little more for that vial. \textbf{But remember, the most important thing is how you feel, not your levels. }Please note that even professional compounding pharmacies can produce dud vials not caught by quality control, as hopefully rare as that may be!

\subsection{Can I compare levels across different tests if I didn’t test at trough?}

\textbf{No.} Not accurately, anyway. This is part of why you should always test at trough. Hours before your normal time for your next injection; that’s what you want. Eliminating as many variables as possible makes the data far more useful to you. If there is nothing else that you take from this guide, please just test at trough.

\subsection{I feel really bad on my trough days. What should I do?}\label{7-3}

In most cases, either the dosage is too low or the frequency is too low. This is most pressing for \textit{estradiol valerate} and \textit{estradiol cypionate}. Adjust your dosage within the range listed or adjust the frequency. Find what works for you. It is also possible with \textit{estradiol valerate} in particular that your dosage might actually be too *high* instead of too low as the high level variability across your cycle may be the culprit for this crashing sensation. In short: swap to \textit{estradiol enanthate} if you can.

\subsection*{Injection Woes}
\addcontentsline{toc}{subsection}{\textemdash{} Injection Woes}

\subsection{The injection is harder to do when it’s cold. What should I do?}

Warm up the vial before drawing, then warm the syringe before injecting. Rolling the barrel of the syringe between your hands should be plenty to warm up the fluid. Forming this as a habit all the time should improve your injection consistency.

\subsection{The injection hurts more when it’s cold. What should I do?}

Warm up your leg before injecting. Relaxing the muscles with a massage or a hot shower (specifically: increasing the temperature with the water aimed at your leg before you get out) before injecting can help.

\subsection{I bled after my injection. Will I die?}

No. This means that you likely just hit a vein or a capillary which can happen sometimes. You might experience some light bruising or increased soreness. Using a cute bandage will make it heal faster.

\subsection{There was some air in my syringe. Will I die?}\label{7-7}

No. While you obviously do not want to inject just air and it can affect dosage if there is too much in the syringe, a small amount of air under 0.1ml is almost certainly not going to cause issue for you. It might actually be recommended in some cases. For instance, the \textit{air lock technique} (a standard technique for injecting fluids that are irritating or can stain, not crucial knowledge for HRT) generally involves injecting 0.1-0.3ml of air, so you have nothing to be worried about. You aren’t doing intravenous injections.

\subsection{Some of the fluid leaked out. Was my injection wasted and/or will I die?}

No. Leakage can happen for any number of reasons and is rarely enough to make a difference, so you do not need to do another injection. For the future, make sure to leave the needle in for 5-10 seconds before retracting and then apply pressure afterwards. You might consider using the air lock technique mentioned above if you are particularly concerned about leakage.

\subsection{Sometimes I am really sore after an injection. Will I die?}

No. Assuming you have otherwise followed all of the suggestions within this guide, sometimes the deposit of fluid hits an uncomfortable place for one reason or another. Better luck next time. \textbf{Make sure you alternate injection spots!} You do not want scar tissue to build up over the long term, and if a spot is already sore, you do not want to make it more sore.

\subsection{I am experiencing a lot of itchiness and irritation after injecting. Will I die?}

Probably not. There are a number of possible causes. Infection is the most concerning cause, but is unlikely in most cases. \textbf{Immediately go to a doctor if you are experiencing a fever, severe pain, muscle aches, pus, red streaks, or other signs of infection. }In most cases however, irritation like itchiness, redness, light swelling, warmth, etc are the result of using a vial whose estrogen and oil have separated (“crashed out of solution”). See below. It’s possible that you may be having a reaction to the carrier oil, but if you are suddenly experiencing issues after some injections without any issue, it is most likely that the vial contents are out of solution.

\subsection{My vial has crystals in it. Can I still use it?}

It most likely means your vial got too cold. Warm it up and gently shake to reincorporate. If the crystals are not going away, then it’s possible the vial contents have separated entirely. With a lot more heat and stirring the crystals might reincorporate, but it is simplest and safest to replace the vial if you can.

 

\section{PROGESTERONE}

\subsection{Do I want to take progesterone?}


\subsection{What is the difference between “progesterone” vs “progestin” / ”progestogen”?}

The class of hormones, both natural and synthetic, that activate the progesterone receptor are “proges\textbf{togens}”. The natural, bioidentical, and most important progestogen is “proges\textbf{terone}”. Synthetic progestogens are “proges\textbf{tins}”. These three terms are mistakenly used interchangeably in scientific literature and in clinical settings, likely causing much of the broader confusion regarding the role of progesterone in HRT, despite the fact that they are \textbf{not }equivalent.

\subsection{Do I want progesterone or a progestin?}

Progesterone. You want bioidentical progesterone.

\subsection{What’s wrong with progestins?}

Progestins, most typically \textit{medroxyprogesterone}, \textit{medroxyprogesterone acetate}, or \textit{levonorgestrel}, are generally associated with the negative side effects and long term risks (breast cancer, blood clots, depression, etc) that are falsely attributed to progesterone. They are not bioidentical which means they do not behave the same as progesterone and thus cannot be directly compared.


Progesterone is believed to play a role in breast development and libido in particular, but as mentioned it’s a key hormone aside from its outward appearance effects. It does also have some antigonadotropic (i.e., it contributes to testosterone suppression) properties which can be sometimes relevant.

\subsection{Does it matter when I start progesterone?}

It is unknown. There is some belief that starting too early may harm breast development long term, but this is purely theoretical and contrary anecdotal evidence makes the answer unclear. The conservative estimate is waiting roughly a year into HRT (until Tanner Stage 3 or 4) in the possible chance that it does matter.

\subsection{How is progesterone normally taken?}

Aside from topical applications, the main form is via a pill. It is prescribed as an oral pill but is most effective when taken as a suppository. Topical sprays and creams can also work very well.

\subsection{Are you serious that progesterone should be taken as a suppository?}

Progesterone metabolizes entirely differently when taken orally vs rectally due to passing through the liver when taken orally. Oral progesterone primarily converts to \textit{allopregnanolone} which can cause heavy drowsiness, whereas rectal progesterone primarily converts to progesterone itself which is what we want (although some still converts). Some people take additional oral progesterone as a sleep aid, but please note that too much \textit{allopregnanolone }can sometimes lead to negative mental health side effects.

\subsection{How do I take progesterone as a suppository?}

Just a bit of water on the pill should work, then dry off and wash your hands. Obviously, don’t go to the bathroom for the next hour or so, so doing it before bed is best. If you are having issues with it not dissolving then you can try piercing the capsule but usually should be no issue. Be aware that if you use large homebrew suppositories made using coconut oil, the large volume of coconut oil will not want to stay in you.

\subsection{How much progesterone should I take?}

For pills, Standard dosage is 100-200mg daily at night. It is a rather arbitrary dosage; 200mg is the max that most doctors will prescribe. Some people take more than 200mg on occasion, but be aware that spiking your levels may lead to an unpleasant crash. See question below.

For topical applications, nobody can tell you with certainty due to the high variability of the delivery medium, nor is there any clear guidance on desired levels, or even frequency (likely daily), as progesterone is simply understudied. Because of this, I would advise titrating your dosage so that you understand how progesterone affects you.

\subsection{Is there any benefit to “cycling” progesterone?}\label{8-11}

No. Some people do this to mimic a cis woman’s menstrual cycle, but there is no reason to believe there is any benefit to this and it may cause negative PMS symptoms. The only exception is if you have good reason to suspect that you have an intersex condition involving a uterus that you are managing. I discourage it otherwise. See Question \ref{11-10}.

\subsection{How long should I take progesterone for?}

For as long as you plan to take estrogen and for as long as you want to. So, probably forever.

Sometimes people (or doctors) arbitrarily say to only take progesterone for X years. There is zero theoretical or empirical reason to suggest that this is sound advice. It's about as coherent as if someone (or a doctor) asked how long a trans person planned to take HRT for\textemdash{}oh wait never mind they do ask that.

\subsection{Can progesterone convert into \textit{dihydrotestosterone} (DHT)?}

No. Well, strictly speaking yes, but also no. It is largely a myth, although \href{https://whsah.co/posts/rethinking-progesterone-and-androgens/}{as outlined in detail by alix in this article}, for cases of people with \textit{nonclassical congenital adrenal hyperplasia} (ncCAH) progesterone can cause some negative side effects of increased androgenic activity. In those cases, discontinuing progesterone is recommended along with seeking out a formal diagnosis/treatment for potential adrenal disorders.

\subsection{Is there any benefit to topical progesterone applications in addition to pills?}

Maybe. It's a possible alternative to pills, especially in the case of someone with a peanut allergy since the most common pill manufacturer uses peanut oil, but again dosage is unclear. Some people find more progesterone fun, if nothing else. Be safe and have fun.

For clarity: Apply creams to your inner thigh region (elsewhere if directed), or optionally on scrotal skin (it's thin and highly vascular) in the case of sprays. And no, applying progesterone to your breasts directly is unlikely to make them grow bigger or faster compared to otherwise. 

\subsection{Can I snort progesterone powder?}

Please don’t. It’s hell on your sinuses. It isn’t hard to make your own topical progesterone spray and there are guides out there. Do that instead. It’s significantly more effective, consistent, and safer.

\subsection{Where can I get progesterone?}

Progesterone tends to be more expensive through DIY sources due to the higher mass of hormones required, so ideally get it through pharmaceutical sources covered by insurance. There is also the option of grey market foreign pharmacies, which are simply pharmacies in another country, although these often require some hurdles to purchase from. Topical progesterone creams are available OTC in some locations, although it is not always the most economical depending on the concentration.

\subsection{I would like to read more about progesterone in an HRT context. What resources should I read?}\label{8-17}


It should be noted that for the entire category of progestogens there are countless myths and falsehoods invented whole cloth by both proponents and detractors alike which does not make discerning truth from the already-sloppy scholarship any easier. Fantastical claims of magical benefits and fearmongering of alleged risks based on nothing are both equally unhelpful, although the later is worse in my opinion when comes from a medical authority, whether neglectful or malicious.

\subsection{Does progesterone interact with any other drugs related to HRT?}\label{8-18}

If you are taking 5$\alpha$-Reductase Inhibitors like \textit{finasteride} and \textit{dutasteride} (See Section \ref{AA} “ANTIANDROGENS”, or keep reading), these can affect how progesterone naturally breaks down into \textit{allopregnanolone} which can cause adverse mood effects in some people, irrespective of how you are taking progesterone. It is not fully clear how much the administration route for the 5$\alpha$-Reductase Inhibitors (i.e., topical vs oral) makes a difference, but lower systemic absorption via topical application may mitigate these side effects. It is recommended to not take either of those if you are someone affected by this interaction, but it is not in all cases anyway. Note that these depressive effects may be felt for up to a month after stopping. 
 

\section{TESTOSTERONE}\label{T}

\subsection{Why don’t we want zero testosterone?}


\subsection{Are there ever cases where I would want to supplement testosterone?}\label{9-2}

Yes. If you are experiencing the issues of the above and your estrogen levels are otherwise good, it’s possible that you might want to supplement with a microdose of testosterone. If you wanted to improve your erectile function, minimize any atrophy before bottom surgery, or otherwise wanted to experiment with your hormones to see what feels best for you, then that might be a reason to explore testosterone in a different context that you can hopefully appreciate more compared to pre-HRT.

\subsection{If I wanted to supplement testosterone, how would I do it?}

There’s a few possibilities. Testosterone comes in either injections or topical gels/creams, similar to estrogen as already discussed. Topical is more likely what you are going to be prescribed. Topical applications have the downsides that we have discussed for estrogen, but those are less of a concern here when precise levels are less important.

\subsection{What are the topical forms of testosterone?}

There is gel and cream. Gel is typically what will be prescribed, but some compounding pharmacies are able to make low-penetrating cream if someone wanted just topical application on the genitals. The latter is harder to get and generally more expensive, however.

\subsection{Does it matter where I apply the testosterone?}

It depends on if you have gel or cream. If you have the kind of localized cream as mentioned above, you would apply it as directly as mentioned. Otherwise, shoulders and upper arms are where gel should go. Make sure not to touch things until long after it dries!

\subsection{How much and how often should I apply testosterone?}

Season to taste. This largely depends on how you are feeling. If you have too much, you might start to experience side effects of testosterone (e.g., oily skin and body hair), but only you can say what is preferred for you. A weekly injection of 5-10mg of \textit{testosterone cypionate} might work for you, but in the case of 1\% topical gels which are often disbursed in 25/50mg packets, there is more variability. You almost never want even half a packet, and definitely not daily. I would suggest starting with much less than you think to see how you feel.

\subsection{Where would I get testosterone?}

If you are an American, you would have to get a prescription or ask any juicer at your closest Planet Fitness. Elsewhere, it depends on what gym chain is closest to you. Disclaimer: This is a joke. See Question \ref{6-11} “Is DIY legal?”

\subsection{Are other steroids equivalent to testosterone in an HRT context?}

Anabolic-androgenic steroids, i.e., drugs that are structurally similar to testosterone, are not all equivalent. Commonly used black market steroids like \textit{trenbolone acetate} have a laundry list of undesirable side effects, but steroids like \textit{nandrolone decanoate }are occasionally used for postmenopausal cis women due to their relatively low androgenic properties which make them very favorable for transfeminine individuals. Regardless, in America it is unlikely you will be prescribed anything other than testosterone itself, if you are able to get a prescription at all.

\subsection{What is the relationship between testosterone and \textit{dihydrotestosterone} (DHT)?}

\textit{Dihydrotestosterone} is primarily synthesized from testosterone via the 5$\alpha$-Reductase enzyme with around 5\% of testosterone in your body being converted. Generally speaking, if testosterone levels are suppressed (or if you have had bottom surgery) then there should not be much left to convert, but systemic levels won’t be zero because it is still locally produced. Depending on your body, this would be the main reason that you might want to consider supplementing with a 5$\alpha$-Reductase Inhibitor antiandrogen as discussed in the following section. As a reminder, \textit{dihydrotestosterone }is the hormone that is responsible Quel voie d'administration prendre pour mon oestrogène ?

Par injection. C'est le moyen le plus facile, prédicible, sûr, efficace et abordable de faire une transition hormonale. On dit même que pour certains, le moment de l'injection devient un rituel attendu, et que pour d'autres,  
\section{ANTIANDROGENS}\label{AA}

\subsection{What are “antiandrogens”?}

\textit{Antiandrogens, }commonly also referred to as “T blockers” or just “blockers”, as the name(s) may suggest prevent androgens (that’s what testosterone is) from acting on your body. There are many types of antiandrogens and they are commonly prescribed as part of an HRT regimen. They are needed if someone still produces testosterone and is not doing a form of HRT conducive to monotherapy, such as injections, but they are usually not desirable. It also should be noted that (most) antiandrogens do not reduce testosterone levels in any way that matters but instead simply reduce/negate effects on the body. This is relevant when interpreting lab results and such.

\subsection{Why wouldn’t I want antiandrogens?}

The main issue with most antiandrogens is that they generally have very undesirable side effects that are superfluous if testosterone is suppressed in the first place by having enough estrogen, so those side effects are being experienced despite\textemdash{}in most cases, at least\textemdash{}being rendered unnecessary by a reasonably-dosed monotherapy regimen. Bottom surgery of any kind also makes antiandrogens unnecessary in most cases.

\subsection{When might I want antiandrogens?}

If you are not most cases, if you desire peace of mind, or if your insurance requires a prescription on file before they will cover a procedure, then you may want antiandrogens. The medications used as antiandrogens might have other effects that may be desirable outside of their antiandrogen properties depending on your health situation. Additionally, if you are supplementing androgens, you may want a \textit{dihydrotestosterone }(DHT) blocker to minimize side effects related to body hair and hair loss, but be aware that this may not be the case if you are not using bioidentical testosterone (e.g. \textit{nandrolone decanoate}) because not all androgens behave the same.


Quel voie d'administration prendre pour mon oestrogène ?

Par injection. C'est le moyen le plus facile, prédicible, sûr, efficace et abordable de faire une transition hormonale. On dit même que pour certains, le moment de l'injection devient un rituel attendu, et que pour d'autres,  
The main medications taken as general testosterone blockers in an HRT context are \textit{spironolactone}, \textit{bicalutamide}, and \textit{cyproterone acetate}. The main medications taken to block the conversion of testosterone into \textit{dihydrotestosterone} (DHT) called “5$\alpha$-Reductase Inhibitors” (5-ARI) are \textit{finasteride} and \textit{dutasteride}. There are also GnRH agonists like \textit{leuprolide} and \textit{triptorelin}, but both of those are more often used as puberty blockers in minors, although in parts of Europe they are used for adults as well.

\subsection{When might I want to take \textit{spironolactone}?}

Due to the heroic dosages and significant negative side effects required for it to function as an antiandrogen in most cases, the only time I would ever recommend taking \textit{spironolactone} would be if you would benefit from its other effects such as its antimineralocorticoid (i.e., blocking \textit{aldosterone}) properties as it relates to blood pressure management or edema. \textbf{If you insist on taking \textit{spironolactone}, please do not take more than 100mg daily.} It has a bad reputation for a reason. “The Devil”, as it were.

In case you are unfamiliar, some of the many side effects include: brain fog, lethargy, poor memory, increased urination frequency, low blood pressure, low sodium / electrolyte imbalance, etc. In other words, \textit{spironolactone} is a blood pressure lowering dieurtic that is a mediocre antiandrogen which is typically prescribed at high dosages in an otherwise-healthy population for questionably-effective off-label use. In any other healthcare context this would (or SHOULD!) be highly unadvisable given the undesirable side effect profile and the widely-available preferable alternatives that already exist, but that's the state of trans healthcare for you.

\subsection{When might I want to take \textit{bicalutamide}?}

If you are going to take an antiandrogen, \textit{bicalutamide} is likely the one to take. It is generally well tolerated, barring 1\% cases of abnormal liver function test results and symptoms of liver dysfunction, but otherwise performs the job with relatively minimal side effects. \textbf{If you take \textit{bicalutamide}, ensure regular liver function tests to make sure that your results are in range. }The liver risks are dependent on your body rather than cumulative so any problem would likely present itself within the first year. Otherwise, there should be no issues. 

\subsection{When might I want to take \textit{cyproterone acetate}?}

Likely never. Take \textit{bicalutamide} instead.

The long term risk profile is poor and there is no situation that I can think of in which I would recommend this over an alternative solution. You can do everything \textit{cyproterone acetate} can by just taking more estrogen and adding progesterone to your regimen.

\subsection{When might I want to take \textit{dutasteride}?}

If you are extremely concerned about possible hair loss and/or want to maximize your chances for hair regrowth, you may want to take \textit{dutasteride}. If your testosterone is otherwise suppressed then it theoretically shouldn’t have much benefit as your \textit{dihydrotestosterone} levels should be relatively low, but bodies can be complicated, so it may be something of interest to you. Also, see Question \ref{11-14}.

It should be noted that \textit{dutasteride} can cause adverse mood effects in some people, in which case stopping is strongly recommended. Note as well that these depressive effects may be felt for up to a month after stopping. 

\subsection{When might I want to take \textit{finasteride}?}

If \textit{dutasteride }is not something prescribed to you or if your insurance mandates \textit{finasteride} specifically to cover a hair treatment. Otherwise, \textit{dutasteride} is preferred as it is more effective and better tolerated.

It should be noted that \textit{finasteride} can cause adverse mood effects in some people, in which case stopping is strongly recommended. Note as well that these depressive effects may be felt for up to a month after stopping.

\subsection{Where can I get antiandrogens?}

Aside from being prescribed them by your doctor or perhaps available over-the-counter, there is also the option of grey market foreign pharmacies. These are simply pharmacies in another country, although these often take some hurdles to purchase from. \textit{Dutasteride} and \textit{finasteride }are generally the easiest to get over-the-counter because of their commonality as hair loss medication.

 

\section{MYTHS AND MISCS}\label{MM}

\subsection*{Common Questions}
\addcontentsline{toc}{subsection}{\textemdash{} Common Questions}

\subsection{Should I be worried about blood clots?}\label{11-1}

Yes and no. It is true that there is a correlation between estrogen dosages/levels and blood clot risk, but this is primarily related to the route of administration and the type of estrogen. Synthetic estrogens are the rightful cause of scorn and do lead to significantly increased blood clot risk, but bioidentical estrogens are not as concerning. In particular, the route of administration makes a major difference. Oral bioidentical estrogen passes through the liver which is what causes the increased blood clot risk. Injections bypass the liver, and there's no evidence to suggest nor reason to believe that injections of bioidentical estrogen provide any significant risk increase beyond the innate differences between testosterone and estrogen. The pervasive fearmongering towards all estrogen has persisted for decades despite these differences.

\textbf{If you are undergoing surgery, please know that pausing hormones out of concern for blood clots is no longer recommended by WPATH.} Many surgeons still include it in their pre-surgery guidelines out of concern for blood clots, but this is torture that has been disproven and even WPATH doesn't recommend it anymore. Remarkable, I know. Per \href{https://www.tandfonline.com/doi/pdf/10.1080/26895269.2022.2100644}{WPATH SOC 8 Statement 12.19}: \blockquote{After careful examination, investigators have found no perioperative increase in the rate of VTE [KT: \textit{venous thromboembolism}, i.e. a blood clot] among transgender individuals undergoing surgery, while being maintained on sex steroid treatment throughout when compared with that among patients whose sex steroid treatment was discontinued preoperatively (Gaither et al., 2018; Hembree et al., 2009; Kozato et al., 2021; Prince \& Safer, 2020).} I should put this in another question entirely, but to not break links, it would have to be at the bottom of a section and I think this is too important for that, so I note it here. A very important clarification that I should have had sooner.

\subsection{Is it okay to use nicotine while on HRT?}\label{11-2}


However, to be abundantly clear, \textbf{this does not mean that you cannot or should not take estrogen. The downsides of not taking estrogen at all far exceed the downsides of using nicotine.} This section is simply seeking to make you aware of any increased risks and potentially slower transition as a very strong recommendation and encouragement to quit. One step at a time.

\subsection{Is there benefit to starting at a low dosage vs a high dosage?}

To the best of knowledge, no. Sex hormones are not like other drugs that need to be titrated to manage side effects as we know the dosages that work for the majority of people, so personally I view “starter dosages” and “antiandrogen first” regimens as medical abuse. Some people believe that mimicking the slow timeline of puberty might be best (even though there are far more things happening than just estrogen levels), but there’s no evidence to support this. An orchiectomy day one might be best for all we know, but who is going to do that the moment they decide they are trans and/or want to start HRT?


\subsection{Does body weight affect dosage?}

No. Because there is no “optimal” blood level for estrogen and because the therapeutic range of acceptable levels is so wide, body weight does not meaningfully affect dosage for HRT. It is for the same reason that slight deviations in dosage are unlikely to affect how you feel. There is no such thing as being “too light” or “too heavy” for HRT in any capacity.

Adjusting your dosage in increments of 0.1mg is a difference that should not be expected to be perceived simply because our bodies are not sensitive enough to such exact measurements, let alone the high possibility of imprecision when performing an injection that makes that certainty of this measurement unlikely. In other words, the accuracy of your dosage is more important than the precision.

\subsection{Is there such a thing as starting estrogen too late?}

\textbf{No.} No matter when you start, estrogen is able to do a LOT and a proper regimen will be able to have powerful results. Sex hormones are some of the strongest hormones in our body in terms of our appearance. Everybody always wishes that they could’ve started sooner, but that’s no reason not to start now. Even if you’ve been on estrogen for years, there is still a benefit to be had in improving the quality of your regimen.


\textbf{No.} There is not an arbitrary time where estrogen suddenly stops working. Various numbers are given and usually it’s either 1) entirely made up or 2) pointing to a study that only went for X years. Doctors in particular love to tell trans women not to expect more than B cup breasts (which isn’t even how breast sizing \textit{works}, but I digress) or for any growth after 2 years, but this is simply not true. There are cases of people who restarted estrogen after stopping for many years and still experiencing new growth.

\subsection{I haven’t seen any changes in years on injections. Would swapping back to pills make a difference?}


\subsection{Is low energy and low libido normal on HRT?}

Generally, no. How libido is expressed changes in the beginning, but the vast majority of the time that someone experiences abnormally low libido it’s because they haven’t gotten their hormones sorted. The same goes for low energy. Get your hormones squared away, and barring that, check your diet/vitamins next. Make sure you don’t randomly have critically low vitamin D levels or something like that. It happens more often than you think.



\subsection{Do we want to mimic the estrogen cycle of cis women?}\label{11-10}

Quel voie d'administration prendre pour mon oestrogène ?

Par injection. C'est le moyen le plus facile, prédicible, sûr, efficace et abordable de faire une transition hormonale. On dit même que pour certains, le moment de l'injection devient un rituel attendu, et que pour d'autres,  
Similar to the last question, it’s important to understand what is happening. The unique hormone curve produced by your particular ester, your dosage, and your frequency can cause changes in your mood as your estrogen levels oscillate between injections. Some trans women liken this phenomenon to a period, but the underlying cause for these physiological changes is different and is usually a sign that your regimen needs tweaking so that you feel the best that you can as suffering is not virtuously feminine. Pain and discomfort are not requirements for womanhood nor should we assert ourselves based on bioessentialist arguments. The exception here are the intersex trans women who have a uterus and literally are having a period, in which case: yeah duh. See Question \ref{11-35}.

\subsection{Can too much estrogen convert to testosterone?}

\textbf{No.} Aromatase is the enzyme responsible for converting testosterone into estrogen, but there is no mechanism to convert estrogen into testosterone. This cannot happen. This is a completely false myth and you should be immediately wary of the knowledge level of anyone who says it to you. Unfortunately, it is doctors who repeat this myth the most.

\subsection{Does bottom surgery cause an increase in testosterone?}

No. This is not a thing. There is not a magic mechanism that suddenly causes testosterone to increase the moment that testicles are removed. Even if magic was stored in the balls, this simply isn’t how hormone production works. “Well, your adrenals…” They don’t work like that either. The only possible rare exception would be undiagnosed adrenal hyperandrogenism conditions that were suppressed by an antiandrogen like \textit{spironolactone }prior to surgery which might show itself after antiandrogens are ceased. Please stop repeating this myth.

\subsection{How do I prevent/revert hair loss?}\label{11-14}

Mechanically, it is pretty simple. A standard HRT regimen alone is borderline magic (don’t ask where the magic is stored) in this regard already, but the inclusion of 5$\alpha$-Reductase Inhibitors (5-ARI) as discussed in Section \ref{AA} “ANTIANDROGENS” is recommended in more extreme cases to completely halt any loss. Topical minoxodil 5\% is the only thing that works to firm up your hairline beyond hormones alone, but keep in mind that aside from miracle cases, you’re only saving dying/dormant follicles. Dead follicles don’t come back.

If this alone is insufficient for you, hair transplant technology has improved significantly. The Follicular Unit Extraction (FUE) procedure is what you want to look into. Here is where in the future I will link a guide written by an expert on getting insurance to cover that, once she writes it. This is peer pressure. Watch this space.


Probably. HRT causes gradual body recomposition, so you can help encourage your body to shift through exercise. Keep in mind that this process is VERY SLOW, so it is crucial that you eat enough to fuel how patient you have to be. The growth hormones from muscle stimulation via strength training also play a role in breast development, so it’s probably a good thing even aside from the rest of the obvious health benefits of exercise.

This is NOT just the writer’s barely-disguised fetish; strength training is important for your health! I mention this because a lot of trans women believe that touching a dumbbell will make them look like the hulk. I get it, but if you have no testosterone in you and you aren’t on steroids, then you aren’t going to look like that. Let alone the constant time, effort, and diligence required to even get close.

\subsection{What should I exercise then?}\label{11-16}

Cardio is useful for living which is important. Lower body exercises will fill out your hips and glutes to accentuate your figure. Upper body exercises will improve your posture and support your breasts which will make them look bigger. In other words, everything. You’re on estrogen. Have you seen cis women athletes? Exercise will feminize you.

\href{https://docs.google.com/document/d/1-NyE5EY5TTaRRMhk7HlTbKJ7HifjEsA4jlDO1qKQVl0/edit?tab=t.0}{This guide was shared with me} \textcolor{red}{(Warning: Google Docs link)} and looks to be a good starting place. I will note that there aren't particular exercises that feminize vs masculinize as bodies don't work like that, but you may wish to focus more on lower body exerices and flexibility more than the typical lifter.

\subsection{Can estrogen really cause height shrinkage?}

Yes. It is possible that it’s related to water content changes within tendons and ligaments, but it is not something that has been studied so the cause is fully speculation. Scientists: free study idea!

\subsection{Can estrogen really cause foot shrinkage?}

Yes. See above.

\subsection{Can estrogen really cause any other kinds of shrinkage?}

Well, “use it or lose it” like they always say.

\subsection*{Sexual Health}
\addcontentsline{toc}{subsection}{\textemdash{} Sexual Health}

\subsection{How do I improve erectile function on HRT?}\label{11-20}

Aside from using it regularly, ways to improve erectile function include: 1) Improving your fitness and physical health, particularly your cardiovascular ability; 2) consider medication like \textit{tadalafil} or \textit{sildenafil}; and 3) consider testosterone supplementation (see Section \ref{T} “TESTOSTERONE”).

If you would like to read a longer explanation for how erectile function works, \href{https://stainedglasswoman.substack.com/p/how-to-maintain-your-penis-function}{this Substack article} provides a good overview of the topic.

\subsection{How do I increase cum/pre-cum volume on HRT?}

Don’t be embarrassed, it’s a common question. Sunflower lecithin and pygeum increase both of those. It seems to also make a difference for vaginal wetness and arousal for those who have had bottom surgery, but data and anecdotes are limited so it’s hard to say. Otherwise just be sure you drink enough water and have your nutrition in check.

\subsection{Can I lactate on HRT?}

Yes. Domperidone, fenugreek, sunflower lecithin, ample estrogen, and ample progesterone. Get a pump. Knock yourself out.

It should be noted that domperidone has side effects and risks associated with it, and that ability to lactate does not affect breast development. Newman-Goldfarb protocols would be what you want to look into.

\subsection{Can HRT change your senses and your perceptions, i.e. smell?}

You very likely were dissociated and depressed for years prior to starting HRT. The world is more vibrant now because you are no longer dissociating 24/7. The wonders of modern medicine!

It can, however, directly change your eye prescription. That can definitely happen.

\subsection{Can HRT change your sexuality?}

Similar to being dissociated as with above, HRT often incurs a lot more openness and acceptance with yourself which can cause a shift in how your sexuality presents itself. It is largely a semantics argument as to whether that is chemical or behavioral. A matter of perspective. 

\subsection{Should I be on PrEP?}

\textbf{Yes.}

\subsection*{Medical Malpractice}
\addcontentsline{toc}{subsection}{\textemdash{} Medical Malpractice}

\subsection{I heard that injections are actually less stable because you do them less frequently. Is that true?}

Only if you follow the dipshit WPATH SOC 8 guidelines that list a recommended regimen of \textit{estradiol valerate} or \textit{estradiol cypionate} in the range of 5-30mg every two weeks which, to be abundantly clear, you absolutely should never do in a million years. “Do no harm”, my ass. 

\subsection{But my doctor said-?}

The average doctor has essentially no training in anything related to trans healthcare, and \href{https://www.endocrine.org/news-and-advocacy/news-room/2017/endocrinologists-want-training-in-transgender-care }{4/5 endocrinologists have never had any formal training in trans healthcare}. It is most likely that you are their first trans patient and that they are inexperienced in the practical elements of managing a trans patient. Even among doctors who care a lot, they are often limited by conservative standards of care that they are forced to follow which do not always align with the care best for you. See above.

Please also be aware of “trans broken arm syndrome”, aka the tendency of doctors to blame everything on HRT. If your arm is broken, it's probably not “because of those hormones”!

And I should put this as a separate question but I don't want to break the formatting: in line with medical malpractice, there is no situation in which it is reasonable for a doctor to request to see or feel your breasts to “monitor growth” or for any other reason. It is far less common these days, thankfully, but it is sexual assault and completely unacceptable.

\subsection{My doctor won’t prescribe me injections. What do I do?}

Attempt to convince them, replace them, or seek DIY sources. Do not let a gatekeeping medical establishment prevent you from receiving the appropriate care that you deserve. \textbf{The most crucial aspect of interfacing with the medical system while trans is that you have to advocate for yourself. }This is compounded with disability, ethnicity, and other afflictions that scare doctors like womanhood.

\subsection{How does HRT for menopausal cis women relate to HRT for trans women?}\label{11-29}

While we generally have different goals and crucially have very different dosage requirements, there is an immense amount of overlap in experience for trans women and menopausal cis women. Medical misogyny in the form of incompetence, dismissiveness, antagonism, and/or misinformation is something that we unfortunately both experience. It is for this reason that it is paramount to build solidarity on this front. To give an example of what I mean, \href{https://www.youtube.com/watch?v=W0XW6av2wLQ}{the first 30-40 minutes of this interview} will likely sound extremely familiar to you if you would like to raise your blood pressure. The interviewee herself notes the connection too! The WHI ruined the lives of countless women.

\subsection*{Intersexuality and Comorbidities}
\addcontentsline{toc}{subsection}{\textemdash{} Intersexuality and Comorbidities}

\subsection{What’s up with Ehlers-Danlos Syndrome?}

This connective tissue disorder doesn’t actually relate to HRT but a lot of trans people have it so congrats in case this is how you learned that you do too. Aside from general cardiovascular long term concerns to maybe look into, keep up with strength training so that your joints work. Look into that elsewhere though. See Question \ref{11-16}.

\subsection{What kind of intersex things should I keep in mind?}

Throughout this guide, I have mentioned intersex conditions vaguely. Below is a short list of things that might be useful for you to know in your travels for yourself or for a friend. 

\subsection{What’s up with Klinefelter Syndrome?}

This is a relatively (considering chromosomal mutations) common intersex-related condition that some trans women might not realize that they have as the two can overlap. It generally presents as low testosterone at the start of puberty. Good for you to know the name, just in case.

\subsection{What’s up with Persistent Müllerian Duct Syndrome (PMDS)?}

Another “I’m putting this here because this might be the first time you’ve even heard of the term” intersex-related condition that can affect some trans women, however few that may be since we don’t have numbers. The possible presence of an underdeveloped uterus leads to some possible complications and oddities. You probably extra want to have progesterone to avoid uterine cancer risks.

\subsection{What's up with ovotesticular syndrome?}

This intersex condition in particular can cause early level fluctuations which made lead to confusing test results due to the presence of both ovarian and testicular tissues, either separate or combined in an \textit{ovotestis}. This presents in many different ways which HRT can interact with as you begin suppressing \textit{luteinizing hormone} (LH). A uterus may or may not be present, multiple sets of gonads could be present, and/or it could look outwardly typical.

\subsection{What’s the difference between intestinal cramps and uterine cramps?}\label{11-35}

These are commonly misattributed in early transition as a symptom of intersex conditions. Intestinal cramps are widespread and diffuse across your abdomen, whereas uterine cramps are highly concentrated in a location somewhere below your belly button and tend to be sharp stabs/contractions in rapid succession. Like the inside of your body is used as a stress ball. Very different!

\subsection{What about other intersex conditions?}

I have listed a few notable ones, but there are far more expressions and ways of testing them that go far beyond the scope of this guide. Anecdotally, prevalence is higher than average among trans people so basic familiarity with this is useful.

\subsection*{Oddball Questions}
\addcontentsline{toc}{subsection}{\textemdash{} Oddball Questions}

\subsection{Many DIY sources only take crypto. Is that required? How does that work?}

There are other guides that cover this in better depth than I can on how to use crypto safely, including some vendors who have their own guides. But yes, crypto is often required for a lot of reasons. “Crypto” means a lot of things, but using it as a currency was the original point after all. It’s mostly just a pain in the ass. Monero (XMR) is good.

\subsection{What about Selective Estrogen Receptor Modulator (SERM) drugs for nonbinary regimens?}

Some people use SERMs as a part of a transition that is not looking to feminize as much for a more androgynous look, but it’s pretty much entirely uncharted waters thus why their mention is otherwise absent from this guide. You’re on your own if that’s something you want to explore, so please be safe. I don’t personally rate them very highly as I have not seen much to suggest that they work well for how people usually think or want them to work, at least not without a lot more caveats, but obviously there are people who like them. It's just not something I feel comfortable giving recommendations for.

The various proposed nonbinary regimens are often highly individualized because they are specific to what a persons' particular goals are. All HRT should be individualized to a degree, but there is often more variation in desired outcomes when people ask about androgyny. Hormonally, it is nontrivial. Everything stated in this guide should be treated solely as a starting place if you are wanting to experiment with something more complicated, but do remember that there is much more to achieving transition goals than just hormones alone.

\subsection{Are things like “herbal HRT” or “phytoestrogens” legitimate?}

\textbf{No.} If someone is telling you they have “herbal HRT”, they are telling you they have snake oil. The only thing that is going to feminize you is estrogen, not plant estrogens. No amount of “natural” products are a replacement for estrogen itself. This isn’t a common scam and you probably already know, but just in case you run into it, now you know for sure. If it smells like bullshit, it’s probably bullshit. Unless we’re talking about bug steroids in which case yeah those are actually cool. Won’t feminize you though.

\subsection{Is the Reddit Doctor that people constantly talk about Good?}

No.

\subsection{I hear DIY estrogen is made in a bathtub. Is that true?}

No. I honestly have no idea where or why this joke started that people now take seriously, but there’s no step in any process where a bathtub would even be considered. Don’t believe everything you read online. I don’t even know what you could even theoretically do with a bathtub, unless you think estrogen vials are full of the bathwater of trans women. I don’t know why you would think that though. It’s obviously cum.

\subsection{How does HRT affect fertility?}\label{11-42}

It is important to understand that this is extremely understudied so exact figures cannot be stated, and given the seriousness of pregnancy, I urge you to practice safe sex and lean on the side of caution where possible. HRT itself can, and likely will, make you infertile eventually, but only through full suppression of the HPG axis (See Question \ref{2-3}) over a long time span. In other words, if you haven't had bottom surgery of any kind and you are on an HRT regimen that is less capable of HPG axis suppression (such as pills), then this is more of a consideration.

\textbf{If the HPG axis is not suppressed then it is fully possible to impregnate someone}, and the timeline for sperm maturation is long enough that this is true even after the HPG axis has been initially suppressed for \textbf{multiple months}. Please take this very seriously. Full HPG axis suppression for at minimum six months, perhaps closer to a year out of an abundance of caution, is recommended.

\subsection{Is infertility from HRT reversible?}\label{11-43}

It is theoretically possible to reverse HRT-induced infertility, assuming you weren't already infertile prior to HRT (a large assumption!), but there are not many documented cases of this so the full efficacy of fertility restoration after long-term HRT is unknown. The process would involve restarting the HPG axis with a variety of medications along with entirely stopping HRT, which would in essence require a hormonal detransition for likely six months at minimum, and even then sperm quality is not certain or guaranteed. It is not something that should be planned for, to say the least, so planning around it would be wise. A sperm bank would be recommended before or early in HRT, financially permitting, if potential biological children are a priority and if a future relationship where that is possible/desired is likely.



\section{CREATINE}

\subsection{What is creatine?}

Creatine is an organic compound in your muscles and in your brain. It recycles ADP into ATP which is important for energy production in your body, especially initial high burst applications before other energy systems take over.

\subsection{Isn’t it like a steroid or something that bodybuilders use?}

No. Bodybuilders and athletes like it because having more energy means more activity before getting tired. They aren’t the only ones who use it since it is basically the \#1 supplement in terms of things that are actually useful and are actually researched. 

\subsection{How is creatine related to HRT?}

It isn’t! But it’s something I yell about because I think it’s good and I am tired of repeating myself because people keep asking and you’re reading this anyway, aren’t you? I love a captive audience. My standup routine is at the bottom.

\subsection{Okay well why should I take creatine then?}

What a great question! It’s good for your brain and your muscles. Creatine is often found in relatively low concentrations for many people depending on their diet, especially people who don’t eat meat. There is compelling research about various chronic fatigue and post-viral conditions (long COVID in particular) being related to depleted creatine reserves in the brain, so some people find cognitive benefits from supplementing it. It isn’t magic but it is dirt cheap so it is worth trying in my opinion.

\subsection{What are the forms?}

Just \textit{creatine monohydrate} powder is what you want. The pills tend to be low dosage and are up charging you anyway, while gummies often destroy the creatine in the creation of the gummy. A lot of brands include creatine in some sort of mix but the pure stuff is usually cheaper.

\subsection{How do I take it then?}

The general recommendation is 5-10g daily dissolved in some sort of liquid. It dissolves best in things that aren’t just water. It’s mostly flavorless, so just throw a scoop or two in your coffee or a smoothie and call it a day. It can be a little chalky or gritty depending on the quantity and the fluid.

\subsection{Does it matter when I take it?}

Not really. It doesn’t have an immediate effect like that which is why it’s silly that it’s microdosed in pre-workout mixes. Take it whenever it’s convenient for you.

\subsection{How does it work then?}

It builds up in your body to a maximum level of saturation over a week or two. Then you just maintain that and reap the rewards (of maybe feeling better).

\subsection{Do I have to do a “loading” phase of taking more at first?}

Probably not. Unless you’re on some sort of intense training time crunch or something, this probably doesn’t matter at all. Just take whatever is convenient with some regularity.

\subsection{What are the side effects?}

Slight weight gain may be possible because of increased water weight in your muscles (which to be clear is Good, so don't be alarmed). If you don’t take it with water, or if you take too much at once, you might get a tummy ache. Ouchie.

\subsection{Who shouldn’t take it?}

People with kidney issues. Not because it causes them, but because creatinine (Different spelling! Creatine becomes creatinine) is used as a marker in lab tests for a number of kidney issues and supplementing might give a false positive. Just keep it in mind.

\subsection{Do you have any brand recommendations?}

No. It shouldn’t really matter. Just get whatever seems reputable and is a reasonable price. I’d give a recommendation for the one I like but when I asked the brand for affiliate link they turned me down, so their loss! No free clout.

\subsection{You seriously put creatine into this document, huh?}

Yeah it’s pretty funny. It’s not my fault that I joked about it and people told me it legitimately helped them because now I feel obligated to keep talking about it!!!

 

\section{CLOSING REMARKS}

If any of the following are true:

\begin{itemize}
\item you are still mad at me despite the disclaimer;

\item you spotted an issue or typo;

\item you have a clarifying question that should be put into the text;

\item you have an objection that hopefully isn’t an Uhm Ackshually;

\item you wish to sing my praises;

\item you wish to pledge fealty; 

\item you wish to send tithes my way;
\end{itemize}

Then please feel free to contact me and I’ll see what we can do. Bluesky is the easiest contact point, and you can DM me for my Signal. Otherwise, thank you for reading and I hope it helps.

\textbf{If you would like to donate to support this project,} \href{https://cash.app/Katitties}{CashApp}, \href{https://ko-fi.com/katitties}{Ko-Fi}, and \href{https://account.venmo.com/u/katitties}{Venmo} all work. I appreciate it!

And lastly: \textbf{The most important thing that you can do as a trans person is to live.} For as much as this document is a manual, it is in equal measure a message to you as a trans person that your existence is a gift upon the world, your presence is a blessing on those around you, and that you deserve to be treated with respect. Even if you do nothing else, your life is a feat worth praising. Thank you.



\section*{FRIENDS OF PGHRT}\label{FOPGHRT}
\addcontentsline{toc}{section}{FRIENDS OF PGHRT}

Across this document is a scattering of links to other guides and resources. Below is a consolidation of them which will also include more links to external resources as time goes on, ideally by other trans people. For the privacy minded or noided, note that some of these are Google Docs links.

\begin{enumerate}
  \item \href{https://startwith4mgestradiolenanthateweeklyandtestatonetothreemonths.com/}{SW4EEWATAOTTM} - TL;DR for PGHRT
  \item \href{https://hrtcafe.net/}{HRT Cafe} - HRT Resource Aggregator
  \item \href{https://transfemscience.org/}{Transfeminine Science} - Informational resource for trans medical literature
  \item \href{http://estrannai.se}{Estrannai.se} - Estradiol Pharmacokinetics Playground
  \item \href{https://globoho.moe/}{Globoho.moe} - Thailand Orchiectomy Medical Tourism Travel Guide 
  \item Julia's FUE Guide - COMING SOON, I'M BULLYING HER TO WRITE FASTER
  \item \href{https://docs.google.com/document/d/1-NyE5EY5TTaRRMhk7HlTbKJ7HifjEsA4jlDO1qKQVl0/edit?tab=t.0}{Sky's Feminine Figure Beginner Program} - An exercise regimen geared towards trans fems
  \item \href{https://docs.google.com/document/d/114sztSw1aVWM2pXLDl9NrHklyvewz3EmFiHiisjM71k/edit?tab=t.0}{Sky's Diet 101} - A guide towards adjusting weight in a healthy way
  \item \href{https://stainedglasswoman.substack.com/p/how-to-maintain-your-penis-function}{How to Maintain Erectile Function on HRT} - A longer form explanation on the "use it or lose it" phenomenon
  \item \href{hhttps://docs.google.com/document/d/1DXFxzN0XTudPZez_SO61fpqncRLPH_Be_QG_8Pcz9LU/edit?pli=1&tab=t.0}{Biohax Guide Googleslop Edition} - Trans Masc DIY Guide
\end{enumerate}

\section*{ABOUT THE AUTHOR}
\addcontentsline{toc}{section}{ABOUT THE AUTHOR}

Katie Tightpussy is an award-winning author and professional trans woman with nearly a decade of experience in the field of transgender. Her accomplishments include transiferating her sex through the novel technique of cross-sex hormone injections, being physically unable to shut up, and utilizing a very fortunate set of hyperfixations as they relate to transbobulation of the humors. She spends her days in the idyllic rural countryside of Los Angeles scheming of new ways to achieve world domination and enjoys riding her bicycle. Media inquiries can reach her agent at \href{http://katietightpussy.com}{katietightpussy.com}.

 

\section*{DISCLOSURES}
\addcontentsline{toc}{section}{DISCLOSURES}

No robot girls were harmed in the making of this document, including any usage of generative large language models. The author does not endorse any reproduction without attribution nor scraping of this work. Leave those poor robot girls alone.

The author declares an attraction towards women and acknowledges a potential conflict of interest for the existence of more beautiful trans women in the world.

 

\section*{ACKNOWLEDGEMENTS}
\addcontentsline{toc}{section}{ACKNOWLEDGEMENTS}

Though the text is primarily my voice, this document would not be even half as good without the contributions, feedback, and suggestions from others involved at every step along the way. A good reminder as ever that transition is not something best done alone.

Many thanks to Q, R, RM, and S in alphabetical order for close review and generally being fun nerds to talk to; love y’all. Special thanks to CB and J for close review that also inspired some very good bits. Thanks to KG for additional intersex information. Thanks to w [sic] for additional injection resources. Thanks to BIR collectively for a plethora of crucial nerd nitpicks. Appreciation for general review from C, JTP, K, S, and V. Thanks to everyone on Bluesky who encouraged me to write this up in the first place, and everyone over the years sharing knowledge. And of course: much appreciation to all HRT nerds, even when we disagree, since we’re all trying to do the best for our community where we’ve otherwise been let down. Keep up the good work everyone. 

Shout out to my IB Chemistry HL teacher many years ago who quite reasonably doubted my studiousness even though I’m now putting much of that knowledge to use for the art of transsexuality; go figure. 

 

\section*{CHANGELOG}
\addcontentsline{toc}{section}{CHANGELOG}

\noindent \href{https://github.com/Juicysteak117/pghrt/}{Source code available here on GitHub.}

\noindent Full Compilation Datetime: \DTMnow

\noindent(There aren't LaTeXML bindings for \texttt{datetime2}, \texttt{hanging}, or \texttt{hyphenat}, so the formatting is slightly ugly. If you'd really like to help me out, please write those bindings!!!)

\noindent 2025-08-20: Initial release. 15.9k words.

\noindent 2025-08-20: A lot of typos and minor verbiage tweaks. Added Question \ref{8-18}.

\noindent 2025-08-21: Typos grow on trees. Added Question \ref{5-27}.

\noindent 2025-08-21: More tweaks. Opted to remove “WHY PROG” from Question \ref{8-17}. 17.0k words.

\noindent 2025-08-22: Nitpicks, clarifications, and typos. 17.2k words.

\noindent 2025-08-24: A few more twinks sorry tweaks. 17.3k words.

\noindent 2025-08-27: How long until remaining typos are embarrasing? 17.3k words.

\noindent 2025-08-28: Reduced ambiguity in a few areas. 17.4k words.

\noindent 2025-08-29: Additional clarity for frequencies in Section \ref{td}. 17.5k words.

\noindent 2025-09-01: Sisyphus boulder meme captioned fixing typos dot png. 17.5k words.

\noindent 2025-09-07: Added donation links per request. That's very kind. 17.5k words.

\noindent 2025-09-07: Few more tweaks. Clarified an additional common progestin. 17.6k words.

\noindent 2025-09-19: Added Question \ref{4-16} plus tweaks. 17.7k words.

\noindent 2025-09-23: A wide variety of clarifications up and down the line. 18.1k words.

\noindent 2025-09-24: Added an important note about surgery to Question \ref{11-1}. 18.3k words.

\noindent 2025-09-24: “Katie my doctor told me-” It never ends. 18.5k words.

\noindent 2025-09-26: Another pass of clarification edits. Yes I should have a git diff. Sorry that I don't. I thought I'd be done by now anyway! 18.7k words.

\noindent 2025-09-30: Added some cross references for clarity. 18.7k words.

\noindent 2025-10-02: More cross references. Likely will do another pass. 18.8k words.

\noindent 2025-10-02: Added a big bold warning about recapping to Question \ref{5-13} because SOMEONE didn't watch the video smh. 18.9k words.

\noindent 2025-10-10: Added Question \ref{11-42} and Question \ref{11-43} per request. Honestly I just forgot about fertility being a thing lol. Also added the Friends Of PGHRT postword section. 19.5k words.

\noindent 2025-10-10: Added Gretchen's Version (.txt) and fixed formatting. 19.5k words.

\noindent 2025-10-11: Added permalinks to everything, yay! And finally made a git repo. Look at me being a big girl, wow. 19.5k words.

\noindent 2025-10-11: Added an external link to Question \ref{11-20}. 19.6k words.

\end{document}