\documentclass{article}
\usepackage{hyperref}
\usepackage{float}
\usepackage{csquotes}
\usepackage[style=iso]{datetime2}
\usepackage[usenames,dvipsnames]{color}
\usepackage{booktabs}
  \setlength\heavyrulewidth{0.20ex}
  \setlength\cmidrulewidth{0.10ex}
  \setlength\lightrulewidth{0.10ex}

\usepackage[font=normalsize,labelfont={bf}]{caption}
  \captionsetup[table]{aboveskip=3pt}

\hypersetup{
    colorlinks=true,
    linkcolor=blue,
    filecolor=magenta,      
    urlcolor=magenta,
 }
 
\usepackage{graphicx}
\graphicspath{ {../img/} }
\newcommand{\ts}{\textsuperscript}
\renewcommand{\abstractname}{CLAUSE DE NON-RESPONSABILITE}
\title{UN GUIDE PRATIQUE POUR LE THS FEMINISANT}
\author{\href{https://katea.gay/}{Katie Tightpussy}}
\date{\today}
\setcounter{section}{-1}
\urlstyle{same}

\begin{document}
% TODO: regarder les changements sur le document principal et les intérger
% TODO: œ au lieu de oe
% TODO: relire le document pour le sens général des phrases
% TODO: relire le document pour les fautes d'orthographe
% TODO: vérifier que les titres sont traduits


\maketitle
\tableofcontents
\begin{abstract}
  Je ne suis pas médecin. Je ne travaille pas dans le domaine de la médecine. Je ne suis pas une professionnelle du domaine médical ou d'un domaine associé. Je suis une amateure qui offre des opinions d'amateure basées sur mon mes expérience et mes études. Toutes les informations et les affirmations ci-dessous doivent donc être traitées comme des opinions et non pas des vérités scientifiques établies ou des avis médicaux. Ce guide priorise les éléments découverts par la communauté, la qualité de la recherche scientifique sur ce sujet nous faisant défaut. En bref, c'est votre dos.
\end{abstract}


\section{AVANT-PROPOS}

Le but de ce document est de cataloguer mes pensées et mes opinions à propos du traitement hormonal de subsitution (THS) féminisant, car je pense que les différents wikis déjà existants manquent d'efficacité. Ce sont des ressources inestimables, mais de mon point de vue, ces wikis ne sont pas destinés aux personnes qui cherchent des conseils clairs et rapidement traduisibles ne action, mais plutôt à ceux qui cherchent à connaître des processus biologiques liés à la transition. Mon but ici est de fournir une référence exhaustive et rapide qui contient des réponses à toutes les questions que je vois souvent sur comment faire son THS de manière efficace et sûre, grâce à des connaissances que j'ai accummulé à travers les années, dans le but de démystifier ce processus qui sauve des vies, autant pour les personnes qui cherchent à savoir si le THS est pour eux que pour pour les transexuels vétérans. Je pars du principe que tu as déjà une idée de ce que fait le THS. Au cas où tu l'ignorais: le THS a beaucoup plus d'efftes que tu ne le penses. Ce qui est une bonne chose. \textbf{Changer son sexe, c'est fun et cool. Vraiment une expérience que je recommande.} Tu mérites des soins de transition de qualité, et tu es la personne la plus capable pour prendre les décisions qui te conviendront le mieux. J'espère que ce document t'aidera pour prendre tes décisions, et un jalon dans ton apprentissage si tu souhaite continuer à accumuler des connaissances dans le domaine.

Aussi, reste loin des sub-reddits trans. Fais-moi confiance là-dessus, ok? Au minimum évite /r/mtf, lui est vraiment horrible. Ce ne sont pas des endroits sains ou de sources de sagesse. Ces endroits te versent directement des vers pourris dans le cerveau, pendant des années. C'est le meilleur conseil que je puisse te donner.

Pour les mecs, certaines sections de ce documents sont tout de même très à propos, mais il y a évidememnt de grosses différences de but et de résultats. \href{https://docs.google.com/document/d/1DXFxzN0XTudPZez\_SO61fpqncRLPH\_Be\_QG\_8Pcz9LU/edit?tab=t.0}{Ce guide pour le THS masculinisant} \textcolor{red}{(Attention : lien Google Docs (et guide en anglais))} à l'air vraiment bien, mais je ne l'ai pas complètement examiné, donc reste vigilant. Ils devraient vraiment faire un mec trans comme Katie Tightpussy. Un type qui s'appelerait genre Oliver Longdick, ou Xavier pourquoi pas. 

\textbf{Si jamais tu veux aider ce projet,} \href{https://cash.app/Katitties}{CashApp}, \href{https://ko-fi.com/katitties}{Ko-Fi}, et \href{https://account.venmo.com/u/katitties}{Venmo} marchent tous. Merci beaucoup!

\subsection*{Comment utiliser ce document}

Ce document est organisé comme une série de questions/résponses, de telle sorte que chaque question et chaque section forment une suite logique. Je t'encourage fortement à le lire séquentiellement, ça se lit comme une conversation qui devrait répondre à la totalité de vos questions (incluant celles que tu ne savais pas que tu te posais), même si c'est très long. Prends ton temps et lis le à ton rythme. 

Sinon, tu peux utiliser la table des matières pour naviguer directement vers une section ou une question particulière, ce qui est très utile si ce n'est pas ta première visite. Je recommande de garder ce document dans un coin pour y revenir à chaque fois que tu as des questions à propos du THS. C'est un peu lourd à absorber en un coup, donc pas de soucis si ça te prends quelques relectures pour tout intégrer, ne te presse pas. 

\noindent\textbf{\href{pghrt.pdf}{Ce document peut aussi être téléchargé en pdf en cliquant ici. Vraiment, télécharge-le et garde-le quelque part, on sait jamais.}}

\noindent\href{pghrtgretchensversion.txt}{Si tu préfères, tu peux lire une version de ce texte en .txt style années 90-2000.} Par contre je ne garantis pas qu'il soit à jour. (Le texte est en anglais)

\noindent\textbf{Si jamais tu es intéressé.e pour faire une traduction ou une version alternative de ce guide, n'hésite pas à me contacter !}



\section*{DEDICACE}
\addcontentsline{toc}{section}{DEDICACE}

Ce document est dédicacé à toutes nos soeurs qui ne sont plus. Puisse-t-on porter la lumière de leur torche dans un jour nouveau.
%  TODO: j'ai fait une traduction littérale de la deuxième phrase, je sais pas trop comment ça rend ceci dit

\section{INTRODUCTION}

\subsection{Est-ce que la prise d'oestrogènes est risquée ?}

Avec des hormones modernes, et donc bioidentiques, le THS est tout ce qu'il y a de plus sûr. Au final, tout ce que tu fais, c'est changer le carburant principal qui fais tourner ton corps pour obtenir un équilibre un peu différent avec des hormones qui étaient déjà dans ton corps. Même si c'est un peu plus complexe que ça quand on va chercher à optimiser le processus, l'idée de base de changer sa biologie permet pas mal de largeurs. Le corps est assez malléable et vas pouvoir ajuster ton traitement vers quelque chose qui te fais aller mieux.

\subsection{Quel voie d'administration prendre pour mon oestrogène ?}

Par injection. C'est le moyen le plus facile, prédicible, sûr, efficace et abordable de faire une transition hormonale. On dit même que pour certains, le moment de l'injection devient un rituel attendu, et que pour d'autres,  

\noindent\underline{\textbf{Rappel important : Peu importe la manière dont tu prends de l'oestrogène, c'est toujours mieux que de ne pas en prendre.}}

\subsection{Pourquoi est-ce que tu ne recommandes pas les pilules, les patches ou les gels ?}

La raison principale est que toutes les autres voies d'administrations ont des inconvénients que les injections n'ont pas. C'est pas qu'elles ne fonctionnent pas, mais tu mérite mieux que de devoir supporter des inconvénients parfois durs à vivre. Mais je le répète : \textbf{toutes les formes de THS permettent d'avoir des résultats satisfaisants}. Ca ne veut pour autant pas dire que toutes les formes de THS se valent.

\subsection{Est-ce que le dosage d'oestrogène est équivalent selon la voie d'adminstration choisie ?}

Non. Et c'est suffisamment important pour ne pas être relègué dans la Section \ref{MM} "Mythes et autres". \textbf{On ne peut pas comparer les dosages d'oestrogènes sans prendre en compte la voie d'administration choisie}. 1mg en pilule n'est pas la même chose qu'1mg en injections par exemple. Les différentes voies d'administration ont des propriétés différentes, ce qui affecte la quantité d'oestrogènes absorbée dans le corps (\text{"biodisponibilité"}), à quelle vitesse, et donc la demi-vie corresponante.

\subsection{C'est quoi une demi-vie ?}

Pour faire simple, la \textit{demi-vie} d'une substance est le temps qu'il faut pour que la moitié de cette substance soit éliminée. Dans le contexte du THS, c'est ce qui détermine coombien de temps un dosage reste actif dasn ton système, et donc la fréquence de prise d'hormones. C'est ce qu'on va appeler ton cycle hormonal, qui forme une courbe. Tes niveaux montent, arrivent à un pic, puis retombent. Les propriétés de cette courbe (comment tes niveaux d'oestrogène changent en fonction du temps) sont importants.

\subsection{C'est quoi le soucis avec les pilules ?}

Leur problème principal, c'est qu'elles sont associées à un risque plus élevé de caillots snaguins et de coagulation du foie. On peut diminuer ces risques en partie en prenant ces pilule de façon sublinguale ou buccalle (en faisant dissoudre la pilule sous ta langue ou entre ta gencive et ta joue, respectivement), plutôt qu'oralement (en avalant la pilule normalement), pour éviter de passer directement par le foie pour métaboliser l'oetrogène. Cependant, même avec les méthodes des voies sublinguales ou buccales, il est fréquent de tout de même avaler une partie du contenu de la pilule, donc on peut imaginer qu'il reste un risque résiduel. Attention, le risque reste très peu élevé, (e.g. \textit{le paracetamol} est plus risqué que l'oestrogène par un ordre de grandeur), mais \textbf{ces complications sont d'autant plus impactante qu'elles se composent avec les risques liés à la prise de nicotine sous oestrogène}. Pour en savoir plus, tu peux consulter la Question \ref{11-2}. 

A part ça, il y a d'autres soucis liés aux pilules qui viennent principalement de deux de ses caractéristiques : 1) elles ont une demi-vie très courte et une mauvaise biodisponibilité, et 2) elles nécessitent souvent de les accompagner avec des anti-androgènes. Le premier point fait que les pilules sont difficiles à utiliser en monothérapie (dont on dicutera plus en détail ensuite), comparé aux injections. Le deuxième fait qu'on se retrouve avec un assortimment d'effets secondaires liés aux antiandrogènes que tu prendrais avec (voir section \ref{AA} "ANTIANDROGENES"). Prises ensembles, ces caractéristiques et autres degrés de variabilités font qu'on a plus souvent avec les pilules des mauvais régimes remplis d'effets secondaires (comme par exemple un manque d'énergie/libido et des lenteurs à obtenir des résultats) par rapport à d'autres vois d'administration. Aussi, il est plus difficile de stocker des pilules et selon la façon dont tu les obtiens, plus chères que les fioles d'injections. A noter aussi que si tu essaye de faire importer de grandes quantités de pilules depuis des distributeurs étrangers pourrait faire tiquer les douanes et mener à une saisie des médicaments, ce qui t'expose à une perte financière, voire des risques de poursuites judicières. \textbf{Si on te demande, tu ne sais pas qui a commandé toutes ces pilules.}

\textbf{Si, pour quelque raison que ce soit, tu utilises des pilules, s'il te plaît prends 4-8mg en voie sublinguale en espaçant tes prises dans la journée.} Tout ce qui est en dessous de 4mg n'est probablement pas un dosage suffisant.

\subsection{Qu'est ce qui ne va pas avec les patches ?}

\begin{itemize}
  \item C'est cher (plus que les pilules);
  \item Plus difficile de s'en procurer en DIY (seulement via des marchés gris);
  \item Te demandent généralement un antiandrogène (voir la  Section \ref{AA} “ANTIANDROGENES”);
  \item Peuvent causer une irritation de la peau;
  \item Nécessistent d'être appliqués 24h/24;
  \item Se décollent souvent (si ça t'arrive, demande des pansements qui couvrent le patch à la pharmacie);
  \item Ne permettent pas une absorbtion uniforme (à cause de la chaleur par exemple);
  \item Difficile de faire des stocks (difficile à acheter en grandes quantités);
  \item Permettent souvent pas de dépasser des niveaux équivalents à une ménopause même avec plusieurs patches en même temps.
\end{itemize}

\subsection{Et le gel alors ?}

\begin{itemize}
  \item Difficile à doser avec précision, ce qui mène à des niveaux non-uniformes;
  \item Requires regular application of goop due to a relatively short half-life;
  \item Requièrent une application régulière de truc poisseux à cause d'une demi-vie relativement courte;
  \item Peut-être sale (poisseux);
  \item Tu risques d'exposer les autres par contact
  \item Te demandent généralement un antiandrogène (voir la  Section \ref{AA} “ANTIANDROGENES”);
\end{itemize}

Il faut cependant noter que le gel nécessite peu d'éléments pour en produire soi-même ce qui peut-être avantageux selon les circonstances.

\subsection{Pourquoi pas les implants?}

\begin{itemize}
  \item Généralement beaucoup plus cher que n'importe quelle autre option;
  \item Peu de possibilités d'en obtenir;
  \item Les périodes d'ajustements de doses sont très espacées;
  \item Un implant déféctueux peut te faire avoir des niveaux insuffisants;
  \item Un implant cassé ou écrasé peut causer des niveaux trop hauts de façon inattendue;
  \item Impossible à obtenir en DIY.
\end{itemize}

Ce dernier point en particulier veut dire qu'il n'est quasiment pas possible de s'en procurer. C'est peut-être même la première fois que tu en entends parler. Tu vois le soucis avec ça ?

\subsection{Et pour les sprays ?}

Ils sont encore assez peu utilisés et on a peu de retours donc il n'y a pas grand-chose à dire à leur propos, mais ils ont pas mal d'avantages / inconvénients en commun avec le gel. Je note ça là essentiellement pour rappeler qu'ils existent.

\subsection{La différence est vraiment si énorme que ça ?}

\textbf{Oui.} C'est tellement le cas que j'ai écrit tout ce truc juste pour arrêter de me répéter tout le temps et envoyer un lien vers ici à la place. Un régime d'injection bien dosé est la meilleure forme d'oetrogène qu'on a pour obtenir des niveaux corrects en monothérapie.

 

\section{POURQUOI LES INJECTIONS}

\subsection{Qu'est ce qui fait que les injections, c'est si bien ?}

La régularité. La régularité est le truc important pour un THS. La régularité, ça veut dire que ton régime est stable, et la stabilité, c'est bien. Même les pires formes d'injections (on détaillera ça plus tard) te créront un cycle hormonal plus prédictible que toutes les autres voies d'administration, ce qui a beaucoup d'avantages.

\subsection{Est-ce qu'on a besoin d'antiandrogènes avec des injections ?}

En général, non. Si tu as un cycle d'injection qui est correctement dosé et espacé, qui donne des niveaux d'oetrogène suffisamment hauts tout le temps pour arrêter la production naturelle de testosterone, rendant les antiandrogènes inutiles, ce qui est préférable dans la plupart des cas. C'est ce qu'on appelle la \textit{“monotherapie”}.

\subsection{La monthérapie, ça marche comment ?}\label{2-3}

Pour faire simple, ton cerveau se fout un peu de savoir quel type d'hormone sexuelle est dans ton corps, du moment que tu en as en assez grande quantité. Si tu as tout le temps assez d'hormones dans ton corps, il ne voit pas d'intérêt à en produire plus. C'est sur cette "régularité" que les injections résussissent là où les autres voies d'administration sont plus en peine. Faire une monothérapie avec des pilules, par exemple, est quasiment impossible dans la plupart des situations. Si tu veux que je précise un peu, si on regarde l'axe HPG, la production \textit{d'hormone lutéinisante} (LH) et \textit{d'hormone folliculo-stimulante} (FSH) est supprimée par les niveaux élevées \textit{d'estradiol}, ce qui inhibe la production de GnRH et donc par extension de la production de testosterone dans les testicules. 

\subsection{Comment ça se fait que les injections soient moins risquées pour la santé ?}

Parce qu'on n'a pas besoin de les utiliser avec des antiandrogènes (voir la Section \ref{AA} “ANTIANDROGENES”), le risque à long terme associé à l'utilisation des antiandrogènes sur la santé est évité. Si tu utilises de l'oestrogène bioidentique et qu'elle ne passe pas par ton foie (voir la question \ref{11-1}), alors tu es au plus proche possible d'une prodcution naturelle d'oetrogènes par ton corps, ce qui enlève la plupart des risques pour ta santé

\subsection{Mais y'a pas des risques associés au fait de s'injecter une substance dans le corps ?}

En soi oui, mais avec un peu de technique, tout ce que tu risques c'est un bleu (see Section \ref{ts} “TECHNIQUES ET MATERIEL”). C'est un peu comme faire du vélo, une fois que tu sais faire, faut vraiment le vouloir pour faire n'importe quoi.
% TODO: vérifier que c'est le même nom de section

\subsection{D'accord, mais en quoi les injections sont plus simple que les autres voies d'administration ?}

Parce qu'une fois que tu as un régime stable, t'es tranquille. Tu t'en occupe moins souvent (e.g. une injection par semaine contre plusieurs pilules par jour), tu as des dosages précis, l'apport d'oestrogènes ne s'arrête pas en plein milieu de cycle à cause d'un patch qui se décolle, et tu n'as pas besoin de te déplacer pour aller voir un médecin souvent.

\subsection{Comment ça se fait que les injections soient aussi abordables ?}

Pour faire simple, c'est parce que tu as besoin de moins de principe actif. Une fiole de 5 mL contient à peu près un an de THS ne contient que 200mg d'oestrogènes, alors que l'équivalent en pilules par exemple (4mg * 365 jours = 1460 mg) en contient beaucoup plus. C'est une comparaison qui a ses limites, mais ça donne une idée. Une autre comparaison marrante : tu peux faire tenir une fiole qui contient 1 à 2 ans d'oestrogènes dans une boîte de 3 mois de pilules. 

\subsection{But I don’t have insurance / my insurance won’t cover it / pills are cheaper than injections with my insurance / injections are not available in my country / my doctor won’t prescribe injections?}
\subsection{Je ne peux pas en acheter / me faire rembourser en pharmacie, je fais comment ? (ou mon médecin ne veut pas m'en prescrire)}

Va voir la section \ref{sv} "OU TROUVER DES INJECTABLES". Ca va changer ta vie, et probablement te radicaliser dans la foulée.
% TODO: nom de section à regarder

\subsection{Est-ce que je peux passer aux injections même après quelques mois / années sous THS ?}

\textbf{Oui.} Je ne garantis rien, mais beaucoup de gens racontent avoir vu une différence notable après être passées aux injections même après quelques années sous THS. On parle d'un plus grand développement mammaire, une meilleure santé mentale, moins de soucis liés aux effets secondaires des anti-androgènes ou aux autres voies d'administration des oestrogènes. Vraiment, saute le pas, ça en vaut la peine.
% (Note de traduction: je fais partie de ces personnes qui ont vu une différence notable en passant aux injections)

\subsection{Oui mais j'ai peur des piqûres...}

On va pas se mentir, les piqûres dont peur, surtout au début. Beaucoup de gens n'aiment pas ça, parce qu'instintivement tu ne veux pas faire des trous dans ton corps, mais avec une bonne technique et de bons instruments, ça ne fait presque pas mal. Il y a énormément de gens qui avaient des cas sérieux de bélénophobie (peur des aiguilles) et qui maintenant trouvent l'expérience de l'injection ennuyante. C'est une peur commune et normale, mais c'est surmontable et ça vaut le coup de la surmonter. "Oh ce n'est pas si terrible que ça" est une réaction assez commune. Comme le dit le mantra: fais le en ayant peur, ça ira. 

\subsection{Est que la sensation des injections est semblables à une prise de sang ou à un vaccin ?}

Non. Une prise de sang utilise des aiguilles beaucoup plus larges et on pique dans un endroit beaucoup plus sensible, tout en te drainant du sang, ce qui est déplaisant en temps normal. Les vaccins contiennent des organismes qui causent des réactions immunitaires douloureuses parce que c'est des vaccins. Les injections d'oestrogènes ajoutent une petite quantité d'hormones dans ton corps qui te fait te sentir bien parce qu'après tu as des hormones dans ton corps. Je suis sûre que tu vois la différence. L'acte de s'injecter soi même peut aussi être plus facile que ce soit quelqu'un d'autre qui le fasse, selon ta personnalité.

\subsection{Est-ce qu'il y a des outils pour faciliter les injections ?}

Oui. Il existe des auto-injecteurs qui peuvent être très utiles si tu as des soucis moteurs par exemple. Pour plus d'informations, va voir la question \ref{5-21}, ou continue de lire.

\subsection{Tu comprends pas, \textit{je} suis spécial.e et je ne peux pas faire d'injections parce que j'ai des os en verre et une peau en papier et\textemdash{}?}

Je comprends que tu aies peur, mais si tu ne veux pas faire d'injections sous aucune circonstance et que tu n'as pas de contrindication comme le fait d'être hémophile, alors ne le fait pas. Tu peux juste dire ça, c'est ok. Quand tu changera d'avis, ce guide sera toujours là. Et si tu ne changes pas d'avis, pas de problème.
 

\section{TYPES ET DOSAGES}\label{td}

\subsection*{Vocabulaire clé}
\addcontentsline{toc}{subsection}{\textemdash{} Vocabulaire clé}

\subsection{Quels sont les différents types d'oestrogène injectable ?}

Les quatre types principaux utilisés pour le THS sont \textit{l'estradiol valerate} (EV), \textit{l'estradiol cypionate} (EC), \textit{l'estradiolm enanthate} (EEn) et \textit{l'estradiol undecylate} (EUn). Chacun de ces composés est un "ester" de \textit{l'estradiol} et va être converti en \textit{estradiol} dans ton corps.

A noter que dans certaines régions, les pilules sont vendues sous le nom \textit{estradiol valerate}, qui peut porter à confusion. Cette section ne se réfère qu'à sa forme injectable.

\subsection{Quelles sont les différences entre les types d'oetrogènes injectables ?}

La seule différence intéressante entre les esters est que chacun a une demi-vie différente, ce qui change leur courbe hormonale résultante affectant le dosage et sa fréquence.

\subsection{Un type d'oestrogène injectable est-il meillleur pour la féminisation qu'un autre ?}

\textbf{Non.} Les différences affectent uniquement le dosage et la fréquence de prise, ce qui change l'exprérience de manière qualitative. Ca peut rendre un ester préférable à un autre, mais les 4 types fonctionnent convenablement et partagent tous les avantages des injections.

\subsection{Quel type d'oestrogène choisir si j'ai le choix ?}

Si tu as le choix, \textit{l'estradiol enanthate} est préférable pour la plupart des gens au vu des niveaux exceptionnelement stables qu'il permet d'avoir, avec l'inconvénient que dans la plupart des pays ce choix est seulement possible si tu le fais en DIY (voir la section \ref{sv} "OBTENIR DES FIOLES"). Si tu passes par un docteur, tu auras probablement l'option de prendre de \textit{l'estradiol cypionate}, mais probablement seulement peu concentrée, ce qui peut être ennuyant en fonction de ta tolérance pour des gros volumes d'injections. L'injectible le plus prescrit (surtout aux USA), \textit{l'estradiol valerate}, permet toujours d'avoir des bons résultats, mais il est un peu ennuyant sur certains points qui font qu'il n'est pas préférable d'en prendre si tu as le choix (i.e. quand tu fais du DIY). Continue à lire.
% Du coup si je localise, il faudrait que je regarde selon les pays francophones comment ça fonctionne. En France, pas d'injections, Canada oui, Belgique ? et pour les autres pays francophones ?

\subsection{C'est quoi la concentration ?}

Les fioles d'oestrogènes sont faites à partir d'oestrogènes contenues dans une solution organique. La \textit{concentration} d'une fiole est la quantité d'oestrogène contenu dans cette solution. C'est un ratio de la masse par rapport au volume de la fiole. En d'autres termes : pour chaque mililitres d'huile (la mesure de volume), il y a tant de miligrammes d'oestrogènes (la mesure de masse). Tu va souvent voire des concentrations listées selon le volume total de la fiole (e.g. 200mg/5mL) mais il est toujours préférable de simplifier cette fraction (donc 40mg/mL) dans ce cas. \textbf{Les concentrations communes sont 5mg/mL, 10mg/mL, 20mg/mL, 40mg/mL, et occasionnellement 50mg/mL.}

\subsection{Qu'est ce qu'on veut dire par "dosage et fréquence ?"}

\textit{Le dosage} et la \textit{fréquence} sont les deux facteurs qui déterminent ton cycle hormonal. \textit{Le dosage} est la quantité d'oestrogènes que tu mets dans ton corps (mesuré en mg), et la \textit{fréquence} est à quelle fréquence tu mets de l'oestrogènes dans ton corps (mesuré en jours ou en semaines). Tu vas souvent entendre le mot "régime d'oestrogène", ce qui référence tout ce qui est lié au THS que tu prends et à quelle fréquence.

\subsection{Comment est-ce que je sais quel dose prendre ?}

Tu obtiens ton dosage en multipliant la concentration de ta fiole par le volume que tu es en train d'injecter \[Concentration (mg/mL) * volume (mL) = dosage (mg)\] \textbf{J'insiste sur le fait que le volume à lui seul ne suffit pas pour avoir un dosage correct.} Si tu veux une analogie, tu peux le voir avec la pâtisserie: tu ne peux pas dire à quelqu'un "fais cuire ça au four pendant 45 minutes", sans lui dire à quel température faire chauffer le four.

\subsection{Est-ce que je pourrais avoir un exemple de calcul de dosage ?}

Le calcul est super simple, promis ! J'ai mis une table de référence dessous qui compare en fonction de la concentration et du volume pour quelques dosages communs. Pas la peine d'être plus précis.e que 2 chiffres après la virgule. De toute façon, tu n'auras jamais une seringue assez précise pour doser 0.153mL par exemple. C'est une marge d'erreur acceptable qui ne fera aucune différence pour ce qu'on cherche à faire.

\begin{table}[]
\centering
\caption{Exemples de Dosages pour des couples Concentrations / Volume Fréqents}
\label{tab:concentrations}
\begin{tabular}{@{}lllll@{}}
    \toprule
    \multicolumn{1}{c}{} & \multicolumn{4}{c}{Concentrations (mg/ml)} \\
    \cmidrule(rl){2-5}
            & 5    & 10  & 20 & 40    \\
            \cmidrule(rl){2-5}
Dosage (mg) & \multicolumn{4}{c}{Volume (mL)}  \\
    \cmidrule(r){1-1} \cmidrule(lr){2-5} 
4        & 0.8  & 0.4 & 0.2  & 0.1      \\
5        & 1    & 0.5 & 0.25 & 0.13   \\
6        & 1.2  & 0.6   & 0.3  & 0.15     \\
7        & 1.4  & 0.7 & 0.35  & 0.18  \\
8        & 1.6  & 0.8   & 0.4  & 0.2    \\
9        & 1.8  & 0.9 & 0.45  & 0.23 \\
10       & 2    & 1   & 0.5  & 0.25   \\
    \bottomrule
\end{tabular}
\end{table}

\textbf{Comment lire ce graphe :} Commence par prendre la dose que tu veux sur la gauche, et en choisissant avec la colonne correspondant à la concentration de ta fiole, tu obtiens le volume correspondant sur la droite. Tu peux déjà remarquer que les volumes nécessaires pour les fioles concentrées à 5mg/mL ne sont pas terribles. C'est parce que les fioles concentrées à 5mg/mL ne sont pas terribles.

\subsection{Comment est-ce que je convertis les dosages entre différents esters ?}

\textbf{Tu ne le fais pas.} Comme ils se comportent de manière différentes, on ne peut pas faire de "conversion" de dosages. Si tu changes d'ester, tu dois (re)commencer à un dosage typique pour ce nouvel ester et ensuite retrouver ton cycle de croisière à partir de là. Tu peux faire des comparaison entre ester, mais on n'a aucune méthode pour convertir des doses.

\subsection{Comment est-ce que je peux comparer les différentes courbes en fonction du dosage et de l'ester choisi ?}

Si tu veux fouiller le sujet, je recommande \href{http://estrannai.se}{estrannai.se} que je trouve très bon. Garde en tête que ce n'est pas nécessaire, mais c'est un bon outil pour faire quelques comparaisons. \href{https://estrannai.se/\#i0__cu,7,7,1-cu,5,7,3-cu,5,7,2}{Voici un exemple de comparaison entre quelques dosage hebdomadaires typiques} qu'on va détailler un peu plus dès à présent.

\textbf{A noter que les dosages que je vais lister ci-dessous devraient suffire pour une dose minimum dans la plupart des cas.} Commence avec le dosage le plus bas, et augmente si besoin. Plus ne veut pas nécessairement dire mieux, mais on va voir ça plus en détail plus tard. Ces dosages fonctionneront peu importe comment tu as obtenu ta fiole.

\subsection*{A la rencontre de nos Esters}
\addcontentsline{toc}{subsection}{\textemdash{} A la rencontre de nos Esters}

\subsection{Comment est-ce que je dose mon \textit{estradiol valerate}?}

Tu peux soit predre une plus petite dose deux fois par semaine, ou une plus grosse dose une fois par semaine avec de \textit{l'estradiol valerate}. C'est en fonction de ta tolérance et de ton confort. En gros, il te faut 1mg pour chaque jour de ton cycle pour une injection tous les 3 à 7 jours. \textbf{Si tu fais une injection hebdomadaire, je recommande entre 6 et 8 mg}, mais 4 à 5mg tous les 5 jours est tout aussi bien. \textbf{Tu ne dois pas dépasser 7 jours entre deux injections.} Un cycle hebdomadaire pousse déjà suffisamment la demi-vie de l'ester. Toute fréquence plus basse est hautement déconseillée pour éviter des effets secondaires liés à la variance (voir la Question \ref{7-3}).
% TODO: je ne veux pas créer d'anxiété sur le ATTENTION NE PAS DEPASSER 7 JOURS SINON CONSEQUENCES, faudra peut-être rephraser

Je rappelle que dans certains endroits, les pilules sont vendues sous le nom \textit{estradiol valerate}, ce qui peut porter à confusion, cette section ne réfère qu'à la forme injectable du produit.

\subsection{Qu'est-ce qui caractérise la courbe hormonale de \textit{l'estradiol valerate} ?}

\textit{L'Estradiol valerate} est l'ester qui a le plus une courbe en forme de pointe. Il monte rapidement à un pic quelques jours après l'injection et retombe durement et rapidement. Cette relative instabilité peut être déplaisante en fonction de ta sensibilité, mais tu peux amoindrir les effets en ajustant le dosage et la fréquence.

 \begin{figure}[H]
     \centering
     \includegraphics[width=1\linewidth]{ev.png}
     \caption{Serum Estradiol (pg / ml) d'Estradiol Valerate en fonction du temps (jours) }
     \label{fig:ev}
 \end{figure}

\subsection{Comment est-ce que je dose mon \textit{estradiol cypionate}?}

\textit{L'Estradiol cypionate} peut être prise de façon hebdomadaire sans soucis. \textbf{On prend généralement une dose entre 5 et 7 mg.} Je ne recommande pas une fréquence plus basse qu'une fois tous les 7 jours (e.g. une fois tous les 10 jours) parce que l'oestrogène sera moins efficace que prise de façon hebdomadaire, vu qu'il faut une dose de plus en plus haute pour avoir des niveaux acceptables. Tout ce qui dépasse les 7 jours entre prise peut mener à des effets secondaires à cause de la variance (voir la Question \ref{7-3}). 

\subsection{Qu'est-ce qui caractérise la courbe hormonale de \textit{l'estradiol cypionate} ?}

\textit{L'Estradiol cypionate} pardonne un peu plus que \textit{l'estradiol valerate}. La courbe ne progesse pas aussi vite et varie beaucoup moins entre sa partie haute et sa partie basse, mais on remarque tout de même une montée et descente visibles sur un cycle hebdomadaire typique.

 \begin{figure}[H]
     \centering
     \includegraphics[width=1\linewidth]{ec.png}
     \caption{Serum Estradiol (pg / ml) d'Estradiol Cypionate en fonction du temps (jours) }
     \label{fig:ec}
 \end{figure}

\subsection{Comment est-ce que je dose mon \textit{estradiol enanthate}?}

\textit{L'Estradiol enanthate} peut se prendre sur cycle hebdomadaire, qui peut être étendu à une injection tous les 10 jours si ça t'arrange. Tu peux techniquement l'étendre plus que ça, mais je ne le recommande pas parce que tes niveaux seront de plus en plus instables. \textbf{On prend généralement une dose hebdomadaire entre 4 et 6mg}, ou 5 à 7mg pour une injection tous les 10 jours. Je préfère la solution hebdomaire de toute façons pour des raisons de facilité de planification, les extensions jusqu'à 10 jours n'apportant rien de particulier. 

\subsection{Qu'est-ce qui caractérise la courbe hormonale de \textit{l'estradiol enanthate} ?}

\textit{L'Estradiol enanthate} est le standard pour l'oestrogène injectable. Il a une courbe extrêmement plate (i.e. qui a peu de variance) sur le cycle hebdomadaire typique. Ca permet d'avoir des niveaux très réguliers sans avoir trop de soucis liés aux effets secondaires liés à la variance (voir la Question \ref{7-3}).

 \begin{figure}[H]
     \centering
     \includegraphics[width=1\linewidth]{een.png}
     \caption{Serum Estradiol (pg / ml) d'Estradiol Enanthate en fonction du temps (jours) }
     \label{fig:een}
 \end{figure}

\subsection{Comment est-ce que je dose mon \textit{estradiol undecylate}?}

\textit{L'Estradiol undecylate} est capable de dépasser de loin le cycle hebdomadaire pour aller vers le mensuel ou le trimestriel. Cependant, on n'a pas de standard ou de base pour déterminer une dose recommandée. Les éléments qui affectent la façon dont une injection d'oestrogène sont absorbées (\textit{"la pharmacocinétique"}), qui sont négligeables pour les autres esters sont plus influents pour \textit{l'estradiol undecylate}. A cause de ça, cet ester est toujours un terrain très expérimental qui dépasse le sujet de ce guide. Tu peux aller consulter un grimoire de sorcière pour savoir comment s'injecter sous la pleine lune.
% La traduction de la vanne est pas top ?

\subsection{Qu'est-ce qui caractérise la courbe hormonale de \textit{l'estradiol undecylate} ?}

On sait pas trop. Il y a trop peu de données pour avoir une vision claire, et il y a beaucoup de variables à prendre en considération. C'est quelque chose que tu peux rechercher et expérimenter si ça d'intéresse, mais c'est un saut dans l'inconnu et tu dois d'abord comprendre les risques induits par le fait d'être un rat de laboratoire, donc je ne recommande pas à moins de savoir ce que tu fais.

 \begin{figure}[H]
     \centering
     \includegraphics[width=1\linewidth]{moon.png}
     \caption{La lune}
     \label{fig:moon}
 \end{figure}

 

\section{PRISES DE SANG ET NIVEAUX}

\subsection*{Obtenir des résulats}
\addcontentsline{toc}{subsection}{\textemdash{} Obtenir des résulats}

\subsection{A quelle fréquence dois-je tester mes niveaux ?}

Tant que tu es en train d'affiner ton dosage, tu dois tester relativement fréquemment. Pour suivre chaque ajustement à ton régime, tu devrais laisser 1 ou 2 mois à tes niveaux pour se stabiliser, et tester dès que as atteint un rythme de croisière.

\subsection{Dois-je tester mes niveaux avant de commencer mon THS ?}

Ce n'est pas nécessaire, parce que tes niveaux de testosterone seront trop hauts et ceux d'oestrogène trop bas, donc ce n'est pas une donnée vraiment intéressante. Cependant, des prises de sang de routines (i.e. bilan lipidique et autres) sont recommandés pour ta santé de toute façon. L'exception étant si tu penses avoir une condition intersexe qui peut affecter ton régime de THS vu que c'est quelque chose qu'on peut des fois détecter dans une prise de sang préliminaire.

\subsection{Do I have to test my levels if I haven’t changed my dosage in a long time?}
\subsection{Est-ce que j'ai besoin de tester mes niveaux si je n'ai pas changé mon dosage depuis longtemps ?}

Il n'y a pas vraiment de raison de le faire, vu que tu n'as rien changé, rien n'a changé. Si tu veux avoir l'esprit tranquille ou si tu as changé des aspects de ta routine ou ta façon d'acheter ton THS, et de temps en temsp ton docteur va te le demander, mais tu ne devrais pas voir des différences énormes. Par contre, si tu es en train d'expérimenter avec \textit{l'estradiol undecylate}, tu devrais faire une prise de sang au moins tous les 3 mois quoi qu'il arrive. 

\subsection{Je ne peux pas avoir d'ordonnance pour une prise de sang, que faire ?}

Tu as quand même le droit de demander une prise de sang, qui ne sera par contre pas remboursée.
% TODO: localiser ici, je sais que c'est possible, mais ça serait top de dire comment
% Autre point intéressant, on pourrait parler du fait d'être en DIY et être suivi.e par un médecin, c'est possible 

\subsection{Je ne peux pas me permettre de payer une prise de sang, est-ce que je peux faire un THS quand même ?}

Même s'il est évident qu'il vaut mieux avoir l'information que de ne pas l'avoir, un THS est extrêmement sûr et pour des doses habituelles, ne posera pas de soucis. Tu devras juste plus te baser sur comment tu te sens et ce que tu observe de ton corps.

\subsection{Qu'est-ce que je dois tester ?}

\textit{L'estradiol} (E2) et \textit{testosterone totale} (T) au minimum parce que c'est les choses principales qui nous intéressent. Les niveaux de \textit{Sex hormone binding globulin} (SHBG), de \textit{dihydrotestosterone} (DHT), \textit{l'estrone} (E1), et de \textit{prolactine}(PRL) peuvent aussi être utiles si tu a des complications parce qu'elles peuvent être utiles pour établir un diagnostic. Les niveaux de \textit{d'hormone lutéinisante} (LH) et \textit{d'hormone folliculo-stimulante} (FSH) peuvent te dire si ton axe HPG est inactif, ce qui est la fondation de la monothérapie (voir la question Question \ref{2-3}). Mais je le répète : \textbf{\textit{L'estradiol} and \textit{Testosterone totale} sont les éléments principaux à regarder.} 
% TODO: traduire SHBG ?

\subsection{A quel moment de mon cycle dois-je faire ma prise de sang ?}

A la fin de ton cycle (\textit{"au bout"}). Tu veux être au plus bas possible parce que c'est l'information la plus utile. On peut même sire que c'est la seule information utile, vu que le fait d'avoir des niveaux régulier est notre problème principal. Par exemple : si tu fais une injection chaque jeudi après-midi, fais ta prise de sang le jeudi dans la matinée ou en début d'après-midi juste avant ta prochaine injection.

\subsection{Mon docteur m'a dit de faire mes prises de sang au milieu de mon cycle / juste après mon injection, est-ce que je fais ça ?}

\textbf{Non.} La mesure du taux d'oestrogène en milieu de cycle ne donne aucune information utile et ne permet que de savoir quel ester tu utilises. En restant gentille, ce genre de demandes est dûe à l'incompétence née de standards de soins datés et conservateurs. En l'étant moins, c'est une volonté malveillante d'asssurer que tu aies des niveaux d'oestrogène trop bas, ce va te donner des mauvais résultats, ou te rendre malade. \textbf{Je recommande de mesurer au dernier jour du cycle quand même.} 

\subsection*{Interpréter les Résultats}
\addcontentsline{toc}{subsection}{\textemdash{} Interpréter les Résultats}

\subsection{Quel est le taux d'oestrogène que je dois viser ?}

C'est probablement la question la plus controversée pour ce qui concerne la transition. Pour faire court, tu veux avoir des niveaux suffisants pour te sentir bien, et suffisants pour supprimer la production de testosterone si c'est ce que tu cherches, mais des niveaux plus hauts que ça sont au mieux du gâchis d'horomones, et au pire contreproductifs. C'est cependant une fourchette très large, et avec autant de variables il y a toujours un élément personnel. En d'autre mots, tu veux avoir suffisamment d'oestrogènes pour que tu te sentes bien, c'est à peu près tout.

\subsection{Est-ce que des niveaux d'oestrogène plus hauts me permettent de mieux me féminiser / de le faire plus rapidement ?}

\textbf{Non.} Des niveaux d'oestrogènes plus hauts que nécessaires sont préférés par certains parce qu'iels se sentent mieux comme ça, mais ça n'a aucun intérêt au niveau de la féminisation. En fait, des niveaux trop hauts peuvent causer des troubles de l'humeur ou d'autres effets secondaires non désirés. \textbf{Minimiser les niveaux de testostérone (jusqu'à un certain point) est bien plus important que de maximiser les niveaux d'oestrogène.}

\subsection{D'accord, mais en pratique je veux voir quel chiffre sur mon résultat de prise de sang ?}

En rappelant que le nombre exact n'est pas si important que ça, et que le chiffre sera toujours un peu plus important que ce que tu as dans ton corps au dernier jour du cyclé à cause de la latence, et que ce chiffre sera dans un nuage de possibilité basé sur un certain nombre de facteurs, \textbf{Je recommande 200pg/mL (730pmol/L) au minimum au dernier jour du cycle.} C'est une recommandation plutôt large, qui prévoit une bonne marge vu que la suppression de l'axe HPG arrive bien en-dessous de ce niveau. La plupart des gens préfèrent être autour de ce niveau, et certains préfèrent un peu plus ou moins. Je ne pense pas que ça soit un chiffre sur lequel trop se fixer, parce que c'est très dépendant de toi, et au final, le plus important c'est que tu te sens bien. \textbf{Cependant, au-delà de 300pg/mL (1100pmol/L) au dernier jour du cycle, tu as un niveau certainement trop haut pour tes besoins.} Il y a des exceptions à cela, mais tu n'en fais probablement pas partie. Mais bon, fais ce qui te fias sentir le mieux. Aussi, va voir la Question \ref{11-1}.

\subsection{Quel est le taux de testostérone que je dois viser ?}

La suppression de la testostrérone (T) est un prérequis pour une féminisation adéquate, donc descendre en-dessous de 50 ng/dL (1.7 nmol/L) es généralement suffisant. \textbf{A noter que des taux de testostérone approchant 0 ne sont pas désirés.} Voir la section \ref{T} "TESTOSTERONE" 

\subsection{J'ai naturellement des taux de T très hauts / très bas. Est-ce qu'il faut que je change quelque chose à mon dosage ?}

Probablement pas. les taux de testostérone qu'on voit avant le début d'un THS sont quasiment toujours plus hauts que ce qu'on cherche pour une féminisation et vont être réduits quoi qu'il arrive (voir la Question \ref{2-3}). L'exception est si tu as une condition intersexe de quelque variété que ce soit, ce qui va peut-être demander un ajustement plus fin que les préconisation listées dans ce guide et qui dépasse l'objet de celui-ci. Tu n'auras peut-être pas besoin de changer quoi ce soit, mais tu te sentira peut-être mieux si tu le fais. Voir la Question \ref{9-2}
% TODO: Pas sûr que supprimé soit le meilleur mot. 

\subsection{I have had bottom surgery. Do my estrogen levels need to be different?}
% TODO: A faire

Vu que la diminution des niveaux de testostérone n'est plus un problème pour toi, tu peux probablement t'en tirer avec des niveaux d'oestrogènes plus bas que pour les autres, mais \textbf{tu as toujours besoin d'oestrogène.} Puisque tu ne produis plus d'hormones sexuelles, il est crucial que tu maintienne des niveaux d'hormones suffisants pour rster en bonne santé. N'avoir plus aucune hormones ou presques va induire des symptômes de ménopause, raison pour laquelles les femmes cis plus âgées prennent aussi un THS parfois après la ménopause. Ajuste tes niveaux comme tu le sens.

\subsection{Est-ce qu'il existe des raisons pour lesquelles une prise de sang pourrait mener à des résultats imprécis ?}

Selon comment le test est conduit, des suppléments de biotine peuvent donner l'impression que les niveaux d'\textit{estradiol} (E2) (parmis d'autres, mais c'est l'\textit{estradiol} qui nous intéresse) sont anormalement hauts. Il n'est pas toujours possible de savoir avec quel méthode le test est fait, donc il vaut mieux arrêter de prendre tes suppléments de biotine quelque jours avant ta prise de sang. Il est possible aussi qu'il y ait eu une erreur avec l'équipement ou l'échantillon, même si c'est beaucoup moins probable 

\subsection{Est-ce qu'on voit quel ester / voit d'administration j'ai choisi sur mes résultats de prise de sang ?}\label{4-16}

Non. Il n'y a pas de moyen de savoir quel type d'oetrogène tu prends juste à partir d'une prise de sang. Tous les ester se transforment en \textit{estradiol} dans ton sang, ce qui est le but recherché, et c'est la même chose pour les pilules, gels, patches, sprays ou quoi que ce soit d'autre que tu pourrais utiliser. Au final, tout n'est qu'oetrogène. 
 

\section{TECHINQUES ET MATERIEL} \label{ts}

\subsection*{Sites d'injections \& Sécurité}
\addcontentsline{toc}{subsection}{\textemdash{} Sites d'injections \& Sécurité}

\subsection{How do I safely perform an injection?}
\subsection{Comment est-ce que je fais une injection en toute sécurité}

\href{https://rebellyon.info/Le-guide-d-auto-injection-sous-cutanee-du-25225}{\textit{Le FLIRT et STRIP ont créé un guide d'auto injection}}, que tu peux lire ou imprimer, et que je recommande fortement.

Sinon, Je recommande ces deux vidéos (en anglais):

\begin{enumerate}
  \item \href{https://www.youtube.com/watch?v=cBabaGC2Dok}{\textit{"Comment s'auto injecter en intramusculaire (IM)”}}
  \item \href{https://www.youtube.com/watch?v=YfNlAZLxLyw}{\textit{“Une technique d'injection IM sans douleur (pour le moment pour moi)”}}
\end{enumerate}

Between these two videos, you should be fully equipped to properly inject with minimal pain. I suggest studying them and revisiting as needed. \textbf{One key thing to emphasize is to inject with the bevel facing up to reduce pain.} In other words: the needle has a clearly defined point, and you want that to be what touches your skin first. You want a nice straight line of travel. You can think about how your hand/wrist rotates if that helps you visualize the motion, but realistically it'll be intuitive muscle memory that you'll learn naturally.
Avec ce guide et ces deux vidéos, tu devrais avoir tout ce qu'il faut pour t'auto injecter sans trop avoir mal. Prends le temps d'étudier ça et reviens voir quand tu en a besoin. \textbf{Le truc le plus important quand tu t'injecte est d'avoir le biseau vers le haut pour réduire la douleur.} Je m'explique : l'aiguille se termine en un point, et tu veux que ça soit ça qui touche ta peau en premier. Tu veux que le parcours de ton aiguille suive une ligne bien droite depuis le trou que tu vas créer. Tu peux imaginer le mouvement de ta main si ça t'aide, mais en étant réaliste, c'est une mémoire musculaire que tu va apprendre avec l'expérience.
% TODO: Je pense que j'ai pas compris ce que j'ai traduit la 

\textbf{Rappelles-toi : s'auto injecter s'apprend ! } Tu vas t'améliorer avec le temps, et ça prendra pas longtemps avant que tu saches faire. Ca va le faire.

\subsection{Est-ce que je dois m'injecter exactement comme ça ?}

Non, tu peux personnaliser comme tu veux. Au final, quand il s'agit de se faire des trous, il y a plein de mainières d'y arriver. Trouve la manière qui marche le mieux pour toi. Faire un mouvement rapide de piqué marche le mieux habituellement, mais si tu préfère aller plus lentement, ça marche aussi. 

\subsection{Comment est-ce que je fais pour passer outre l'anxiété au moment de l'injection ?}

Je suggère de créer un rituel autour de l'injection. En fromant une routine, le processus va devenir une seconde nature pour toi. Si ça marche pour toi de te distraire en mettant de la musique, en ayant une conversation, en regardant une série, ou en faisant autre chose qui marche pour toi et qui laisse la mémoire musculaire prendre le dessus, c'est super ! Sinon, tu peux te faire aider par un.e ami.e ou un.e proche pour faire tes premières injections, ça aide aussi. La première injection est celle qui fait le plus peur. Habituellement, les gens disent "Ah c'est tout ?", parce que c'est jamais aussi terrible que ce qu'on croit.

\subsection{Est-ce que l'endoit où je m'injecte a une importance ?}

Oui et non. Il faut rester dans des zones sûres, mais à part ça, ça dépend surtout de ta mobilité, du volumes que tu injectes, de ton combo seringue/aiguille et de ce que tu préfères. En tout cas, \textbf{assure-toi de changer de site d'injection a chaque fois.} Alterne le côté de ton corps que tu choisis, par exemple, si tu injecte sur ta jambe droite une semaine, utilise ta jambe gauche la suivante. C'est pour éviter d'avoir des risques à long terme.

\subsection{Quels sont les sites d'injection sûrs ?}

Les opinions varient en fonction des autorités médicales, et la composition de ton corps a un impact. Je recommande de t'injecter sur le côté de la jambe comme on le voit sur le guide et les vidéos, c'est le plus simple pour la plupart des gens et permet de faire tout le temps des injections sans douleur une fois que tu as pris l'habitude de la technique, mais certaines personnes préfèrent le faire sur la fesse ou le ventre. \href{https://vertisis.com/articles/how-to-self-administer-a-subcutaneous-injection}{Ce cite/vidéo (en anglais)} montre d'autres sites d'injections qui peuvent être acceptables en fonction de l'équipement utilisé. Expérimente et trouve ce qu'il te plaît le plus.

\subsection{Qu'est-ce que "intramusculaire" (IM) et "sous-cutané" (SubQ/SC) veulent dire ?}

Tu vas souvent entendre ces termes dans le contexte des injections. \textit{intramusculaire} veut dire injecté dans le muscle et \textit{sous-cutané} veut dire injecté dans le tissu gras en-dessous de ta peau.

\subsection{Quelle est la différence entre une injection intra-musculaire (IM) et sous-cutanée (SubQ/SC) ?}

\textbf{En ce qui concerne le THS, il n'a aucune différence significative entre une injection sous-cutanée et intramusculaire.} Les injections sous-cutanées sont absorbées plus lentement que les injections intramusculaire, mais c'est généralement pas une différence suffisamment significative pour impacter ton dosage. A noter aussi que ton injection est rarement totalement déposée dans le muscle ou dans le tissu sous-cutané, ce qui brouille d'autant plus les différences possible entre les deux types d'injections.

En passant, les sources pharmaceutiques pour les fioles d'oestrogènes vont souvent dire que leur produit est déstiné à des injections intramusculaires seulement, parce qu'elles n'ont reçu uniquement une autorisation pour cet usage. Ca n'a cependant aucune importance.

\subsection{Est-ce que je devrais plutôt faire une injection intramusculaire (IM) ou  sous-cutanée (SubQ/SC) ?}

\textbf{Tu te poses la mauvaise question.} \textbf{Une injection est une injection.} Les injections sous-cutanées sont souvent recommandées parce que les gens croient qu'elles permettent de faire des injections moins douloureuses par le simple fait d'être sous-cutanées, mais ça en fait en fait aucune différence. \textbf{Les avantages dont les gens parlent à ce sujet ne sont pas inhérent au site de dépôt d'injection; ils sont dependants d'autres facteurs qui eux, affectent les douleurs liés à l'injection.} La meilleure question serait "Comment puis-je minimiser la douleur liée à une injection ?", j'y viens, je réponds à deux autres questions d'abord. 

\subsection{Est-ce que l'angle d'injection / ma méthode préférée d'injection à une incidence ?}

Non. Je me répète, mais la partie la plus importante pour faire une injection est que tu perce ta peau avec une aiguille pour déposer un fluide dans ton corps. Si le fluide ne ressort pas (ou en tout cas pas trop) et ça ne fait pas mal (ou en tout cas pas trop), alors tu as fait un travail formidable. \textbf{Je ne peux pas le répéter assez, la "division" intramusculaire / sous-cutané importe peu n'impacte pas de façon significative l'efficacité de l'oestrogène injectable. } Le cas de l'\textit{estradiol undecylate} est le seul où le site de dépôt à l'air de vraiment affecter l'absorption, mais même là, on ne connait pas vraiment les détails. En bref : reste concentré sur les choses qui comptent, et pas celles qui ne changent rien.
% TODO: "division" faute de mieux, je troube pas ça top

\subsection{Est-ce que je dois créer une aspiration ?}

Non. "l'aspiration", c'est quand tu tire sur le piston de la seringue juste après avoir percé ta peau, avant d'injecter du fluide pour s'assurer que tu n'ijecte rien dans un vaisseau sanguin. La nécessité d'une telle procedure est controversée, mais en ce qui concerne les procédures standard d'injection d'hormones, les avantages ne couvrent pas les inconvénients. Les sites d'injections standards ont peu de risques de toucher un vaisseau sanguin, risque encore amoindri par la petite taille des aiguilles, donc cette pratique n'est plus recommandée par les organisations médicales.

\subsection{Comment puis-je minimiser la douleur liée à une injection ?}

A part accumuler de l'expérience pour améliorer ta technique, le facteur principal d'inconfort pendant une injection est la combinaison aiguille / seringue que tu utilises. \textbf{Pour minimiser l'inconfort, prends le calibre d'aiguille le plus élevé que ton huile de solvant est capable de tolérer, avec une seringue et une aiguille de taille appropriée.} Tu dois te demander : "Quel calibre et quelle longeur d'aiguille choisir pour mon injection ?" Pour répondre à cela, parlons un peu du fonctionnement des aiguilles.

\subsection*{Connaitre tes aiguilles}
\addcontentsline{toc}{subsection}{\textemdash{} Connaitre tes aiguilles}

\subsection{Un calibre d'aiguille, c'est quoi ?}

\textit{Le calibre d'aiguille} est une mesure de la largeur de ton aiguille. Plus le nombre est haut, plus l'aiguille est fine. Une aiguille 25G est plus fine qu'une aiguille 20G, par exemple. Les aiguilles de calibre élevé sont généralement plus courtes car les aiguilles trop longues ont tendance à se plier, donc leur longeur à un maximum atteint plus rapidement. Sans surprise, les aiguilles les plus fines font le moins mal. A noter que le calibre de ton (ou tes) aiguille(s) n'affectent pas ton THs de quelque manière que ce soit; ça va seulement impacter le confort et la facilité d'injection.

\subsection{C'est quoi les aiguilles/seringues "Luer lock" / "insuline ?"}\label{5-13}

\textit{Les seringues Luer lock} ont une aiguille qui se séparent de la seringue, avec une aiguille de prélèvement, et une autre pour injecter. Les \textit{seringues à insuline} ont une aiguille fixée, ce qui veut dire que tu vas utiliser la même seringue pour prélever et pour injecter. Quand c'est possible, on préfère utiliser des seringues à insuline pour minimiser l'espace mort (voir la Question \ref{5-26}).

\textbf{Recommandations de sécurité : remettre le bouchon d'une aiguille n'est généralement pas recommandé par risque de te piquer, mais si jamais tu dois le faire (par exemple quand tu change d'aiguille après avoir prélevé ton oestrogène), ne mets JAMAIS de force avec ta main sur le chemin de l'aiguille.}

Il est possible que le bouchon se casse et que tu te fasse mal si jamais tu place le bouchon incorrectement. Mets ton bouchon sur une suface horizontale et utilises l'aiguille pour le rammasser gentimment et appuie le bouchon sur un mur ou tire le sur les côtés pour le fermer. Il n'y a pas de risque d'infection mirobiologique en faisant une auto-injection, donc utilise cette mise en garde comme tu veux, mais remettre un bouchon est une action très sérieuse si tu injecte quelqu'un d'autre. Pour la gestion des déchets, va voir la Question \ref{5-27}.

\subsection{Avec quel calibre dois-je prélever le fluide ?}

Si tu utilises des seringues Luer lock, il vaut mieux utiliser un calibre plus bas que celui que tu vas utiliser pour l'injection pour minimiser le temps de prélèvement du fluide de la fiole. Un calibre trop peu élevé peut mener à la fragmentation de la fiole (voir Question \ref{5-23}), donc je recommande de prendre au moins une aiguille 21-23G. Si tu es patient.e et moins de volume à injecter, alors il vaut mieux utiliser un calibre plus élevé pour éviter au maximum une fragmentation de la fiole. A noter que l'aiguille ne devient pas vraiment moins perforante après avoir percé la fiole. La question ne se pose pas avec les seringues à insuline vu que tu ne peux pas changer l'aiguille.

\subsection{Quelle longueur d'aiguille dois-je utiliser pour le prélèvement ?}

Si tu utilises des seringues Luer lock, la longueur de l'aiguille de prélèvement n'importe pas sauf pour le fait qu'une aiguille trop longue peut être compliqué. Autement dit, pas besoin de s'embêter avec ça. La question ne se pose pas avec les seringues à insuline vu que tu ne peux pas changer l'aiguille.

\subsection{Avec quel calibre dois-je m'injecter ?}\label{5-16}

C'est une question difficile à répondre, avec une réponse très subjective, qui va dépendre de 4 facteurs principaux : 1) l'huile solvante que tu t'injecte; 2) si la fiole contient un cosolvant; 3) ta patiente à avoir une aiguille dans ta jambe plus longtemps; 4) ta volonté/possibilité de pousser plus fort sur le piston de la seringue. C'ests une question de confort. Une huile plus épaisse demande plus de temps et d'efforts si tu utilise un calibre plus élevés, mais un calibre élevé fait significativement moins mal. \textbf{Une aiguille 25G devrait être le calibre minimum à utiliser pour minimiser l'inconfort lié à l'injection.} La plupart des huiles solvantes peuvent être utilisées confortablement avec des aiguille jusqu'à 27G, et la l'huile MCT en particulier peut totalement être utilisée avec des aiguilles de 30G (voir la Question \ref{6-16}).

\subsection{Quelle longueur d'aiguille dois-je utiliser pour l'injection ?}

\textbf{Je recommande des aiguilles de 12,5mm à 25mm (0,5” à 1”), en fonction du calibre que tu as choisi.} En dessous de 12,5mm (0,5"), tu augmente les risques de fuites. Des aiguilles de 6,5mm (0,25") peuvent suffire si tu as la bonne technique et si le fluide à injecter le permet, mais une aiguillle de 12,5 mm (0,5") est un pari plus sûr. Une aiguille plus grande que 25mm (1") est plus douloueurse et intimidante sans contrepartie.

\subsection{Est-ce que la taille de la seringue compte ?}

\textbf{Oui, la taille compte.} Il y a deux raisons à cela. 1) Des seringues de gros volumes ont tendance à être moins précises, ce qui peut mener à un dosage incorrect, et 2) la physique fait que les seringues de gros volumes sont plus difficiles à utiliser pour l'injection. Pour garder une bonne précision sur ton dosage, il ne faut pas utiliser une seringue beaucoup plus large que le volume que tu injectes (i.e., pour des injections de moins de 0.1mL, prends des seringues de moins de 1mL). \textbf{Les seringues de 3mL sont à éviter si possible.} Evidemment si c'est tout ce que tu as utilise-les, mais elles ne sont pas vraiment faites pour ce genre de choses. Me demande pas pourquoi les pharmacien.nes ont l'air de ne donner que celles-ci. Une blague cruelle peut-être. 

\subsection{Où est-ce que je peux acheter des seringues et des aiguilles ?}

Ca dépend de ta juridiction locale, vu que certains endroits interdisent la vente d'aiguille et de seringues aux public pour criminaliser les gens qui prennent des drogues. Sinon, les fournisseurs vétérinaires, de pharmacie, ou des producteurs autorisés en vendent, et sont des endroits à regader. \textbf{Amazon n'est pas un vendeur recommandé}, parce que la qualité de leur seringue ne peut pas être garantie.

En France, les pharmacies et parapharmacies en vendent (et en donnent gratuitement dans le cadre de programmes de réduction des risques), et tu as le droit d'en acheter. Si tu préfères gérer ça en beaucoup d'autres pharmacies en ligne en proposent pour un prix raisonnable.

\subsection{Est-ce que je peux réutiliser des aiguilles ou des seringues ?}

\textbf{Non. Ne réutilise jamais d'aiguille ni de seringues.} Ne les partage pas non plus. tu le sais déjà probablement, masi je le rappelle ici parce que c'est vraiment dangereux !

\subsection{Comment faire si je veux faire des injections mais que j'ai du mal à le faire moi-même ?}\label{5-21}

Tu aimeras peut-être les auto-injecteurs. Comme le nom l'indique, les auto-injecteurs font l'injection pour toi. Les auto-injecteurs comme le \href{https://unionmedico.com/90-super-grip/}{\textit{UnionMedico 45/90 Super Grip}} peuvent prendre des seringues de 1mL ce qui peut rendre l'injection moins difficile (mais il faudra toujours appuyer sur le piston), alors que des auto-injecteurs comme le \href{https://www.owenmumford.com/us/medical-devices/autoject-2}{\textit{Owen Mumford Autoject 2}} cachent complètement l'aiguillle de la seringue à insuline, et poussent automatiquement le piston. Il y a aussi des modèles 3D imprimables disponibles en ligne. Je n'ai utilisé aucun de ces produits, et je ne peux m'avancer sur leur qualités respectives.

\subsection*{Le B-A BA d'une fiole}
\addcontentsline{toc}{subsection}{\textemdash{} Le B-A BA d'une fiole}

\subsection{Qu'est-ce que je dois regarder quand j'inspecte une fiole ?}

A part pour des signes de déchirement (voir plus bas), cherche des signes de décoloration, de séparation, de contamination, de cristallisation, de fissures dans le verres, de fibres ou de cheveux dans la fiole, etc. Une fiole bien faite ne doit pas être différente par rapport à d'habitude. \textbf{Inspecte toujours une fiole avant usage. N'utilise jamais une fiole qui n'a pas l'air OK.} 

\subsection{Qu'est-ce que le "déchirement"}\label{5-23}

\textit{Le déchirement} arrive quand un bout du bouchon en caoutchouc se casse et tombe dans la fiole. Ca peut arriver à cause de l'usage d'un calibre trop peu élevé, des ponctions répétées exactement au même endroit, ou à cause de trop de ponctions (i.e. une injection de faible volume avec une fiole trsè volumnineuse). \textbf{Une fiole déchirée doit être jetée immédiatement.} \href{https://www.youtube.com/watch?v=w5F0SLoMjC8}{\textit{La technique du 45-90°}} peut aussi être utilisée pour éviter de déchirer ta fiole.
% TODO: J'ai pas utilisé le même nom plus haut, je pense toujours pas que c'est le bon nom à utiliser

Le risque lié à cela est de t'injecter des bouts de caoutchouc. Si tu vois des gros morceaux, il y a probablement des plus petits que tu ne vois pas. Le but du bouchon est de protéger le contenu de la fiole des éléments extérieurs, donc une fiole avec un trou est plus à risque d'oxydation et d'infection bactérienne. \textbf{Note: Fais attention à retirer la capuche en métal ou en plastique du dessus de la fiole.} Ca peut paraître évident, mais certains design de fioles peuvent porter à confusion.

\subsection{J'ai combien de temps avant qu ema file arrive à expiration ?}

Une fiole scellée peut tenir des années sans soucis si elle est stockée à température constante loin de la lumière. Le plus gros soucis lié à l'âge de ta fiole est l'oxydation de l'huile solvante, en supposant qu ela fiole a été stérilisée comme il faut. Une fiole ouverte qui a des conservateurs (voir Question \ref{6-17}) devrait tenir le temps que tu la finisse complètement. Le message "jeter après 28 jours" qui est marqué sur les fioles vendues en pharmacie répondent à un minimum garanti par les producteurs sur la stérilité du produit, pas sa durée de vie maximum.

\subsection{Comment est-ce que je stocke ma fiole ?}

Dans un endroit loin d'une source de lumière et à température stable. Les chaleurs élevées et les UV peuvent dégrader l'huile solvante, et les températures bassent peuvent causer une cristallisation. Les cristaux peuvent être dissous et réincorporés, mais ça peut causer des irritation s'ils ne sont pas complètement dissous. Le conseil vaut pour les fioles ouvertes et les fioles scellées.

\subsection{C'est quoi "l'espace mort"}\label{5-26}

\textit{L'espace mort} est le petit volume de fluide qu'on perd quand on fait une injection. C'est du fluide qui est bloqué dans la seringue ou dans l'aiguille. Avec une seringue Luer lock standard, ça peut être jusqu'à 0.1mL, alors que pour une aiguille à insuline on est plutôt autour des 0.003mL. Il vaut mieux réduire l'espace mort au plus possible pour raisons économiques, mais aussi parce qu'au bout du compte, ça fait beaucoup d'oestrogène gaché. \href{https://hrtcafe.net/Calc/}{Ce calculateur} peut être utile pour estimer combien tu gache d'oestrogène en fonction de l'équipement utilisé.

A noter que si tu changes d'aiguille entre le prélèvement et l'injection, il faut un peu tirer sur le piston avant d'enlever l'aiguille de prélèvement pour que le fluide bloqué dans l'aiguille ne soit pas gaché. C'est pas grand-chose, mais ça peut faire une différence. Va voir la Question \ref{7-7} pour une autre stratégie si l'espace mort t'inquiètes.

\subsection{Qu'est-ce que je fais de mes seringues / aiguilles usagées ?}\label{5-27}

Place tout ton matériel d'injection dans un conteneur à seringues (soit un conteneur fait pour ça ou en réutilisant un pot en plastique dur comme un pot à protéine en poudre ou de détergent). Quand le pot est plein au trois-quarts, scelle le de façon à ce qu'on ne puisse pas l'ouvrir accidentellement. Marque clairement "Seringues usagées" dessus et jete-le en accord avec la juridiction de l'endroit où tu vis. \textbf{A noter que les aiguilles / seringues ne doivent pas être jetées dans la poubelle ou la poubelle de recyclage.} Ta ville/région/pays a pribablement un site web dédié qui explique comment se débarasser des matières dangereuses.

En France, tu peux demander à une pharmacie de te remettre gratuitement une boîte à aiguilles, que tu leur rendras une fois rempli et scellée pour être jetée ensuite. 
Il est d'ailleurs offciellement interdit d'utiliser autre chose qu'une boîte fournie par une pharmacie pour jeter des seringues usagées.


\section{OBTENIR DES FIOLES}\label{sv}

\subsection{Où est-ce que je peux obtenir des fioles pour mes injections?}

En gros, tu as deux options : \textit{en pahramacie} ou \textit{via le DIY}. Pour en demander en pharmacie, il te faudra généralement une ordonnance, vu que dans la plupart des pays, on ne peut pas obtenir d'oestrogène sans ordonnace, et quand c'est possible, ça ne concerne pas les injections. Le \textit{DIY}, c'est toutes les autres manières d'en obtenir.
A noter en France que comme il n'y a pas d'autorisation de mise sur le marché pour les injections d'oestrogènes, il n'est possible d'obtenir des fioles en pharmacie. Pour nous, c'est le DIY ou rien. 

\subsection{Est-ce que je devrais plutôt aller voir en pharmaice ou directement récupérer mes fioles en DIY ?}

Le choix t'appartient, mais des fois il n'y a pas de choix à faire. Il y a des avantages et des inconvénients dans les deux cas. Evidemment, rien ne t'empâche de récupérer de l'oestrogène de plusieurs sources différentes pour cumuler les avantages. Dans la plupart des cas, c'est même recommandé.

\subsection*{En pharmacie}
\addcontentsline{toc}{subsection}{\textemdash{} En pharmacie}

\subsection{Quels sont les avantages et les inconvénients d'aller en pharmacie ?}

\begin{itemize}
  \item Tu peux généralment faire confiance aux processus de contrôle de qualité et aux certifications;
  \item C'est couvert par l'assurance maladie;
  \item Ca peut être plus pratique, en fonction de ta chance avec les docteurs;
  \item The product most likely will be consistent;
  \item Le produit a plus de chance de respecter ses spécifications;
  \item \textbf{Donner au moins l'illustion d'utiliser les routes prévues par les institutions sont souvent demandés pour avoir des remboursements pour tes chirurgies.}
\end{itemize}

\subsection{Quels sont les inconvénients d'aller à la pharmacie ?}

\begin{itemize}
  \item Peu (ou pas) de choix d'esters;
  \item Possiblement des temps d'attente très longs (des mois, voire des années);
  \item Il te faudra probablement une ordonnance (selon les pays);
  \item L'assurance maladie de ton pays ne couvre peut-être pas tous les coûts;
  \item Dans certains cas, tu ne peux pas t'en procurer dans ton pays;
  \item On peut refuser de te faire une odronnance de façon arbitraire;
  \item On peut refuser de te renouveller ton ordonnance de façon arbitraire;
  \item Il peut y avoir des ruptures de stocks;
  \item Il est possible que tu sois obligé.e de suivre les recommandations de la WPATH, voire pire;
  \item Plus difficile de faire des stocks;
  \item Ton accès est assujetti aux caprices de la situation politique de ton pays, ce qui veut dire que ton dossier médical contiendra probablement l'information que tu es trans.
\end{itemize}

\subsection*{En DIY}
\addcontentsline{toc}{subsection}{\textemdash{} En DIY}

\subsection{C'est quoi les avantages de récupérer ses fioles en DIY}

\begin{itemize}
\item Généralement beaucoup moins cher dans la plupart des pays;
\item Disponible partout dans le monde;
\item En obtenir te fait attendre des mois voire des années de moins si tu es sur une liste d'attente (les seuls délais sont liés à la production et l'envoi);
\item Facile de faire un stock;
\item Large choix d'esters;
\item Pas besoin de s'embêter avec le système médical;
\item C'est probalement fait avec amour.
\end{itemize}

\subsection{Quels sont les inconvénients du DIY ?}

\begin{itemize}
  \item Ne sont certainement pas faits dans une pièce certifiée propre;
  \item La qualité dépend de la source;
  \item Peut être peu pratique en fonction de la source; % TODO: reword
  \item Nécessite de faire confiance à ta source;
  \item Nécessite de trouver une source;
  \item Les sources ont plus de chance de disparaître que ta pharmacie locale;
  \item Les temps de livraisons de produits peuvent varier;
  \item Très probable que tu aies besoin de passer par des cryptomonnaies, ce qui est ennuyant;
  \item Tu ne peux pas te faire rembourser pas l'assurance maladie.
\end{itemize}
De plus, comme dit plus haut, si tu as besoin de l'assurance pour te faire rembourser tes chirurgies, il te faudra probablement un temps minimum avec une prescription de THS. Ce qui peut (ou non) être un problème pour toi.

\subsection{Quels sont les types d'oestrogènes injectable qu'on peut uniquement obtenir en DIY ?}

Principalement l'\textit{estradiol enanthate}. Le secteur médiacal va quasiment toujours te prescrire de l'\textit{estradiol valerate}, mais pas toujours concentré à 40mg/mL. On prescrit des fois de l'\textit{estradiol cypionate}, mais ramrement plus concentré que 5mg/mL ou 10mg/mL, ce qui est embêtant pour faire un dosage correct. Les avantages de l'utilisation de l'\textit{estradiol enanthate} sont à eux seuls des bonnes raisons de considérer le DIY, mais tu peux obtenir n'importe quel ester à 40mg/mL via le DIY.

\subsection{C'est quoi exactement les sources en DIY ?}

Les sources de DIY sont généralement des fabricants professionnels, des groupes d'entraide, ou toi-même si tu te sens l'esprit entrepreneurial !

\subsection{Comment je fais pour obtenir des fioles en DIY ?}

T'es un flic ? Je vais pas te dire ça. C'est pas le but de ce quide. Il y a d'autre ressources pour obtenir ce genre d'information. Reste concentré.e.

\subsection{Comment ça se fait que le DIY puisse être moins cher que d'aller à la pharmacie ?}

Une fiole coûte en moyenne 10€ à produire, en comptant les amortissement et le salaire, en comptant large. La plupart du coût des sources de DIY "commerciales" sont liés à tout ce qu'il faut pour envoyer les fioles en gardant son anonymité. Les sources non-commerciales n'ont sûrement pas de tels problèmes. Pour les laboratoires pharmaceutiques, ils n'ont juste pas de raison de vendre moins cher qu'ils le font aujourd'hui.  

\subsection{Est-ce que le DIY est légal ?}\label{6-11}

Dans beacoup d'endroits, l'oestrogène n'est pas une substance interdite, et la testostérone peut ou non être criminalisée. Il est cependant très rare d'être inculpé.e dans des affaires criminelles autour de la possession d'hormones. \textbf{Je n'endosse aucune responsabilité à ce propos.} 
% TODO: Légalité en France ? La vente est interdite, mais je ne sais pas pour la possession

\subsection{Est-ce que le DIY est dangereux ?}

Le "DIY" en trant que catégorie large de sources n'est ni sûr ni dangereux, mais toutes les sources ne se valent pas. Quand on parle de s'injecter des choses de manière sûre, la vraie question à poser est : est-ce que tu as confiance dans la personne qui a produit cette fiole pour avoir suivi les bonnes techniques d'aseptisation, de façon à ce que la fiole ne contienne que ce que tu veux et rien d'autre ? Quand tu achète à un labo pharmaceutique, tu fais confiance aux réglementations et aux lois qui les surveillent. A défaut de ça, pour les sources en DIY, la confiance est obtenue en expliquant / montrant les processus, des tests fait par des labos indépendants pour vérifier la pureté et la concentration, et via la réputation.  

\subsection{Qu'est-ce que je dois regarder pour savoir si une source de DIY est digne de confiance ?}

Fais confiance à ton cerveau et ton instinct.

\begin{itemize}
  \item Est-ce que la personne est prêt.e à parler de son processus / est-il listé quelque part ? (e.g. est-ce que la possière est filtrée ? La réponse devrait être oui !!!)
  \item Est-ce qu'iel a l'air compétent.e ?
  \item Le produit a-t-il été testé ? 
  \item Are they a trusted member of the community?
  \item Est-ce que cette personne est réputé.e comme digne de confiance dans la communauté ?
  \item Est-ce que quelqu'un / une institution en qui tu as confiance peut attester de la qualité ? (i.e. inspections, témoignages, retours, etc.)
  \item Les erreurs arrives, mais est-ce qu'iel prend en la responsabilité quand elles arrivent où est-ce qu'iel essaie de taire les reproches ? 
  \item Pour les soures commerciales, comment se résolvent les soucies de commandes ?
  \item Pour les sources commerciales, est-ce que tu paye pour un produit qui n'a pas encore été fabriqué sans qu'il soit indiqué que c'est une précommande ? (Tu ne devrais jamais précommander !)
  \item Est-ce que les fioles ont des conservateurs ?
  \item Iel produit depuis combien de temps ? (Iel pourrait ne pas te le dire pour de bonnes raisons !)
  \item Combien de fioles sont produites ? (Iel pourrait ne pas te le dire pour de bonnes raisons !)
  \item Est-ce que tu ne le sens juste \textit{pas} ?
\end{itemize}

C'est juste une partie de la variété de questions que tu peux poser pour savoir si tu peux avoir confiance dans le fait que ta source s'intéresse autant que toi à la qualité du produit.

\subsection{Est-ce que je devrais avoir des standards différents selon mes sources de DIY ?}

Probablement, oui. Les fabricants commerciaux devraient être tenus à un standard plus haut, vu que tu leur donne de l'argent en échange du produit, donc iels peuvent se permettre de bien le faire. Tu ne peux porbablement pas te permettre d'être aussi sévère avec un projet d'entraide qui distribue des fioles gratuitement, ce qui ne veut pas dire que le produit sera meilleur ou moins bien. A toi de voir ! 

\subsection*{Anatomie d'une Fiole}
\addcontentsline{toc}{subsection}{\textemdash{} Anatomie d'une Fiole}

\subsection{Qu'est-ce que je devrais chercher dans une fiole ?}

Les ingrédients d'une fioles peuvent être classés dans deux catégories : les \textit{"principes actifs"} et les \textit{"excipients"}. Le \textit{principe actif} dans notre cas est l'ester d'oestrogène, et les \textit{excipients} sont tous les autres. Il y a généralement trois ou quatre ingrédients : 1) l'ester d'oestrogène; 2) l'huile solvante; 3) le conservateur; et optionnellement 4) un ou des cosolvent(s). On a déjà fait un paragraphe sur les esters dnas la Section \ref{td} "TYPES ET DOSAGES". Les fioles de laboratoires pharmaceutiques ont presque toujours ces quatre ingrédients.

\subsection{Quelle huile solvante devrais-je avoir dans ma fiole ?}\label{6-16}

C'est une question de préférence, de tolérance personnelle, et possiblement d'allergies. \textbf{La variable principale qui t'intéresse est la viscosité qui affecte le confort d'injection et la praticité de celle-ci.} Comme disuté (dans la Question \ref{5-16}), les huiles moins épaisses peuvent te permettre d'utiliser des calibres d'aguilles plus haut sans plus de difficulté au prélèvement et à l'injection. \textbf{Les huiles solvantes les plus utilisées pour un THS sont l'huile MCT et l'huile de castor.} L'huile de castor est l'huile la plus épaisse couramment utilisée, mais elle tend à causer le moins d'irritation donc les labos pharmaceutiques utilisent en général celle-ci. L'huile MCT est moins épaisse, mais certaines personnes la trouve irritante que les autres et n'est disponible qu'en DIY. L'huile de graines de coton ou de grains de raisins sont occasionnelement utilisées, mais généralement pas par des fabricants de THS. D'autres huiles comme l'huile de tournesol ou de sésame sont utilisées des fois, mais ne sont généralement pas recommandées. Selon tes circonstances, cette question peut ne pas en être vraiment une puisque tu n'as peut-être pas le choix.

\subsection{Quels conservateurs devrais-je avoir dans ma fiole ?}\label{6-17}

Le conservateur le plus utilisé dans les fioles d'injectables est le \textit{benzyl alcohol} (BA) en concentration basse. C'est nécessaire, et il n'y a pas de débat là-dessus. \textbf{Tu ne dois jamais utiliser une fiole qui n'a pas de conservateurs.} Pour les gens qui en sont allergiques, le \textit{cholobutanol} est un conservateur alternatif utilisé communément, mais presque jamais par les sources de DIY, ce qui veut dire que tu vas devoir aller chercher le truc vendu par un labo qui marche pour toi. 

\subsection{Quels cosolvents devrais-je avoir dans ma fiole ?}

Le cosolvant principalement utilisé est le \textit{benzyl benzoate} (BB) qui réduit la viscosité de la solution. C'est techniquement optionnel, mais il est généralement recommandé pour avoir une variance moins élevée entre les lots, et des fois même nécessaire en fonction de l'huile solvante utilisée et de la concentration désirée. Certaines personnes trouvent ce composant irritant, mais c'est pas le cas pour tout le monde.
 

\section{GESTION DES PROBLEMES}

\subsection*{Incertitude sur les Dosages}
\addcontentsline{toc}{subsection}{\textemdash{} Incertitude sur les Dosages}

\subsection{Mes niveaux ne sont pas aussi hauts/bas que je pensais, pourquoi ?}

Il y a plusieurs possibilités. Rappelle-toi d'abord que les estimations sont des modèles qui ne prennent pas en compte une pléthore de facteurs qui peuvent causer des déviations. Rappelle-toi aussi qu'il faut plusieurs injections avant d'arriver à une stabilité, donc si tu viens tout juste de changer ton dosage, c'est peut-être pour ça. Vérifie plusieurs fois avec un.e ami.e que tu injecte autant que ce que tu penses. C'est plus commun qu'on ne pourrait le penser, mais pour les fioles obtenues en DIY il est possible que la concentration soit plus basse qu'annoncée à cause d'équipement imprecis ou dû au manque d'expérience du fabricant. Dans ce cas, injecte un petit peu plus si tu prends depuis cette fiole. \textbf{Mais rappelle-toi, le plus important c'est comment tu te sens, pas tes niveaux.} Note aussi que même les labos pharmaceutiques produisent des fois des fioles pourries qui ne sont pas repérées par les contrôle de qualité, même si c'est rare ! 

\subsection{Est-ce que je peux comparer mes niveaux à travers plusieurs tests si je ne les ai pas fait le dernier jour du cycle ?}

\textbf{Non.} Pas de façon précise en tout cas. C'est une des raisons pour lesquelles tu dois toujours faire tes tests au dernier jour du cycle, quelques heures avant ta prochaine injection; c'est ça que tu veux. retirer le plus de variables possible est ce qui rend les données plus utiles pour toi. Si c'est la seule chose que tu gardes de ce guide, c'est ça : teste au dernier jour du cycle.

\subsection{Je me sens vraiment pas bien sur mon dernier jour de cycle, qu'est-ce que je dois faire ?}\label{7-3}

Dans la plupart des cas, ton dosage est trop bas ou ta fréquence d'injection trop basse. C'est particulièrement vrai dans le cas de l'\textit{estradiol valerate} et de l'\textit{estradiol cypionate}. Ajuste ton dosage dans la fourchette donnée plus haut et/ou ajuste ta fréquence d'injection. Trouve ce qui marche pour toi. C'est aussi possible avec l'\textit{estradiol valerate} en particulier que ton dosage soit trop *haut* plutôt que trop bas et que la variance tout au long du cycle soit la raison de cette sensation de crash. En bref : prends de l'\textit{estradiol enanthate} si tu peux.

\subsection*{Difficultés au Moment de l'Injection}
\addcontentsline{toc}{subsection}{\textemdash{} Difficultés au Moment de l'Injection}

\subsection{Mon injection est plus dure à faire quand il fait froid, qu'est-ce qu'il faut que je fasse ?}

Réchauffe la fiole avant de prélever, et réchauffe la seringue avant d'injecter. Fais tourner la fiole dans tes mains pour réchauffer le fluide, et fais pareil avec le cylindre de la seringue. Prends cette habitude pour toutes tes injections pour garantir une meilleur uniformité dans tes injections.  

\subsection{Mon injection fait plus mal quand il fait froid, qu'est-ce que je fais ?}

Réchauffe ta jambe avant d'injecter. Relaxe tes muscles avec un massage ou une douche bien chaude (spécifiquement : réchauffe ta jambre avec de l'eau chaude juste avant de sortir de la douche). Ca va aider.

\subsection{J'ai saigné après mon injection. Est-ce que je vais mourir ?}

Non. Ca veut juste dire que tu as touché un capillaire ou une veine, ça arrive de temps en temps. Tu vas probablement avoir un bleu ou des démangeasons. Ca passera plus vite avec un pansement mignon.

\subsection{Il y avait un peu d'air dans ma seringue. Est-ce que je vais mourir ?}\label{7-7}

Non plus. Même si évidemment tu ne veux pas injecter que de l'air, et que ça peut affecter ton dosage s'il y'a trop d'air dans la seringue, un petit peu d'air (moins de 0.1mL) ne va pas causer de problèmes. Ca peut même être recommandé dans certains cas. Par exemple, la technique de l'\textit{air lock} (une tehcnique standard pour injecter des fluides irritants ou qui tâchent, pas une connaissance super important pour le THS) nécessite d'injecter entre 0.1mL et 0.3mL d'air, donc pas besoin de s'inquiéter. Tu n'es pas en train de faire une injection intraveineuse.

\subsection{Un peu de fluide est sorti du site d'injection. Est-ce que j'ai raté mon injection/Est-ce que je vais mourir ?}

Non. Une fuite peut arriver pour plein de raisons et c'est rarement un volume suffisant pour être significatif, donc pas besoin de faire une nouvelle injection. Pour la prochaine fois, pense à laisser l'aiguille 5 à 10 secondes avant de la retirer et applique une pression après avoir reitré l'aiguille. Si les fuites t'inquiètent, tu peux utiliser la technique de l'air lock mentionnée plus tôt.

\subsection{Des fois, j'ai mal après une injection. Est-ce que je vais mourir ?}

Non. En supposant que tu as suivi les recommandations de ce guide, des fois tu injecte juste dans un endroit infortable pour une raison ou une autre. Tu auras plus de chance la prochaine fois. \textbf{Assure-toi d'alterner les sites d'injection !} Tu ne veux pas que le tissu cicatriciel s'accumule sur le long terme, et si un endroit te fais déjà mal, tu ne veux pas empirer la chose.

\subsection{Le site de ma dernière injection est irrité / gratte. Est-ce que je vais mourir ?}

Probalement pas. Il y plein de raisons possible à ça. La plus inquiétante est l'infection, mais elle est peu probable. \textbf{Va voir un.e docteur.e immédiatement si tu as des symptômes de fièvre, de douleurs importantes, de douleurs musculaire, du pus, des marques rouges, ou autres marques d'infection.} Dans la plupart des cas cependant, des rougeurs, un léger gonflement, etc. sont dûs au fait que l'huile et l'oestrogène se sont dissociés. Plus d'informations plus bas. C'est aussi possible que tu aies juste une réaction à l'huile solvante, mais si jamais tu as des soucis après plusieurs injections sans aucun problèmes, c'est probablement que le contenu de la fiole s'est dissocié.

\subsection{Il y a des cristaux dans ma fiole, est-ce que je peux toujours l'utiliser ?}

C'est probablement que ta fiole est trop froide. Réchauffe la un peu dans tes mains et secoue-la doucement pour récincorporer les cristaux. Si les cristaux ne disparaissent pas, alors il est possible que le contenu de la fiole se soit dissocié. Avec beaucoup de chaleur et de temps à mélanger, les cristaux peuvent se réincorporer, mais c'est plus simple et plus sûr de changer de fiole si tu peux.
 

\section{PROGESTERONE}

\subsection{Do I want to take progesterone?}


\subsection{What is the difference between “progesterone” vs “progestin” / ”progestogen”?}

The class of hormones, both natural and synthetic, that activate the progesterone receptor are “proges\textbf{togens}”. The natural, bioidentical, and most important progestogen is “proges\textbf{terone}”. Synthetic progestogens are “proges\textbf{tins}”. These three terms are mistakenly used interchangeably in scientific literature and in clinical settings, likely causing much of the broader confusion regarding the role of progesterone in HRT, despite the fact that they are \textbf{not }equivalent.

\subsection{Do I want progesterone or a progestin?}

Progesterone. You want bioidentical progesterone.

\subsection{What’s wrong with progestins?}

Progestins, most typically \textit{medroxyprogesterone}, \textit{medroxyprogesterone acetate}, or \textit{levonorgestrel}, are generally associated with the negative side effects and long term risks (breast cancer, blood clots, depression, etc) that are falsely attributed to progesterone. They are not bioidentical which means they do not behave the same as progesterone and thus cannot be directly compared.


Progesterone is believed to play a role in breast development and libido in particular, but as mentioned it’s a key hormone aside from its outward appearance effects. It does also have some antigonadotropic (i.e., it contributes to testosterone suppression) properties which can be sometimes relevant.

\subsection{Does it matter when I start progesterone?}

It is unknown. There is some belief that starting too early may harm breast development long term, but this is purely theoretical and contrary anecdotal evidence makes the answer unclear. The conservative estimate is waiting roughly a year into HRT (until Tanner Stage 3 or 4) in the possible chance that it does matter.

\subsection{How is progesterone normally taken?}

Aside from topical applications, the main form is via a pill. It is prescribed as an oral pill but is most effective when taken as a suppository. Topical sprays and creams can also work very well.

\subsection{Are you serious that progesterone should be taken as a suppository?}

Progesterone metabolizes entirely differently when taken orally vs rectally due to passing through the liver when taken orally. Oral progesterone primarily converts to \textit{allopregnanolone} which can cause heavy drowsiness, whereas rectal progesterone primarily converts to progesterone itself which is what we want (although some still converts). Some people take additional oral progesterone as a sleep aid, but please note that too much \textit{allopregnanolone }can sometimes lead to negative mental health side effects.

\subsection{How do I take progesterone as a suppository?}

Just a bit of water on the pill should work, then dry off and wash your hands. Obviously, don’t go to the bathroom for the next hour or so, so doing it before bed is best. If you are having issues with it not dissolving then you can try piercing the capsule but usually should be no issue. Be aware that if you use large homebrew suppositories made using coconut oil, the large volume of coconut oil will not want to stay in you.

\subsection{How much progesterone should I take?}

For pills, Standard dosage is 100-200mg daily at night. It is a rather arbitrary dosage; 200mg is the max that most doctors will prescribe. Some people take more than 200mg on occasion, but be aware that spiking your levels may lead to an unpleasant crash. See question below.

For topical applications, nobody can tell you with certainty due to the high variability of the delivery medium, nor is there any clear guidance on desired levels, or even frequency (likely daily), as progesterone is simply understudied. Because of this, I would advise titrating your dosage so that you understand how progesterone affects you.

\subsection{Is there any benefit to “cycling” progesterone?}\label{8-11}

No. Some people do this to mimic a cis woman’s menstrual cycle, but there is no reason to believe there is any benefit to this and it may cause negative PMS symptoms. The only exception is if you have good reason to suspect that you have an intersex condition involving a uterus that you are managing. I discourage it otherwise. See Question \ref{11-10}.

\subsection{How long should I take progesterone for?}

For as long as you plan to take estrogen and for as long as you want to. So, probably forever.

Sometimes people (or doctors) arbitrarily say to only take progesterone for X years. There is zero theoretical or empirical reason to suggest that this is sound advice. It's about as coherent as if someone (or a doctor) asked how long a trans person planned to take HRT for\textemdash{}oh wait never mind they do ask that.

\subsection{Can progesterone convert into \textit{dihydrotestosterone} (DHT)?}

No. Well, strictly speaking yes, but also no. It is largely a myth, although \href{https://whsah.co/posts/rethinking-progesterone-and-androgens/}{as outlined in detail by alix in this article}, for cases of people with \textit{nonclassical congenital adrenal hyperplasia} (ncCAH) progesterone can cause some negative side effects of increased androgenic activity. In those cases, discontinuing progesterone is recommended along with seeking out a formal diagnosis/treatment for potential adrenal disorders.

\subsection{Is there any benefit to topical progesterone applications in addition to pills?}

Maybe. It's a possible alternative to pills, especially in the case of someone with a peanut allergy since the most common pill manufacturer uses peanut oil, but again dosage is unclear. Some people find more progesterone fun, if nothing else. Be safe and have fun.

For clarity: Apply creams to your inner thigh region (elsewhere if directed), or optionally on scrotal skin (it's thin and highly vascular) in the case of sprays. And no, applying progesterone to your breasts directly is unlikely to make them grow bigger or faster compared to otherwise. 

\subsection{Can I snort progesterone powder?}

Please don’t. It’s hell on your sinuses. It isn’t hard to make your own topical progesterone spray and there are guides out there. Do that instead. It’s significantly more effective, consistent, and safer.

\subsection{Where can I get progesterone?}

Progesterone tends to be more expensive through DIY sources due to the higher mass of hormones required, so ideally get it through pharmaceutical sources covered by insurance. There is also the option of grey market foreign pharmacies, which are simply pharmacies in another country, although these often require some hurdles to purchase from. Topical progesterone creams are available OTC in some locations, although it is not always the most economical depending on the concentration.

\subsection{I would like to read more about progesterone in an HRT context. What resources should I read?}\label{8-17}


It should be noted that for the entire category of progestogens there are countless myths and falsehoods invented whole cloth by both proponents and detractors alike which does not make discerning truth from the already-sloppy scholarship any easier. Fantastical claims of magical benefits and fearmongering of alleged risks based on nothing are both equally unhelpful, although the later is worse in my opinion when comes from a medical authority, whether neglectful or malicious.

\subsection{Does progesterone interact with any other drugs related to HRT?}\label{8-18}

If you are taking 5$\alpha$-Reductase Inhibitors like \textit{finasteride} and \textit{dutasteride} (See Section \ref{AA} “ANTIANDROGENS”, or keep reading), these can affect how progesterone naturally breaks down into \textit{allopregnanolone} which can cause adverse mood effects in some people, irrespective of how you are taking progesterone. It is not fully clear how much the administration route for the 5$\alpha$-Reductase Inhibitors (i.e., topical vs oral) makes a difference, but lower systemic absorption via topical application may mitigate these side effects. It is recommended to not take either of those if you are someone affected by this interaction, but it is not in all cases anyway. Note that these depressive effects may be felt for up to a month after stopping. 
 

\section{TESTOSTERONE}\label{T}

\subsection{Why don’t we want zero testosterone?}


\subsection{Are there ever cases where I would want to supplement testosterone?}\label{9-2}

Yes. If you are experiencing the issues of the above and your estrogen levels are otherwise good, it’s possible that you might want to supplement with a microdose of testosterone. If you wanted to improve your erectile function, minimize any atrophy before bottom surgery, or otherwise wanted to experiment with your hormones to see what feels best for you, then that might be a reason to explore testosterone in a different context that you can hopefully appreciate more compared to pre-HRT.

\subsection{If I wanted to supplement testosterone, how would I do it?}

There’s a few possibilities. Testosterone comes in either injections or topical gels/creams, similar to estrogen as already discussed. Topical is more likely what you are going to be prescribed. Topical applications have the downsides that we have discussed for estrogen, but those are less of a concern here when precise levels are less important.

\subsection{What are the topical forms of testosterone?}

There is gel and cream. Gel is typically what will be prescribed, but some compounding pharmacies are able to make low-penetrating cream if someone wanted just topical application on the genitals. The latter is harder to get and generally more expensive, however.

\subsection{Does it matter where I apply the testosterone?}

It depends on if you have gel or cream. If you have the kind of localized cream as mentioned above, you would apply it as directly as mentioned. Otherwise, shoulders and upper arms are where gel should go. Make sure not to touch things until long after it dries!

\subsection{How much and how often should I apply testosterone?}

Season to taste. This largely depends on how you are feeling. If you have too much, you might start to experience side effects of testosterone (e.g., oily skin and body hair), but only you can say what is preferred for you. A weekly injection of 5-10mg of \textit{testosterone cypionate} might work for you, but in the case of 1\% topical gels which are often disbursed in 25/50mg packets, there is more variability. You almost never want even half a packet, and definitely not daily. I would suggest starting with much less than you think to see how you feel.

\subsection{Where would I get testosterone?}

If you are an American, you would have to get a prescription or ask any juicer at your closest Planet Fitness. Elsewhere, it depends on what gym chain is closest to you. Disclaimer: This is a joke. See Question \ref{6-11} “Is DIY legal?”

\subsection{Are other steroids equivalent to testosterone in an HRT context?}

Anabolic-androgenic steroids, i.e., drugs that are structurally similar to testosterone, are not all equivalent. Commonly used black market steroids like \textit{trenbolone acetate} have a laundry list of undesirable side effects, but steroids like \textit{nandrolone decanoate }are occasionally used for postmenopausal cis women due to their relatively low androgenic properties which make them very favorable for transfeminine individuals. Regardless, in America it is unlikely you will be prescribed anything other than testosterone itself, if you are able to get a prescription at all.

\subsection{What is the relationship between testosterone and \textit{dihydrotestosterone} (DHT)?}

\textit{Dihydrotestosterone} is primarily synthesized from testosterone via the 5$\alpha$-Reductase enzyme with around 5\% of testosterone in your body being converted. Generally speaking, if testosterone levels are suppressed (or if you have had bottom surgery) then there should not be much left to convert, but systemic levels won’t be zero because it is still locally produced. Depending on your body, this would be the main reason that you might want to consider supplementing with a 5$\alpha$-Reductase Inhibitor antiandrogen as discussed in the following section. As a reminder, \textit{dihydrotestosterone }is the hormone that is responsible Quel voie d'administration prendre pour mon oestrogène ?

Par injection. C'est le moyen le plus facile, prédicible, sûr, efficace et abordable de faire une transition hormonale. On dit même que pour certains, le moment de l'injection devient un rituel attendu, et que pour d'autres,  
\section{ANTIANDROGENS}\label{AA}

\subsection{What are “antiandrogens”?}

\textit{Antiandrogens, }commonly also referred to as “T blockers” or just “blockers”, as the name(s) may suggest prevent androgens (that’s what testosterone is) from acting on your body. There are many types of antiandrogens and they are commonly prescribed as part of an HRT regimen. They are needed if someone still produces testosterone and is not doing a form of HRT conducive to monotherapy, such as injections, but they are usually not desirable. It also should be noted that (most) antiandrogens do not reduce testosterone levels in any way that matters but instead simply reduce/negate effects on the body. This is relevant when interpreting lab results and such.

\subsection{Why wouldn’t I want antiandrogens?}

The main issue with most antiandrogens is that they generally have very undesirable side effects that are superfluous if testosterone is suppressed in the first place by having enough estrogen, so those side effects are being experienced despite\textemdash{}in most cases, at least\textemdash{}being rendered unnecessary by a reasonably-dosed monotherapy regimen. Bottom surgery of any kind also makes antiandrogens unnecessary in most cases.

\subsection{When might I want antiandrogens?}

If you are not most cases, if you desire peace of mind, or if your insurance requires a prescription on file before they will cover a procedure, then you may want antiandrogens. The medications used as antiandrogens might have other effects that may be desirable outside of their antiandrogen properties depending on your health situation. Additionally, if you are supplementing androgens, you may want a \textit{dihydrotestosterone }(DHT) blocker to minimize side effects related to body hair and hair loss, but be aware that this may not be the case if you are not using bioidentical testosterone (e.g. \textit{nandrolone decanoate}) because not all androgens behave the same.


Quel voie d'administration prendre pour mon oestrogène ?

Par injection. C'est le moyen le plus facile, prédicible, sûr, efficace et abordable de faire une transition hormonale. On dit même que pour certains, le moment de l'injection devient un rituel attendu, et que pour d'autres,  
The main medications taken as general testosterone blockers in an HRT context are \textit{spironolactone}, \textit{bicalutamide}, and \textit{cyproterone acetate}. The main medications taken to block the conversion of testosterone into \textit{dihydrotestosterone} (DHT) called “5$\alpha$-Reductase Inhibitors” (5-ARI) are \textit{finasteride} and \textit{dutasteride}. There are also GnRH agonists like \textit{leuprolide} and \textit{triptorelin}, but both of those are more often used as puberty blockers in minors, although in parts of Europe they are used for adults as well.

\subsection{When might I want to take \textit{spironolactone}?}

Due to the heroic dosages and significant negative side effects required for it to function as an antiandrogen in most cases, the only time I would ever recommend taking \textit{spironolactone} would be if you would benefit from its other effects such as its antimineralocorticoid (i.e., blocking \textit{aldosterone}) properties as it relates to blood pressure management or edema. \textbf{If you insist on taking \textit{spironolactone}, please do not take more than 100mg daily.} It has a bad reputation for a reason. “The Devil”, as it were.

In case you are unfamiliar, some of the many side effects include: brain fog, lethargy, poor memory, increased urination frequency, low blood pressure, low sodium / electrolyte imbalance, etc. In other words, \textit{spironolactone} is a blood pressure lowering dieurtic that is a mediocre antiandrogen which is typically prescribed at high dosages in an otherwise-healthy population for questionably-effective off-label use. In any other healthcare context this would (or SHOULD!) be highly unadvisable given the undesirable side effect profile and the widely-available preferable alternatives that already exist, but that's the state of trans healthcare for you.

\subsection{When might I want to take \textit{bicalutamide}?}

If you are going to take an antiandrogen, \textit{bicalutamide} is likely the one to take. It is generally well tolerated, barring 1\% cases of abnormal liver function test results and symptoms of liver dysfunction, but otherwise performs the job with relatively minimal side effects. \textbf{If you take \textit{bicalutamide}, ensure regular liver function tests to make sure that your results are in range. }The liver risks are dependent on your body rather than cumulative so any problem would likely present itself within the first year. Otherwise, there should be no issues. 

\subsection{When might I want to take \textit{cyproterone acetate}?}

Likely never. Take \textit{bicalutamide} instead.

The long term risk profile is poor and there is no situation that I can think of in which I would recommend this over an alternative solution. You can do everything \textit{cyproterone acetate} can by just taking more estrogen and adding progesterone to your regimen.

\subsection{When might I want to take \textit{dutasteride}?}

If you are extremely concerned about possible hair loss and/or want to maximize your chances for hair regrowth, you may want to take \textit{dutasteride}. If your testosterone is otherwise suppressed then it theoretically shouldn’t have much benefit as your \textit{dihydrotestosterone} levels should be relatively low, but bodies can be complicated, so it may be something of interest to you. Also, see Question \ref{11-14}.

It should be noted that \textit{dutasteride} can cause adverse mood effects in some people, in which case stopping is strongly recommended. Note as well that these depressive effects may be felt for up to a month after stopping. 

\subsection{When might I want to take \textit{finasteride}?}

If \textit{dutasteride }is not something prescribed to you or if your insurance mandates \textit{finasteride} specifically to cover a hair treatment. Otherwise, \textit{dutasteride} is preferred as it is more effective and better tolerated.

It should be noted that \textit{finasteride} can cause adverse mood effects in some people, in which case stopping is strongly recommended. Note as well that these depressive effects may be felt for up to a month after stopping.

\subsection{Where can I get antiandrogens?}

Aside from being prescribed them by your doctor or perhaps available over-the-counter, there is also the option of grey market foreign pharmacies. These are simply pharmacies in another country, although these often take some hurdles to purchase from. \textit{Dutasteride} and \textit{finasteride }are generally the easiest to get over-the-counter because of their commonality as hair loss medication.

 

\section{MYTHS AND MISCS}\label{MM}

\subsection*{Common Questions}
\addcontentsline{toc}{subsection}{\textemdash{} Common Questions}

\subsection{Should I be worried about blood clots?}\label{11-1}

Yes and no. It is true that there is a correlation between estrogen dosages/levels and blood clot risk, but this is primarily related to the route of administration and the type of estrogen. Synthetic estrogens are the rightful cause of scorn and do lead to significantly increased blood clot risk, but bioidentical estrogens are not as concerning. In particular, the route of administration makes a major difference. Oral bioidentical estrogen passes through the liver which is what causes the increased blood clot risk. Injections bypass the liver, and there's no evidence to suggest nor reason to believe that injections of bioidentical estrogen provide any significant risk increase beyond the innate differences between testosterone and estrogen. The pervasive fearmongering towards all estrogen has persisted for decades despite these differences.

\textbf{If you are undergoing surgery, please know that pausing hormones out of concern for blood clots is no longer recommended by WPATH.} Many surgeons still include it in their pre-surgery guidelines out of concern for blood clots, but this is torture that has been disproven and even WPATH doesn't recommend it anymore. Remarkable, I know. Per \href{https://www.tandfonline.com/doi/pdf/10.1080/26895269.2022.2100644}{WPATH SOC 8 Statement 12.19}: \blockquote{After careful examination, investigators have found no perioperative increase in the rate of VTE [KT: \textit{venous thromboembolism}, i.e. a blood clot] among transgender individuals undergoing surgery, while being maintained on sex steroid treatment throughout when compared with that among patients whose sex steroid treatment was discontinued preoperatively (Gaither et al., 2018; Hembree et al., 2009; Kozato et al., 2021; Prince \& Safer, 2020).} I should put this in another question entirely, but to not break links, it would have to be at the bottom of a section and I think this is too important for that, so I note it here. A very important clarification that I should have had sooner.

\subsection{Is it okay to use nicotine while on HRT?}\label{11-2}


However, to be abundantly clear, \textbf{this does not mean that you cannot or should not take estrogen. The downsides of not taking estrogen at all far exceed the downsides of using nicotine.} This section is simply seeking to make you aware of any increased risks and potentially slower transition as a very strong recommendation and encouragement to quit. One step at a time.

\subsection{Is there benefit to starting at a low dosage vs a high dosage?}

To the best of knowledge, no. Sex hormones are not like other drugs that need to be titrated to manage side effects as we know the dosages that work for the majority of people, so personally I view “starter dosages” and “antiandrogen first” regimens as medical abuse. Some people believe that mimicking the slow timeline of puberty might be best (even though there are far more things happening than just estrogen levels), but there’s no evidence to support this. An orchiectomy day one might be best for all we know, but who is going to do that the moment they decide they are trans and/or want to start HRT?


\subsection{Does body weight affect dosage?}

No. Because there is no “optimal” blood level for estrogen and because the therapeutic range of acceptable levels is so wide, body weight does not meaningfully affect dosage for HRT. It is for the same reason that slight deviations in dosage are unlikely to affect how you feel. There is no such thing as being “too light” or “too heavy” for HRT in any capacity.

Adjusting your dosage in increments of 0.1mg is a difference that should not be expected to be perceived simply because our bodies are not sensitive enough to such exact measurements, let alone the high possibility of imprecision when performing an injection that makes that certainty of this measurement unlikely. In other words, the accuracy of your dosage is more important than the precision.

\subsection{Is there such a thing as starting estrogen too late?}

\textbf{No.} No matter when you start, estrogen is able to do a LOT and a proper regimen will be able to have powerful results. Sex hormones are some of the strongest hormones in our body in terms of our appearance. Everybody always wishes that they could’ve started sooner, but that’s no reason not to start now. Even if you’ve been on estrogen for years, there is still a benefit to be had in improving the quality of your regimen.


\textbf{No.} There is not an arbitrary time where estrogen suddenly stops working. Various numbers are given and usually it’s either 1) entirely made up or 2) pointing to a study that only went for X years. Doctors in particular love to tell trans women not to expect more than B cup breasts (which isn’t even how breast sizing \textit{works}, but I digress) or for any growth after 2 years, but this is simply not true. There are cases of people who restarted estrogen after stopping for many years and still experiencing new growth.

\subsection{I haven’t seen any changes in years on injections. Would swapping back to pills make a difference?}


\subsection{Is low energy and low libido normal on HRT?}

Generally, no. How libido is expressed changes in the beginning, but the vast majority of the time that someone experiences abnormally low libido it’s because they haven’t gotten their hormones sorted. The same goes for low energy. Get your hormones squared away, and barring that, check your diet/vitamins next. Make sure you don’t randomly have critically low vitamin D levels or something like that. It happens more often than you think.



\subsection{Do we want to mimic the estrogen cycle of cis women?}\label{11-10}

Quel voie d'administration prendre pour mon oestrogène ?

Par injection. C'est le moyen le plus facile, prédicible, sûr, efficace et abordable de faire une transition hormonale. On dit même que pour certains, le moment de l'injection devient un rituel attendu, et que pour d'autres,  
Similar to the last question, it’s important to understand what is happening. The unique hormone curve produced by your particular ester, your dosage, and your frequency can cause changes in your mood as your estrogen levels oscillate between injections. Some trans women liken this phenomenon to a period, but the underlying cause for these physiological changes is different and is usually a sign that your regimen needs tweaking so that you feel the best that you can as suffering is not virtuously feminine. Pain and discomfort are not requirements for womanhood nor should we assert ourselves based on bioessentialist arguments. The exception here are the intersex trans women who have a uterus and literally are having a period, in which case: yeah duh. See Question \ref{11-35}.

\subsection{Can too much estrogen convert to testosterone?}

\textbf{No.} Aromatase is the enzyme responsible for converting testosterone into estrogen, but there is no mechanism to convert estrogen into testosterone. This cannot happen. This is a completely false myth and you should be immediately wary of the knowledge level of anyone who says it to you. Unfortunately, it is doctors who repeat this myth the most.

\subsection{Does bottom surgery cause an increase in testosterone?}

No. This is not a thing. There is not a magic mechanism that suddenly causes testosterone to increase the moment that testicles are removed. Even if magic was stored in the balls, this simply isn’t how hormone production works. “Well, your adrenals…” They don’t work like that either. The only possible rare exception would be undiagnosed adrenal hyperandrogenism conditions that were suppressed by an antiandrogen like \textit{spironolactone }prior to surgery which might show itself after antiandrogens are ceased. Please stop repeating this myth.

\subsection{How do I prevent/revert hair loss?}\label{11-14}

Mechanically, it is pretty simple. A standard HRT regimen alone is borderline magic (don’t ask where the magic is stored) in this regard already, but the inclusion of 5$\alpha$-Reductase Inhibitors (5-ARI) as discussed in Section \ref{AA} “ANTIANDROGENS” is recommended in more extreme cases to completely halt any loss. Topical minoxodil 5\% is the only thing that works to firm up your hairline beyond hormones alone, but keep in mind that aside from miracle cases, you’re only saving dying/dormant follicles. Dead follicles don’t come back.

If this alone is insufficient for you, hair transplant technology has improved significantly. The Follicular Unit Extraction (FUE) procedure is what you want to look into. Here is where in the future I will link a guide written by an expert on getting insurance to cover that, once she writes it. This is peer pressure. Watch this space.


Probably. HRT causes gradual body recomposition, so you can help encourage your body to shift through exercise. Keep in mind that this process is VERY SLOW, so it is crucial that you eat enough to fuel how patient you have to be. The growth hormones from muscle stimulation via strength training also play a role in breast development, so it’s probably a good thing even aside from the rest of the obvious health benefits of exercise.

This is NOT just the writer’s barely-disguised fetish; strength training is important for your health! I mention this because a lot of trans women believe that touching a dumbbell will make them look like the hulk. I get it, but if you have no testosterone in you and you aren’t on steroids, then you aren’t going to look like that. Let alone the constant time, effort, and diligence required to even get close.

\subsection{What should I exercise then?}\label{11-16}

Cardio is useful for living which is important. Lower body exercises will fill out your hips and glutes to accentuate your figure. Upper body exercises will improve your posture and support your breasts which will make them look bigger. In other words, everything. You’re on estrogen. Have you seen cis women athletes? Exercise will feminize you.

\href{https://docs.google.com/document/d/1-NyE5EY5TTaRRMhk7HlTbKJ7HifjEsA4jlDO1qKQVl0/edit?tab=t.0}{This guide was shared with me} \textcolor{red}{(Warning: Google Docs link)} and looks to be a good starting place. I will note that there aren't particular exercises that feminize vs masculinize as bodies don't work like that, but you may wish to focus more on lower body exerices and flexibility more than the typical lifter.

\subsection{Can estrogen really cause height shrinkage?}

Yes. It is possible that it’s related to water content changes within tendons and ligaments, but it is not something that has been studied so the cause is fully speculation. Scientists: free study idea!

\subsection{Can estrogen really cause foot shrinkage?}

Yes. See above.

\subsection{Can estrogen really cause any other kinds of shrinkage?}

Well, “use it or lose it” like they always say.

\subsection*{Sexual Health}
\addcontentsline{toc}{subsection}{\textemdash{} Sexual Health}

\subsection{How do I improve erectile function on HRT?}\label{11-20}

Aside from using it regularly, ways to improve erectile function include: 1) Improving your fitness and physical health, particularly your cardiovascular ability; 2) consider medication like \textit{tadalafil} or \textit{sildenafil}; and 3) consider testosterone supplementation (see Section \ref{T} “TESTOSTERONE”).

If you would like to read a longer explanation for how erectile function works, \href{https://stainedglasswoman.substack.com/p/how-to-maintain-your-penis-function}{this Substack article} provides a good overview of the topic.

\subsection{How do I increase cum/pre-cum volume on HRT?}

Don’t be embarrassed, it’s a common question. Sunflower lecithin and pygeum increase both of those. It seems to also make a difference for vaginal wetness and arousal for those who have had bottom surgery, but data and anecdotes are limited so it’s hard to say. Otherwise just be sure you drink enough water and have your nutrition in check.

\subsection{Can I lactate on HRT?}

Yes. Domperidone, fenugreek, sunflower lecithin, ample estrogen, and ample progesterone. Get a pump. Knock yourself out.

It should be noted that domperidone has side effects and risks associated with it, and that ability to lactate does not affect breast development. Newman-Goldfarb protocols would be what you want to look into.

\subsection{Can HRT change your senses and your perceptions, i.e. smell?}

You very likely were dissociated and depressed for years prior to starting HRT. The world is more vibrant now because you are no longer dissociating 24/7. The wonders of modern medicine!

It can, however, directly change your eye prescription. That can definitely happen.

\subsection{Can HRT change your sexuality?}

Similar to being dissociated as with above, HRT often incurs a lot more openness and acceptance with yourself which can cause a shift in how your sexuality presents itself. It is largely a semantics argument as to whether that is chemical or behavioral. A matter of perspective. 

\subsection{Should I be on PrEP?}

\textbf{Yes.}

\subsection*{Medical Malpractice}
\addcontentsline{toc}{subsection}{\textemdash{} Medical Malpractice}

\subsection{I heard that injections are actually less stable because you do them less frequently. Is that true?}

Only if you follow the dipshit WPATH SOC 8 guidelines that list a recommended regimen of \textit{estradiol valerate} or \textit{estradiol cypionate} in the range of 5-30mg every two weeks which, to be abundantly clear, you absolutely should never do in a million years. “Do no harm”, my ass. 

\subsection{But my doctor said-?}

The average doctor has essentially no training in anything related to trans healthcare, and \href{https://www.endocrine.org/news-and-advocacy/news-room/2017/endocrinologists-want-training-in-transgender-care }{4/5 endocrinologists have never had any formal training in trans healthcare}. It is most likely that you are their first trans patient and that they are inexperienced in the practical elements of managing a trans patient. Even among doctors who care a lot, they are often limited by conservative standards of care that they are forced to follow which do not always align with the care best for you. See above.

Please also be aware of “trans broken arm syndrome”, aka the tendency of doctors to blame everything on HRT. If your arm is broken, it's probably not “because of those hormones”!

And I should put this as a separate question but I don't want to break the formatting: in line with medical malpractice, there is no situation in which it is reasonable for a doctor to request to see or feel your breasts to “monitor growth” or for any other reason. It is far less common these days, thankfully, but it is sexual assault and completely unacceptable.

\subsection{My doctor won’t prescribe me injections. What do I do?}

Attempt to convince them, replace them, or seek DIY sources. Do not let a gatekeeping medical establishment prevent you from receiving the appropriate care that you deserve. \textbf{The most crucial aspect of interfacing with the medical system while trans is that you have to advocate for yourself. }This is compounded with disability, ethnicity, and other afflictions that scare doctors like womanhood.

\subsection{How does HRT for menopausal cis women relate to HRT for trans women?}\label{11-29}

While we generally have different goals and crucially have very different dosage requirements, there is an immense amount of overlap in experience for trans women and menopausal cis women. Medical misogyny in the form of incompetence, dismissiveness, antagonism, and/or misinformation is something that we unfortunately both experience. It is for this reason that it is paramount to build solidarity on this front. To give an example of what I mean, \href{https://www.youtube.com/watch?v=W0XW6av2wLQ}{the first 30-40 minutes of this interview} will likely sound extremely familiar to you if you would like to raise your blood pressure. The interviewee herself notes the connection too! The WHI ruined the lives of countless women.

\subsection*{Intersexuality and Comorbidities}
\addcontentsline{toc}{subsection}{\textemdash{} Intersexuality and Comorbidities}

\subsection{What’s up with Ehlers-Danlos Syndrome?}

This connective tissue disorder doesn’t actually relate to HRT but a lot of trans people have it so congrats in case this is how you learned that you do too. Aside from general cardiovascular long term concerns to maybe look into, keep up with strength training so that your joints work. Look into that elsewhere though. See Question \ref{11-16}.

\subsection{What kind of intersex things should I keep in mind?}

Throughout this guide, I have mentioned intersex conditions vaguely. Below is a short list of things that might be useful for you to know in your travels for yourself or for a friend. 

\subsection{What’s up with Klinefelter Syndrome?}

This is a relatively (considering chromosomal mutations) common intersex-related condition that some trans women might not realize that they have as the two can overlap. It generally presents as low testosterone at the start of puberty. Good for you to know the name, just in case.

\subsection{What’s up with Persistent Müllerian Duct Syndrome (PMDS)?}

Another “I’m putting this here because this might be the first time you’ve even heard of the term” intersex-related condition that can affect some trans women, however few that may be since we don’t have numbers. The possible presence of an underdeveloped uterus leads to some possible complications and oddities. You probably extra want to have progesterone to avoid uterine cancer risks.

\subsection{What's up with ovotesticular syndrome?}

This intersex condition in particular can cause early level fluctuations which made lead to confusing test results due to the presence of both ovarian and testicular tissues, either separate or combined in an \textit{ovotestis}. This presents in many different ways which HRT can interact with as you begin suppressing \textit{luteinizing hormone} (LH). A uterus may or may not be present, multiple sets of gonads could be present, and/or it could look outwardly typical.

\subsection{What’s the difference between intestinal cramps and uterine cramps?}\label{11-35}

These are commonly misattributed in early transition as a symptom of intersex conditions. Intestinal cramps are widespread and diffuse across your abdomen, whereas uterine cramps are highly concentrated in a location somewhere below your belly button and tend to be sharp stabs/contractions in rapid succession. Like the inside of your body is used as a stress ball. Very different!

\subsection{What about other intersex conditions?}

I have listed a few notable ones, but there are far more expressions and ways of testing them that go far beyond the scope of this guide. Anecdotally, prevalence is higher than average among trans people so basic familiarity with this is useful.

\subsection*{Oddball Questions}
\addcontentsline{toc}{subsection}{\textemdash{} Oddball Questions}

\subsection{Many DIY sources only take crypto. Is that required? How does that work?}

There are other guides that cover this in better depth than I can on how to use crypto safely, including some vendors who have their own guides. But yes, crypto is often required for a lot of reasons. “Crypto” means a lot of things, but using it as a currency was the original point after all. It’s mostly just a pain in the ass. Monero (XMR) is good.

\subsection{What about Selective Estrogen Receptor Modulator (SERM) drugs for nonbinary regimens?}

Some people use SERMs as a part of a transition that is not looking to feminize as much for a more androgynous look, but it’s pretty much entirely uncharted waters thus why their mention is otherwise absent from this guide. You’re on your own if that’s something you want to explore, so please be safe. I don’t personally rate them very highly as I have not seen much to suggest that they work well for how people usually think or want them to work, at least not without a lot more caveats, but obviously there are people who like them. It's just not something I feel comfortable giving recommendations for.

The various proposed nonbinary regimens are often highly individualized because they are specific to what a persons' particular goals are. All HRT should be individualized to a degree, but there is often more variation in desired outcomes when people ask about androgyny. Hormonally, it is nontrivial. Everything stated in this guide should be treated solely as a starting place if you are wanting to experiment with something more complicated, but do remember that there is much more to achieving transition goals than just hormones alone.

\subsection{Are things like “herbal HRT” or “phytoestrogens” legitimate?}

\textbf{No.} If someone is telling you they have “herbal HRT”, they are telling you they have snake oil. The only thing that is going to feminize you is estrogen, not plant estrogens. No amount of “natural” products are a replacement for estrogen itself. This isn’t a common scam and you probably already know, but just in case you run into it, now you know for sure. If it smells like bullshit, it’s probably bullshit. Unless we’re talking about bug steroids in which case yeah those are actually cool. Won’t feminize you though.

\subsection{Is the Reddit Doctor that people constantly talk about Good?}

No.

\subsection{I hear DIY estrogen is made in a bathtub. Is that true?}

No. I honestly have no idea where or why this joke started that people now take seriously, but there’s no step in any process where a bathtub would even be considered. Don’t believe everything you read online. I don’t even know what you could even theoretically do with a bathtub, unless you think estrogen vials are full of the bathwater of trans women. I don’t know why you would think that though. It’s obviously cum.

\subsection{How does HRT affect fertility?}\label{11-42}

It is important to understand that this is extremely understudied so exact figures cannot be stated, and given the seriousness of pregnancy, I urge you to practice safe sex and lean on the side of caution where possible. HRT itself can, and likely will, make you infertile eventually, but only through full suppression of the HPG axis (See Question \ref{2-3}) over a long time span. In other words, if you haven't had bottom surgery of any kind and you are on an HRT regimen that is less capable of HPG axis suppression (such as pills), then this is more of a consideration.

\textbf{If the HPG axis is not suppressed then it is fully possible to impregnate someone}, and the timeline for sperm maturation is long enough that this is true even after the HPG axis has been initially suppressed for \textbf{multiple months}. Please take this very seriously. Full HPG axis suppression for at minimum six months, perhaps closer to a year out of an abundance of caution, is recommended.

\subsection{Is infertility from HRT reversible?}\label{11-43}

It is theoretically possible to reverse HRT-induced infertility, assuming you weren't already infertile prior to HRT (a large assumption!), but there are not many documented cases of this so the full efficacy of fertility restoration after long-term HRT is unknown. The process would involve restarting the HPG axis with a variety of medications along with entirely stopping HRT, which would in essence require a hormonal detransition for likely six months at minimum, and even then sperm quality is not certain or guaranteed. It is not something that should be planned for, to say the least, so planning around it would be wise. A sperm bank would be recommended before or early in HRT, financially permitting, if potential biological children are a priority and if a future relationship where that is possible/desired is likely.



\section{CRÉATINE}

\subsection{Qu'est-ce que c'est la créatine?}

La créatine est un composé organique dans les muscles et le cerveau. Elle recycle l'ADP en ATP ce qui est important pour la production d'énergie dans le corps, en particulier lors d'efforts intenses initiaux avant que d'autres système énergétiques ne prennent le relais. 

\subsection{C'est pas genre un stéroïde ou un truc que les bodybuilders utilisent?}

Non. Les bodybuilders et les athlètes l'aiment parce qu'avoir plus d'énergie permet de faire plus de choses avant d'être fatigués. Ce ne sont pas les seuls à l'utiliser parce qu'en fait c'est le 1\ts{er} supplément en terme d'effet vraiment efficace et qui sont vraiment soutenus par la recherche.

\subsection{En quoi la créatine est liée au THS ?}

Il n'y a aucun lien ! Par contre c'est un sujet sur lequel je radote souvent parce que je pense que c'est une bonne chose et je suis fatiguée de me répéter parce que les gens continuent de me poser des questions à ce sujet et tu lis déjà ça de toutes façon non ? J'adore une audience captive. Ma routine de stand-up est en bas.

\subsection{Okay d'accord pour quoi je devrait prendre de la crétine alors ?}

Quelle splendide question ! C'est bon pour ton cerveau et tes muscles. La créatine se retrouve sounvent en concentration relativement faibles pour la plupart des gens suivant leur régime alimentaire, surtout pour les personnes qui ne mangent pas de viande. Il y des éléments de recherche convaincants qui lient diverses conditions de fatigue chronique ou d'état post-infection virale (COVID long en particulier) avec des réserves vides de créatine dans le cerveau, pour cela certaines personne observent des améliorations des fonctions congnitives en se supplémentant. Ce n'est pas de la magie mais c'est vraiment pas cher donc ça vaut le coup d'essayer à mon avis.

\subsection{Sous quelle forme ?}

Tu veux juste de la poudre de \textit{créatine monohydrate}. Les pilules ont en général des dosage faibles et se font payer plus de toutes façon, de leur côté les bonbons gélifiés détruisent sonvent la crétine lors de leur création. Beaucoup de marques ont de la créatine dans divers mélanges mais la version pure est souvent moins chère.

\subsection{Comment je la prend du coup ?}

La recommendation habituelle est de prendre 5-10g chaque jour en la dissolvant dans un liquide. Elle se dissout mieux dans les choses qui ne sont pas que de l'eau. C'est quasiment sans goût donc tu peux juste mettre une portion ou deux dans ton café ou ton smoothie. Il se peut de qu'en buvant il y ait une texture sableuse ou granuleuse suivant la quantité de poudre et le liquide utilisé. 

\subsection{Est-ce-que c'est important quand je la prend ?}

Pas vraiment. Il n'y a pas d'effet immédiat qui pourrait le justifier ce qui rend le fait que ce soit micro-dosé dans des mélange pré-entraînement assez marrant. Prend la quand c'est le plus pratique pour toi.

\subsection{Comment ça marche alors ?}

Cela s'accumule dans ton corps jusqu'à atteindre un niveau maximum de saturation après une ou deux semaines. Il suffit ensuite de maintenir cet état et profiter des bénéfices (de peut être te sentir mieux).

\subsection{Est-ce-que j'ai besoin de faire une phase de "charge" où j'en prend plus au début ?}

Probablement non. Si tu n'es pas dans une période d'entraînement intense ou un truc du genre ça ne changera rien. Contente toi de la prendre quand c'est le plus pratique pour toi de manière régulière.

\subsection{Quels sont les effets secondaires ?}

Il se peut que tu observes un légère augmentation de poids à cause de l'augmentation du poids de l'eau dans tes muscles (ce qui, soyons clair, est un bonne chose donc pas besoin de s'alarmer). Si tu ne la prend pas dans de l'eau ou si tu en prend trop d'un coup tu aura peut être des maux d'estomac. Ouille.
Slight weight gain may be possible because of increased water weight in your muscles (which to be clear is Good, so don't be alarmed). If you don’t take it with water, or if you take too much at once, you might get a tummy ache. Ouchie.

\subsection{Qui ne devrait pas en prendre ?}

Les gens avec des problèmes de foie. Pas parce que ça en cause mais parce que la créatinine (l'orthographe est différente ! la crétine de vient de la créatinine) est utilisé dans les résultats d'analyses comme marqueur pour un éventail de problème du foie et prendre ce supplément peut créer des faux positifs. Important à garder en tête.

\subsection{Est-ce-qye tu as une marques que tu recommandes ?}

No. Celle que tu prends ne devrait pas avoir d'importance. Prend un truc qui semble avoir une bonne réputation avec un prix raisonnable. Je pourrait recommander celle que j'aime bien mais quand j'ai demandé à la marque un lien d'affiliation elle n'a pas voulu, tant pis pour eux ! Pas de publicité gratuite. 

\subsection{T'as vraiment mis la créatine dans ce document hein ?}

Ouais c'est assez marrant. C'est pas ma faute si j'en ai parlé comme ça et que des gens m'ont dit que ça les avait vraiment aidé, donc maintenant je me sens obligée de continuer d'en parler !!!

 

\section{OBSERVATIONS FINALES}

Si n'importe quelle des affirmation ci-dessous est toujours vraie :

\begin{itemize}
\item tu m'en veux toujours malgré les avertissements;

\item tu as remarqué un problème ou une faute de frappe;

\item tu as une question dont la clarification devrait être mise dans le texte;

\item tu as une objection qui je l'espère n'est pas une Euh Alors En Fait;

\item tu veux chanter mes louanges;

\item tu veux me jurer loyauté; 

\item tu veux m'envoyer une dîme;
\end{itemize}

Alors tu peux me contacter et je verrais ce que je peux faire. Bluesky est l'endroit où ce sera le plus facile, et tu peux me MP pour mon Signal. Sinon, merci d'avoir lu et j'espère que cela a été utile.

\textbf{Si tu veux donner pour soutenir ce projet,} \href{https://cash.app/Katitties}{CashApp}, \href{https://ko-fi.com/katitties}{Ko-Fi}, et \href{https://account.venmo.com/u/katitties}{Venmo} sont disponibles. Merci beaucoup !

Et finalement : \textbf{la chose la plus important que tu puisse faire en tant que personne trans c'est de vivre.} Ce document est tout autant un manuel qu'il est un message à toi en tant que personne trans pour te dire que ton existence est un don pour ce monde, ta présence est une bénédiction pour celleux autour de toi, et tu mérites d'être traîtée avec respect. Même si tu ne fais rien d'autre ta vie est une prouesse digne d'éloges. Merci.



\section*{AMIS DE PGHRT}\label{FOPGHRT}
\addcontentsline{toc}{section}{AMIS DE PGHRT}

Tout au long de ce document sont éparpillés des liens vers d'autres guides et ressources. La liste ci-dessous est une consolidation de ce liens avec quelques autres vers des ressources externes qui seront ajoutés au fur et à mesure, idéallement par d'autres personnes trans. Pour les personnes sensibles à leur vie privée ou qui veulent absolument éviter : il y a des liens Google Docs dans la liste.

\begin{enumerate}
  \item \href{https://startwith4mgestradiolenanthateweeklyandtestatonetothreemonths.com/}{SW4EEWATAOTTM} - TL;DR de PGHRT
  \item \href{https://hrtcafe.net/}{HRT Cafe} - Aggrégateur de Ressources THS
  \item \href{https://transfemscience.org/}{Transfeminine Science} - Ressource d'information pour la litérature médicale trans
  \item \href{http://estrannai.se}{Estrannai.se} - Bac à sable de Pharmacocinétique de l'Estradiol
  \item \href{https://globoho.moe/}{Globoho.moe} - Guide de Tourisme Médical pour l'Orchiectomie en Thaïlande
  \item Guide sur le FUE de Julia - ARRIVE BIENTÔT, JE L'EMBÊTE POUR QU'ELLE ÉCRIVE POUR PLUS VITE
  \item \href{https://docs.google.com/document/d/1-NyE5EY5TTaRRMhk7HlTbKJ7HifjEsA4jlDO1qKQVl0/edit?tab=t.0}{Sky's Feminine Figure Beginner Program} - Un programme d'exercises physique à destination des trans fems
  \item \href{https://docs.google.com/document/d/114sztSw1aVWM2pXLDl9NrHklyvewz3EmFiHiisjM71k/edit?tab=t.0}{Sky's Diet 101} - Un guide pour ajuster son poids d'un manière saine
  \item \href{https://stainedglasswoman.substack.com/p/how-to-maintain-your-penis-function}{How to Maintain Erectile Function on HRT} - Une explication plus détaillée du phénomène "utilise la ou perd la"
  \item \href{hhttps://docs.google.com/document/d/1DXFxzN0XTudPZez_SO61fpqncRLPH_Be_QG_8Pcz9LU/edit?pli=1&tab=t.0}{Biohax Guide Googleslop Edition} - Guide de DIY Trans Masc
\end{enumerate}

\section*{À PROPOS DE L'AUTRICE}
\addcontentsline{toc}{section}{À PROPOS DE L'AUTRICE}

Katie Tightpussy est une autrice primée et une femme trans professionnelle avec près d'une décénie d'expérience dans le champs de transgenre. Ses succès incluent transifier son sexe par le biais d'une technique novatirce d'injections d'hormones transexuelles, être physiquement incapable de la fermer, et utiliser une ensemble très pratique d'hyperfixations dans leur relation avec la transbobulation des humeurs. Elle passes ses journée dans la campagne rurale idyllique de Los Angeles devisant de nouvelles manière d'atteindre la domination mondiale et se plait à faire du vélo. Les demande de renseignement des média peuvent s'addresser à son agent via \href{http://katietightpussy.com}{katietightpussy.com}. 


 

\section*{DIVULGATIONS}
\addcontentsline{toc}{section}{DIVULGATIONS}

Aucune meuf robot n'a été blessée ou endommagée lors de la réalisation de ce document, cela inclut tout utilisation de modèles de langage génératif (LLM). L'autrice n'approuve aucune reproduction qui serait fait sans attribution ainsi que tout grattage de ce contenu. Laissez ces pauvres robots tranquille.

L'autrice déclare une attraction envers les femmes et reconnaît un possible conflit d'intérêt vis-à-vis de l'existence d'un plus grand nombre de femme trans magnifique dans le monde.

 

\section*{REMERCIEMENTS}
\addcontentsline{toc}{section}{REMERCIEMENTS}

Bien que ce texte soit mien il serait loin d'être aussi bon sans les contributions, retours et suggestion d'autres personnes impliquées tout au long de la rédaction. Il s'agit d'un rappel au combien important qu'une transition n'est que meilleure quand on est accompagné.

Merci beaucooup à Q, R, RM et S dans l'ordre alphabétique pour leurs revues précises et pour être des nerd sympa avec qui parler plus généralement ; je vous adore. Remerciements très spéciaux à CB et J pour leurs revues détaillées qui ont inspirés des passages marrants. Merci à KG pour les informations supplémentaires sur les personnes intersexes. Merci à w [sic] pour des ressources supplémentaires sur l'injection. Merci collectif à BIR pour la pléthore de pinaillages de nerd très importants. Appréciations pour les revues générales de C, JTP, K, S et V. Merci à toutes les personnes sur Bluesky qui m'ont encouragée à écrire ce guide et à tout le mond qui ont pu partager leurs connaissances au travers des années. Évidemment pour finir : merci aux nerds du THS, même quand nous avons des désaccords, car ce que nous voulons par dessus tout c'est le meilleur pour notre communauté trop souvent abandonnée. Continuons comme ça.

Dédicace à mon professeur d'IB Chemistry HL il y bien longtemps qui doutait de manière très raisonnable de mon assiduité alors que je met en application un très grande partie de ces connaissances au service de l'art de la transexualité ; va savoir pourquoi.



\section*{MISES À JOUR}
\addcontentsline{toc}{section}{MISES À JOUR}

\noindent \href{https://github.com/Juicysteak117/pghrt/}{Code source disponible sur ce dépot Github.}

\noindent Date de la dernière génération : \DTMnow

\noindent(Il n'y a pas de binding LaTeXML pour \texttt{datetime2}, \texttt{hanging}, ou \texttt{hyphenat}, donc le formattage est un peu moche. Si tu veux vraiment m'aider j'adorerais que tu écrives ces bindings !!!)

\noindent 2025-1?-??: Publication initiale en français. ?k mots.

\end{document}